 Gabriel Garc?a M?rquez Cien a?os de soledad Llovi? cuatro a?os, once meses y dos d?as. Hubo ?pocas de llovizna en que todo mundo se puso sus ropas de pon tifical y se compuso una cara de convaleciente para ce lebrar la escampada, pero pronto se acostumbraron a interpretar las pausas como anuncios de recrudecimiento. Se desempedraba el cielo en unas tempestades de estro picio, y el norte mandaba unos huracanes que desporti llaron techos y derribaron paredes, y desenterraron de ra?z las ?ltimas cepas de las plantaciones. Como ocurri? durante la peste del insomnio, que ?rsula se dio a recor dar por aquellos d?as, la propia calamidad iba inspirando defensas contra el tedio. Aureliano Segundo fue uno de los que m?s hicieron para no dejarse vencer por la ociosi dad. Hab?a ido a la casa por alg?n asunto casual la noche en que el se?or Brown convoc? la tormenta, y Fernanda trat? de auxiliarlo con un paraguas medio desvarillado que encontr? en un armario. "No hace fal ta", dijo ?l. "Me quedo aqu? hasta que escampe." No era, por supuesto, un compromiso ineludible, pero estuvo a punto de cumplirlo al pie de la letra. Como su ropa estaba en casa de Petra Cotes, se quitaba cada tres d?as la que llevaba puesta, y esperaba en calzoncillos mientras la lavaban. Para no aburrirse, se entreg? a la tarea de componer los numerosos desperfectos de la casa. Ajust? bisagras, aceit? cerraduras, atornill? aldabas y nivel? fallebas. Du rante varios meses se le vio vagar con una caja de herra mientas que debieron olvidar los gitanos en los tiempos de Jos? Arcadio Buend?a, y nadie supo si fue por la gimnasia involuntaria, por el tedio invernal o por la abs tinencia obligada, que la panza s? le fue desinflando poco a poco como un pellejo, y la cara de tortuga beat?fica se le hizo menos sangu?nea y menos protuberante la papada, hasta que todo ?l termin? por ser menos paqui d?rmico y pudo amarrarse otra vez los cordones de los zapatos. Vi?ndolo montar picaportes y desconectar re lojes, Fernanda se pregunt? si no estar?a incurriendo tambi?n en el vicio de hacer para deshacer, como el coronel Aureliano Buend?a con los pescaditos de oro, Amaranta con los botones y la mortaja, Jos? Arcadio Segundo con los pergaminos y rsula con los recuerdos. Pero no era cierto. Lo malo era que la lluvia lo trastor naba todo, y las m?quinas m?s ?ridas echaban flores por entre los engranajes si no se les aceitaba cada tres d?as, y se oxidaban los hilos de los brocados y le nac?an algas de azafr?n a la ropa mojada. La atm?sfera era tan h?meda, que los peces hubieran podido entrar pol las puertas y salir por las ventanas, navegando en el aire de los aposentos. Una ma?ana despert? ?rsula sin tiendo que se acababa en un soponcio de placidez, y ya hab?a pedido que le llevaran al padre Antonio Isabel, aunque fuera en andas, cuando Santa Sof?a de la Piedad descubri? que ten?a la espalda adoquinada de sanguijue las. Se las desprendieron una por una, achicharr?ndolas con tizones, antes de que terminaran de desangrarla. Fue necesario excavar canales para desaguar la casa, y des embarazarla de sapos y caracoles, de modo que pudieran secarse los pisos, quitar los ladrillos de las patas de las camas y caminar otra vez con zapatos. Entretenido con las m?ltiples minucias que reclamaban su atenci?n, Au reliano Segundo no se di? cuenta de que se estaba vol viendo viejo, hasta una tarde en que se encontr? con templando el atardecer prematuro desde un mecedor, y pensando en Petra Cotes sin estremecerse. No habr?a tenido ning?n inconveniente en regresar al amor ins?pido de Fernanda, cuya belleza se hab?a reposado con la ma durez, pero la lluvia lo hab?a puesto a salvo de toda emergencia pasional, y le hab?a infundido la serenidad esponjosa de la inapetencia. Se divirti? pensando en las cosas que hubiera podido hacer en otro tiempo con aquella lluvia que ya iba para un a?o. Hab?a sido uno de los primeros que llevaron l?minas de cinc a Macondo, mucho antes de que la compa??a bananera las pusiera de moda, s?lo por techar con ellas el dormitorio de Petra Cotes, y solazarse con la impresi?n de intimidad profun da que en aquella ?poca le produc?a la crepitaci?n de la lluvia. Pero hasta esos recuerdos locos de su juventud estrafalaria lo dejaban imp?vido, como si en la ?ltima parranda hubiera agotado sus cuotas de salacidad, y s?lo le hubiera quedado el premio maravilloso de poder evo carlas sin amargura ni arrepentimientos. Hubiera podido pensarse que el diluvio le hab?a dado la oportunidad de sentarse a reflexionar, y que el traj?n de los alicates y las alcuzas le hab?a despertado la a?oranza tard?a de tantos oficios ?tiles como hubiera podido hacer y no hizo en la vida, pero ni lo uno ni lo otro era cierto, porque la ten taci?n se sedentarismo y domesticidad que lo andaba rondando no era fruto de la recapacitaci?n ni el escar 6p. 6

 miento. Le llegaba de mucho m?s lejos, desenterrada por el trinche de la lluvia, de los tiempos en que le?a en el cuarto de Melqu?ades las prodigiosas f?bulas de los tapi ces volantes y las ballenas que se alimentaban de barcos con tripulaci?n. Fue por esos d?as que en un descuido de Fernanda apareci? en el corredor el peque?o Aureliano, y su abuelo conoci? el secreto de su identidad. Le cort? el pelo, lo visti?, lo ense?? a perderle el miedo a la gente, y muy pronto se vio que era un leg?timo Aureliano Buen d?a, con sus p?mulos altos, su mirada de asombro y su aire solitario. Para Fernanda fue un descanso. Hac?a tiempo que hab?a medido la magnitud de su soberbia, pero no encontraba c?mo remediarla, porque mientras m?s pensaba en las soluciones, menos racionales le pa rec?an. De haber sabido que Aureliano Segundo iba a tomar las cosas como las tom?, con una buena compla cencia de abuelo, no le habr?a dado tantas vueltas ni tantos plazos, sino que desde el a?o anterior se hubiera liberado de la mortificaci?n. Para Amaranta ?rsula, que ya hab?a mudado los dientes, el sobrino fue como un juguete providencial que la consol? del tedio de la lluvia. Aureliano Segundo se acord? entonces de la enciclopedia inglesa que nadie hab?a vuelto a tocar en el antiguo dor mitorio de Meme. Empez? por mostrarles las l?minas a los ni?os, en especial las de animales, y m?s tarde los mapas y las fotograf?as de pa?ses remotos y personajes c?lebres. Como no sab?a ingl?s, y como apenas pod?a distinguir las ciudades m?s conocidas y las personalidades m?s corrientes, se dio a inventar nombres y leyendas para satisfacer la curiosidad insaciable de los ni?os. Fernanda cre?a de veras que su esposo estaba esperan do a que escampara para volver con la concubina. En los primeros meses de la lluvia temi? que ?l intentara deslizarse hasta su dormitorio, y que ella iba a pasar por la verg?enza de revelarle que estaba incapacitada para la reconciliaci?n desde el nacimiento de Amaranta ?rsula. ?sa era la causa de su ansiosa correspondencia con los m?dicos invisibles, interrumpida por los frecuentes de sastres del correo. Durante los primeros meses, cuando se supo que los trenes se descarrilaban en la tormenta, una carta de los m?dicos invisibles le indic? que se estaban perdiendo las suyas. M?s tarde, cuando se in terrumpieron los contactos con sus corresponsales ignotos, hab?a pensado seriamente en ponerse la m?scara de tigre que us? su marido en el carnaval sangriento, para ha cerse examinar con un nombre ficticio por los m?dicos de la compa??a bananera. Pero una de las tantas per sonas que pasaban a menudo por la casa llevando las noticias ingratas del diluvio, le hab?a dicho que la com pa??a estaba desmantelando sus dispensarios para lle v?rselos a tierras de escampada. Entonces perdi? la es peranza. Se resign? a aguardar que pasara la lluvia y se normalizara el correo, y mientras tanto se aliviaba de sus dolencias secretas con recursos de inspiraci?n, porque hubiera preferido morirse a ponerse en manos del ?nico m?dico que quedaba en Macondo, el franc?s extravagan te que se alimentaba con hierba para burros. Se hab?a aproximado a ?rsula, confiado en que ella conociera al g?n paliativo para sus quebrantos. Pero la tortuosa cos tumbre de no llamar las cosas por su nombre, la llev? a poner lo anterior en lo posterior, y a sustituir lo parido por lo expulsado, y a cambiar flujos por ardores para que todo fuera menos vergonzoso, de manera que ?rsula concluy? razonablemente que los trastornos no eran ute rinos sino intestinales, y le aconsej? que tomara en ayu nas una papeleta de calomel. De no haber sido por ese padecimiento que nada hubiera tenido de pudendo para alguien que no estuviera tambi?n enfermo de pudibundez, y de no haber sido por la p?rdida de las cartas, a Fer nanda no le habr?a importado la lluvia, porque a fin de cuentas toda la vida hab?a sido para ella como si estu viera lloviendo. No modific? los horarios ni perdon? los ritos. Cuando todav?a estaba la mesa alzada sobre ladrillos y puestas las sillas sobre tablones para que los comensales no se mojaran los pies, ella segu?a sirviendo con manteles de lino y vajillas chinas, y prendiendo los candelabros en la cena, porque consideraba que las cala midades no pod?an tomarse de pretexto para el relaja miento de las costumbres. Nadie hab?a vuelto a asomar se a la calle. Si de Fernanda hubiera dependido no habr?a vuelto a hacerlo jam?s, no s?lo desde que empez? a llover sino desde mucho antes, puesto que ella consi deraba que las puertas se hab?an inventado para ce rrarlas, y que la curiosidad por lo que ocurr?a en la calle era cosa de rameras. Sin embargo, ella fue la primera en asomarse cuando avisaron que estaba pasando el en tierro del coronel Gerineldo M?rquez, aunque lo que vio entonces por la ventana entreabierta la dej? en tal estado de aflicci?n, que durante mucho tiempo estuvo arrepin ti?ndose de su debilidad. No hab?a podido concebirse un cortejo m?s desolado. Hab?an puesto el ata?d en una carreta de bueyes sobre la cual construyeron un cobertizo de hojas de banano, pero la presi?n de la lluvia era tan intensa y las calles estaban tan empantanadas, que a cada paso se atollaban las ruedas y el cobertizo estaba a punto de desbaratarse. Los chorros de agua triste que ca?an sobre el ata?d iban ensopando la bandera que le hab?an puesto encima, y que era en realidad la bandera sucia de sangre y de p?lvora, repudiada por los veteranos m?s dignos. S6bre el ata?d hab?an puesto tambi?n el sable con borlas de cobre y seda, el mismo que el coronel Gerineldo M?r quez colgaba en la percha de la sala para entrar inerme al costurero de Amaranta. Detr?s de la carreta, algunos descalzos y todos con los pantalones a media pierna, chapaleaban en el fango los ?ltimos sobrevivientes de la capitulaci?n de Neerlandia, llevando en una mano el bas t?n de car reto y en la otra una corona de flores de papel descoloridas por la lluvia. Aparecieron como una visi?n irreal en la calle que todav?a llevaba el nombre del co ronel Aureliano Buend?a, y todos miraron la casa al pasar, y doblaron por la esquina de la plaza, donde tu vieron que pedir ayuda para sacar la carreta atascada. ?rsula se hab?a hecho llevar a la puerta por Santa Sof?a de la Piedad. Sigui? con tanta atenci?n las peripecias del 7p. 7

 entierro, que nadie dud? de que lo estaba viendo, sobre todo porque su alzada mano de arc?ngel anunciador se mov?a con los cabeceos de la carreta. ?Adi?s, Gerineldo, hijo m?o ?grit??. Sal?dame a mi gente y dile que nos vemos cuando escampe. Aureliano Segundo la ayud? a volver a la cama, y con la misma informalidad con que la trataban siempre le pregunt? el significado de su despedida. ?Es verdad ?dijo ella?. Nada m?s estoy esperando que pase la lluvia para morirme. El estado de las calles alarm? a Aureliano Segundo. Tard?amente preocupado por la suerte de sus animales, se ech? encima un lienzo encerado y fue a casa de Petra Cotes. La encontr? en el patio, con el agua a la cintura, tratando de desencallar el cad?ver de un caballo. Aure liano Segundo la ayud? con una tranca, y el enorme cuerpo tumefacto dio una vuelta de campana y fue arras trado por el torrente de barro l?quido. Desde que empez? la lluvia, Petra Cotes no hab?a hecho m?s que desemba razar su patio de animales muertos. En las primeras semanas le mand? recados a Aureliano Segundo para que tomara providencias urgentes, y ?l hab?a contesta do que no hab?a prisa, que la situaci?n no era alarmante, que ya se pensar?a en algo cuando escampara. Le mand? a decir que los potreros se estaban inundando, que el ganado se fugaba hacia las tierras altas donde no hab?a qu? comer, y que estaban a merced del tigre y la peste. "No hay nada qu? hacer", le contest? Aureliano Segun do. "Ya nacer?n otros cuando escampe." Petra Cotes los hab?a visto morir a racimadas, y apenas si se daba abasto para destazar a los que se quedaban atollados. Vio con una impotencia sorda c?mo el diluvio fue extermi nando sin misericordia una fortuna que en un tiempo se tuvo como la m?s grande y s?lida de Macondo, y de la cual no quedaba sino la pestilencia. Cuando Aureliano Segundo decidi? ir a ver lo que pasaba, s?lo encontr? el cad?ver del caballo y una mula escu?lida entre los escombros de la caballeriza; Petra Cotes lo vio llegar sin sorpresa, sin alegr?a ni resenti miento, y apenas se permiti? una sonrisa ir?nica. ??A buena hora! ?dijo. Estaba envejecida, en los puros huesos, y sus lanceola dos ojos de animal carn?voro se hab?an vuelto tristes y mansos de tanto mirar la lluvia. Aureliano Segundo se qued? m?s de tres meses en su casa, no porque entonces se sintiera mejor all? que en la de su familia, sino por que necesit? todo ese tiempo para tomar la decisi?n de echarse otra vez encima el pedazo de lienzo encerado. "No hay prisa", dijo, como hab?a dicho en la otra casa. "Esperemos que escampe en las pr?ximas horas." En el curso de la primera semana se fue acostumbrando a los desgastes que hab?an hecho el tiempo y la lluvia en la salud de su concubina, y poco a poco fue vi?ndola como era antes, acord?ndose de sus desafueros jubilosos y de la fecundidad de delirio que su amor provocaba en los animales, y en parte por amor y en parte por inter?s, una noche de la segunda semana la despert? con caricias apremiantes. Petra Cotes no reaccion?. "Duerme tran quilo", murmur?. "Ya los tiempos no est?n para estas co sas." Aureliano Segundo se vio a s? mismo en los espe jos del techo, vio la espina dorsal de Petra Cotes como una hilera de carretes ensartados en un mazo de nervios marchitos, y comprendi? que ella ten?a raz?n, no por los tiempos, sino por ellos mismos, que ya no serv?an para esas cosas. Aureliano Segundo regres? a la casa con sus ba?les, convencido de que no s?lo ?rsula, sino todos los habi tantes de Macondo, estaban esperando que escampara pa ra morirse. Los hab?a visto al pasar, sentados en las salas con la mirada absorta y los brazos cruzados, sintiendo transcurrir un tiempo entero, un tiempo sin desbravar, porque era in?til dividirlo en meses y a?os, y los d?as en horas, cuando no pod?a hacerse nada m?s que contemplar la lluvia. Los ni?os recibieron alborozados a Aureliano Segundo, quien volvi? a tocar para ellos el acorde?n asm?tico. Pero el concierto no les llam? tanto la atenci?n como las sesiones enciclop?dicas, de modo que otra vez volvieron a reunirse en el dormitorio de Meme, donde la imaginaci?n de Aureliano Segundo convirti? el diri gible en un elefante volador que buscaba un sitio para dormir entre las nubes. En cierta ocasi?n encontr? un hombre de a caballo que a pesar de su atuendo ex?tico conservaba un aire familiar, y despu?s de mucho exami narlo lleg? a la conclusi?n de que era un retrato del coronel Aureliano Buend?a. Se lo mostr? a Fernanda, y tambi?n ella admiti? el parecido del jinete no s?lo con el coronel sino con todos los miembros de la familia, aunque en verdad era un guerrero t?rtaro. As? se le fue pasando el tiempo, entre el coloso de Rodas y los encantadores de serpientes, hasta que su esposa le anunci? que no que daban m?s de seis kilos de carne salada y un saco de arroz en el granero. ??Y ahora qu? quieres que haga? ?pregunt? ?l. 8p. 8
continuous tone image

 ?Yo no s? ?contest? Fernanda?. Eso es asunto de hombres. ?Bueno ?dijo Aureliano Segundo?, algo se har? cuando escampe. Sigui? m?s interesado en la enciclopedia que en el problema dom?stico, aun cuando tuvo que conformarse con una piltrafa y un poco de arroz en el almuerzo. "Ahora es imposible hacer nada", dec?a. "No puede llover toda la vida." Y mientras m?s largas le daba a las urgencias del granero, m?s intensa se iba haciendo la in dignaci?n de Fernanda, hasta que sus protestas eventua les, sus desahogos poco frecuentes, se desbordaron en un torrente incontenible, desatado, que empez? una ma?ana como el mon?tono bord?n de una guitarra, y que a me dida que avanzaba el d?a fue subiendo de tono, cada vez m?s rico, m?s espl?ndido. Aureliano Segundo no tuvo conciencia de la cantaleta hasta el d?a siguiente, despu?s del desayuno, cuando se sinti? aturdido por un abejorreo que era entonces m?s fluido y alto que el rumor de la lluvia, y era Fernanda que se paseaba por toda la casa doli?ndose de que la hu bieran educado como una reina para terminar de sirvien ta en una casa de locos, con un marido holgaz?n, id?latra, libertino, que se acostaba bocarriba a esperar que le llo vieran panes del cielo, mientras ella se destroncaba los r??ones tratando de mantener a flote un hogar empara petado con alfileres, donde hab?a tanto que hacer, tanto que soportar y corregir desde que amanec?a Dios hasta la hora de acostarse, que llegaba a la cama con los ojos llenos de polvo de vidrio, y sin embargo, nadie le hab?a dicho nunca buenos d?as Fernanda, qu? tal noche pasaste Fernanda, ni le hab?an preguntado aunque fuera por cor tes?a por qu? estaba tan p?lida ni por qu? despertaba con esas ojeras de violeta, a pesar de que ella no espera ba, por supuesto, que aquello saliera del resto de una familia que al fin y al cabo la hab?a tenido siempre como un estorbo, como el trapito de bajar la olla, como un monigote pintado en la pared, y que siempre andaban desbarrando contra ella por los rincones, llam?ndola san turrona, llam?ndola farisea, llam?ndola lagarta, y hasta Amaranta, que en paz descanse, hab?a dicho de viva voz que ella era de las que confund?an el recto con las t?m poras, bendito sea Dios, qu? palabras, y ella hab?a aguan tado todo con resignaci?n por las intenciones del Santo Padre, pero no hab?a podido soportar m?s cuando el mal vado de Jos? Arcadio Segundo dijo que la perdici?n de la familia hab?a sido abrirle las puertas a un cachaca, imag?nese, una cachaca mandona, v?lgame Dios, un ca chaca hija de la mala saliva, de la misma ?ndole de los cachacos que mand? el gobierno a matar trabajadores, d?game usted, y se refer?a nada menos que a ella, la ahi jada del Duque de Alba, una dama con tanta alcurnia que le revolv?a el h?gado a las esposas de los presidentes, una fijodalga de sangre como ella que ten?a derecho a firmar con once apellidos peninsulares, y que era el ?nico mortal en ese pueblo de bastardos que no se sent?a em berejenada frente a diecis?is cubiertos, para que luego el ad?ltero de su marido dijera muerto de risa que tantas cucharas y tenedores, y tantos cuchillos y cucharitas no era cosa de cristianos sino de ciempi?s, y la ?nica que pod?a determinar a ojos cerrados cu?ndo se serv?a el vino blanco, y de qu? lado y en qu? copa, y cu?ndo se serv?a el vino rojo, y de qu? lado y en qu? copa, y no como la montuna de Amaranta, que en paz descanse, que cre?a que el vino blanco se serv?a de d?a y el vino rojo de noche, y la ?nica en tocio el litoral que pod?a vana gloriarse de no haber hecho del cuerpo sino en bacinillas de oro, para que luego el coronel Aureliano Buend?a, que en paz descanse, tuviera el atrevimiento de preguntar con su mala bilis de mas?n de d?nde hab?a merecido ese privilegio, si era que ella no cagaba mierda sino astro melias, imag?nese, con esas palabras, y para que Renata, su propia hija, que por indiscreci?n hab?a visto sus aguas mayores en el dormitorio, contestara que de verdad la bacinilla era de mucho oro y de mucha her?ldica, pero que lo que ten?a dentro era pura mierda, mierda f?sica, y peor todav?a que las otras porque era mierda de ca chaca, imag?nese, su propia hija, de modo que nunca se hab?a hecho ilusiones con el resto de la familia, pero de todos modos ten?a derecho a esperar un poco de m?s consideraci?n de parte de su esposo, puesto que bien o mal era su c?nyuge de sacramento, su autor, su leg?timo perjudicador, que se ech? encima por voluntad libre y soberana la grave responsabilidad de sacarla del solar paterno, donde nunca se priv? ni se doli? de nada, donde tej?a palmas f?nebres por gusto de entretenimiento, puesto que su padrino hab?a mandado una carta con su firma y el sello de su anillo impreso en el lacre, s?lo para decir que las manos de su ahijada no estaban hechas para me nesteres de este mundo, como no fuera tocar el clavicor dio, y sin embargo el insensato de su marido la hab?a sacado de su casa con todas las admoniciones y adverten cias y la hab?a llevado a aquella paila de infierno donde no se pod?a respirar de calor, y antes de que ella acabara de guardar sus dietas de Pentecost?s ya se hab?a ido con sus ba?les trashumantes y su acorde?n de perdulario a holgar en adulterio con una desdichada a quien bastaba con verle las nalgas, bueno, ya estaba dicho, a quien 9p. 9
continuous tone image

 bastaba con verle menear las nalgas de potranca para adivinar que era una, que era una, todo lo contrario de ella que era una dama en el palacio o en la pocilga, o en la mesa o en la cama, una dama de naci?n, temerosa de Dios, obediente de sus leyes y sumisa a sus designios, y con quien no pod?a hacer, por supuesto, las maromas y vagabundinas que hac?a con la otra, que por supuesto se prestaba a todo, como las matronas francesas, y peor a?n pens?ndolo bien, porque ?stas al menos ten?an la honradez de poner un foco colorado en la puerta, semejantes por quer?as, imag?nese, ni m?s faltaba, con la hija ?nica y bienamada de do?a Renata Argote y don Fernando del Carpio, y sobre todo de ?ste, por supuesto, un santo var?n, un cristiano de los grandes, Caballero de la Orden del Santo Sepulcro, de ?sos que reciben directamente de Dios el privilegio de conservarse intactos en la tumba, con la piel tersa como raso de novia y los ojos vivos y di?fanos como las esmeraldas. ?Eso s? no es cierto ?la interrumpi? Aureliano Se gundo?. Cuando lo trajeron ya apestaba. Hab?a tenido la paciencia de escucharla un d?a entero, hasta sorprenderla en una falta. Fernanda no le hizo caso, pero baj? la voz. Esa noche, durante la cena, el exasperante zumbido de la cantaleta hab?a derrotado al rumor de la lluvia. Aureliano Segundo comi? muy poco, con la cabeza baja, y se retir? temprano al dormitorio. En el desayuno del d?a siguiente Fernanda estaba tr?mu la, con aspecto de haber dormido mal, y parec?a des ahogada por completo de sus rencores. Sin embargo, cuando su marido pregunt? si no ser?a posible comerse un huevo tibio, ella no contest? simplemente que desde la semana anterior se hab?an acabado los huevos, sino que elabor? una virulenta diatriba contra los hombres que se pasaban el tiempo ador?ndose el ombligo y luego te n?an la cachaza de pedir h?gados de alondra en la mesa. Aureliano Segundo llev? a los ni?os a ver la enciclopedia, como siempre, y Fernanda fingi? poner orden en el dor mitorio de Meme, s?lo para que ?l la oyera murmurar que por supuesto, se necesitaba tener la cara dura para decirles a los pobres inocentes que el coronel Aureliano Buend?a estaba retratado en la enciclopedia. En la tarde, mientras los ni?os hac?an la siesta, Aureliano Segundo se sent? en el corredor, y hasta all? lo persigui? Fernan da, provoc?ndolo, atorment?ndolo, girando en torno a ?l con su implacable zumbido de moscard?n, diciendo que por supuesto, mientras ya no quedaban m?s que piedras para comer, su marido se sentaba como un sult?n de Persia a contemplar la lluvia, porque no era m?s que eso, un mampol?n, un mantenido, un bueno para nada, m?s flojo que el algod?n de borla, acostumbrado a vivir de las mujeres. Aureliano Segundo la oy? m?s de dos horas, impasible, como si fuera sordo. No la interrumpi? hasta muy avanzada la tarde cuando no pudo soportar m?s la resonancia de bombo que le atormentaba la cabeza. ?C?llate ya, por favor ?suplic?. Fernanda, por el contrario, levant? el tono. "No tengo por qu? callarme", dijo. "El que no quiera o?rme que se vaya." Entonces Aureliano Segundo perdi? el dominio. Se incorpor? sin prisa, como si s?lo pensara estirar los huesos, y con una furia perfectamente regulada y met? dica fue agarrando uno tras otro los tiestos de begonias, las macetas de hel?chos, los potes de or?gano, y uno tras otro los fue despedazando contra el suelo. Fernanda se asust?, pues en realidad no hab?a tenido hasta entonces una conciencia clara de la tremenda fuerza interior de la cantaleta, pero ya era tarde para cualquier tentativa de rectificaci?n. Embriagado por el torrente incontenible del desahogo, Aureliano Segundo rompi? el cristal de la vidriera, y una por una, sin apresurarse, fue sacando las piezas de la vajilla y las hizo polvo contra el piso. Siste m?tico, sereno, con la misma parsimonia con que hab?a empapelado la casa de billetes, fue rompiendo luego con tra las paredes la cristaler?a de Bohemia, los floreros pintados a mano, los cuadros de las doncellas en barcas cargadas de rosas, los espejos de marcos dorados, y todo cuanto era rompible desde la sala hasta el granero, y ter min? con la tinaja de la cocina que se revent? en el centro del patio con una explosi?n profunda. Luego se lav? las manos, se ech? encima el lienzo encerado, y antes de me dia noche volvi? con unos tiesos colgajos de carne salada, varios sacos de arroz y ma?z con gorgojo, y unos desmi rriados racimos de pl?tano. Desde entonces no volvieron a faltar las cosas de comer. Amaranta ?rsula y el peque?o Aureliano hab?an de re cordar el diluvio como una ?poca feliz. A pesar del rigor de Fernanda, chapaleaban en los pantanos del patio, ca zaban lagartos para descuartizarlos y jugaban a envene nar la sopa ech?ndole polvo de alas de mariposas en los descuidos de Santa Sof?a de la Piedad. ?rsula era su juguete m?s entretenido. La tuvieron por una gran mu ?eca decr?pita que llevaban y tra?an por los rincones, disfrazada con trapos de colores y la cara pintada con holl?n y achiote, y una vez estuvieron a punto de destri parle los ojos como le hac?an a los sapos con las tijeras de podar. Nada les causaba tanto alborozo como sus des varios. En efecto, algo debi? ocurrir en su cerebro en el ?ltimo a?o de la lluvia, porque poco a poco fue perdien do el sentido de la realidad, y confund?a el tiempo actual con ?pocas remotas de su vida, hasta el punto de que en una ocasi?n pas? tres d?as llorando sin consuelo por la muerte de Petronila Iguar?n, su bisabuela, enterrada des de hac?a m?s de un siglo. Se hundi? en un estado de confusi?n tan disparatado, que cre?a que el peque?o Aure liano era su hijo el coronel por los tiempos en que lo llevaron a conocer el hielo, y que el Jos? Arcadio que estaba entonces en el seminario era el primog?nito que se fue con los gitanos. T?nto habl? de la familia, que los ni?os aprendieron a organizarle visitas imaginarias con seres que no s?lo hab?an muerto desde hac?a mucho tiempo, sino que hab?an existido en ?pocas distintas. Sentada en la cama con el pelo cubierto de ceniza y la cara tapada con un pa?uelo rojo, ?rsula era feliz en me dio de la parentela irreal que los ni?os describ?an sin omisi?n de detalles, como si de verdad la hubieran co nocido. ?rsula conversaba con sus antepasados sobre acontecimientos anteriores a su propia existencia, gozaba 10p. 10

 con las noticias que le daban y lloraba con ellos por muertos mucho m?s recientes que los mismos contertu lios. Los ni?os no tardaron en advertir que en el curso de esas visitas fantasmales ?rsula planteaba siempre una pregunta destinada a establecer qui?n era el que hab?a llevado a la casa durante la guerra un San Jos? de yeso de tama?o natural para que lo guardaran mientras pa saba la lluvia. Fue as? como Aureliano Segundo se acor d? de la fortuna encerrada en alg?n lugar que s?lo ?rsula conoc?a, pero fueron in?tiles las preguntas y las maniobras astutas que se le ocurrieron, porque en los laberintos de su desvar?o ella parec?a conservar un mar gen de lucidez para defender aquel secreto, que s?lo ha b?a de revelar a qui?n demostrara ser el verdadero due ?o del oro sepultado. Era tan h?bil y tan estricta, que cuando Aureliano Segundo instruy? a uno de sus com pa?eros de parranda para que se hiciera pasar por el propietario de la fortuna, ella lo enred? en un interro gatorio minucioso y sembrado de trampas sutiles. Con vencido de que ?rsula se llevar?a el secreto a la tumba, Aureliano Segundo contrat? una cuadrilla de excavadores con el pretexto de que construyeran canales de desag?e en el patio y en el traspatio, y ?l mismo sonde? el suelo con barretas de hierro y con toda clase de detectores de metales, sin encontrar nada que se pareciera al oro en tres meses de exploraciones exhaustivas. M?s tarde recurri? a Pilar Ternera con la experanza de que las barajas vie ran m?s que los cavadores, pero ella empez? por expli carle que era in?til cualquier tentativa mientras no fue ra ?rsula quien cortara el naipe. Confirm? en cambio la existencia del tesoro, con la precisi?n de que eran siete mil doscientas catorce monedas enterradas en tres sacos de lona con jaretas de alambre de cobre, dentro de un c?rculo con un radio de ciento veintid?s metros, tomando como centro la cama de ?rsula, pero advirti? que no se r?a encontrado antes de que acabara de llover y los soles de tres junios consecutivos convirtieran en polvo los ba rrizales. La profusi?n y la meticulosa vaguedad de los datos le parecieron a Aureliano Segundo tan semejantes a las f?bulas espiritistas, que insisti? en su empresa a pesar de que estaban en agosto y habr?a sido necesario esperar por lo menos tres a?os para satisfacer las condi ciones del pron?stico. Lo primero que le caus? asombro, aunque al mismo tiempo aument? su confusi?n, fue el comprobar que hab?a exactamente ciento veintid?s me tros de la cama de rusula a la cerca del traspatio. Fer nanda temi? que estuviera tan loco como su hermano gemelo cuando lo vio haciendo las mediciones, y peor a?n cuando orden? a las cuadrillas de excavadores pro fundizar un metro m?s en las zanjas. Presa de un delirio exploratorio comparable apenas al del bisabuelo cuando buscaba la ruta de los inventos, Aureliano Segundo perdi? las ?ltimas bolsas de grasa que le quedaban, y la antigua semejanza con el hermano gemelo se fue otra vez acen tuando, no s?lo por el escurrimiento de la figura, sino por el aire distante y la actitud ensimismada. No volvi? a ocuparse de los ni?os. Com?a a cualquier hora, emba rrado de pies a cabeza, y lo hac?a en un rinc?n de la cocina, contestando apenas a las preguntas ocasionales de Santa Sof?a de la Piedad. Vi?ndolo trabajar en aque lla forma, como nunca so?? que pudiera hacerlo, Fer nanda crey? que su temeridad era diligencia, y que sup. [11]
continuous tone image

 codicia era abnegaci?n y que su tosudez era perseveran cia, y le remordieron las entra?as por la virulencia con que hab?a despotricado contra su desidia. Pero Aure liano Segundo no estaba entonces para reconciliaciones misericordiosas. Hundido hasta el cuello en una ci?naga de ramazones muertas y flores podridas, volte? al dere cho y al rev?s el suelo del jard?n despu?s de haber ter minado con el patio y el traspatio, y barren? tan pro fundamente los cimientos de la galer?a oriental de la casa, que una noche despertaron aterrorizados por lo que parec?a ser un cataclismo, t?nto por las trepidaciones como por el pavoroso crujido subterr?neo, y era que tres aposentos se estaban desbarrancando y se hab?a abierto una grieta de escalofr?o desde el corredor hasta el dor mitorio de Fernanda. Aureliano Segundo no renunci? por eso a la exploraci?n. Aun cuando ya se hab?an extinguido las ?ltimas esperanzas y lo ?nico que parec?a tener alg?n sentido eran las predicciones de las barajas, reforz? los cimientos mellados, resan? la grieta con argamasa, y con tinu? excavando en el costado occidental. All? estaba to dav?a la segunda semana del junio siguiente, cuando la lluvia empez? a apaciguarse y las nubes se fueron alzan do, y se vio que de un momento a otro iba a escampar. As? fue. Un viernes a las dos de la tarde se alumbr? el mundo con un sol bobo, bermejo y ?spero como polvo de ladrillo, y casi tan fresco como el agua, y no volvi? a llover en diez a?os. Macondo estaba en ruinas. En los pantanos de las ca lles quedaban muebles despedazados, esqueletos de ani males cubiertos de lirios colorados, ?ltimos recuerdos de las hordas de advenedizos que se fugaron de Macondo tan atolondradamente como hab?an llegado. Las casas paradas con t?nta urgencia durante la fiebre del banano, hab?an sido abandonadas. La compa??a bananera des mantel? sus instalaciones. De la antigua ciudad alambra da s?lo quedaban los escombros. Las casas de madera, las frescas terrazas donde transcurr?an las serenas tardes de naipes, parec?an arrasadas por una anticipaci?n del vien to prof?tico que a?os despu?s hab?a de borrar a Ma condo de la faz de la tierra. El ?nico rastro humano que dej? aquel soplo voraz, fue un guante de Patricia Brown en el autom?vil carcomido por las trinitarias. La regi?n encantada que explor? Jos? Arcadio Buend?a en los tiem pos de la fundaci?n, y donde luego prosperaron las plantaciones de banano, era un tremedal de cepas putre factas, en cuyo horizonte remoto se alcanz? a ver por varios a?os la espuma silenciosa del mar. Aureliano Se gundo padeci? una crisis de aflicci?n el primer domingo en que visti? ropas secas y sali? a reconocer el pueblo. Los sobrevivientes de la cat?strofe, los mismos que ya viv?an en Macondo antes de que fuera sacudido por el hurac?n de la compa??a bananera, estaban sentados en mitad de la calle gozando de los primeros soles. Todav?a conservaban en la piel el verde de alga y el olor de rin c?n que les imprimi? la lluvia, pero en el fondo de sus corazones parec?an satisfechos de haber recuperado el pueblo en que nacieron. La Calle de los Turcos era otra vez la de antes, la de los tiempos en que los ?rabes de pantuflas y argollas en las orejas, que recorr?an el mun do cambiando guacamayas por chucher?as, hallaron en Macondo un buen recodo para descansar de su milenaria condici?n de gente transhumante. Al otro lado de la llu via, la mercanc?a de los bazares estaba cay?ndose a pe dazos, los g?neros abiertos en la puerta estaban veteados de musgo, los mostradores socavados por el comej?n y las paredes carcomidas por la humedad, pero los ?rabes de la tercera generaci?n estaban sentados en el mismo lugar y en la misma actitud de sus padres y sus abuelos, taci turnos, imp?vidos, invulnerables al tiempo y al desastre, tan vivos o tan muertos como estuvieron despu?s de la peste del insomnio y de las treinta y dos guerras del co ronel Aureliano Buend?a. Era tan asombrosa su fortaleza de ?nimo frente a los escombros de las mesas de juego, los puestos de fritangas, las casetas de tiro al blanco y el callej?n donde se interpretaban los sue?os y se adivi naba el porvenir, que Aureliano Segundo les pregunt? con su informalidad habitual de qu? recursos misteriosos se hab?an valido para no naufragar en la tormenta, c?mo diablos hab?an hecho para no ahogarse, y uno tras otro, de puerta en puerta, le devolvieron una sonrisa ladina y una mirada de ensue?o, y todos le dieron sin ponerse de acuerdo la misma respuesta: ?Nadando. Petra Cotes era tal vez el ?nico nativo que ten?a co raz?n de ?rabe. Hab?a visto los ?ltimos destrozos de sus establos y caballerizas arrastrados por la tormenta, pero hab?a logrado mantener la casa en pie. En el ?ltimo a?o le hab?a mandado recados apremiantes a Aureliano Se gundo, y ?ste le hab?a contestado que ignoraba cu?ndo volver?a a su casa pero que en todo caso llevar?a un ca j?n de monedas de oro para empedrar el dormitorio. Entonces ella hab?a escarbado en su coraz?n, buscando fuerza que le permitiera sobrevivir a la desgracia, y ha b?a encontrado una rabia reflexiva y justa, con la cual hab?a jurado restaurar la fortuna despilfarrada por el amante y acabada de exterminar por el diluvio. Fue una decisi?n tan inquebrantable, que Aureliano Segundo vol vi? a su casa ocho meses despu?s del ?ltimo recado, y la encontr? verde, desgre?ada, con los p?rpados hundi dos y la piel escarchada por la sarna, pero estaba escri biendo n?meros en pedacitos de papel, para hacer una rifa. Aureliano Segundo se qued? at?nito, y estaba tan escu?lido y tan solemne, que Petra Cotes no crey? que quien hab?a vuelto a buscarla fuera el amante de toda la vida, sino el hermano gemelo. ?Est?s loca ?dijo ?l?. A menos que pienses rifar los huesos. Entonces ella le dijo que se asomara al dormitorio, y Aureliano Segundo vio la mula. Estaba con el pellejo pe gado a los huesos, como la due?a, pero tan viva y resuel ta como ella. Petra Cotes la hab?a alimentado con su rabia, y cuando no tuvo m?s hierba, ni ma?z, ni ra?ces, la alberg? en su propio dormitorio y le dio a comer las s?banas de percal, los tapices persas, los sobrecamas de peluche, las cortinas de terciopelo y el palio bordado con hilos de oro y borlones de seda de la cama episcopal. 12p. 12


