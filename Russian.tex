Франкенштейн, или Современный Прометей , ПИСЬМО ПЕРВОЕ 1 В Англию, м- Сзвилл, . , Санкт-Петербург, 11 дек. 17.. . , . 11th, 17 — Ты порадуешься, когда услышишь, что предприятие, вызывавшее у тебя столь мрачные предчувствия, началось вполне благоприятно. . Я прибыл сюда вчера; и спешу прежде всего заверить мою милую сестру, что у меня все благополучно и что я все более убеждаюсь в счастливом исходе моего дела. , . Я нахожусь уже далеко к северу от Лондона; прохаживаясь по улицам Петербурга, я ощущаю на лице холодный северный ветер, который меня бодрит и радует. , , , . Поймешь ли ты это чувство? ? Ветер, доносящийся из краев, куда я стремлюсь, уже дает мне предвкушать их ледяной простор. , , . Под этим ветром из обетованной земли мечты мои становятся живее и пламенней. , . Тщетно стараюсь я убедить себя, что полюс – это обитель холода и смерти; он предстает моему воображению как царство красоты и радости. ; . Там, Маргарет, солнце никогда не заходит; его диск, едва подымаясь над горизонтом, излучает вечное сияние. , , , . Там – ибо ты позволишь, мне хоть несколько доверять бывалым мореходам – кончается власть мороза и снега, и по волнам спокойного моря можно достичь страны, превосходящей красотою и чудесами все страны, доныне открытые человеком. — , , — ; , , . Природа и богатства этой неизведанной страны могут оказаться столь же диковинными, как и наблюдаемые там небесные явления. , . Чего только нельзя ждать от страны вечного света! ? Там я смогу открыть секрет дивной силы, влекущей к себе магнитную стрелку; а также проверить множество астрономических наблюдений; одного такого путешествия довольно, чтобы их кажущиеся противоречия раз навсегда получили разумное объяснение. . Я смогу насытить свое жадное любопытство зрелищем еще никому не ведомых краев и пройти по земле, где еще не ступала нога человека. , . Вот что влечет меня – побеждая страх перед опасностью и смертью и наполняя меня, перед началом трудного пути, той радостью, с какой ребенок вместе с товарищами своих игр плывет в лодочке по родной реке, на открытие неведомого. , , , . Но если даже все эти надежды не оправдаются, ты не можешь отрицать, что я окажу неоценимую услугу человечеству, если хотя бы проложу северный путь в те края, куда ныне нужно плыть долгие месяцы, или открою тайну магнита, – ведь если ее вообще можно открыть, то лишь с помощью подобного путешествия. , , , , ; , , , . Эти размышления развеяли тревогу, с какой я начал писать тебе, и наполнили меня возвышающим душу восторгом, ибо ничто так не успокаивает дух, как обретение твердой цели – точки, на которую устремляется наш внутренний взор. , , — . Эта экспедиция была мечтой моей юности. . Я жадно зачитывался книгами о путешествиях, предпринятых в надежде достичь северной части Тихого океана по полярным морям. . Ты, вероятно, помнишь, что истории путешествий и открытий составляли всю библиотеку нашего доброго дядюшки Томаса. ' . Образованием моим никто не занимался; но я рано пристрастился к чтению. , . Эти тома я изучал днем и ночью и все более сожалел, что мой отец, как я узнал, еще будучи ребенком, перед смертью строго наказал моему дяде не пускать меня в море. , , , ' . Мечты о море поблекли, когда я впервые познакомился с творениями поэтов, восхитившими мою душу и вознесшими ее к небесам. , , . Я сам стал поэтом и целый год прожил в Эдеме, созданном моей фантазией. Я вообразил, что и мне суждено место в храме, посвященном Гомеру и Шекспиру. ; . Тебе известна постигшая меня неудача и то, как тяжело я пережил это разочарование. . Но как раз в то время я унаследовал состояние нашего кузена, и мысли мои вновь обратились к мечтам моего детства. , . Вот уже шесть лет, как я задумал свое нынешнее предприятие. . Я до сих пор помню час, когда решил посвятить себя этой великой цели. , , . Прежде всего я начал закалять себя. . Я сопровождал китоловов в северные моря; я довольно подвергал себя холоду, голоду, жажде и недосыпанию. Днем я часто работал больше матросов, а по ночам изучал математику, медицину и те области физических наук, которые более всего могут понадобиться мореходу. - ; , , , ; , , . Я дважды нанимался подшкипером на гренландские китобойные суда и отлично справлялся с делом. - , . Должен признаться, что я почувствовал гордость, когда капитан предложил мне место своего первого помощника и долго уговаривал меня согласиться: так высоко он оценил мою службу. , . А теперь скажи, милая Маргарет: неужели я не достоин свершить нечто великое? , , ? Моя жизнь могла бы пройти в довольстве и роскоши; но всем соблазнам богатства я предпочел славу. , . О, если б прозвучал для меня чей-нибудь ободряющий голос! , ! Мужество и решение мои непоколебимы; но надежда и бодрость временами мне изменяют. ; , . Я отправляюсь в долгий и трудный путь, где потребуется вся моя стойкость. Мне надо будет не только поддерживать бодрость в других, во иногда и в себе, когда все остальные падут духом. , : , , . Сейчас лучшее время года для путешествия по России. . Здешнее сани быстро несутся по снегу; этот способ передвижения приятен и, по-моему, много удобнее английской почтовой кареты. ; , , , . Холод не страшен, если ты закутан в меха; такой одеждой я уже обзавелся, ибо ходить по палубе – совсем не то, что часами сидеть на месте, не согревая кровь движением. , — , , . Я вовсе не намерен замерзнуть на почтовом тракте между Петербургом и Архангельском. - . . В последний из названных мной городов я отправлюсь через две-три недели; и там думаю нанять корабль, а это легко сделать, уплатив за владельца страховую сумму; хочу также набрать нужное мне число матросов из тех, кто знаком с китоловным промыслом. ; , , -. Я думаю пуститься в плавание не раньше июня, а когда возвращусь? ; ? Ах, милая сестра, что могу я ответить на это? , , ? В случае удачи мы не увидимся много месяцев, а может, и лет. , , , , . В случае неудачи ты увидишь меня скоро – или не увидишь никогда. , , . Прощай, моя милая, добрая Маргарет. , , . Пусть Бог благословит тебя и сохранит мне жизнь, чтобы я мог еще не раз отблагодарить тебя за твою любовь и заботу. , , . Любящий тебя брат. Р. , . Уолтон. * * * ПИСЬМО ВТОРОЕ 2 В Англию, м-с Сэвилл . , Архангельск, 28 март 17.. , 28th , 17 — Как долго тянется время для того, кто скован морозом и льдом! , ! Однако я сделал еще один шаг к моей цели. . Я нанял корабль и набираю матросов; те, кого я уже нанял, кажутся мне людьми надежными и, несомненно, отважными. ; . Мне не хватает лишь одного – не хватало всегда, но сейчас я ощущаю отсутствие этого как большое зло. У меня нет друга, Маргарет; никого, кто мог бы разделить со мною радость, если мне суждено счастье успеха; никого, кто поддержал бы меня, если я паду духом. , , , : , ; , . Правда, я буду поверять свои мысли бумаге; но она мало пригодна для передачи чувств. , ; . Мне нужно общество человека, когорый сочувствовал бы мне и понимал с полуслова. , . Ты можешь счесть меня излишне чувствительным, милая сестра, но я с горечью ощущаю отсутствие такого друга. , , . Возле меня нет никого с душою чуткой и вместе с тем бесстрашной, с умом развитым и восприимчивым; нет друга, который разделял бы мои стремления, мог одобрить мои планы или внести в них поправки. , , , , . Как много мог бы подобный друг сделать для исправления недостатков твоего бедного брата! ! Я излишне поспешен в действиях и слишком нетерпелив перед лицом препятствий. . Но еще большим злом является то, что я учился самоучкою: первые четырнадцать лет моей жизни я гонял по полям и читал одни лишь книги о путешествиях из библиотеки нашего дядюшки Томаса. -: ' . В этом возрасте я познакомился с прославленными поэтами моей страны; но слишком поздно убедился я в необходимости знать другие языки, кроме родного, – когда уже не мог извлечь из этого убеждения никакой истинной пользы. ; . Сейчас мне двадцать восемь, а ведь я невежественнее многих пятнадцатилетних школьников. - . Правда, я больше их размышлял и о большем мечтаю; но этим мечтам недостает того, что художники называют «соотношением», и мне очень нужен друг, достаточно разумный, чтобы не презирать меня как пустого мечтателя, и достаточно любящий, чтобы мною руководить. , ( ) ; , . Впрочем, все эти жалобы безполезны; какого друга могу я обрести на океанских просторах или даже здесь, в Архангельске, среди купцов и моряков? , ; , , . Правда, и им, при всей их внешней грубости, не чужды благородные чувства. , , . Мой помощник, например, человек на редкость отважный и предприимчивый; он страстно жаждет славы, или, вернее сказать, преуспеяния на своем поприще. , , ; , , , . Он родом англичанин и, при всех предрассудках своей нации и своего ремесла, не смягченных просвещением, сохранил немало благороднейших человеческих качеств. , , , . Я впервые встретил его на борту, китобойного судна: узнав, что у него сейчас нет работы, я легко склонил его к участию в моем предприятии. ; , . Капитан также отличный человек, выделяющийся среди всех мягкостью и кротостью в обращении. . Именно эти свойства, в сочетании с безупречной честностью и бесстрашием, и привлекли меня. , - , . Моя юность, проведенная в уединении, мои лучшие годы, прошедшие под твоей нежной женской опекой, настолько смягчили мой характер, что я испытываю неодолимое отвращение к грубости, обычно царящей на судах; я никогда не считал ее необходимой. Услыхав о моряке, известном как сердечной добротой, так и умением заставить себя уважать и слушаться, я счел для себя большой удачей заполучить его. , , : , , . Я впервые услышал о нем при довольно романтических обстоятельствах, от женщины, которая обязана ему своим счастьем. , . Вот, вкратце, его история. , , . Несколько лет назад он полюбил русскую девушку из небогатой семьи. Когда он скопил изрядную сумму наградных денег, отец девушки согласился на их брак. , -, . Перед свадьбой он встретился со своей невестой; но она упала к его ногам, заливаясь слезами и прося пощадить её; она призналась, что полюбила другого, а этот другой так беден, что отец ее никогда не дает согласия на брак. ; , , , , , . Мой великодушный друг успокоил ее и, справившись об имени ее возлюбленного, отступился от своих прав. , , . На скопленные им деньги он успел уже купить дом и участок, рассчитывая поселиться там навсегда, – все это, вместе с остатками наградных денег, он отдал своему сопернику на обзаведение и сам испросил у отца девушки согласия на ее брак с любимым. , ; , - , ' . Отец отказался, считая, что уже связан словом с моим другом; тогда тот, видя упорство старика, уехал и не возвращался до тех пор, пока девушка не вышла за избранника своего сердца. , , , , , . «Какое благородство!» –воскликнешь ты. " !" . Да, он именно таков, но при этом совершенно необразован, молчалив в угрюм, что, конечно, делает его поступок еще удивительнее, но вместе с тем уменьшает интерес и сочувствие, которые этот человек мог бы вызывать. ; : , , , , . Но если я порой жалуюсь и мечтаю о дружеской поддержке, которая мне, быть может, не суждена, не думай, что я колеблюсь в своем решении. , , . Оно неизменно, как сама судьба; и плавание мое отложено лишь потому, что к этому нас вынуждает погода. , . Зима была на редкость суровой; но весна обещает быть дружной и очень ранней; так что я, возможно, смогу отправиться в путь раньше, чем предполагал. , , , . Я ни в чем не хочу поступать опрометчиво; ты достаточно знаешь меня, чтобы быть уверенной в моей осмотрительности, когда мне вверена безопасность других. : . Не могу описать тебе моих чувств при мысли о скором отплытии. . Невозможно выразить трепетное чувство, радостное и вместе тревожное, с каким я готовлюсь в путь. , , . Я отправляюсь в неведомые земли, в «края туманов и снегов», но я не намерен убивать альбатроса, поэтому не бойся за меня, – я не вернусь к тебе разбитым и больным, как , " ," ; «Старый мореход». " ." Ты улыбнешься при этой аллюзии; но я открою тебе один секрет. , . Мою страстную тягу к опасным тайнам океана я часто приписывал влиянию этого творения наиболее поэтичного из современных поэтов. , , . В моей душе живет нечто непонятное мне самому. . Я практичен и трудолюбив, я старательный и терпеливый работник, но вместе с тем во все мои планы вплетается любовь к чудесному и вера в чудесное, увлекающие меня вдаль от проторенных дорог, в неведомые моря, в неоткрытые страны, которые я намерен исследовать. — , — , , , , . Вернусь, однако, к вещам, более близким сердцу. . Увижу ли я тебя вновь, когда пересеку безбрежные моря и вернусь назад мимо южной оконечности Африки или Америки? , , ? Не смею ожидать такой удачи, но не хочу думать и о другом. , . Пиши мне при каждой возможности; быть может, письма твои дойдут до меня, когда будут всего нужнее, и поддержат во мне мужество. : . Люблю тебя нежно. . Помни обо мне с любовью, если больше обо мне не услышишь. , . Любящий тебя брат Роберт Уолтон. , * * * ПИСЬМО ТРЕТЬЕ 3 В Англию, м-с Сэвилл . , 7 июля 17.. 7th, 17 — Дорогая сестра! , Пишу эти торопливые строки, чтобы сообщить, что я здоров в проделал немалый путь. — . Это письмо придет в Англию с торговым судном, которое сейчас отправляется из Архангельска; оно счастливее меня, который, быть может, еще много лет не увидит родных берегов. ; , , , . Тем не менее я бодр; мои люди отважны и тверды; их не страшат плавучие льдины, то и дело появляющиеся за бортом как предвестники ожидающих нас опасностей. , , : , , , . Мы достигли уже очень высоких широт; но сейчас разгар лета, и южные ветры, быстро несущие нас к берегам, которых я так жажду достичь, хотя и менее теплы, чем в Англии, но все же веют теплом, какого я здесь не ждал. ; , , , , . До сих пор с нами не произошло ничего настолько примечательного, чтобы об этом стоило писать. . Один-два шквала и пробоина в судне – все это такие события, которые не остаются в памяти опытных моряков; я буду рад, если с нами не приключится ничего хуже этого. , . До свидания, милая Маргарет. , . Будь уверена, что ради тебя и себя самого я не помчусь безрассудно навстречу опасности. , , . Я буду хладнокровен, упорен и благоразумен. , , . Но я все-таки добьюсь успеха. . Почему бы нет? ? Я уже пролагаю путь в неизведанном океане; самые звезды являются свидетелями моего триумфа. , , . Почему бы мне не пройти в дальше, в глубь непокоренной, но послушной стихии? ? Что может преградить путь отваге и воле человека? ? Переполняющие меня чувства невольно рвутся наружу; а между тем пора кончать. . . Да благословит небо мою милую сестру! ! Р. У. .. * * * ПИСЬМО ЧЕТВЕРТОЕ 4 В Англию, м-с Сэвилл . , 5 августа 17.. 5th, 17 — С нами случилось нечто до того странное, что я должен написать тебе, хотя мы, быть может, увидимся раньше, чем ты получишь это письмо. , . В прошлый понедельник (31 июля) мы вошли а область льда, который почти сомкнулся вокруг корабля, едва оставляя нам свободный проход. ( 31st) , , - . Положение наше стало опасным, в особенности из-за густого тумана. , . Поэтому мы легли в дрейф, надеясь на перемену погоды. , . Около двух часов дня туман рассеялся, и мы увидели простиравшиеся во все стороны обширные поля неровного льда, которым, казалось, нет конца. ' , , , , . У некоторых моих спутников вырвался стон, да и сам я ощутил тревогу; но тут наше внимание было привлечено странным зрелищем, заставившим нас забыть о своем положении. , , . Примерно в полумиле от нас мы увидели низкие сани, запряженные собаками и мчавшиеся к северу; в санях, управляя собаками, сидело существо подобное человеку, но гигантского роста. , , , ; , , . Мы следили в подзорные трубы за быстрым бегом саней, пока они не скрылись за ледяными холмами. . Это зрелище немало нас поразило. . Мы полагали, что находимся на расстоянии многих сотен миль от какой бы то ни было земли; но увиденное, казалось, свидетельствовало, что земля не столь уж далека. , , ; , , . Скованные льдом, мы не могли следовать за санями, но внимательно их рассмотрели. , , , , . Часа через два после этого события мы услыхали шум волн; к ночи лед вскрылся, и наш корабль был освобожден. , . Однако мы пролежали в дрейфе до утра, опасаясь столкнуться в темноте с огромными плавучими глыбами, которые появляются при вскрытии льда. , , , . Я воспользовался этими часами, чтобы отдохнуть. . Утром, едва рассвело, я поднялся на палубу и увидел, что все матросы столпились у борта, как видно, переговариваясь с кем-то, находящимся в море. , , , , . Оказалось, что к кораблю, на большой льдине, прибило сани, похожие на виденные нами накануне. , , , , . Из всей упряжки уцелела лишь одна собака, но в санях сидел человек, и матросы убеждали его подняться на борт. ; . Это не был туземец с некоего неведомого острова, каким нам показался первый путник; это был европеец. , , , . Когда я вышел на палубу, боцман сказал: , «Вот идет наш капитан, и он не допустит, чтобы вы погибли в море». " , ." Увидев меня, незнакомец обратился ко мне по-английски, хотя и с иностранным акцентом. , , . «Прежде чем я взойду на ваш корабль, – сказал он, – я прощу сообщить мне, куда он направляется». " ," , " ?" Нетрудно вообразить, мое изумление, когда я услыхал подобный вопрос от погибающего, который, казалось бы, должен был считать мое судно спасительным пристанищем, самым дорогим сокровищем в мире. . Я ответил, однако, что мы исследователи и направляемся к Северному полюсу. , , . Это, по-видимому, удовлетворило его, и он согласился взойти на борт. . Великий Боже! ! Если бы ты только видела, Маргарет, человека, которого еще пришлось уговаривать спастись, твоему удивлению не было бы границ. , , . Он был сильно обморожен и до крайности истощен усталостью и лишениями. , . Никогда я не видел человека в столь жалком состоянии. . Мы сперва отнесли его в каюту, но там, лишенный свежего воздуха, он тотчас же потерял сознание. , . Поэтому мы снова принесли его на палубу и привели в чувство, растирая коньяком; несколько глотков мы влили ему в рот. . Едва он подал признаки жизни, мы завернули его в одеяла и положили возле кухонной трубы. . Мало-помалу он пришел в себя и съел немного супу, который сразу его подкрепил. , . Прошло два дня, прежде чем он смог заговорить, и я уже опасался, что страдания лишили его рассудка. , . Когда он немного оправился, я велел перенести его ко мне я каюту и сам ухаживал за ним, насколько позволяли мне мои обязанности. , . Никогда я не встречал более интересного человека; обычно взгляд его дик и почти безумен; но стоит кому-нибудь ласково обратиться к нему или оказать самую пустячную услугу, как лицо его озаряется, точно лучом, благостной улыбкой, какой я ни у кого не видел. : , , , , , , . Однако большей частью он бывает мрачен и подавлен; а порой скрипит зубами, словно не может более выносить бремя своих страданий. , , . Когда мой гость немного оправился, мне стоило больших трудов удерживать матросов, которые жаждали расспросить его; я не мог допустить, чтобы его донимали праздным любопытством, когда тело и дух его явно нуждались в полном покое. , ; , . Однажды мой помощник все же спросил его: как он проделал столь длинный путь по льду, да еще в таком необычном экипаже? , , . Лицо незнакомца тотчас омрачилось, и он ответил: , , «Я преследовал беглеца». " ." – А беглец путешествует тем же способом? " ?" – Да. "." – В таком случае, мне думается, что мы его видели; накануне того дня, когда мы вас подобрали, мы заметили на льду собачью упряжку, а в санях – человека. " , , , ." Это заинтересовало незнакомца, и он задал нам множество вопросов относительно направления, каким отправился демон, как он его назвал. ' , , , . Немного погодя, оставшись со мной наедине, он сказал: , , , «Я наверняка возбудил ваше любопытство, как, впрочем, и любопытство этих славных людей, но вы слишком деликатны, чтобы меня расспрашивать». " , , , ; ." – Разумеется; бесцеремонно в жестоко было бы докучать вам расспросами. "; ." – Но ведь вы вызволили меня из весьма опасного положения и заботливо возвратили к жизни. " ; ." Затем он спросил, не могли ли те, другие, сани погибнуть при вскрытии льда. . Я ответил, что наверное этого знать нельзя, ибо лед вскрылся лишь около полуночи и путник мог к тому времени добраться до какого-либо безопасного места. , , ; . С тех пор в изнуренное тело незнакомца влились новые силы. . Он непременно хотел находиться на палубе и следить, не появятся ли знакомые нам сани. Однако я убедил его оставаться в каюте, ибо он слишком слаб, чтобы переносить мороз. ; , . Я пообещал, что мы сами последим за этим и немедленно сообщим ему, как только заметим что-либо необычное. . Вот что записано в моем судовом журнале об этом удивительном событии. . Здоровье незнакомца поправляется, но он очень молчалив и обнаруживает тревогу, если в каюту входит кто-либо, кроме меня. . Впрочем, он так кроток и вежлив в обращении, что все матросы ему сочувствуют, хотя и очень мало с ним общаются. , . Что до меня, то я уже люблю его, как брата; его постоянная, глубокая печаль несказанно огорчает меня. , , . В свои лучшие дни он, должно быть, был благородным созданием, если и сейчас, когда дух его сломлен, он так привлекает к себе. , . В одном из писем, милая Маргарет, я писал тебе, что навряд ли обрету друга на океанских просторах; и, однако ж, я нашел человека, которого был бы счастлив иметь своим лучшим другом, если б только его не сломило горе. , , ; , , . Я продолжу свои записи о незнакомце, когда будет что записать. , . * * * 13 августа 17.. 13th, 17 — Моя привязанность к гостю растет с каждым днем. . Он возбуждает одновременно безмерное восхищение и сострадание. . Да и можно ли видеть столь благородного человека, сраженного бедами, не испытывая самой острой жалости? ? Он так кроток и, вместе, так мудр; он так широко образован; а когда говорит, речь его поражает и беглостью и свободой, хотя он выбирает слова с большой тщательностью. , ; , , , . Сейчас он вполне оправился от своего недуга и постоянно находится на палубе, видимо, ожидая появления опередивших его саней. , . Хотя он и несчастен, во не настолько поглощен собственным горем, чтобы не проявлять живого интереса к нашим делам. , , . Он нередко обсуждает их со мной, и я вполне ему доверился. , . Он внимательно выслушал все моя доводы в пользу моего предприятия и входит во все подробности принятых мною мер. . Выказанное им участие подкупило меня, и я заговорил с ним языком сердца; высказал все, что переполняло мою душу, и горячо заверил его, что охотно пожертвовал бы состоянием, жизнью и всеми надеждами ради успеха задуманного дела. , , , , , , . Одна человеческая жизнь – сходная цена за те познания, к которым я стремлюсь, за власть над исконными врагами человечества. ' , . При этих моих словах чело моего собеседника омрачилось. , ' . Сперва я заметив, что он пытается скрыть свое волнение: он закрыл глаза руками; но когда между его пальцев заструились слезы, а из груди вырвался стон, я не мог продолжать. ; , ; . Я умолк – и он заговорил прерывающимся голосом: ; , : «Несчастный! " ! И ты, значит, одержим тем же безумием? ? И ты испил опьяняющего напитка? ? Так выслушай же меня, узнай мою повесть, и ты бросишь наземь чашу с ядом!» ; , !" Эта слова, как ты можешь себе представить, разожгли мое любопытство; но волнение, охватившее незнакомца, оказалось слишком сильным для его истощенного тела, и понадобились долгие часы отдыха и мирных бесед, прежде чем силы его восстановились. , , ; , . Справившись со своими: чувствами, он, казалось, презирал себя за то, что не совладал с собой. Преодолевая мрачное отчаяние, он вновь заговорил обо мне. , ; , . Ом пожелал услышать историю моей юности. . Рассказ о ней не занял много времени, но навел нас на размышления. , . Я поведал ему, как страстно я желаю иметь друга, как жажду более близкого общения с родственной душою, чем до сих пор выпало мне на долю, и убежденно заявил, что без этого дара судьбы человек не может быть счастлив. , , . – Я с вами согласен, – отвечал незнакомец, – мы остаемся как бы незавершенными, пока некто более мудрый и достойный, чем мы сами, – а именно таким должен быть друг, – не поможет нам бороться с нашими слабостями и пороками. " ," ; " , , , , — — . Я некогда имел друга, благороднейшего из людей, и потому способен судить о дружбе. , , , , . У вас есть надежды, перед вами – весь мир, и вам нечего отчаиваться. , , . А вот я – я все утратил и не могу начать жизнь снова. — ." При этих словах лицо его выразило тяжкое, неизбывное горе, тронувшее меня до глубины сердца. , . Но он не произнес более ни слова и удалился в свою каюту. . Даже сломленный духом, он, как никто, умеет чувствовать красоту природы. , . Звездное небо, океан и все ландшафты этих удивительных мест еще имеют над ним силу и способны возвышать его над земным. , , . Такой человек ведет как бы двойную жизнь: он может страдать и сгибаться под тяжестью пережитого; но, уходя в себя, он уподобляется небесному духу; его ограждает сияние, и в этот волшебный круг нет доступа горю и злу. : , , , . Быть может, восторг, с каким я описываю чудесного странника, вызовет у тебя улыбку. ? Но это лишь потому, что ты его не видела. . Книги и уединение возвысили твою душу и сделали тебя требовательной; но ты тем более способна оценить необыкновенные достоинства этого удивительного человека. , ; . Иногда я пытаюсь понять, какое именно качество так возвышает его над любым человеком, доныне мне встречавшимся. . Мне кажется, что это – некая интуиция, способность быстрого, но безошибочного суждения; необычайно ясное и точное проникновение в причины вещей; добавь к этому редкий дар красноречия и голос, богатый чарующими модуляциями. , - , , ; - . * * * 19 августа 17.. 19, 17 — Вчера незнакомец сказал мне: , «Вы, должно быть, догадываетесь, капитан Уолтон, что я перенес неслыханные бедствия. " , , . Когда-то я решил, что память о них умрет вместе со мной, но вы заставили меня изменить мое решение. , . Так же, как и я в свое время, вы стремитесь к истине и познанию, и я горячо желаю, чтобы достижение цели не обернулось для вас злою бедой, как это случилось со мною. , ; , . Не знаю, принесет ли вам пользу рассказ о моих несчастьях, но, видя, что вы идете тем же путем, подвергаете себя тем же опасностям, которые довели меня до нынешнего моего состояния, я полагаю, что из моей повести вы сумеете извлечь мораль, и притом такую, которая послужит вам руководством в случае, успеха и утешением в неудаче. ; , , , , . Приготовьтесь услышать рассказ, который может показаться неправдоподобным. . Будь мы в более привычной обстановке, я опасался бы встретить у вас недоверие, быть может, даже насмешку; но в этих загадочных и суровых краях кажется возможным многое такое, что вызывает смех у непосвященных в тайны природы; не сомневаюсь к тому же, что мое повествование заключает в себе самом доказательства своей истинности» , ; - ; ." Можешь вообразить, как я обрадовался его предложению; но я не мог допустить, чтобы рассказом о своих несчастьях он бередил свои раны. , . Мне не терпелось услышать обещанную повесть отчасти из любопытства, но также из сильнейшего желания помочь ему, если б это оказалось в моих силах. , . Все эти чувства я выразил в своем ответе. . – Благодарю вас за сочувствие, – ответил он, – но оно бесполезно; судьба моя свершилась. " ," , " , ; . Я жду лишь одного события – и тогда обрету покой. , . Я понимаю ваши чувства, – продолжал он, заметив, что я собираюсь прервать его, – но вы ошибаетесь, друг мой, если мне позволено так называть вас; ничто не может изменить моей судьбы; выслушайте мой рассказ, и вы убедитесь, что она решена бесповоротно. ," , ; " , , ; ; , ." Свое повествование он пожелал начать на следующий день, в часы моего досуга. . Я горячо поблагодарил его. . Отныне каждый вечер, если не помешают мои обязанности, я буду записывать услышанное, стараясь как можно точнее придерживаться его слов. , , , , . Если на это не хватит времени, я буду делать хотя бы краткие заметки. , . Эту рукопись ты, несомненно, прочтешь с интересом; но с еще большим интересом я когда-нибудь перечту ее сам, я, видевший его и слышавший повесть из собственных его уст! ; , , — ! Вот в сейчас, когда я приступаю к записям, мне слышится его звучный голос, на меня печально и ласково глядят его блестящие глаза, я вижу выразительные движения его исхудалых рук и лицо, словно озаренное внутренним светом. , , - ; ; , . Необычной и страшной была повесть его жизни; ужасна была буря, настигшая этот славный корабль и разбившая его. , — ! Глава 1 Я житель Женевы; мои родные принадлежат к числу самых именитых граждан республики. , . Мои предки много лет были советниками и синдиками; отец также с честью отправлял ряд общественных должностей. , . Он пользовался уважением всех