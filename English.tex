Preface to the Dover Edition
The republication of this book gave me an opportunity to
correct and bring up to date Symbols, Signals and Noise,1 which I
wrote almost twenty years ago. Because the book deals largely
with Shannon’s work, which remains eternally valid, I found that
there were not many changes to be made. In a few places I altered
tense in referring to men who have died. I did not try to replace
cycles per second (cps) by the more modern term, hertz (hz) nor
did I change everywhere communication theory (Shannon’s term)
to information theory, the term I would use today.
Some things I did alter, rewriting a few paragraphs and about
twenty pages without changing the pagination.
In Chapter X, Information Theory and Physics, I replaced a
background radiation temperature of space of “2° to 4°K” (Heaven
knows where I got that) by the correct value of 3.5°K, as determined
by Penzias and Wilson. To the fact that in the absence of
noise we can in principle transmit an unlimited number of bits per
quantum, I added new material on quantum effects in communication.2
I also replaced an obsolete thought-up example of space
communication by a brief analysis of the microwave transmission
of picture signals from the Voyager near Jupiter, and by an exposition
of new possibilities.
1 Harper Modem Science Series, Harper and Brothers, New York, 1961.
2 See Introduction to Communication Science and Systems, John R. Pierce
and Edward C. Posner, Plenum Publishing Corporation, New York, 1980.
vii
viii Symbols, Signals and Noise
In Chapter VII, Efficient Encoding, I rewrote a few pages concerning
efficient source encoding of TV and changed a few sentences
about pulse code modulation and about vocoders. I also
changed the material on error correcting codes.
In Chapter XI, Cybernetics, I rewrote four pages on computers
and programming, which have advanced incredibly during the last
twenty years.
Finally, I made a few changes in the last short Chapter XIV,
Back to Communication Theory.
Beyond these revisions, I call to the reader’s attention a series
of papers on the history of information theory that were published
in 1973 in the IEEE Transactions on Information Theory3 and
two up-to-date books as telling in more detail the present state
of information theory and the mathematical aspects of com-
munication.2-45
Several chapters in the original book deal with areas relevant
only through application or attempted application of information
theory.
I think that Chapter XII, Information Theory and Psychology,
gives a fair idea of the sort of applications attempted in that area.
Today psychologists are less concerned with information theory
than with cognitive science, a heady association of truly startling
progress in the understanding of the nervous system, with ideas
drawn from anthropology, linguistics and a belief that some powerful
and simple mathematical order must underly human function.
Cognitive science of today reminds me of cybernetics of twenty
years ago.
As to Information Theory and Art, today the computer has replaced
information theory in casual discussions. But, the ideas
explored in Chapter XIII have been pursued further. I will mention
some attractive poems produced by Marie Borroff67, and, es-
3 IEEE Transactions on Information Theory, Vol. IT-19, pp. 3-8, 145-
148, 257-262, 381-389 (1973).
4 The Theory of Information and Coding, Robert J. McEliece, Addison-
Wesley, Reading, MA, 1977.
5 Principles of Digital Communication and Coding, Andrew J. Viterbi and
Jim K. Omura, McGraw Hill, New York, 1979.
6 “Computer as Poet,” Marie Borroff, Yale Alumni Magazine, Jan. 1971.
7 Computer Poems, gathered by Richard Bailey, Potagannissing Press,
1973.
IX
pecially a grammar of Swedish folksongs by means of which Johan
Sundberg produced a number of authentic sounding tunes.8
This brings us back to language and Chapter VI, Language and
Meaning. The problems raised in that chapter have not been resolved
during the last twenty years. We do not have a complete
grammar of any natural language. Indeed, formal grammar has
proved most powerful in the area of computer languages. It is my
reading that attention in linguistics has shifted somewhat to the
phonological aspects of spoken language, to understanding what
its building blocks are and how they interact—matters of great
interest in the computer generation of speech from text. Chomsky
and Halle have written a large book on stress,9 and Liberman and
Prince a smaller and very powerful account.10
So much for changes from the original Signals, Symbols and
Noise. Beyond this, I can only reiterate some of the things I said
in the preface to that book.
When James R. Newman suggested to me that I write a book
about communication I was delighted. All my technical work has
been inspired by one aspect or another of communication. Of
course I would like to tell others what seems to me to be interesting
and challenging in this important field.
It would have been difficult to do this and to give any sense of
unity to the account before 1948 when Claude E. Shannon published
“A Mathematical Theory of Communication.”11 Shannon’s
communication theory, which is also called information theory,
has brought into a reasonable relation the many problems that have
been troubling communication engineers for years. It has created a
broad but clearly defined and limited field where before there were
many special problems and ideas whose interrelations were not well
8 “Generative Theories in Language and Musical Descriptions,” Johan
Sundberg and Bjorn Lindblom, Cognition, Vol. 4, pp. 99-122, 1976.
8 The Sound Pattern of English, N. Chomsky and M. Halle, Harper and
Row, 1968.
10 “On Stress and Linguistic Rhythm,” Mark Liberman and Alan Prince,
Linguistic Inquiry, Vol. 8, No. 2, pp. 249-336, Spring, 1977.
11 The papers, originally published in the Bell System Technical Journal,
are reprinted in The Mathematical Theory of Communication, Shannon and
Weaver, University of Illinois Press, first printing 1949. Shannon presented
a somewhat different approach (used in Chapter IX of this book) in “Communication
in the Presence of Noise,” Proceedings of the Institute of Radio
Engineers, Vol. 37, pp. 10-21, 1949.
X
Symbols, Signals and Noise
understood. No one can accuse me of being a Shannon worshipef
and get away unrewarded.
Thus, I felt that my account of communication must be an
account of information theory as Shannon formulated it. The
account would have to be broader than Shannon’s in that it would
discuss the relation, or lack of relation, of information theory to
the many fields to which people have applied it. The account would
have to be broader than Shannon’s in that it would have to be less
mathematical.
Here came the rub. My account could be less mathematical than
Shannon’s, but it could not be nonmathematical. Information
theory is a mathematical theory. It starts from certain premises
that define the aspects of communication with which it will deal,
and it proceeds from these premises to various logical conclusions.
The glory of information theory lies in certain mathematical
theorems which are both surprising and important. To talk about
information theory without communicating its real mathematical
content would be like endlessly telling a man about a wonderful
composer yet never letting him hear an example of the composer’s
music.
How was I to proceed? It seemed to me that I had to make the
book self-contained, so that any mathematics in it could be understood
without referring to other books or without calling for the
particular content of early mathematical training, such as high
school algebra. Did this mean that I had to avoid mathematical
notation? Not necessarily, but any mathematical notation would
have to be explained in the most elementary terms. I have done
this both in the text and in an appendix; by going back and forth
between the two, the mathematically untutored reader should be
able to resolve any difficulties.
But just how difficult should the most difficult mathematical
arguments be? Although it meant sliding over some very important
points, I resolved to keep things easy compared with, say, the more
difficult parts of Newman’s The World of Mathematics. When the
going is very difficult, I have merely indicated the general nature of
the sort of mathematics used rather than trying to describe its content
clearly.
Nonetheless, this book has sections which will be hard for the
XI
nonmathematical reader. I advise him merely to skim through
these, gathering what he can. When he has gone through the book
in this manner, he will see why the difficult sections are there. Then
he can turn back and restudy them if he wishes. But, had I not
put these difficult sections in, and had the reader wanted the sort
of understanding that takes real thought, he would have been stuck.
As far as I know, other available literature on information theory
is either too simple or too difficult to help the diligent but inexpert
reader beyond the easier parts of this book. I might note also that
some of the literature is confused and some of it is just plain wrong.
By this sort of talk I may have raised wonder in the reader’s
mind as to whether or not information theory is really worth so
much trouble, either on his part, for that matter, or on mine. I can
only say that to the degree that the whole world of science and
technology around us is important, information theory is important,
for it is an important part of that world. To the degree to which an
intelligent reader wants to know something both about that world
and about information theory, it is worth his while to try to get a
clear picture. Such a picture must show information theory neither
as something utterly alien and unintelligible nor as something that
can be epitomized in a few easy words and appreciated without
effort.
The process of writing this book was not easy. Of course it could
never have been written at all but for the work of Claude Shannon,
who, besides inspiring the book through his work, read the original
manuscript and suggested several valuable changes. David Slepian
jolted me out of the rut of error and confusion in an even more
vigorous way. E. N. Gilbert deflected me from error in several
instances. Milton Babbitt reassured me concerning the major
contents of the chapter on information theory and art and suggested
a few changes. P. D. Bricker, H. M. Jenkins, and R. N. Shepard
advised me in the field of psychology, but the views I finally expressed
should not be attributed to them. The help of M. V.
Mathews was invaluable. Benoit Mandelbrot helped me with
Chapter XII. J. P. Runyon read the manuscript with care, and Eric
Wolman uncovered an appalling number of textual errors, and
made valuable suggestions as well. I am also indebted to Prof.
Martin Harwit, who persuaded me and Dover that the book was
xii Symbols, Signals and Noise
worth reissuing. The reader is indebted to James R. Newman for
the fact that I have provided a glossary, summaries at the ends of
some chapters, and for my final attempts to make some difficult
points a little clearer. To all of these I am indebted and not less to
Miss F. M. Costello, who triumphed over the chaos of preparing
and correcting the manuscript and figures. In preparing this new
edition, I owe much to my secretary, Mrs. Patricia J. Neill.
September, 1979 J. R. Pierce
CHAPTER I The World and
Theories
In 1948, Claude E. Shannon published a paper called “A
Mathematical Theory of Communication”; it appeared in book
form in 1949. Before that time, a few isolated workers had from
time to time taken steps toward a general theory of communication.
Now, thirty years later, communication theory, or information
theory as it is sometimes called, is an accepted field of research.
Many books on communication theory have been published, and
many international symposia and conferences have been held.
The Institute of Electrical and Electronic Engineers has a professional
group on information theory, whose Transactions appear
six times a year. Many other journals publish papers on information
theory.
All of us use the words communication and information, and
we are unlikely to underestimate their importance. A modem
philosopher, A. J. Ayer, has commented on the wide meaning and
importance of communication in our lives. We communicate, he
observes, not only information, but also knowledge, error, opinions,
ideas, experiences, wishes, orders, emotions, feelings, moods. Heat
and motion can be communicated. So can strength and weakness
and disease. He cites other examples and comments on the manifold
manifestations and puzzling features of communication in
man’s world.
Surely, communication being so various and so important, a
1
2
Symbols, Signals and Noise
theory of communication, a theory of generally accepted soundness
and usefulness, must be of incomparable importance to all of us.
When we add to theory the word mathematical, with all its implications
of rigor and magic, the attraction becomes almost irresistible.
Perhaps if we learn a few formulae our problems of
communication will be solved, and we shall become the masters
of information rather than the slaves of misinformation.
Unhappily, this is not the course of science. Some 2,300 years
ago, another philosopher, Aristotle, discussed in his Physics a
notion as universal as that of communication, that is, motion.
Aristotle defined motion as the fulfillment, insofar as it exists
potentially, of that which exists potentially. He included in the
concept of motion the increase and decrease of that which can be
increased or decreased, coming to and passing away, and also being
built. He spoke of three categories of motion, with respect to
magnitude, affection, and place. He found, indeed, as he said, as
many types of motion as there are meanings of the word is.
Here we see motion in all its manifest complexity. The complexity
is perhaps a little bewildering to us, for the associations of
words differ in different languages, and we would not necessarily
associate motion with all the changes of which Aristotle speaks.
How puzzling this universal matter of motion must have been
to the followers of Aristotle. It remained puzzling for over two
millennia, until Newton enunciated the laws which engineers still
use in designing machines and astronomers in studying the motions
of stars, planets, and satellites. While later physicists have found
that Newton’s laws are only the special forms which more general
laws assume when velocities are small compared with that of fight
and when the scale of the phenomena is large compared with the
atom, they are a living part of our physics rather than a historical
monument. Surely, when motion is so important a part of our
world, we should study Newton’s laws of motion. They say:
1. A body continues at rest or in motion with a constant velocity
in a straight fine unless acted upon by a force.
2. The change in velocity of a body is in the direction of the force
acting on it, and the magnitude of the change is proportional to
the force acting on the body times the time during which the force
acts, and is inversely proportional to the mass of the body.
The World and Theories 3
3. Whenever a first body exerts a force on a second body, the
second body exerts an equal and oppositely directed force on the
first body.
To these laws Newton added the universal law of gravitation:
4. Two particles of matter attract one another with a force acting
along the line connecting them, a force which is proportional
to the product of the masses of the particles and inversely proportional
to the square of the distance separating them.
Newton’s laws brought about a scientific and a philosophical
revolution. Using them, Laplace reduced the solar system to an
explicable machine. They have formed the basis of aviation and
rocketry, as well as of astronomy. Yet, they do little to answer many
of the questions about motion which Aristotle considered. Newton’s
laws solved the problem of motion as Newton defined it,
not of motion in all the senses in which the word could be used in
the Greek of the fourth century before our Lord or in the English
of the twentieth century after.
Our speech is adapted to our daily needs or, perhaps, to the needs
of our ancestors. We cannot have a separate word for every distinct
object and for every distinct event; if we did we should be forever
coining words, and communication would be impossible. In order
to have language at all, many things or many events must be
referred to by one word. It is natural to say that both men and
horses run (though we may prefer to say that horses gallop) and
convenient to say that a motor runs and to speak of a run in a
stocking or a run on a bank.
The unity among these concepts lies far more in our human
language than in any physical similarity with which we can expect
science to deal easily and exactly. It would be foolish to seek some
elegant, simple, and useful scientific theory of running which would
embrace runs of salmon and runs in hose. It would be equally
fodlish to try to embrace in one theory all the motions discussed
by Aristotle or all the sorts of communication and information
which later philosophers have discovered.
In our everyday language, we use words in a way which is convenient
in our everyday business. Except in the study of language
itself, science does not seek understanding by studying words and
their relations. Rather, science looks for things in nature, including
4
Symbols, Signals and Noise
our human nature and activities, which can be grouped together
and understood. Such understanding is an ability to see what
complicated or diverse events really do have in common (the
planets in the heavens and the motions of a whirling skater on ice,
for instance) and to describe the behavior accurately and simply.
The words used in such scientific descriptions are often drawn
from our everyday vocabulary. Newton used force, mass, velocity,
and attraction. When used in science, however, a particular meaning
is given to such words, a meaning narrow and often new. We
cannot discuss in Newton’s terms force of circumstance, mass
media, or the attraction of Brigitte Bardot. Neither should we
expect that communication theory will have something sensible to
say about every question we can phrase using the words communication
or information.
A valid scientific theory seldom if ever offers the solution to the
pressing problems which we repeatedly state. It seldom supplies
a sensible answer to our multitudinous questions. Rather than
rationalizing our ideas, it discards them entirely, or, rather, it
leaves them as they were. It tells us in a fresh and new way what
aspects of our experience can profitably be related and simply
understood. In this book, it will be our endeavor to seek out the
ideas concerning communication which can be so related and
understood.
When the portions of our experience which can be related have
been singled out, and when they have been related and understood,
we have a theory concerning these matters. Newton’s laws of
motion form an important part of theoretical physics, a field called
mechanics. The laws themselves are not the whole of the theory;
they are merely the basis of it, as the axioms or postulates of
geometry are the basis of geometry. The theory embraces both the
assumptions themselves and the mathematical working out of the
logical consequences which must necessarily follow from the
assumptions. Of course, these consequences must be in accord
with the complex phenomena of the world about us if the theory
is to be a valid theory, and an invalid theory is useless.
The ideas and assumptions of a theory determine the generality
of the theory, that is, to how wide a range of phenomena the
theory applies. Thus, Newton’s laws of motion and of gravitation
The World and Theories 5
are very general; they explain the motion of the planets, the timekeeping
properties of a pendulum, and the behavior of all sorts of
machines and mechanisms. They do not, however, explain radio
waves.
Maxwell’s equations1 explain all (non-quantum) electrical phenomena;
they are very general. A branch of electrical theory called
network theory deals with the electrical properties of electrical
circuits, or networks, made by interconnecting three sorts of idealized
electrical structures: resistors (devices such as coils of thin,
poorly conducting wire or films of metal or carbon, which impede
the flow of current), inductors (coils of copper wire, sometimes
wound on magnetic cores), and capacitors (thin sheets of metal
separated by an insulator or dielectric such as mica or plastic; the
Leyden jar was an early form of capacitor). Because network
theory deals only with the electrical behavior of certain specialized
and idealized physical structures, while Maxwell’s equations describe
the electrical behavior of any physical structure, a physicist
would say that network theory is less general than are Maxwell’s
equations, for Maxwell’s equations cover the behavior not only of
idealized electrical networks but of all physical structures and
include the behavior of radio waves, which lies outside of the scope
of network theory.
Certainly, the most general theory, which explains the greatest
range of phenomena, is the most powerful and the best; it can
always be specialized to deal with simple cases. That is why physicists
have sought a unified field theory to embrace mechanical
laws and gravitation and all electrical phenomena. It might, indeed,
seem that all theories could be ranked in order of generality, and,
if this is possible, we should certainly like to know the place of
communication theory in such a hierarchy.
Unfortunately, life isn’t as simple as this. In one sense, network
theory is less general than Maxwell’s equations. In another sense,
1 In 1873, in his treatise Electrictity and Magnetism, James Clerk Maxwell presented
and fully explained for the first time the natural laws relating electric and
magnetic fields and electric currents. He showed that there should be electromagnetic
waves (radio waves) which travel with the speed of light. Hertz later demonstrated
these experimentally, and we now know that light is electromagnetic waves. Maxwell’s
equations are the mathematical statement of Maxwell’s theory of electricity
and magnetism. They are the foundation of all electric art.
6
Symbols, Signals and Noise
however, it is more general, for all the mathematical results of
network theory hold for vibrating mechanical systems made up of
idealized mechanical components as well as for the behavior of
interconnections of idealized electrical components. In mechanical
applications, a spring corresponds to a capacitor, a mass to an
inductor, and a dashpot or damper, such as that used in a door
closer to keep the door from slamming, corresponds to a resistor.
In fact, network theory might have been developed to explain the
behavior of mechanical systems, and it is so used in the field of
acoustics. The fact that network theory evolved from the study of
idealized electrical systems rather than from the study of idealized
mechanical systems is a matter of history, not of necessity.
Because all of the mathematical results of network theory apply
to certain specialized and idealized mechanical systems, as well as
to certain specialized and idealized electrical systems, we can say
that in a sense network theory is more general than Maxwell’s
equations, which do not apply to mechanical systems at all. In
another sense, of course, Maxwell’s equations are more general
than network theory, for Maxwell’s equations apply to all electrical
systems, not merely to a specialized and idealized class of electrical
circuits.
To some degree we must simply admit that this is so, without
being able to explain the fact fully. Yet, we can say this much.
Some theories are very strongly physical theories. Newton’s laws
and Maxwell’s equations are such theories. Newton’s laws deal
with mechanical phenomena; Maxwell’s equations deal with electrical
phenomena. Network theory is essentially a mathematical
theory. The terms used in it can be given various physical meanings.
The theory has interesting things to say about different physical
phenomena, about mechanical as well as electrical vibrations.
Often a mathematical theory is the offshoot of a physical theory
or of physical theories. It can be an elegant mathematical formulation
and treatment of certain aspects of a general physical theory.
Network theory is such a treatment of certain physical behavior
common to electrical and mechanical devices. A branch of mathematics
called potential theory treats problems common to electric,
magnetic, and gravitational fields and, indeed, in a degree to aerodynamics.
Some theories seem, however, to be more mathematical
than physical in their very inception.
The World and Theories 7
We use many such mathematical theories in dealing with the
physical world. Arithmetic is one of these. If we label one of a
group of apples, dogs, or men 1, another 2, and so on, and if we
have used up just the first 16 numbers when we have labeled all
members of the group, we feel confident that the group of objects
can be divided into two equal groups each containing 8 objects
(16 -r- 2 = 8) or that the objects can be arranged in a square
array of four parallel rows of four objects each (because 16 is a
perfect square; 16 = 4x4). Further, if we line the apples, dogs,
or men up in a row, there are 2,092,278,988,800 possible sequences
in which they can be arranged, corresponding to the 2,092,278,-
988,800 different sequences of the integers 1 through 16. If we used
up 13 rather than 16 numbers in labeling the complete collection
of objects, we feel equally certain that the collection could not be
divided into any number of equal heaps, because 13 is a prime
number and cannot be expressed as a product of factors.
This seems not to depend at all on the nature of the objects.
Insofar as we can assign numbers to the members of any collection
of objects, the results we get by adding, subtracting, multiplying,
and dividing numbers or by arranging the numbers in sequence
hold true. The connection between numbers and collections of
objects seems so natural to us that we may overlook the fact that
arithmetic is itself a mathematical theory which can be applied to
nature only to the degree that the properties of numbers correspond
to properties of the physical world.
Physicists tell us that we can talk sense about the total number
of a group of elementary particles, such as electrons, but we can’t
assign particular numbers to particular particles because the particles
are in a very real sense indistinguishable. Thus, we can’t talk
about arranging such particles in different orders, as numbers can
be arranged in different sequences. This has important consequences
in a part of physics called statistical mechanics. We may
also note that while Euclidean geometry is a mathematical theory
which serves surveyors and navigators admirably in their practical
concerns, there is reason to believe that Euclidean geometry is not
quite accurate in describing astronomical phenomena.
How can we describe or classify theories? We can say that a
theory is very narrow or very general in its scope. We can also
distinguish theories as to whether they are strongly physical or
8
Symbols, Signals and Noise
strongly mathematical. Theories are strongly physical when they
describe very completely some range of physical phenomena,
which in practice is always limited. Theories become more mathematical
or abstract when they deal with an idealized class of
phenomena or with only certain aspects of phenomena. Newton’s
laws are strongly physical in that they afford a complete description
of mechanical phenomena such as the motions of the planets or
the behavior of a pendulum. Network theory is more toward the
mathematical or abstract side in that it is useful in dealing with a
variety of idealized physical phenomena. Arithmetic is very mathematical
and abstract; it is equally at home with one particular
property of many sorts of physical entities, with numbers of dogs,
numbers of men, and (if we remember that electrons are indistinguishable)
with numbers of electrons. It is even useful in reckoning
numbers of days.
In these terms, communication theory is both very strongly
mathematical and quite general. Although communication theory
grew out of the study of electrical communication, it attacks problems
in a very abstract and general way. It provides, in the bit, a
universal measure of amount of information in terms of choice or
uncertainty. Specifying or learning the choice between two equally
probable alternatives, which might be messages or numbers to be
transmitted, involves one bit of information. Communication
theory tells us how many bits of information can be sent per second
over perfect and imperfect communication channels in terms of
rather abstract descriptions of the properties of these channels.
Communication theory tells us how to measure the rate at which
a message source, such as a speaker or a writer, generates information.
Communication theory tells us how to represent, or encode,
messages from a particular message source efficiently for transmission
over a particular sort of channel, such as an electrical
circuit, and it tells us when we can avoid errors in transmission.
Because communication theory discusses such matters in very
general and abstract terms, it is sometimes difficult to use the
understanding it gives us in connection with particular, practical
problems. However, because communication theory has such an
abstract and general mathematical form, it has a very broad field
of application. Communication theory is useful in connection with
The World and Theories 9
written and spoken language, the electrical and mechanical transmission
of messages, the behavior of machines, and, perhaps, the
behavior of people. Some feel that it has great relevance and
importance to physics in a way that we shall discuss much later
in this book.
Primarily, however, communication theory is, as Shannon described
it, a mathematical theory of communication. The concepts
are formulated in mathematical terms, of which widely different
physical examples can be given. Engineers, psychologists, and
physicists may use communication theory, but it remains a mathematical
theory rather than a physical or psychological theory or
an engineering art.
It is not easy to present a mathematical theory to a general
audience, yet communication theory is a mathematical theory,
and to pretend that one can discuss it while avoiding mathematics
entirely would be ridiculous. Indeed, the reader may be startled
to find equations and formulae in these pages; these state accurately
ideas which are also described in words, and I have included
an appendix on mathematical notation to help the nonmathe-
matical reader who wants to read the equations aright.
I am aware, however, that mathematics calls up chiefly unpleasant
pictures of multiplication, division, and perhaps square roots,
as well as the possibly traumatic experiences of high-school classrooms.
This view of mathematics is very misleading, for it places
emphasis on special notation and on tricks of manipulation, rather
than on the aspect of mathematics that is most important to mathematicians.
Perhaps the reader has encountered theorems and
proofs in geometry; perhaps he has not encountered them at all,
yet theorems and proofs are of primary importance in all mathematics,
pure and applied. The important results of information
theory are stated in the form of mathematical theorems, and these
are theorems only because it is possible to prove that they are true
statements.
Mathematicians start out with certain assumptions and definitions,
and then by means of mathematical arguments or proofs they
are able to show that certain statements or theorems are true. This
is what Shannon accomplished in his “Mathematical Theory of
Communication.” The truth of a theorem depends on the validity
10
Symbols, Signals and Noise
of the assumptions made and on the validity of the argument or
proof which is used to establish it.
All of this is pretty abstract. The best way to give some idea of
the meaning of theorem and proof is certainly by means of examples.
I cannot do this by asking the general reader to grapple,
one by one and in all their gory detail, with the difficult theorems
of communication theory. Really to understand thoroughly the
proofs of such theorems takes time and concentration even for one
with some mathematical background. At best, we can try to get
at the content, meaning, and importance of the theorems.
The expedient I propose to resort to is to give some examples
of simpler mathematical theorems and their proof. The first
example concerns a game called hex, or Nash. The theorem which
will be proved is that the player with first move can win.
Hex is played on a board which is an array of forty-nine hexagonal
cells or spaces, as shown in Figure 1-1, into which markers
may be put. One player uses black markers and tries to place them
so as to form a continuous, if wandering, path between the black
area at the left and the black area at the right. The other player uses
white markers and tries to place them so as to form a continuous,
if wandering, path between the white area at the top and the white
area at the bottom. The players play alternately, each placing one
marker per play. Of course, one player has to start first.
Fig. 1-1
The World and Theories 11
In order to prove that the first player can win, it is necessary
first to prove that when the game is played out, so that there is
either a black or a white marker in each cell, one of the players
must have won.
Theorem I: Either one player or the other wins.
Discussion: In playing some games, such as chess and ticktack-
toe, it may be that neither player will win, that is, that the game
will end in a draw. In matching heads or tails, one or the other
necessarily wins. What one must show to prove this theorem is
that, when each cell of the hex board is covered by either a black
or a white marker, either there must be a black path between the
black areas which will interrupt any possible white path between
the white areas or there must be a white path between the white
areas which will interrupt any possible black path between the
black areas, so that either white or black must have won.
Proof: Assume that each hexagon has been filled in with either
a black or a white marker. Let us start from the left-hand comer
of the upper white border, point I of Figure 1-2, and trace out the
boundary between white and black hexagons or borders. We will
proceed always along a side with black on our right and white on
our left. The boundary so traced out will turn at the successive
corners, or vertices, at which the sides of hexagons meet. At a
corner, or vertex, we can have only two essentially different con-
EI
Fig. 1-2
12 Symbols, Signals and Noise
ditions. Either there will be two touching black hexagons on the
right and one white hexagon on the left, as in a of Figure 1-3, or
two touching white hexagons on the left and one black hexagon
on the right, as shown in b of Figure 1-3. We note that in either
case there will be a continuous black path to the right of the
boundary and a continuous white path to the left of the boundary.
We also note that in neither a nor b of Figure 1-3 can the boundary
cross or join itself, because only one path through the vertex has
black on the right and white on the left. We can see that these two
facts are true for boundaries between the black and white borders
and hexagons as well as for boundaries between black and white
hexagons. Thus, along the left side of the boundary there must be
a continuous path of white hexagons to the upper white border,
and along the right side of the boundary there must be a continuous
path of black hexagons to the left black border. As the
boundary cannot cross itself, it cannot circle indefinitely, but must
eventually reach a black border or a white border. If the boundary
reaches a black border or white border with black on its right and
white on its left, as we have prescribed, at any place except comer
II or corner III, we can extend the boundary further with black on
its right and white on its left. Hence, the boundary will reach either
point II or point III. If it reaches point II, as shown in Figure 1-2,
the black hexagons on the right, which are connected to the left
black border, will also be connected to the right black border,
while the white hexagons to the left will be connected to the upper
white border only, and black will have won. It is clearly impossible
for white to have won also, for the continuous band of adjacent
(a) Cb)
Fig. 1-3
The World and Theories 13
black cells from the left border to the right precludes a continuous
band of white cells to the bottom border. We see by similar argument
that, if the boundary reaches point III, white will have won.
Theorem II: The player with the first move can win.
Discussion: By can is meant that there exists a way, if only the
player were wise enough to know it. The method for winning would
consist of a particular first move (more than one might be allowable
but are not necessary) and a chart, formula, or other specification
or recipe giving a correct move following any possible move
made by his opponent at any subsequent stage of the game, such
that if, each time he plays, the first player makes the prescribed
move, he will win regardless of what moves his opponent may
make.
Proof: Either there must be some way of play which, if followed
by the first player, will insure that he wins or else, no matter how
the first player plays, the second player must be able to choose
moves which will preclude the first player from winning, so that he,
the second player, will win. Let us assume that the player with the
second move does have a sure recipe for winning. Let the player
with the first move make his first move in any way, and then, after
his opponent has made one move, let the player with the first
move apply the hypothetical recipe which is supposed to allow the
player with the second move to win. If at any time a move calls for
putting a piece on a hexagon occupied by a piece he has already
played, let him place his piece instead on any unoccupied space.
The designated space will thus be occupied. The fact that by
starting first he has an extra piece on the board may keep his
opponent from occupying a particular hexagon but not the player
with the extra piece. Hence, the first player can occupy the hexagons
designated by the recipe and must win. This is contrary to
the original assumption that the player with the second move can
win, and so this assumption must be false. Instead, it must be
possible for the player with the first move to win.
A mathematical purist would scarcely regard these proofs as
rigorous in the form given. The proof of theorem II has another
curious feature; it is not a constructive proof. That is, it does not
show the player with the first move, who can win in principle, how
to go about winning. We will come to an example of a constructive
14
Symbols, Signals and Noise
proof in a moment. First, however, it may be appropriate to philosophize
a little concerning the nature of theorems and the need
for proving them.
Mathematical theorems are inherent in the rigorous statement
of the general problem or field. That the player with the first move
can win at hex is necessarily so once the game and its rules of play
have been specified. The theorems of Euclidean geometry are
necessarily so because of the stated postulates.
With sufficient intelligence and insight, we could presumably see
the truth of theorems immediately. The young Newton is said to
have found Euclid’s theorems obvious and to have been impatient
with their proofs.
Ordinarily, while mathematicians may suspect or conjecture the
truth of certain statements, they have to prove theorems in order
to be certain. Newton himself came to see the importance of proof,
and he proved many new theorems by using the methods of Euclid.
By and large, mathematicians have to proceed step by step in
attaining sure knowledge of a problem. They laboriously prove one
theorem after another, rather than seeing through everything in a
flash. Too, they need to prove the theorems in order to convince
others.
Sometimes a mathematician needs to prove a theorem to convince
himself, for the theorem may seem contrary to common
sense. Let us take the following problem as an example: Consider
the square, 1 inch on a side, at the left of Figure 1-4. We can specify
any point in the square by giving two numbers, y, the height of
the point above the base of the square, and x, the distance of the
point from the left-hand side of the square. Each of these numbers
will be less than one. For instance, the point shown will be represented
by
x = 0.547000 . . . (ending in an endless sequence of zeros)
y = 0.312000 . . . (ending in an endless sequence of zeros)
Suppose we pair up points on the square with points on the line,
so that every point on the line is paired with just one point on the
square and every point on the square with just one point on the
line. If we do this, we are said to have mapped the square onto
the line in a one-to-one way, or to have achieved a one-to-one mapping
of the square onto the line.
The World and Theories 15
-P
A
y i
—*•
X1
Fig. 1-4
Theorem: It is possible to map a square of unit area onto a line
of unit length in a one-to-one way.2
Proof: Take the successive digits of the height of the point in
the square and let them form the first, third, fifth, and so on digits
of a number x'. Take the digits of the distance of the point P from
the left side of the square, and let these be the second, fourth,
sixth, etc., of the digits of the number x'. Let x' be the distance of
the point P' from the left-hand end of the line. Then the point P'
maps the point P of the square onto the line uniquely, in a one-
to-one way. We see that changing either x or y will change x' to a
new and appropriate number, and changing x’ will change x and
y. To each point x,y in the square corresponds just one point x'
on the line, and to each point x’ on the line corresponds just one
point x,y in the square, the requirement for one-to-one mapping.3
In the case of the example given before
x = 0.547000 . . .
y = 0.312000 . ..
x' = 0.351427000 . . .
In the case of most points, including those specified by irrational
numbers, the endless string of digits representing the point will not
become a sequence of zeros nor will it ever repeat.
Here we have an example of a constructive proof. We show that
we can map each point of a square into a point on a line segment
in a one-to-one way by giving an explicit recipe for doing this.
Many mathematicians prefer constructive proofs to proofs which
2 This has been restricted for convenience; the size doesn’t matter.
3 This proof runs into resolvable difficulties in the case of some numbers such as
'A, which can be represented decimally .5 followed by an infinite sequence of zeros
or .4 followed by an infinite sequence of nines.
16
Symbols, Signals and Noise
are not constructive, and mathematicians of the intuitionist school
reject nonconstructive proofs in dealing with infinite sets, in which
it is impossible to examine all the members individually for the
property in question.
Let us now consider another matter concerning the mapping of
the points of a square on a line segment. Imagine that we move
a pointer along the line, and imagine a pointer simultaneously
moving over the face of the square so as to point out the points
in the square corresponding to the points that the first pointer
indicates on the line. We might imagine (contrary to what we shall
prove) the following: If we moved the first pointer slowly and
smoothly along the line, the second pointer would move slowly and
smoothly over the face of the square. All the points lying in a small
cluster on the line would be represented by points lying in a small
cluster on the face of the square. If we moved the pointer a short
distance along the line, the other pointer would move a short
distance over the face of the square, and if we moved the pointer
a shorter distance along the line, the other pointer would move a
shorter distance across the face of the square, and so on. If this
were true we could say that the one-to-one mapping of the points
of the square into points on the line was continuous.
However, it turns out that a one-to-one mapping of the points
in a square into the points on a line cannot be continuous. As we
move smoothly along a curve through the square, the points on
the line which represent the successive points on the square necessarily
jump around erratically, not only for the mapping described
above but for any one-to-one mapping whatever. Any one-to-one
mapping of the square onto the line is discontinuous.
Theorem: Any one-to-one mapping of a square onto a line must
be discontinuous.
Proof: Assume that the one-to-one mapping is continuous. If
this is to be so then all the points along some arbitrary curve AB
of Figure 1-5 on the square must map into the points lying between
the corresponding points A' and B'. If they did not, in moving along
the curve in the square we would either jump from one end of the
line to the other (discontinuous mapping) or pass through one
point on the line twice (not one-to-one mapping). Let us now
choose a point C to the left of line segment A'B' and D' to the
right of A'B' and locate the corresponding points C and D in the
The World and Theories 17
square. Draw a curve connecting C and D and crossing the curve
from A to B. Where the curve crosses the curve AB it will have a
point in common with AB; hence, this one point of CD must map
into a point lying between A' and B', and all other points which
are not on AB must map to points lying outside of A'B’, either to
the left or the right of A'B'. This is contrary to our assumption that
the mapping was continuous, and so the mapping cannot be
continuous.
We shall find that these theorems, that the points of a square
can be mapped onto a fine and that the mapping is necessarily
discontinuous, are both important in communication theory, so we
have proved one theorem which, unlike those concerning hex, will
be of some use to us.
Mathematics is a way of finding out, step by step, facts which
are inherent in the statement of the problem but which are not
immediately obvious. Usually, in applying mathematics one must
first hit on the facts and then verify them by proof. Here we come
upon a knotty problem, for the proofs which satisfied mathematicians
of an earlier day do not satisfy modem mathematicians.
In our own day, an irascible minor mathematician who reviewed
Shannon’s original paper on communication theory expressed
doubts as to whether or not the author’s mathematical intentions
were honorable. Shannon’s theorems are true, however, and proofs
have been given which satisfy even rigor-crazed mathematicians.
The simple proofs which I have given above as illustrations of
mathematics are open to criticism by purists.
What I have tried to do is to indicate the nature of mathematical
reasoning, to give some idea of what a theorem is and of how it
may be proved. With this in mind, we will go on to the mathematical
theory of communication, its theorems, which we shall not
really prove, and to some implications and associations which
18 Symbols, Signals and Noise
extend beyond anything that we can establish with mathematical
certainty.
As I have indicated earlier in this chapter, communication
theory as Shannon has given it to us deals in a very broad and
abstract way with certain important problems of communication
and information, but it cannot be applied to all problems which
we can phrase using the words communication and information
in their many popular senses. Communication theory deals with
certain aspects of communication which can be associated and
organized in a useful and fruitful way, just as Newton’s laws of
motion deal with mechanical motion only, rather than with all the
named and indeed different phenomena which Aristotle had in
mind when he used the word motion.
To succeed, science must attempt the possible. We have no
reason to believe that we can unify all the things and concepts for
which we use a common word. Rather we must seek that part of
experience which can be related. When we have succeeded in
relating certain aspects of experience we have a theory. Newton’s
laws of motion are a theory which we can use in dealing with
mechanical phenomena. Maxwell’s equations are a theory which
we can use in connection with electrical phenomena. Network
theory we can use in connection with certain simple sorts of electrical
or mechanical devices. We can use arithmetic very generally
in connection with numbers of men, stones, or stars, and geometry
in measuring land, sea, or galaxies.
Unlike Newton’s laws of motion and Maxwell’s equations, which
are strongly physical in that they deal with certain classes of
physical phenomena, communication theory is abstract in that it
applies to many sorts of communication, written, acoustical, or
electrical. Communication theory deals with certain important but
abstract aspects of communication. Communication theory proceeds
from clear and definite assumptions to theorems concerning
information sources and communication channels. In this it is
essentially mathematical, and in order to understand it we must
understand the idea of a theorem as a statement which must be
proved, that is, which must be shown to be the necessary consequence
of a set of initial assumptions. This is an idea which is the
very heart of mathematics as mathematicians understand it.
c h a p t e r II The Origins of
Information Theory
Men have been at odds concerning the value of history. Some
have studied earlier times in order to find a universal system of
the world, in whose inevitable unfolding we can see the future as
well as the past. Others have sought in the past prescriptions for
success in the present. Thus, some believe that by studying scientific
discovery in another day we can learn how to make discoveries.
On the other hand, one sage observed that we learn nothing from
history except that we never learn anything from history, and
Henry Ford asserted that history is bunk.
All of this is as far beyond me as it is beyond the scope of this
book. I will, however, maintain that we can learn at least two
things from the history of science.
One of these is that many of the most general and powerful
discoveries of science have arisen, not through the study of phenomena
as they occur in nature, but, rather, through the study of
phenomena in man-made devices, in products of technology, if you
will. This is because the phenomena in man’s machines are simplified
and ordered in comparison with those occurring naturally, and
it is these simplified phenomena that man understands most easily.
Thus, the existence of the steam engine, in which phenomena
involving heat, pressure, vaporization, and condensation occur in a
simple and orderly fashion, gave tremendous impetus to the very
powerful and general science of thermodynamics. We see this
19
20
Symbols, Signals and Noise
especially in the work of Carnot.1 Our knowledge of aerodynamics
and hydrodynamics exists chiefly because airplanes and ships
exist, not because of the existence of birds and fishes. Our knowledge
of electricity came mainly not from the study of lightning,
but from the study of man’s artifacts.
Similarly, we shall find the roots of Shannon’s broad and elegant
theory of communication in the simplified and seemingly
easily intelligible phenomena of telegraphy.
The second thing that history can teach us is with what difficulty
understanding is won. Today, Newton’s laws of motion seem
simple and almost inevitable, yet there was a day when they were
undreamed of, a day when brilliant men had the oddest notions
about motion. Even discoverers themselves sometimes seem incredibly
dense as well as inexplicably wonderful. One might expect
of Maxwell’s treatise on electricity and magnetism a bold and
simple pronouncement concerning the great step he had taken.
Instead, it is cluttered with all sorts of such lesser matters as once
seemed important, so that a naive reader might search long to find
the novel step and to restate it in the simple manner familiar to us.
It is true, however, that Maxwell stated his case clearly elsewhere.
Thus, a study of the origins of scientific ideas can help us to value
understanding more highly for its having been so dearly won. We
can often see men of an earlier day stumbling along the edge of
discovery but unable to take the final step. Sometimes we are
tempted to take it for them and to say, because they stated many
of the required concepts in juxtaposition, that they must really have
reached the general conclusion. This, alas, is the same trap into
which many an ungrateful fellow falls in his own life. When someone
actually solves a problem that he merely has had ideas about,
he believes that he understood the matter all along.
Properly understood, then, the origins of an idea can help to
show what its real content is; what the degree of understanding
was before the idea came along and how unity and clarity have
been attained. But to attain such understanding we must trace the
actual course of discovery, not some course which we feel discovery
1 N. L. S. Carnot (1796-1832) first proposed an ideal expansion of gas (the Carnot
cycle) which will extract the maximum possible mechanical energy from the thermal
energy of the steam.
The
Origins of Information Theory
21
should or could have taken, and we must see problems (if we can)
as the men of the past saw them, not as we see them today.
In looking for the origin of communication theory one is apt to
fall into an almost trackless morass. I would gladly avoid this
entirely but cannot, for others continually urge their readers to
enter it. I only hope that they will emerge unharmed with the help
of the following grudgingly given guidance.
A particular quantity called entropy is used in thermodynamics
and in statistical mechanics. A quantity called entropy is used in
communication theory. After all, thermodynamics and statistical
mechanics are older than communication theory. Further, in a
paper published in 1929, L. Szilard, a physicist, used an idea of
information in resolving a particular physical paradox. From these
facts we might conclude that communication theory somehow grew
out of statistical mechanics.
This easy but misleading idea has caused a great deal of confusion
even among technical men. Actually, communication theory
evolved from an effort to solve certain problems in the field of
electrical communication. Its entropy was called entropy by mathematical
analogy with the entropy of statistical mechanics. The
chief relevance of this entropy is to problems quite different from
those which statistical mechanics attacks.
In thermodynamics, the entropy of a body of gas depends on its
temperature, volume, and mass—and on what gas it is—just as the
energy of the body of gas does. If the gas is allowed to expand in
a cylinder, pushing on a slowly moving piston, with no flow of heat
to or from the gas, the gas will become cooler, losing some of its
thermal energy. This energy appears as work done on the piston.
The work may, for instance, lift a weight, which thus stores the
energy lost by the gas.
This is a reversible process. By this we mean that if work is done
in pushing the piston slowly back against the gas and so recompressing
it to its original volume, the exact original energy, pressure,
and temperature will be restored to the gas. In such a
reversible process, the entropy of the gas remains constant, while
its energy changes.
Thus, entropy is an indicator of reversibility; when there is no
change of entropy, the process is reversible. In the example dis
22
Symbols, Signals and Noise
cussed above, energy can be transferred repeatedly back and forth
between thermal energy of the compressed gas and mechanical
energy of a lifted weight.
Most physical phenomena are not reversible. Irreversible phenomena
always involve an increase of entropy.
Imagine, for instance, that a cylinder which allows no heat flow
in or out is divided into two parts by a partition, and suppose that
there is gas on one side of the partition and none on the other.
Imagine that the partition suddenly vanishes, so that the gas
expands and fills the whole container. In this case, the thermal
energy remains the same, but the entropy increases.
Before the partition vanished we could have obtained mechanical
energy from the gas by letting it flow into the empty part of
the cylinder through a little engine. After the removal of the partition
and the subsequent increase in entropy, we cannot do this.
The entropy can increase while the energy remains constant in
other similar circumstances. For instance, this happens when heat
flows from a hot object to a cold object. Before the temperatures
were equalized, mechanical work could have been done by making
use of the temperature difference. After the temperature difference
has disappeared, we can no longer use it in changing part of the
thermal energy into mechanical energy.
Thus, an increase in entropy means a decrease in our ability to
change thermal energy, the energy of heat, into mechanical energy.
An increase of entropy means a decrease of available energy.
While thermodynamics gave us the concept of entropy, it does
not give a detailed physical picture of entropy, in terms of positions
and velocities of molecules, for instance. Statistical mechanics does
give a detailed mechanical meaning to entropy in particular cases.
In general, the meaning is that an increase in entropy means a
decrease in order. But, when we ask what order means, we must
in some way equate it with knowledge. Even a very complex
arrangement of molecules can scarcely be disordered if we know
the position and velocity of every one. Disorder in the sense in
which it is used in statistical mechanics involves unpredictability
based on a lack of knowledge of the positions and velocities of
molecules. Ordinarily we lack such knowledge when the arrangement
of positions and velocities is “complicated.”
The Origins of Information Theory
23
Let us return to the example discussed above in which all the
molecules of a gas are initially on one side of a partition in a
cylinder. If the molecules are all on one side of the partition, and
if we know this, the entropy is less than if they are distributed on
both sides of the partition. Certainly, we know more about the
positions of the molecules when we know that they are all on one
side of the partition than if we merely know that they are somewhere
within the whole container. The more detailed our knowledge
is concerning a physical system, the less uncertainty we have
concerning it (concerning the location of the molecules, for
instance) and the less the entropy is. Conversely, more uncertainty
means more entropy.
Thus, in physics, entropy is associated with the possibility of
converting thermal energy into mechanical energy. If the entropy
does not change during a process, the process is reversible. If the
entropy increases, the available energy decreases. Statistical mechanics
interprets an increase of entropy as a decrease in order or,
if we wish, as a decrease in our knowledge.
The applications and details of entropy in physics are of course
much broader than the examples I have given can illustrate, but I
believe that I have indicated its nature and something of its importance.
Let us now consider the quite different purpose and use of
the entropy of communication theory.
In communication theory we consider a message source, such
as a writer or a speaker, which may produce on a given occasion
any one of many possible messages. The amount of information
conveyed by the message increases as the amount of uncertainty
as to what message actually will be produced becomes greater. A
message which is one out of ten possible messages conveys a
smaller amount of information than a message which is one out
of a million possible messages. The entropy of communication
theory is a measure of this uncertainty and the uncertainty, or
entropy, is taken as the measure of the amount of information
conveyed by a message from a source. The more we know about
what message the source will produce, the less uncertainty, the
less the entropy, and the less the information.
We see that the ideas which gave rise to the entropy of physics
and the entropy of communication theory are quite different. One
24
Symbols, Signals and Noise
can be fully useful without any reference at all to the other. Nonetheless,
both the entropy of statistical mechanics and that of
communication theory can be described in terms of uncertainty,
in similar mathematical terms. Can some significant and useful
relation be established between the two different entropies and,
indeed, between physics and the mathematical theory of communication?
Several physicists and mathematicians have been anxious to
show that communication theory and its entropy are extremely
important in connection with statistical mechanics. This is still a
confused and confusing matter. The confusion is sometimes aggravated
when more than one meaning of information creeps into a
discussion. Thus, information is sometimes associated with the idea
of knowledge through its popular use rather than with uncertainty
and the resolution of uncertainty, as it is in communication theory.
We will consider the relation between communication theory
and physics in Chapter X, after arriving at some understanding of
communication theory. Here I will merely say that the efforts to
marry communication theory and physics have been more interesting
than fruitful. Certainly, such attempts have not produced
important new results or understanding, as communication theory
has in its own right.
Communication theory has its origins in the study of electrical
communication, not in statistical mechanics, and some of the
ideas important to communication theory go back to the very
origins of electrical communication.
During a transatlantic voyage in 1832, Samuel F. B. Morse set
to work on the first widely successful form of electrical telegraph.
As Morse first worked it out, his telegraph was much more complicated
than the one we know. It actually drew short and long
lines on a strip of paper, and sequences of these represented, not
the letters of a word, but numbers assigned to words in a dictionary
or code book which Morse completed in 1837. This is (as we
shall see) an efficient form of coding, but it is clumsy.
While Morse was working with Alfred Vail, the old coding was
given up, and what we now know as the Morse code had been
devised by 1838. In this code, letters of the alphabet are represented
by spaces, dots, and dashes. The space is the absence of an electric
The Origins of Information Theory
25
current, the dot is an electric current of short duration, and the
dash is an electric current of longer duration.
Various combinations of dots and dashes were cleverly assigned
to the letters of the alphabet. E, the letter occurring most frequently
in English text, was represented by the shortest possible code
symbol, a single dot, and, in general, short combinations of dots
and dashes were used for frequently used letters and long combinations
for rarely used letters. Strangely enough, the choice was
not guided by tables of the relative frequencies of various letters
in English text nor were letters in text counted to get such data.
Relative frequencies of occurrence of various letters were estimated
by counting the number of types in the various compartments of
a printer’s type box!
We can ask, would some other assignment of dots, dashes, and
spaces to letters than that used by Morse enable us to send English
text faster by telegraph? Our modem theory tells us that we could
only gain about 15 per cent in speed. Morse was very successful
indeed in achieving his end, and he had the end clearly in mind.
The lesson provided by Morse’s code is that it matters profoundly
how one translates a message into electrical signals. This matter
is at the very heart of communication theory.
In 1843, Congress passed a bill appropriating money for the
construction of a telegraph circuit between Washington and Baltimore.
Morse started to lay the wire underground, but ran into
difficulties which later plagued submarine cables even more
severely. He solved his immediate problem by stringing the wire
on poles.
The difficulty which Morse encountered with his underground
wire remained an important problem. Different circuits which
conduct a steady electric current equally well are not necessarily
equally suited to electrical communication. If one sends dots and
dashes too fast over an underground or undersea circuit, they are
run together at the receiving end. As indicated in Figure II-1,
when we send a short burst of current which turns abruptly on and
off, we receive at the far end of the circuit a longer, smoothed-out
rise and fall of current. This longer flow of current may overlap
the current of another symbol sent, for instance, as an absence of
current. Thus, as shown in Figure II-2, when a clear and distinct
26 Symbols, Signals and Noise
SENT
0
RECEIVED
o TIME —►
Fig. Il l
signal is transmitted it may be received as a vaguely wandering
rise and fall of current which is difficult to interpret.
Of course, if we make our dots, spaces, and dashes long enough,
the current at the far end will follow the current at the sending end
better, but this slows the rate of transmission. It is clear that there
is somehow associated with a given transmission circuit a limiting
speed of transmission for dots and spaces. For submarine cables
this speed is so slow as to trouble telegraphers; for wires on poles
it is so fast as not to bother telegraphers. Early telegraphists were
aware of this limitation, and it, too, lies at the heart of communication
theory.
SENT
RECEIVED
Fig II-2
The Origins of Information Theory
27
Even in the face of this limitation on speed, various things can
be done to increase the number of letters which can be sent over
a given circuit in a given period of time. A dash takes three times
as long to send as a dot. It was soon appreciated that one could
gain by means of double-current telegraphy. We can understand
this by imagining that at the receiving end a galvanometer, a
device which detects and indicates the direction of flow of small
currents, is connected between the telegraph wire and the ground.
To indicate a dot, the sender connects the positive terminal of his
battery to the wire and the negative terminal to ground, and the
needle of the galvanometer moves to the right. To send a dash, the
sender connects the negative terminal of his battery to the wire and
the positive terminal to the ground, and the needle of the galvanometer
moves to the left. We say that an electric current in one
direction (into the wire) represents a dot and an electric current
in the other direction (out of the wire) represents a dash. No
current at all (battery disconnected) represents a space. In actual
double-current telegraphy, a different sort of receiving instrument
is used.
In single-current telegraphy we have two elements out of which
to construct our code: current and no current, which we might call
1 and 0. In double-current telegraphy we really have three elements,
which we might characterize as forward current, or current into
the wire; no current; backward current, or current out of the wire;
or as +1,0, — 1. Here the + or — sign indicates the direction of
current flow and the number 1 gives the magnitude or strength of
the current, which in this case is equal for current flow in either
direction.
In 1874, Thomas Edison went further; in his quadruplex telegraph
system he used two intensities of current as well as two
directions of current. He used changes in intensity, regardless of
changes in direction of current flow to send one message, and
changes of direction of current flow regardless of changes in
intensity, to send another message. If we assume the currents to
differ equally one from the next, we might represent the four
different conditions of current flow by means of which the two
messages are conveyed over the one circuit simultaneously as +3,
+ 1, — 1, —3. The interpretation of these at the receiving end is
shown in Table I.
28 Symbols, Signals and Noise
Table I
Current Transmitted
Message
Meaning
1 Message 2
+ 3 on on
+ 1 off on
-1 off off
-3 on off
Figure II-3 shows how the dots, dashes, and spaces of two
simultaneous, independent messages can be represented by a succession
of the four different current values.
Clearly, how much information it is possible to send over a
circuit depends not only on how fast one can send successive
symbols (successive current values) over the circuit but also on how
many different symbols (different current values) one has available
to choose among. If we have as symbols only the two currents +1
or 0 or, which is just as effective, the two currents + 1 and — 1,
we can convey to the receiver only one of two possibilities at a
time. We have seen above, however, that if we can choose among
any one of four current values (any one of four symbols) at a
MESSAGE
' : P_ r L T “ L _ r - L _ n ___
2
ON
OFFJ L
+ 3
+ 1
CURRENT
L
- 3
Fig. II-3
The Origins of Information Theory
29
time, such as +3 or +1 or — 1 or —3, we can convey by means
of these current values (symbols) two independent pieces of information:
whether we mean a 0 or 1 in message 1 and whether we
mean a 0 or 1 in message 2. Thus, for a given rate of sending successive
symbols, the use of four current values allows us to send two
independent messages, each as fast as two current values allow us
to send one message. We can send twice as many letters per minute
by using four current values as we could using two current values.
The use of multiplicity of symbols can lead to difficulties. We
have noted that dots and dashes sent over a long submarine cable
tend to spread out and overlap. Thus, when we look for one symbol
at the far end we see, as Figure II-2 illustrates, a little of several
others. Under these circumstances, a simple identification, as 1 or
0 or else +1 or — 1, is easier and more certain than a more complicated
indentification, as among +3, +1, — 1, —3.
Further, other matters limit our ability to make complicated
distinctions. During magnetic storms, extraneous signals appear
on telegraph lines and submarine cables.2 And if we look closely
enough, as we can today with sensitive electronic amplifiers, we
see that minute, undesired currents are always present. These are
akin to the erratic Brownian motion of tiny particles observed
under a microscope and to the agitation of air molecules and of
all other matter which we associate with the idea of heat and
temperature. Extraneous currents, which we call noise, are always
present to interfere with the signals sent.
Thus, even if we avoid the overlapping of dots and spaces which
is called intersymbol interference, noise tends to distort the received
signal and to make difficult a distinction among many alternative
symbols. Of course, increasing the current transmitted, which
means increasing the power of the transmitted signal, helps to
overcome the effect of noise. There are limits on the power that
can be used, however. Driving a large current through a submarine
cable takes a large voltage, and a large enough voltage can destroy
the insulation of the cable—can in fact cause a short circuit. It is
likely that the large transmitting voltage used caused the failure
of the first transatlantic telegraph cable in 1858.
2 The changing magnetic field of the earth induces currents in the cables. The
changes in the earth’s magnetic field are presumably caused by streams of charged
particles due to solar storms.
30
Symbols, Signals and Noise
Even the early telegraphists understood intuitively a good deal
about the limitations associated with speed of signaling, interference,
or noise, the difficulty in distinguishing among many alternative
values of current, and the limitation on the power that one
could use. More than an intuitive understanding was required,
however. An exact mathematical analysis of such problems was
needed.
Mathematics was early applied to such problems, though their
complete elucidation has come only in recent years. In 1855,
William Thomson, later Lord Kelvin, calculated precisely what
the received current will be when a dot or space is transmitted over
a submarine cable. A more powerful attack on such problems
followed the invention of the telephone by Alexander Graham
Bell in 1875. Telephony makes use, not of the slowly sent off-on
signals of telegraphy, but rather of currents whose strength varies
smoothly and subtly over a wide range of amplitudes with a
rapidity several hundred times as great as encountered in manual
telegraphy.
Many men helped to establish an adequate mathematical treatment
of the phenomena of telephony: Henri Poincare, the great
French mathematician; Oliver Heaviside, an eccentric, English,
minor genius; Michael Pupin, of From Immigrant to Inventor fame;
and G. A. Campbell, of the American Telephone and Telegraph
Company, are prominent among these.
The mathematical methods which these men used were an
extension of work which the French mathematician and physicist,
Joseph Fourier, had done early in the nineteenth century in connection
with the flow of heat. This work had been applied to the study
of vibration and was a natural tool for the analysis of the behavior
of electric currents which change with time in a complicated fashion—as
the electric currents of telephony and telegraphy do.
It is impossible to proceed further on our way without understanding
something of Fourier’s contribution, a contribution which
is absolutely essential to all communication and communication
theory. Fortunately, the basic ideas are simple; it is their proof and
the intricacies of their application which we shall have to omit here.
Fourier based his mathematical attack on some of the problems
of heat flow on a very particular mathematical function called a
The Origins of Information Theory
31
sine wave. Part of a sine wave is shown at the right of Figure II-4.
The height of the wave h varies smoothly up and down as time
passes, fluctuating so forever and ever. A sine wave has no beginning
or end. A sine wave is not just any smoothly wiggling curve.
The height of the wave (it may represent the strength of a current
or voltage) varies in a particular way with time. We can describe
this variation in terms of the motion of a crank connected to a shaft
which revolves at a constant speed, as shown at the left of Figure
II-4. The height h of the crank above the axle varies exactly
sinusoidally with time.
A sine wave is a rather simple sort of variation with time. It can
be characterized, or described, or differentiated completely from
any other sine wave by means of just three quantities. One of these
is the maximum height above zero, called the amplitude. Another
is the time at which the maximum is reached, which is specified
as the phase. The third is the time T between maxima, called the
period. Usually, we use instead of the period the reciprocal of the
period called the frequency, denoted by the letter f. If the period
T of a sine wave is 1/100 second, the frequency /is 100 cycles per
second, abbreviated cps. A cycle is a complete variation from
crest, through trough, and back to crest again. The sine wave is
periodic in that one variation from crest through trough to crest
again is just like any other.
Fourier succeeded in proving a theorem concerning sine waves
which astonished his, at first, incredulous contemporaries. He
showed that any variation of a quantity with time can be accurately
represented as the sum of a number of sinusoidal variations of
Fig. 11-4
32 Symbols, Signals and Noise
different amplitudes, phases, and frequencies. The quantity concerned
might be the displacement of a vibrating string, the height
of the surface of a rough ocean, the temperature of an electric iron,
or the current or voltage in a telephone or telegraph wire. All are
amenable to Fourier’s analysis. Figure II-5 illustrates this in a
simple case. The height of the periodic curve a above the centerline
is the sum of the heights of the sinusoidal curves b and c.
The mere representation of a complicated variation of some
physical quantity with time as a sum of a number of simple sinusoidal
variations might seem a mere mathematician’s trick. Its
utility depends on two important physical facts. The circuits used
in the transmission of electrical signals do not change with time,
and they behave in what is called a linear fashion. Suppose, for
instance, we send one signal, which we will call an input signal,
over the line and draw a curve showing how the amplitude of the
received signal varies with time. Suppose we send a second input
signal and draw a curve showing how the corresponding received
signal varies with time. Suppose we now send the sum of the two
input signals, that is, a signal whose current is at every moment
the simple sum of the currents of the two separate input signals.
Then, the received output signal will be merely the sum of the two
output signals corresponding to the input signals sent separately.
We can easily appreciate the fact that communication circuits
don’t change significantly with time. Linearity means simply that
(b)
Fig. II-5
The Origins of Information Theory
33
if we know the output signals corresponding to any number of
input signals sent separately, we can calculate the output signal
when several of the input signals are sent together merely by adding
the output signals corresponding to the input signals. In a linear
electrical circuit or transmission system, signals act as if they were
present independently of one another; they do not interact. This is,
indeed, the very criterion for a circuit being called a linear circuit.
While linearity is a truly astonishing property of nature, it is by
no means a rare one. All circuits made up of the resistors, capacitors,
and inductors discussed in Chapter I in connection with
network theory are linear, and so are telegraph lines and cables.
Indeed, usually electrical circuits are linear, except when they
include vacuum tubes, or transistors, or diodes, and sometimes
even such circuits are substantially linear.
Because telegraph wires are linear, which is just to say because
telegraph wires are such that electrical signals on them behave
independently without interacting with one another, two telegraph
signals can travel in opposite directions on the same wire at the
same time without interfering with one another. However, while
linearity is a fairly common phenomenon in electrical circuits, it
is by no means a universal natural phenomenon. Two trains can’t
travel in opposite directions on the same track without interference.
Presumably they could, though, if all the physical phenomena
comprised in trains were linear. The reader might speculate on the
unhappy lot of a truly linear race of beings.
With the very surprising property of linearity in mind, let us
return to the transmission of signals over electrical circuits. We
have noted that the output signal corresponding to most input
signals has a different shape or variation with time from the input
signal. Figures II-1 and II-2 illustrate this. However, it can be
shown mathematically (but not here) that, if we use a sinusoidal
signal, such as that of Figure II-4, as an input signal to a linear
transmission path, we always get out a sine wave of the same
period, or frequency. The amplitude of the output sine wave may
be less than that of the input sine wave; we call this attenuation of
the sinusoidal signal. The output sine wave may rise to a peak later
than the input sine wave; we call this phase shift, or delay of the
sinusoidal signal.
34
Symbols, Signals and Noise
The amounts of the attenuation and delay depend on the frequency
of the sine wave. In fact, the circuit may fail entirely to
transmit sine waves of some frequencies. Thus, corresponding to
an input signal made up of several sinusoidal components, there
will be an output signal having components of the same frequencies
but of different relative phases or delays and of different amplitudes.
Thus, in general the shape of the output signal will be
different from the shape of the input signal. However, the difference
can be thought of as caused by the changes in the relative delays
and amplitudes of the various components, differences associated
with their different frequencies. If the attenuation and delay of a
circuit is the same for all frequencies, the shape of the output wave
will be the same as that of the input wave; such a circuit is
distortionless.
Because this is a very important matter, I have illustrated it in
Figure II-6. In a we have an input signal which can be expressed
as the sum of the two sinusoidal components, b and c. In transmission,
b is neither attenuated nor delayed, so the output b' of
the same frequency as b is the same as b. However, the output c'
due to the input c is attenuated and delayed. The total output a',
the sum of b' and c', clearly has a different shape from the input
a. Yet, the output is made up of two components having the same
frequencies that are present in the input. The frequency components
merely have different relative phases or delays and different
relative amplitudes in the output than in the input.
The Fourier analysis of signals into components of various frequencies
makes it possible to study the transmission properties of
a linear circuit for all signals in terms of the attenuation and delay
it imposes on sine waves of various frequencies as they pass
through it.
Fourier analysis is a powerful tool for the analysis of transmission
problems. It provided mathematicians and engineers with a
bewildering variety of results which they did not at first clearly
understand. Thus, early telegraphists invented all sorts of shapes
and combinations of signals which were alleged to have desirable
properties, but they were often inept in their mathematics and
wrong in their arguments. There was much dispute concerning the
efficacy of various signals in ameliorating the limitations imposed
The Origins of Information Theory
35
by circuit speed, intersymbol interference, noise, and limitations
on transmitted power.
In 1917, Harry Nyquist came to the American Telephone and
Telegraph Company immediately after receiving his Ph.D. at Yale
(Ph.D.’s were considerably rarer in those days). Nyquist was a
much better mathematician than most men who tackled the problems
of telegraphy, and he always was a clear, original, and
philosophical thinker concerning communication. He tackled the
problems of telegraphy with powerful methods and with clear
insight. In 1924, he published his results in an important paper,
“Certain Factors Affecting Telegraph Speed.”
(a)
(b)
( c )
(b')
(c')
Fig. 11-6
36 Symbols, Signals and Noise
This paper deals with a number of problems of telegraphy.
Among other things, it clarifies the relation between the speed of
telegraphy and the number of current values such as +1, —1
(two current values) or +3, +1, — 1, —3 (four current values).
Nyquist says that if we send symbols (successive current values)
at a constant rate, the speed of transmission, W, is related to m,
the number of different symbols or current values available, by
W = K log m
Here K is a constant whose value depends on how many successive
current values are sent each second. The quantity log m means
logarithm of m. There are different bases for taking logarithms. If
we choose 2 as a base, then the values of log m for various values
of m are given in Table II.
Table II
m log m
i 0
2 1
3 1.6
4 2
8 3
16 4
To sum up the matter by means of an equation, log x is such a
number that
2i°e * = x
We may see by taking the logarithm of each side that the following
relation must be true:
log 2loe x = log x
If we write M in place of log x, we see that
log 2U = M
All of this is consistent with Table II.
We can easily see by means of an example why the logarithm is
the appropriate function in Nyquist’s relation. Suppose that we
The Origins of Information Theory
37
wish to specify two independent choices of off-or-on, 0-or-l, simultaneously.
There are four possible combinations of two independent
0-or-l choices, as shown in Table III.
Table III
Number of Combination First 0-OR-1
Choice
Second 0-or- 1
Choice
1 0 0
2 0 1
3 1 0
4 1 1
Further, if we wish to specify three independent choices of 0-or-l
at the same time, we find eight combinations, as shown in Table IV.
Table IV
Number of Combination First 0-OR-l
Choice
Second 0-OR-l
Choice
Third 0-OR-1
Choice
1 0 0 0
2 0 0 1
3 0 1 0
4 0 1 1
5 1 0 0
6 1 0 1
7 1 1 0
8 1 1 1
Similarly, if we wish to specify four independent 0-or-l choices,
we find sixteen different combinations, and, if we wish to specify
M different independent 0-or-l choices, we find 2U different
combinations.
If we can specify M independent 0-or-l combinations at once,
we can in effect send M independent messages at once, so surely
the speed should be proportional to M. But, in sending M messages
at once we have 2M possible combinations of the M independent
0-or-l choices. Thus, to send M messages at once, we need to be
able to send 2M different symbols or current values. Suppose that
we can choose among 2M different symbols. Nyquist tells us that
38
Symbols, Signals and Noise
we should take the logarithm of the number of symbols in order
to get the line speed, and
log 2« = M
Thus, the logarithm of the number of symbols is just the number
of independent 0-or-l choices that can be represented simultaneously,
the number of independent messages we can send at once,
so to speak.
Nyquist’s relation says that by going from off-on telegraphy to
three-current (+1,0, — 1) telegraphy we can increase the speed of
sending letters or other symbols by 60 per cent, and if we use four
current values ( + 3, +1, — 1, — 3) we can double the speed. This
is, of course, just what Edison did with his quadruplex telegraph,
for he sent two messages instead of one. Further, Nyquist showed
that the use of eight current values (0, 1, 2, 3, 4, 5, 6, 7, or +7, +5,
+ 3, +1, — 1, —3, —5, —7) should enable us to send four times
as fast as with two current values. However, he clearly realized that
fluctuations in the attenuation of the circuit, interference or noise,
and limitations on the power which can be used, make the use of
many current values difficult.
Turning to the rate at which signal elements can be sent, Nyquist
defined the line speed as one half of the number of signal elements
(dots, spaces, current values) which can be transmitted in a second.
We will find this definition particularly appropriate for reasons
which Nyquist did not give in this early paper.
By the time that Nyquist wrote, it was common practice to send
telegraph and telephone signals on the same wires. Telephony
makes use of frequencies above 150 cps, while telegraphy can be
carried out by means of lower frequency signals. Nyquist showed
how telegraph signals could be so shaped as to have no sinusoidal
components of high enough frequency to be heard as interference
by telephones connected to the same line. He noted that the line
speed, and hence also the speed of transmission, was proportional
to the width or extent of the range or band (in the sense of strip)
of frequencies used in telegraphy; we now call this range of frequencies
the band width of a circuit or of a signal.
Finally, in analyzing one proposed sort of telegraph signal,
The Origins of Information Theory
39
Nyquist showed that it contained at all times a steady sinusoidal
component of constant amplitude. While this component formed
a part of the transmitter power used, it was useless at the receiver,
for its eternal, regular fluctuations were perfectly predictable and
could have been supplied at the receiver rather than transmitted
thence over the circuit. Nyquist referred to this useless component
of the signal, which, he said, conveyed no intelligence, as redundant,
a word which we will encounter later.
Nyquist continued to study the problems of telegraphy, and in
1928 he published a second important paper, “Certain Topics in
Telegraph Transmission Theory.” In this he demonstrated a number
of very important points. He showed that if one sends some
number 2N of different current values per second, all the sinusoidal
components of the signal with frequencies greater than N are
redundant, in the sense that they are not needed in deducing from
the received signal the succession of current values which were sent.
If all of these higher frequencies were removed, one could still
deduce by studying the signal which current values had been
transmitted. Further, he showed how a signal could be constructed
which would contain no frequencies above N cps and from which
it would be very easy to deduce at the receiving point what current
values had been sent. This second paper was more quantitative and
exact than the first; together, they embrace much important material
that is now embodied in communication theory.
R. V. L. Hartley, the inventor of the Hartley oscillator, was
thinking philosophically about the transmission of information at
about this time, and he summarized his reflections in a paper,
“Transmission oflnformation,” which he published in 1928.
Hartley had an interesting way of formulating the problem of
communication, one of those ways of putting things which may
seem obvious when stated but which can wait years for the insight
that enables someone to make the statement. He regarded the
sender of a message as equipped with a set of symbols (the letters
of the alphabet for instance) from which he mentally selects symbol
after symbol, thus generating a sequence of symbols. He observed
that a chance event, such as the rolling of balls into pockets, might
equally well generate such a sequence. He then defined H, the
40 Symbols, Signals and Noise
information of the message, as the logarithm of the number of
possible sequences of symbols which might have been selected and
showed that
H = n log s
Here n is the number of symbols selected, and s is the number of
different symbols in the set from which symbols are selected.
This is acceptable in the light of our present knowledge of
information theory only if successive symbols are chosen independently
and if any of the s symbols is equally likely to be selected.
In this case, we need merely note, as before, that the logarithm of
s, the number of symbols, is the number of independent 0-or-l
choices that can be represented or sent simultaneously, and it is
reasonable that the rate of transmission of information should be
the rate of sending symbols per second n, times the number of
independent 0-or-1 choices that can be conveyed per symbol.
Hartley goes on to the problem of encoding the primary symbols
(letters of the alphabet, for instance) in terms of secondary symbols
(e.g., the sequences of dots, spaces, and dashes of the Morse code).
He observes that restrictions on the selection of symbols (the fact
that E is selected more often than Z) should govern the lengths of
the secondary symbols (Morse code representations) if we are to
transmit messages most swiftly. As we have seen, Morse himself
understood this, but Hartley stated the matter in a way which
encouraged mathematical attack and inspired further work. Hartley
also suggested a way of applying such considerations to continuous
signals, such as telephone signals or picture signals.
Finally, Hartley stated, in accord with Nyquist, that the amount
of information which can be transmitted is proportional to the
band width times the time of transmission. But this makes us
wonder about the number of allowable current values, which is also
important to speed of transmission. How are we to enumerate
them?
After the work of Nyquist and Hartley, communication theory
appears to have taken a prolonged and comfortable rest. Workers
busily built and studied particular communication systems. The
art grew very complicated indeed during World War II. Much new
understanding of particular new communication systems and
The Origins of Information Theory 41
devices was achieved, but no broad philosophical principles were
laid down.
During the war it became important to predict from inaccurate
or “noisy” radar data the courses of airplanes, so that the planes
could be shot down. This raised an important question: Suppose
that one has a varying electric current which represents data concerning
the present position of an airplane but that there is added
to it a second meaningless erratic current, that is, a noise. It may
be that the frequencies most strongly present in the signal are
different from the frequencies most strongly present in the noise.
If this is so, it would seem desirable to pass the signal with the noise
added through an electrical circuit or filter which attenuates the
frequencies strongly present in the noise but does not attenuate
very much the frequencies strongly present in the signal. Then, the
resulting electric current can be passed through other circuits in
an effort to estimate or predict what the value of the original signal,
without noise, will be a few seconds from the present. But what
sort of combination of electrical circuits will enable one best to
predict from the present noisy signal the value of the true signal
a few seconds in the future?
In essence, the problem is one in which we deal with not one but
with £ whole ensemble of possible signals (courses of the plane),
so that we do not know in advance which signal we are dealing
with. Further, we are troubled with an unpredictable noise.
This problem was solved in Russia by A. N. Kolmogoroff. In this
country it was solved independently by Norbert Wiener. Wiener
is a mathematician whose background ideally fitted him to deal
with this sort of problem, and during the war he produced a
yellow-bound document, affectionately called “the yellow peril”
(because of the headaches it caused), in which he solved the difficult
problem.
During and after the war another mathematician, Claude E.
Shannon, interested himself in the general problem of communication.
Shannon began by considering the relative advantages of
many new and fanciful communication systems, and he sought
some basic method of comparing their merits. In the same year
(1948) that Wiener published his book, Cybernetics, which deals
with communication and control, Shannon published in two parts
42 Symbols, Signals and Noise
a paper which is regarded as the foundation of modem communication
theory.
Wiener and Shannon alike consider, not the problem of a single
signal, but the problem of dealing adequately with any signal
selected from a group or ensemble of possible signals. There was
a free interchange among various workers before the publication
of either Wiener’s book or Shannon’s paper, and similar ideas and
expressions appear in both, although Shannon’s interpretation
appears to be unique.
Chiefly, Wiener’s name has come to be associated with the field
of extracting signals of a given ensemble from noise of a known
type. An example of this has been given above. The enemy pilot
follows a course which he choses, and our radar adds noise of
natural origin to the signals which represent the position of the
plane. We have a set of possible signals (possible courses of the
airplane), not of our own choosing, mixed with noise, not of our
own choosing, and we try to make the best estimate of the present
or future value of the signal (the present or future position of the
airplane) despite the noise.
Shannon’s name has come to be associated with matters of so
encoding messages chosen from a known ensemble that they can
be transmitted accurately and swiftly in the presence of noise. As
an example, we may have as a message source English text, not
of our own choosing, and an electrical circuit, say, a noisy telegraph
cable, not of our own choosing. But in the problem treated by
Shannon, we are allowed to choose how we shall represent the
message as an electrical signal—how many current values we shall
allow, for instance, and how many we shall transmit per second.
The problem, then, is not how to treat a signal plus noise so as to
get a best estimate of the signal, but what sort of signal to send
so as best to convey messages of a given type over a particular sort
of noisy circuit.
This matter of efficient encoding and its consequences form the
chief substance of information theory. In that an ensemble of
messages is considered, the work reflects the spirit of the work of
Kolmogoroff and Wiener and of the work of Morse and Hartley
as well.
It would be useless to review here the content of Shannon’s
The Origins of Information Theory
43
work, for that is what this book is about. We shall see, however,
that it sheds further light on all the problems raised by Nyquist
and Hartley and goes far beyond those problems.
In looking back on the origins of communication theory, two
other names should perhaps be mentioned. In 1946, Dennis Gabor
published an ingenious paper, “Theory of Communication.” This,
suggestive as it is, missed the inclusion of noise, which is at the
heart of modern communication theory. Further, in 1949, W. G.
Tuller published an interesting paper, “Theoretical Limits on the
Rate of Transmission of Information,” which in part parallels
Shannon’s work.
The gist of this chapter has been that the very general theory of
communication which Shannon has given us grew out of the study
of particular problems of electrical communication. Morse was
faced with the problem of representing the letters of the alphabet
by short or long pulses of current with intervening spaces of no
current—that is, by the dots, dashes, and spaces of telegraphy. He
wisely chose to represent common letters by short combinations
of dots and dashes and uncommon letters by long combinations;
this was a first step in efficient encoding of messages, a vital part
of communication theory.
Ingenious inventors who followed Morse made use of different
intensities and directions of current flow in order to give the sender
a greater choice of signals than merely off-or-on. This made it
possible to send more letters per unit time, but it made the signal
more susceptible to disturbance by unwanted electrical disturbances
called noise as well as by inability of circuits to transmit
accurately rapid changes of current.
An evaluation of the relative advantages of many different sorts
of telegraph signals was desirable. Mathematical tools were needed
for such a study. One of the most important of these is Fourier
analysis, which makes it possible to represent any signal as a sum
of sine waves of various frequencies.
Most communication circuits are linear. This means that several
signals present in the circuit do not interact or interfere. It can be
shown that while even linear circuits change the shape of most
signals, the effect of a linear circuit on a sine wave is merely to
make it weaker and to delay its time of arrival. Hence, when a
44
Symbols, Signals and Noise
complicated signal is represented as a sum of sine waves of various
frequencies, it is easy to calculate the effect of a linear circuit on
each sinusoidal component separately and then to add up the
weakened or attenuated sinusoidal components in order to obtain
the over-all received signal.
Nyquist showed that the number of distinct, different current
values which can be sent over a circuit per second is twice the total
range or band width of frequencies used. Thus, the rate at which
letters of text can be transmitted is proportional to band width.
Nyquist and Hartley also showed that the rate at which letters of
text can be transmitted is proportional to the logarithm of the
number of current values used.
A complete theory of communication required other mathematical
tools and new ideas. These are related to work done by
Kolmogoroff and Wiener, who considered the problem of an
unknown signal of a given type disturbed by the addition of noise.
How does one best estimate what the signal is despite the presence
of the interfering noise? Kolmogoroff and Wiener solved this
problem.
The problem Shannon set himself is somewhat different. Suppose
we have a message source which produces messages of a given type,
such as English text. Suppose we have a noisy communication
channel of specified characteristics. How can we represent or
encode messages from the message source by means of electrical
signals so as to attain the fastest possible transmission over the
noisy channel? Indeed, how fast can we transmit a given type of
message over a given channel without error? In a rough and general
way, this is the problem that Shannon set himself and solved.
CHAPTER A Mathematical
Model
A mathematical theory which seeks to explain and to predict
the events in the world about us always deals with a simplified
model of the world, a mathematical model in which only things
pertinent to the behavior under consideration enter.
Thus, planets are composed of various substances, solid, liquid,
and gaseous, at various pressures and temperatures. The parts of
their substances exposed to the rays of the sun reflect various
fractions of the different colors of the light which falls upon them,
so that when we observe planets we see on them various colored
features. However, the mathematical astronomer in predicting the
orbit of a planet about the sun need take into account only the total
mass of the sun, the distance of the planet from the sun, and the
speed and direction of the planet’s motion at some initial instant.
For a more refined calculation, the astronomer must also take into
account the total mass of the planet and the motions and masses
of other planets which exert gravitational forces on it.
This does not mean that astronomers are not concerned with
other aspects of planets, and of stars and nebulae as well. The
important point is that they need not take these other matters into
consideration in computing planetary orbits. The great beauty and
power of a mathematical theory or model lies in the separation of
the relevant from the irrelevant, so that certain observable behavior
45
46
Symbols, Signals and Noise
can be related and understood without the need of comprehending
the whole nature and behavior of the universe.
Mathematical models can have various degrees of accuracy or
applicability. Thus, we can accurately predict the orbits of planets
by regarding them as rigid bodies, despite the fact that no truly
rigid body exists. On the other hand, the long-term motions of our
moon can only be understood by taking into account the motion
of the waters over the face of the earth, that is, the tides. Thus, in
dealing very precisely with lunar motion we cannot regard the
earth as a rigid body.
In a similar way, in network theory we study the electrical
properties of interconnections of ideal inductors, capacitors, and
resistors, which are assigned certain simple mathematical properties.
The components of which the actual useful circuits in radio,
TV, and telephone equipment are made only approximate the
properties of the ideal inductors, capacitors, and resistors of network
theory. Sometimes, the difference is trivial and can be disregarded.
Sometimes it must be taken into account by more refined
calculations.
Of course, a mathematical model may be a very crude or even
an invalid representation of events in the real world. Thus, the
self-interested, gain-motivated “economic man” of early economic
theory has fallen into disfavor because the behavior of the economic
man does not appear to correspond to or to usefully explain
the actual behavior of our economic world and of the people in it.
In the orbits of the planets and the behavior of networks, we
have examples of idealized deterministic systems which have the
sort of predictable behavior we ordinarily expect of machines.
Astronomers can compute the positions which the planets will
occupy millennia in the future. Network theory tells us all the
subsequent behavior of an electrical network when it is excited by
a particular electrical signal.
Even the individual economic man is deterministic, for he will
always act for his economic gain. But, if he at some time gambles
on the honest throw of a die because the odds favor him, his
economic fate becomes to a degree unpredictable, for he may lose
even though the odds do favor him.
We can, however, make a mathematical model for purely chance
A Mathematical Model 47
events, such as the drawing of some number, say three, of white
or black balls from a container holding equal numbers of white
and black balls. This model tells us, in fact, that after many trials
we will have drawn all white about Vs of the time, two whites and
a black about Vs of the time, two blacks and a white about % of
the time, and all black about % of the time. It can also tell us how
much of a deviation from these proportions we may reasonably
expect after a given number of trials.
Our experience indicates that the behavior of actual human
beings is neither as determined as that of the economic man nor
as simply random as the throw of a die or as the drawing of balls
from a mixture of black and white balls. It is clear, however, that
a deterministic model will not get us far in the consideration of
human behavior, such as human communication, while a random
or statistical model might.
We all know that the actuarial tables used by insurance companies
make fair predictions of the fraction of a large group of men
in a given age group who will die in one year, despite the fact that
we cannot predict when a particular man will die. Thus a statistical
model may enable us to understand and even to make some sort
of predictions concerning human behavior, even as we can predict
how often, on the average, we will draw three black balls by chance
from an equal mixture of white and black balls.
It might be objected that actuarial tables make predictions concerning
groups of people, not predictions concerning individuals.
However, experience teaches us that we can make predictions
concerning the behavior of individual human beings as well as of
groups of individuals. For instance, in counting the frequency of
usage of the letter E in all English prose we will find that E constitutes
about 0.13 of all the letters appearing, while W, for instance,
constitutes only about 0.02 of all letters appearing. But, we also
find almost the same proportions of E’s and W’s in the prose
written by any one person. Thus, we can predict with some confidence
that if you, or I, or Joe Doakes, or anyone else writes a long
letter, or an article, or a book, about 0.13 of the letters he uses will
be E’s.
This predictability of behavior limits our freedom no more than
does any other habit. We don’t have to use in our writing the same
48
Symbols, Signals and Noise
fraction of E’s, or of any other letter, that everyone else does. In
fact, several untrammeled individuals have broken away from the
common pattern. William F. Friedman, the eminent cryptanalyst
and author of The Shakesperian Cipher Examined, has supplied
me with the following examples.
Gottlob Burmann, a German poet who lived from 1737 to 1805,
wrote 130 poems, including a total of 20,000 words, without once
using the letter R. Further, during the last seventeen years of his
life, Burmann even omitted the letter from his daily conversation.
In each of five stories published by Alonso Alcala y Herrera in
Lisbon in 1641 a different vowel was suppressed. Francisco Navar-
rete y Ribera (1659), Fernando Jacinto de Zurita y Haro (1654),
and Manuel Lorenzo de Lizarazu y Berbuizana (1654) provided
other examples.
In 1939, Ernest Vincent Wright published a 267-page novel,
Gadsby, in which no use is made of the letter E. I quote a paragraph
below:
Upon this basis I am going to show you how a bunch of bright young
folks did find a champion; a man with boys and girls of his own; a man
of so dominating and happy individuality that Youth is drawn to him as
is a fly to a sugar bowl. It is a story about a small town. It is not a gossipy
yarn; nor is it a dry, monotonous account, full of such customary “fill-ins”
as “romantic moonlight casting murky shadows down a long, winding
country road.” Nor will it say anything about tinklings lulling distant
folds; robins carolling at twilight, nor any “warm glow of lamplight” from
a cabin window. No. It is an account of up-and-doing activity; a vivid
portrayal of Youth as it is today; and a practical discarding of that worn-
out notion that “a child don’t know anything.”
While such exercises of free will show that it is not impossible
to break the chains of habit, we ordinarily write in a more conventional
manner. When we are not going out of our way to demonstrate
that we can do otherwise, we customarily use our due
fraction of 0.13 E’s with almost the consistency of a machine or a
mathematical rule.
We cannot argue from this to the converse idea that a machine
into which the same habits were built could write English text.
However, Shannon has demonstrated how English words and text
A Mathematical Model 49
can be approximated by a mathematical process which could be
carried out by a machine.
Suppose, for instance, that we merely produce a sequence of
letters and spaces with equal probabilities. We might do this by
putting equal numbers of cards marked with each letter and with
the space into a hat, mixing them up, drawing a card, recording
its symbol, returning it, remixing, drawing another card, and so
on. This gives what Shannon calls the zero-order approximation
to English text. His example, obtained by an equivalent process,
goes:
1. Zero-order approximation (symbols independent and equi-
probable)
XFOML RXKHRJFFJUJ ZLPWCFWKCYJ FFJEYVKCQSGHYD
QPAAMKBZAACIBZLHJQD.
Here there are far too many Zs and Ws, and not nearly enough
E’s and spaces. We can approach more nearly to English text by
choosing letters independently of one another, but choosing E
more often than W or Z. We could do this by putting many E’s
and few W’s and Z’s into the hat, mixing, and drawing out the
letters. As the probability that a given letter is an E should be .13,
out of every hundred letters we put into the hat, 13 should be E’s.
As the probability that a letter will be W should be .02, out of each
hundred letters we put into the hat, 2 should be W’s, and so on.
Here is the result of an equivalent procedure, which gives what
Shannon calls a first-order approximation of English text:
2. First-order approximation (symbols independent but with
frequencies of English text).
OCRO HLI RGWR NMIELWIS EU LL NBNESEBYA TH
EEI ALHENHTTPA OOBTTVA NAH BRL
In English text we almost never encounter any pair of letters
beginning with Q except QU. The probability of encountering QX
or QZ is essentially zero. While the probability of QU is not 0, it is
so small as not to be listed in the tables I consulted. On the other
hand, the probability of TH is .037, the probability of OR is .010
and the probability of WE is .006. These probabilities have the
following meaning. In a stretch of text containing, say, 10,001
50
Symbols, Signals and Noise
letters, there are 10,000 successive pairs of letters, i.e., the first and
second, the second and third, and so on to the next to last and the
last. Of the pairs a certain number are the letters TH. This might
be 370 pairs. If we divide the total number of times we find TH,
which we have assumed to be 370 times, by the total number of
pairs of letters, which we have assumed to be 10,000, we get the
probability that a randomly selected pair of letters in the text will
be TH, that is, 370/10,000, or .037.
Diligent cryptanalysts have made tables of such digram probabilities
for English text. To see how we might use these in constructing
sequences of letters with the same digram probabilities
as English text, let us assume that we use 27 hats, 26 for digrams
beginning with each of the letters and one for digrams beginning
with a space. We will then put a large number of digrams into the
hats according to the probabilities of the digrams. Out of 1,000
digrams we would put in 37 TH’s, 10 WE’s, and so on.
Let us consider for a moment the meaning of these hats full of
digrams in terms of the original counts which led to the evaluations
of digram probabilities.
In going through the text letter by letter we will encounter every
T in the text. Thus, the number of digrams beginning with T, all
of which we put in one hat, will be the same as the number of T’s.
The fraction these represent of the total number of digrams counted
is the probability of encountering T in the text; that is, .10. We
might call this probability p(T)
Pi T) = .10
We may note that this is also the fraction of digrams, distributed
among the hats, which end in T as well as the fraction that begin
with T.
Again, basing our total numbers on 1,001 letters of text, or
1,000 digrams, the number of times the digram TH is encountered
is 37, and so the probability of encountering the digram TH, which
we might call p(T, H) is
p(T, H) = .037
Now we see that 0.10, or 100, of the digrams will begin with T
and hence will be in the T hat and of these 37 will be TH. Thus,
A Mathematical Model 51
the fraction of the T digrams which are TH will be 37/100, or 0.37.
Correspondingly, we say that the probability that a digram beginning
with T is TH, which we might call/>t(H), is
/>t(H) = .37
This is called the conditional probability that the letter following a
T will be an H.
One can use these probabilities, which are adequately represented
by the numbers of various digrams in the various hats, in
the construction of text which has both the same letter frequencies
and digram frequencies as does English text. To do this one draws
the first digram at random from any hat and writes down its letters.
He then draws a second digram from the hat indicated by the
second letter of the first digram and writes down the second letter
of this second digram. Then he draws a third digram from the hat
indicated by the second letter of the second digram and writes
down the second letter of this third digram, and so on. The space
is treated just like a letter. There is a particular probability that a
space will follow a particular letter (ending a “word”) and a
particular probability that a particular letter will follow a space
(starting a new “word”).
By an equivalent process, Shannon constructed what he calls a
second-order approximation to English; it is:
3. Second-order approximation (digram structure as in English).
ON IE ANTSOUTINYS ARE T IN CT O RE ST BE S
DEAMY ACHIN D ILONASIVE TUCOOWE AT TEASONARE
FUSO TIZIN ANDY TOBE SEACE CTISBE
Cryptanalysts have even produced tables giving the probabilities
of groups of three letters, called trigram probabilities. These can
be used to construct what Shannon calls a third-order approximation
to English. His example goes:
4. Third-order approximation (trigram structure as in English).
IN NO 1ST LAT WHEY CRATICT FROURE BIRS
GROCID PONDENOME OF DEMONSTURES OF THE
REPTAGIN IS REGOACTIONA OF CRE
When we examine Shannon’s examples 1 through 4 we see an
increasing resemblance to English text. Example 1, the zero-order
52
Symbols, Signals and Noise
approximation, has no wordlike combinations. In example 2, which
takes letter frequencies into account, ocro and nah somewhat
resemble English words. In example 3, which takes digram frequencies
into account, all the “words” are pronounceable, and on, are,
be, at, and andy occur in English. In example 4, which takes
trigram frequencies into account, we have eight English words and
many English-sounding words, such as grocid, pondenome, and
DEMONSTURES.
G. T. Guilbaud has carried out a similar process using the
statistics of Latin and has so produced a third-order approximation
(one taking into account trigram frequencies) resembling Latin,
which I quote below:
IBUS CENT IPITIA VETIS IPSE CUM VIVIVS
SE ACETITI DEDENTUR
The underlined words are genuine Latin words.
It is clear from such examples that by giving a machine certain
statistics of a language, the probabilities of finding a particular
letter or group of 1, or 2, or 3, or n letters, and by giving the
machine an ability equivalent to picking a ball from a hat, flipping
a coin, or choosing a random number, we could make the machine
produce a close approximation to English text or to text in some
other language. The more complete information we gave the
machine, the more closely would its product resemble English or
other text, both in its statistical structure and to the human eye.
If we allow the machine to choose groups of three letters on the
basis of their probability, then any three-letter combination which
it pfoduces must be an English word or a part of an English word
and any two letter “word” must be an English word. The machine
is, however, less inhibited than a person, who ordinarily writes
down only sequences of letters which do spell words. Thus, he
misses ever writing down pompous pondenome, suspect ilonasive,
somewhat vulgar grocid, learned demonstures, and wacky but
delightful deamy. Of course, a man in principle could write down
such combinations of letters but ordinarily he doesn’t.
We could cure the machine of this ability to produce un-English
words by making it choose among groups of letters as long as the
longest English word. But, it would be much simpler merely to
A Mathematical Model 53
supply the machine with words rather than letters and to let it
produce these words according to certain probabilities.
Shannon has given an example in which words were selected
independently, but with the probabilities of their occurring in
English text, so that the, and, man, etc., occur in the same proportion
as in English. This could be achieved by cutting text into
words, scrambling the words in a hat, and then drawing out a
succession of words. He calls this a first-order word approximation.
It runs as follows:
5. First-order word approximation. Here words are chosen independently
but with their appropriate frequencies.
REPRESENTING AND SPEEDILY IS AN GOOD APT
OR COME CAN DIFFERENT NATURAL HERE HE THE
A IN CAME THE TO OF TO EXPERT GRAY COME
TO FURNISHES THE LINE MESSAGE HAD BE THESE
There are no tables which give the probability of different pairs
of words. However, Shannon constructed a random passage in
which the probabilities of pairs of words were the same as in
English text by the following expedient. He chose a first pair of
words at random in a novel. He then looked through the novel for
the next occurrence of the second word of the first pair and added
the word which followed it in this new occurrence, and so on.
This process gave him the following second-order word approximation
to English.
6. Second-order word approximation. The word transition probabilities
are correct, but no further structure is included.
THE HEAD AND IN FRONTAL ATTACK ON AN
ENGLISH WRITER THAT THE CHARACTER OF THIS
POINT IS THEREFORE ANOTHER METHOD FOR THE
LETTERS THAT THE TIME OF WHO EVER TOLD THE
PROBLEM FOR AN UNEXPECTED.
We see that there are stretches of several words in this final
passage which resemble and, indeed, might occur in English text.
Let us consider what we have found. In actual English text, in
that text which we send by teletypewriter, for instance, particular
letters occur with very nearly constant frequencies. Pairs of letters
54
Symbols, Signals and Noise
and triplets and quadruplets of letters occur with almost constant
frequencies over long stretches of the text. Words and pairs of
words occur with almost constant frequencies. Further, we can by
means of a random mathematical process, carried out by a machine
if you like, produce sequences of English words or letters exhibiting
these same statistics.
Such a scheme, even if refined greatly, would not, however,
produce all sequences of words that a person might utter. Carried
to an extreme, it would be confined to combinations of words
which had occurred; otherwise, there would be no statistical data
available on them. Yet I may say, “The magenta typhoon whirled
the farded bishop away,” and this may well never have been said
before.
The real rules of English text deal not with letters or words alone
but with classes of words and their rules of association, that is, with
grammar. Linguists and engineers who try to make machines for
translating one language into another must find these rules, so that
their machines can combine words to form grammatical utterances
even when these exact combinations have not occurred before
(and also so that the meaning of words in the text to be translated
can be deduced from the context). This is a big problem. It is easy,
however, to describe a “machine” which randomly produces endless,
grammatical utterances of a limited sort.
Figure III-l is a diagram of such a “machine.” Each numbered
box represents a state of the machine. Because there is only a finite
number of boxes or states, this is called a finite-state machine.
From each box a number of arrows go to other boxes. In this
particular machine, only two arrows go from each box to each of
two other boxes. Also, in this case, each arrow is labeled Vi. This
indicates that the probability of the machine passing from, for
instance, state 2 to state 3 is Vi and the probability of the machine
passing from state 2 to state 4 is Vi.
To make the machine run, we need a sequence of random
choices, which we can obtain by flipping a coin repeatedly. We can
let heads (H) mean follow the top arrow and tails (T), follow the
bottom arrow. This will tell us to pass to a new state. When we do
this we print out the word, words, or symbol written in that state
box and flip again to get a new state.
Fig. I l l l
/) Mathematical Model
L/i L/i
56
Symbols, Signals and Noise
As an example, if we started in state 7 and flipped the following
sequence of heads and tails: T H H H T T H T T T H H H
H, the “machine would print out”
THE COMMUNIST PARTY INVESTIGATED THE CONGRESS.
THE COMMUNIST PARTY PURGED THE CONGRESS AND
DESTROYED THE COMMUNIST PARTY AND FOUND
EVIDENCE OF THE CONGRESS.
This can go on and on, never retracing its whole course and
producing “sentences” of unlimited length.
Random choice according to a table of probabilities of sequences
of symbols (letters and space) or words can produce material
resembling English text. A finite-state machine with a random
choice among allowed transitions from state to state can produce
material resembling English text. Either process is called a stochastic
process, because of the random element involved in it.
We have examined a number of properties of English text. We
have seen that the average frequency of E’s is commonly constant
for both the English text produced by one writer and, also, for the
text produced by all writers. Other more complicated statistics,
such as the frequency of digrams (TH, WE, and other letter pairs),
are also essentially constant. Further, we have shown that English-
like text can be produced by a sequence of random choices, such
as drawings of slips of paper from hats, or flips of a coin, if the
proper probabilities are in some way built into the process. One
way of producing such text is through the use of a finite-state
machine, such as that of Figure III-l.
We have been seeking a mathematical model of a source of
English text. Such a model should be capable of producing text
which corresponds closely to actual English text, closely enough
so that the problem of encoding and transmitting such text is
essentially equivalent to the problem of encoding and transmitting
actual English text. The mathematical properties of the model must
be mathematically defined so that useful theorems can be proved
concerning the encoding and transmission of the text is produces,
theorems which are applicable to a high degree of approximation
to the encoding of actual English text. It would, however, be asking
too much to insist that the production of actual English text conform
with mathematical exactitude to the operation of the model.
A Mathematical Model 57
The mathematical model which Shannon adopted to represent
the production of text (and of spoken and visual messages as well)
is the ergodic source. To understand what an ergodic source is, we
must first understand what a stationary source is, and to explain
this is our next order of business.
The general idea of a stationary source is well conveyed by the
name. Imagine, for instance, a process, i.e., an imaginary machine,
that produces forever after it is started the sequences of characters
A E A E A E A E A E , etc.
Clearly, what comes later is like what has gone before, and
stationary seems an apt designation of such a source of characters.
We might contrast this with a source of characters which, after
starting, produced
A E A A E E A A A E E E, etc.
Here the strings of A’s and E’s get longer and longer without end;
certainly this is not a stationary source.
Similarly, a sequence of characters chosen at random with some
assigned probabilities (the first-order letter approximation of example
1 above) constitutes a stationary source and so do the
digram and trigram sources of examples 2 and 3. The general idea
of a stationary source is clear enough. An adequate mathematical
definition is a little more difficult.
The idea of stationarity of a source demands no change with
time. Yet, consider a digram source, in which the probability of
the second character depends on what the previous character is.
If we start such a source out on the letter A, several different
letters can follow, while if we start such a source out on the letter
Q, the second letter must be U. In general, the maimer of starting
the source will influence the statistics of the sequence of characters
produced, at least for some distance from the start.
To get around this, the mathematician says, let us not consider
just one sequence of characters produced by the source. After all,
our source is an imaginary machine, and we can quite well imagine
that it has been started an infinite number of times, so as to produce
an infinite number of sequences of characters. Such an infinite
number of sequences is called an ensemble of sequences.
These sequences could be started in any specified manner. Thus,
58
Symbols, Signals and Noise
in the case of a digram source, we can if we wish start a fraction,
0.13, of the sequences with E (this is just the probability of E in
English text), a fraction, 0.02, with W (the probability of W), and
so on. If we do this, we will find that the fraction of E’s is the same,
averaging over all the first letters of the ensemble of sequences, as
it is averaging over all the second letters of the ensemble, as it is
averaging over all the third letters of the ensemble, and so on. No
matter what position from the beginning we choose, the fraction
of E’s or of any other letter occurring in that position, taken over
all the sequences in the ensemble, is the same. This independence
with respect to position will be true also for the probability with
which TH or WE occurs among the first, second, third, and subsequent
pairs of letters in the sequences of the ensemble.
This is what we mean by stationarity. If we can find a way of
assigning probabilities to the various starting conditions used in
forming the ensemble of sequences of characters which we allow
the source to produce, probabilities such that any statistic obtained
by averaging over the ensemble doesn’t depend on the distance
from the start at which we take an average, then the source is said
to be stationary. This may seem difficult or obscure to the reader,
but the difficulty arises in giving a useful and exact mathematical
form to an idea which would otherwise be mathematically useless.
In the argument above we have, in discussing the infinite ensemble
of sequences produced by a source, considered averaging
over-all first characters or over-all second or third characters (or
pairs, or triples of characters, as other examples). Such an average
is called an ensemble average. It is different from a sort of average
we talked about earlier in this chapter, in which we lumped
together all the characters in one sequence and took the average
over them. Such an average is called a time average.
The time average and the ensemble average can be different.
For instance, consider a source which starts a third of the time with
A and produces alternately A and B, a third of the time with B and
produces alternately B and A, and a third of the time with E and
produces a string of E’s. The possible sequences are
1. A B A B A B A B , etc.
2. B A B A B A B A , etc.
3. E E E E E E E E , etc.
A Mathematical Model 59
We can see that this is a stationary source, yet we have the
probabilities shown in Table V.
Table V
Probability
°f
Tim e Average
Tim e A verage
Tim e A verage
Sequence (1)
Sequence (2)
Sequence (3)
Ensem ble
A verage
A Vi Vi 0 Vs
B Vi Vi 0 Vs
E 0 0 1 Vs
When a source is stationary, and when every possible ensemble
average (of letters, digrams, trigrams, etc.) is equal to the corresponding
time average, the source is said to be ergodic. The
theorems of information theory which are discussed in subsequent
chapters apply to ergodic sources, and their proofs rest on the
assumption that the message source is ergodic.1
While we have here discussed discrete sources which produce
sequences of characters, information theory also deals with continuous
sources, which generate smoothly varying signals, such as
the acoustic waves of speech or the fluctuating electric currents
which correspond to these in telephony. The sources of such signals
are also assumed to be ergodic.
Why is an ergodic message source an appropriate and profitable
mathematical model for study? For one thing, we see by examining
the definition of an ergodic source as given above that for an
ergodic source the statistics of a message, for instance, the frequency
of occurrence of a letter, such as E, or of a digram, such
as TH, do not vary along the length of the message. As we analyze
a longer and longer stretch of a message, we get a better and better
estimate of the probabilities of occurrence of various letters and
letter groups. In other words, by examining a longer and longer
stretch of a message we are able to arrive at and refine a mathematical
description of the source.
Further, the probabilities, the description of the source arrived
at through such an examination of one message, apply equally
well to all messages generated by the source and not just to the
1 Some work has been done on the encoding of nonstationary sources, but it is
not discussed in this book.
60
Symbols, Signals and Noise
particular message examined. This is assured by the fact that the
time and ensemble averages are the same.
Thus, an ergodic source is a particularly simple kind of probabilistic
or stochastic source of messages, and simple processes are
easier to deal with mathematically than are complicated processes.
However, simplicity in itself is not enough. The ergodic source
would not be of interest in communication theory if it were not
reasonably realistic as well as simple.
Communication theory has two sides. It has a mathematically
exact side, which deals rigorously with hypothetical, exactly ergodic
sources, sources which we can imagine to produce infinite ensembles
of infinite sequences of symbols. Mathematically, we are
free to investigate rigorously either such a source itself or the
infinite ensemble of messages which it can produce.
We use the theorems of communication theory in connection
with the transmission of actual English text. A human being is not
a hypothetical, mathematically defined machine. He cannot produce
even one infinite sequence of characters, let alone an infinite
ensemble of sequences.
A man does, however, produce many long sequences of characters,
and all the writers of English together collectively produce a
great many such long sequences of characters. In fact, part of this
huge output of very long sequences of characters constitutes the
messages actually sent by teletypewriter.
We will, thus, think of all the different Americans who write out
telegrams in English as being, approximately at least, an ergodic
source of telegraph messages and of all Americans speaking over
telephones as being, approximately at least, an ergodic source of
telephone signals. Clearly, however, all men writing French plus
all men writing English could not constitute an ergodic source. The
output of each would have certain time-average probabilities for
letters, digrams, trigrams, words, and so on, but the probabilities
for the English text would be different from the probabilities for
the French text, and the ensemble average would resemble neither.
We will not assert that all writers of English (and all speakers
of English) constitute a strictly ergodic message source. The statistics
of the English we produce change somewhat as we change
subject or purpose, and different people write somewhat differently.
A Mathematical Model 61
Too, in producing telephone signals by speaking, some people
speak softly, some bellow, and some bellow only when they are
angry. What we do assert is that we find a remarkable uniformity
in many statistics of messages, as in the case of the probability of
E for different samples of English text. Speech and writing as
ergodic sources are not quite true to the real world, but they are
far truer than is the economic man. They are true enough to be
useful.
This difference between the exactly ergodic source of the mathematical
theory of communication and the approximately ergodic
message sources of the real world should be kept in mind. We must
exercise a reasonable caution in applying the conclusions of the
mathematical theory of communication to actual problems. We are
used to this in other fields. For instance, mathematics tells us that
we can deduce the diameter of a circle from the coordinates or
locations of any three points on the circle, and this is true for
absolutely exact coordinates. Yet no sensible man would try to
determine the diameter of a somewhat fuzzy real circle drawn on
a sheet of paper by trying to measure very exactly the positions of
three points a thousandth of an inch apart on its circumference.
Rather, he would draw a line through the center and measure the
diameter directly as the distance between diametrically opposite
points. This is just the sort of judgment and caution one must
always use in applying an exact mathematical theory to an inexact
practical case.
Whatever caution we invoke, the fact that we have used a random,
probabilistic, stochastic process as a model of man in his role
of a message source raises philosophical questions. Does this mean
that we imply that man acts at random? There is no such implication.
Perhaps if we knew enough about a man, his environment,
and his history, we could always predict just what word he would
write or speak next.
In communication theory, however, we assume that our only
knowledge of the message source is obtained either from the
messages that the source produces or perhaps from some less-than-
complete study of man himself. On the basis of information so
obtained, we can derive certain statistical data which, as we have
seen, help to narrow the probability as to what the next word or
62
Symbols, Signals and Noise
letter of a message will be. There remains an element of uncertainty.
For us who have incomplete knowledge of it, the message
source behaves as if certain choices were made at random, insofar
as we cannot predict what the choices will be. If we could predict
them, we should incorporate the knowledge which enables us to
make the predictions into our statistics of the source. If we had
more knowledge, however, we might see that the choices which we
cannot predict are not really random, in that they are (on the basis
of knowledge that we do not have) predictable.
We can see that the view we have taken of finite-state machines,
such as that of Figure III-1, has been limited. Finite-state machines
can have inputs as well as outputs. The transition from a particular
state to one among several others need not be chosen randomly;
it could be determined or influenced by various inputs to the
machine. For instance, the operation of an electronic digital computer,
which is a finite-state machine, is determined by the program
and data fed to it by the programmer.
It is, in fact, natural to think that man may be a finite-state
machine, not only in his function as a message source which produces
words, but in all his other behavior as well. We can think if
we like of all possible conditions and configurations of the cells of
the nervous system as constituting states (states of mind, perhaps).
We can think of one state passing to another, sometimes with the
production of a letter, word, sound, or a part thereof, and sometimes
with the production of some other action or of some part of
an action. We can think of sight, hearing, touch, and other senses
as supplying inputs which determine or influence what state the
machine passes into next. If man is a finite-state machine, the
number of states must be fantastic and beyond any detailed mathematical
treatment. But, so are the configurations of the molecules
in a gas, and yet we can explain much of the significant behavior
of a gas in terms of pressure and temperature merely.
Can we someday say valid, simple, and important things about
the working of the mind in producing written text and other things
as well? As we have seen, we can already predict a good deal
concerning the statistical nature of what a man will write down on
paper, unless he is deliberately trying to behave eccentrically, and,
even then, he cannot help conforming to habits of his own.
Such broad considerations are not, of course, the real purpose
A Mathematical Model 63
or meat of this chapter. We set out to find a mathematical model
adequate to represent some aspects of the human being in his role
as a source of messages and adequate to represent some aspects
of the messages he produces. Taking English text as an example,
we noted that the frequencies of occurrence of various letters are
remarkably constant, unless the writer deliberately avoids certain
letters. Likewise, frequencies of occurrence of particular pairs,
triplets, and so on, of letters are very nearly constant, as are
frequencies of various words.
We also saw that we could generate sequences of letters with
frequencies corresponding to those of English text by various random
or stochastic processes, such as, cutting a lot of text into letters
(or words), scrambling the bits of paper in a hat, and drawing them
out one at a time. More elaborate stochastic processes, including
finite-state machines, can produce an even closer approximation
to English text.
Thus, we take a generalized stochastic process as a model of a
message source, such as, a source producing English text. But, how
must we mathematically define or limit the stochastic sources we
deal with so that we can prove theorems concerning the encoding
of messages generated by the sources? Of course, we must choose
a definition consistent with the character of real English text.
The sort of stochastic source chosen as a model of actual message
sources is the ergodic source. An ergodic source can be regarded
as a hypothetical machine which produces an infinite number of
or ensemble of infinite sequences of characters. Roughly, the nature
or statistics of the sequences of characters or messages produced
by an ergodic source do not change with time; that is, the source
is stationary. Further, for an ergodic source the statistics based on
one message apply equally well to all messages that the source
generates.
The theorems of communication theory are proved exactly for
truly ergodic sources. All writers writing English text together
constitute an approximately ergodic source of text. The mathematical
model—the truly ergodic source—is close enough to the
actual situation so that the mathematics we base on it is very
useful. But we must be wise and careful in applying the theorems
and results of communication theory, which are exact for a mathematical
ergodic source, to actual communication problems.
CHAPTER i v Encoding and
Binary Digits
A source of information may be English text, a man speaking,
the sound of an orchestra, photographs, motion picture films, or
scenes at which a television camera may be pointed. We have seen
that in information theory such sources are regarded as having the
properties of ergodic sources of letters, numbers, characters, or
electrical signals. A chief aim of information theory is to study how
such sequences of characters and such signals can be most effectively
encoded for transmission, commonly by electrical means.
Everyone has heard of codes and the encoding of messages.
Romantic spies use secret codes. Edgar Allan Poe popularized
cryptography in The Gold Bug. The country is full of amateur
cryptanalysts who delight in trying to read encoded messages that
others have devised.
In this historical sense of cryptography or secret writing, codes
are used to conceal the content of an important message from these
for whom it is not intended. This may be done by substituting for
the words of the message other words which are listed in a code
book. Or, in a type of code called a cipher, letters or numbers may
be substituted for the letters in the message according to some
previously agreed upon secret scheme.
The idea of encoding, of the accurate representation of one
thing by another, occurs in other contexts as well. Geneticists
believe that the whole plan for a human body is written out in the
64
Encoding and Binary Digits
65
chromosomes of the germ cell. Some assert that the “text” consists
of an orderly linear arrangement of four different units, or “bases,”
in the DNA (desoxyribonucleic acid) forming the chromosome.
This text in turn produces an equivalent text in RNA (ribonucleic
acid), and by means of this RNA text proteins made up of
sequences of twenty amino acids are synthesized. Some cryptana-
lytic effort has been spent in an effort to determine how the four-
character message of RNA is reencoded into the twenty-character
code of the protein.
Actually, geneticists have been led to such considerations by the
existence of information theory. The study of the transmission of
information has brought about a new general understanding of the
problems of encoding, an understanding which is important to any
sort of encoding, whether it be the encoding of cryptography or the
encoding of genetic information.
We have already noted in Chapter II that English text can be
encoded into the symbols of Morse code and represented by short
and long pulses of current separated by short and long spaces. This
is one simple form of encoding. From the point of view of information
theory, the electromagnetic waves which travel from an FM
transmitter to the receiver in your home are an encoding of the
music which is transmitted. The electric currents in telephone
circuits are an encoding of speech. And the sound waves of speech
are themselves an encoding of the motions of the vocal tract which
produce them.
Nature has specified the encoding of the motions of the vocal
tract into the sounds of speech. The communication engineer,
however, can choose the form of encoding by means of which he
will represent the sounds of speech by electric currents, just as he
can choose the code of dots, dashes, and spaces by means of which
he represents the letters of English text in telegraphy. He wants to
perform this encoding well, not poorly. To do this he must have
some standard which distinguishes good encoding from bad encoding,
and he must have some insight into means for achieving good
encoding. We learned something of these matters in Chapter II.
It is the study of this problem, a study that might in itself seem
limited, which has provided through information theory new ideas
important to all encoding, whether cryptographic or genetic. These
66
Symbols, Signals and Noise
new ideas include a measure of amount of information, called
entropy, and a unit of measurement, called the bit.
I would like to believe that at this point the reader is clamoring
to know the meaning of “amount of information” as measured in
bits, and if so I hope that this enthusiasm will carry him over a
considerable amount of intervening material about the encoding
of messages.
It seems to me that one can’t understand and appreciate the
solution to a problem unless he has some idea of what the problem
is. You can’t explain music meaningfully to a man who has never
heard any. A story about your neighbor may be full of insight, but
it would be wasted on a Hottentot. I think it is only by considering
in some detail how a message can be encoded for transmission that
we can come to appreciate the need for and the meaning of a
measure of amount of information.
It is easiest to gain some understanding of the important problems
of coding by considering simple and concrete examples. Of
course, in doing this we want to learn something of broad value,
and here we may foresee a difficulty.
Some important messages consist of sequences of discrete characters,
such as the successive letters of English text or the successive
digits of the output of an electronic computer. We have seen,
however, that other messages seem inherently different.
Speech and music are variations with time of the pressure of air
at the ear. This pressure we can accurately represent in telephony
by the voltage of a signal traveling along a wire or by some other
quantity. Such a variation of a signal with time is illustrated in a
of Figure IV-1. Here we assume the signal to be a voltage which
varies with time, as shown by the wavy line.
Information theory would be of limited value if it were not
applicable to such continuous signals or messages as well as to
discrete messages, such as English text.
In dealing with continuous signals, information theory first
invokes a mathematical theorem called the sampling theorem,
which we will use but not prove. This theorem states that a continuous
signal can be represented completely by and reconstructed
perfectly from a set of measurements or samples of its amplitude
which are made at equally spaced times. The interval between such
67
(a)
(b)
TIME
Fig. IV-1
samples must be equal to or less than one-half of the period of the
highest frequency present in the signal. A set of such measurements
or samples of the amplitude of the signal a, Figure IV-1, is represented
by a sequence of vertical lines of various heights in b of
Figure IV-1.
We should particularly note that for such samples of the signal
to represent a signal perfectly they must be taken frequently
enough. For a voice signal including frequencies from 0 to 4,000
cycles per second we must use 8,000 samples per second. For a
television signal including frequencies from 0 to 4 million cycles
per second we must use 8 million samples per second. In general,
if the frequency range of the signal is/cycles per second we must
use at least 2/ samples per second in order to describe it perfectly.
Thus, the sampling theorem enables us to represent a smoothly
varying signal by a sequence of samples which have different
amplitudes one from another. This sequence of samples is, however,
still inherently different from a sequence of letters or digits.
There are only ten digits and there are only twenty-six letters, but
a sample can have any of an infinite number of amplitudes. The
amplitude of a sample can lie anywhere in a continuous range of
values, while a character or a digit has only a limited number of
discrete values.
The manner in which information theory copes with samples
having a continuous range of amplitudes is a topic all in itself, to
which we will return later. Here we will merely note that a signal
68
Symbols, Signals and Noise
need not be described or reproduced perfectly. Indeed, with real
physical apparatus a signal cannot be reproduced perfectly. In the
transmission of speech, for instance, it is sufficient to represent the
amplitude of a sample to an accuracy of about 1 per cent. Thus,
we can, if we wish, restrict ourselves to the numbers 0 to 99 in
describing the amplitudes of successive speech samples and represent
the amplitude of a given sample by that one of these hundred
integers which is closest to the actual amplitude. By so quantizing
the signal samples, we achieve a representation comparable to the
discrete case of English text.
We can, then, by sampling and quantizing, convert the problem
of coding a continuous signal, such as speech, into the seemingly
simpler problem of coding a sequence of discrete characters, such
as the letters of English text.
We noted in Chapter II that English text can be sent, letter by
letter, by means of the Morse code. In a similar manner, such
messages can be sent by teletypewriter. Pressing a particular key
on the transmitting machine sends a particular sequence of electrical
pulses and spaces out on the circuit. When these pulses and
spaces reach the receiving machine, they activate the corresponding
type bar, and the machine prints out the character that was transmitted.
Patterns of pulses and spaces indeed form a particularly useful
and general way of describing or encoding messages. Although
Morse code and teletypewriter codes make use of pulses and spaces
of different lengths, it is possible to transmit messages by means
of a sequence of pulses and spaces of equal length, transmitted at
perfectly regular intervals. Figure IV-2 shows how the electric
current sent out on the line varies with time for two different
patterns, each six intervals long, of such equal pulses and spaces.
Sequence a is a pulse-space-space-pulse-space-pulse. Sequence b
is pulse-pulse-pulse-space-pulse-pulse.
The presence of a pulse or a space in a given interval specifies
one of two different possibilities. We could use any pair of symbols
to represent such patterns of pulses or spaces as those of Figure
IV-2: yes, no; +, —; 1,0. Thus we could represent pattern a as
follows:
Encoding and Binary Digits 69
pulse space space pulse space pulse
Yes N o N o Yes N o Yes
+—
— +— +
1 0 0 1 0 1
The representation by 1 or 0 is particularly convenient and
important. It can be used to relate patterns of pulses to numbers
expressed in the binary system of notation.
When we write 315 we mean
3 X 102 + 1 X 101 + 5 X 1
= 3 x 100 + 1 X 10 + 5 X 1
= 315
In this ordinary decimal system of representing numbers we make
use of the ten different digits: 0, 1, 2, 3, 4, 5, 6, 7, 8, 9. In the
binary system we use only two digits, 0 and 1. When we write 1 0
0 1 0 1 we mean
Ix 25 + 0x 24 + 0x 23 + l x 2 z + 0 x 2 + l x l
= 1x32 + 0x16 + 0x8 + 1x4 + 0 x 2 + 1 x 1
= 37 in decimal notation
It is often convenient to let zeros precede a number; this does
not change its value. Thus, in decimal notation we can say,
0016 = 16
1
H
Z Ui (T (T D U
(a)
(b)
Fig. IV -2
70 Symbols, Signals and Noise
Or in binary notation
001010 = 1010
In binary numbers, each 0 or 1 is a binary digit. To describe the
pulses or spaces occurring in six successive intervals, we can use
a sequence of six binary digits. As a pulse or space in one interval
is equivalent to a binary digit, we can also refer to a pulse group
of six binary digits, or we can refer to the pulse or space occurring
in one interval as one binary digit.
Let us consider how many patterns of pulses and spaces there
are which are three intervals long. In other words, how many
three-digit binary numbers are there? These are all shown in
Table VI.
Table VI
0 0 0 (0 )
0 01 (1 )
0 1 0 (2 )
O i l (3 )
100 (4 )
101 (5 )
1 10 (6 )
111 (7 )
The decimal numbers corresponding to these sequences of l’s
and 0’s regarded as binary numbers are shown in parentheses to
the right.
We see that there are 8 (0 and 1 through 7) three-digit binary
numbers. We may note that 8 is 23. We can, in fact, regard an
orderly listing of binary digits n intervals long as simply setting
down 2n successive binary numbers, starting with 0. As examples,
in Table VII the numbers of different patterns corresponding to
different numbers n of binary digits are tabulated.
We see that the number of different patterns increases very
rapidly with the number of binary digits. This is because we double
the number of possible patterns each time we add one digit. When
we add one digit, we get all the old sequences preceded by a 0 plus
all the old sequences preceded by a 1.
The binary system of notation is not the only alternative to the
Encoding and Binary Digits
71
Table VII
n (N um ber o f Binary D igits) N um ber o f Patterns (2n)
1 2
2 4
3 8
4 16
5 32
10 1,024
20 1,048,576
decimal system. The octal system is very important to people who
use computers. We can regard the octal system as made up of the
eight digits 0, 1,2, 3, 4, 5, 6, 7.
When we write 356 in the octal system we mean
3 x 8 2 + 5 x 8 + 6 x l
= 3x64 + 5 x8 + 6x1
= 236 in decimal notation
We can convert back and forth between the octal and the binary
systems very simply. We need merely replace each successive block
of three binary digits by the appropriate octal digit, as, for instance,
binary 010 111 O il 110
octal 2 7 3 6
People who work with binary notation in connection with computers
find it easier to remember and transcribe a short sequence
of octal digits than a long group of binary digits. They learn to
regard patterns of three successive binary digits as an entity, so that
they will think of a sequence of twelve binary digits as a succession
of four patterns of three, that is, as a sequence of four octal digits.
It is interesting to note, too, that, just as a pattern of pulses and
spaces can correspond to a sequence of binary digits, so a sequence
of pulses of various amplitudes (0, 1, 2,3, 4, 5,6,7) can correspond
to a sequence of octal digits. This is illustrated in Figure IV-3. In
a, we have the sequence of off-on, 0-1 pulses corresponding to the
binary number 010111011110. The corresponding octal number is
2736, and in b this is represented by a sequence of four pulses of
current having amplitudes 2, 7, 3, 6.
72 Symbols, Signals and Noise
0 1 0
n
2
1 1 I | 0 1 1 1 1 0
11 1
1
7 1 3
6
(a)
8
6
4
2
O
,
1 I
Fig. IV -3
(b)
Conversion from binary to decimal numbers is not so easy. On
the average, it takes about 3.32 binary digits to represent one
decimal digit. Of course we can assign four binary digits to each
decimal digit, as shown in Table VIII, but this means that some
patterns are wasted; there are more patterns than we use.
It is convenient to think of sequences of 0’s and l’s or sequences
of pulses and spaces as binary numbers. This helps us to under-
Table VIII
B inary N um ber D ecim al D igit
0000 0
0001 1
0010 2
0011 3
0100 4
0101 5
0110 6
0111 7
1000 8
1001 9
1010 not used
1011 not used
1100 not used
1101 not used
1110 not used
m i not used
Encoding and Binary Digits
73
stand how many sequences of a different length there are and how
numbers written in the binary system correspond to numbers
written in the octal or in the decimal system. In the transmission
of information, however, the particular number assigned to a
sequence of binary digits is irrelevent. For instance, if we wish
merely to transmit representations of octal digits, we could make
the assignments shown in Table IX rather than those in Table VI.
Table IX
Sequence o f B inary D igits O ctal D igit R epresented
000 5
001 7
010 1
Oil 6
100 0
101 4
110 2
111 3
Here the “binary numbers” in the left column designate octal
numbers of different numerical value.
In fact, there is another way of looking at such a correspondence
between binary digits and other symbols, such as octal digits, a way
in which we do not regard the sequence of binary digits as part of
a binary number but rather as means of choosing or designating
a particular symbol.
We can regard each 0 or 1 as expressing an elementary choice
between two possibilities. Consider, for instance, the “tree of
choice” shown in Figure IV-4. As we proceed upward from the root
to the twigs, let 0 signify that we take the left branch and let 1
signify that we take the right branch. Then O il means left,
right, right and takes us to the octal digit 6, just as in Table IX.
Just as three binary digits give us enough information to determine
one among eight alternatives, four binary digits can determine
one among sixteen alternatives, and twenty binary digits can determine
one among 1,048,576 alternatives. We can do this by assigning
the required binary numbers to the alternatives in any order
we wish.
74 Symbols, Signals and Noise
5 7 1 6 0 4 2 3
The alternatives which we wish to specify by successions of
binary digits need not of course be numbers at all. In fact, we began
by considering how we might encode English text so as to transmit
it electrically by sequences of pulses and spaces, which can be
represented by sequences of binary digits.
A bare essential in transmitting English text letter by letter is
twenty-six letters plus a space, or twenty-seven symbols in all. This
of course allows us no punctuation and no Arabic numbers.
We can write out the numbers (three, not 3) if we wish and use
words for punctuation, (stop, comma, colon, etc.).
Mathematics says that a choice among 27 symbols corresponds
to about 4.75 binary digits. If we are not too concerned with
efficiency, we can assign a different 5-digit binary number to each
character, which will leave five 5-digit binary numbers unused.
My typewriter has 48 keys, including shift and shift lock. We
might add two more “symbols” representing carriage return and
line advance, making a total of 50. I could encode my actions in
typing, capitalization, punctuation, and all (but not insertion of the
paper) by a succession of choices among 50 symbols, each choice
corresponding to about 5.62 binary digits. We could use 6 binary
digits per character and waste some sequences of binary digits.
This waste arises because there are only thirty-two 5-digit binary
numbers, which is too few, while there are sixty-four 6-digit binary
numbers, which is too many. How can we avoid this waste? If we
have 50 characters, we have 125,000 possible different groups of 3
ordered characters. There are 131,072 different combinations of
Encoding and Binary Digits
75
17 binary digits. Thus, if we divide our text into blocks of 3 successive
characters, we can specify any possible block by a 17-digit
binary number and have a few left over. If we had represented each
separate character by 6 binary digits, we would have needed 18
binary digits to represent 3 successive characters. Thus, by this
block coding, we have cut down the number of binary digits we use
in encoding a given length of text by a factor 17/18.
Of course, we might encode English text in quite a different way.
We can say a good deal with 16,384 English words. That’s quite a
large vocabulary. There are just 16,384 fourteen-digit binary numbers.
We might assign 16,357 of these to different useful words and
27 to the letters of the alphabet and the space, so that we could
spell out any word or sequence of words we failed to include in
our word vocabulary. We won’t need to put a space between words
to which numbers have been assigned; it can be assumed that a
space goes with each word.
If we have to spell out words very infrequently, we will use about
14 binary digits per word in this sort of encoding. In ordinary
English text there are on the average about 4.5 letters per word.
As we must separate words by a space, when we send the message
character by character, even if we disregard capitalization and
punctuation, we will require on the average 5.5 characters per
word. If we encode these using 5 binary digits per character, we
will use on the average 27.5 binary digits per word, while in encoding
the message word by word we need only 14 binary digits
per word.
How can this be so? It is because, in spelling out the message
letter by letter, we have provided means for sending with equal
facility all sequences of English letters, while, in sending word by
word, we restrict ourselves to English words.
Clearly, the average number of binary digits per word required to
represent English text depends strongly on how we encode the text.
Now, English text is just one sort of message we might want to
transmit. Other messages might be strings of numbers, the human
voice, a motion picture, or a photograph. If there are efficient and
inefficient ways of encoding English text, we may expect that there
will be efficient and inefficient ways of encoding other signals
as well.
76 Symbols, Signals and Noise
Indeed, we may be led to believe that there exists in principle
some best way of encoding the signals from a given message source,
a way which will on the average require fewer binary digits per
character or per unit time than any other way.
If there is such a best way of encoding a signal, then we might
use the average number of binary digits required to encode the
signal as a measure of the amount of information per character or
the amount of information per second of the message source which
produced the signal.
This is just what is done in information theory. How it is done
and further reasons for so doing will be considered in the next
chapter.
Let us first, however, review very briefly what we have covered
in this chapter. In communication theory, we regard coding very
broadly, as representing one signal by another. Thus a radio wave
can represent the sounds of speech and so form an encoding of
these sounds. Encoding is, however, most simply explained and
explored in the case of discrete message sources, which produce
messages consisting of sequences of characters or numbers. Fortunately,
we can represent a continuous signal, such as the current
in a telephone line, by a number of samples of its amplitude, using,
each second, twice as many samples as the highest frequency
present in the signal. Further we can if we wish represent the amplitude
of each of these samples approximately by a whole number.
The representation of letters or numbers by sequences of off-or-
on signals, which can in turn be represented directly by sequences
of the binary digits 0 and 1, is of particular interest in communication
theory. For instance, by using sequences of 4 binary digits
we can form 16 binary numbers, and we can use 10 of these to
represent the 10 decimal digits. Or, by using sequences of 5 binary
digits we can form 32 binary numbers, and we can use 27 of these
to represent the letters of the English alphabet plus the space. Thus,
we can transmit decimal numbers or English text by sending
sequences of off-or-on signals.
We should note that while it may be convenient to regard the
sequences of binary digits so used as binary numbers, the numerical
value of the binary number has no particular significance; we can
choose any binary number to represent a particular decimal digit.
Encoding and Binary Digits
77
If we use 10 of the 16 possible 5-digit binary numbers to encode
the 10 decimal digits, we never use (we waste) 6 binary numbers.
We could, but never do, transmit these sequences as sequences of
off-or-on signals. We can avoid such waste by means of block
coding, in which we encode sequences of 2, 3, or more decimal
digits or other characters by means of binary digits. For instance,
all sequences of 3 decimal digits can be represented by 10 binary
digits, while it takes a total of 12 binary digits to represent separately
each of 3 decimal digits.
Any sequence of decimal digits may occur, but only certain
sequences of English letters ever occur, that is, the words of the
English language. Thus, it is more efficient to encode English words
as sequences of binary digits rather than to encode the letters of
the words individually. This again emphasizes the gain to be made
by encoding sequences of characters, rather than encoding each
character separately.
All of this leads us to the idea that there may be a best way of
encoding the messages from a message source, a way which calls
for the least number of binary digits.
CHAPTER V Entropy
In the last chapter, we have considered various ways in which
messages can be encoded for transmission. Indeed, all communication
involves some sort of encoding of messages. In the electrical
case, letters may be encoded in terms of dots or dashes of electric
current or in terms of several different strengths of current and
directions of current flow, as in Edison’s quadruplex telegraph. Or
we can encode a message in the binary language of zeros and ones
and transmit it electrically as a sequence of pulses or absences
of pulses.
Indeed, we have shown that by periodically sampling a continuous
signal such as a speech wave and by representing the amplitudes
of each sample approximately by the nearest of a set of
discrete values, we can represent or encode even such a continuous
wave as a sequence of binary digits.
We have also seen that the number of digits required in encoding
a given message depends on how it is encoded. Thus, it takes fewer
binary digits per character when we encode a group or block of
English letters than when we encode the letters one at a time.
More important, because only a few combinations of letters form
words, it takes considerably fewer digits to encode English text
word by word than it does to encode the same text letter by letter.
Surely, there are still other ways of encoding the messages produced
by a particular ergodic source, such as a source of English
text. How many binary digits per letter or per word are realty
needed? Must we try all possible sorts of encoding in order to find
78
Entropy 79
out? But, if we did try all forms of encoding we could think of, we
would still not be sure we had found the best form of encoding,
for the best form might be one which had not occurred to us.
Is there not, in principle at least, some statistical measurement
we can make on the messages produced by the source, a measure
which will tell us the minimum average number of binary digits
per symbol which will serve to encode the messages produced by
the source?
In considering this matter, let us return to the model of a message
source which we discussed in Chapter III. There we regarded
the message source as an ergodic source of symbols, such as letters
or words. Such an ergodic source has certain unvarying statistical
properties: the relative frequencies of symbols; the probability that
one symbol will follow a particular other symbol, or pair of symbols,
or triplet of symbols; and so on.
In the case of English text, we can speak in the same terms of
the relative frequencies of words and of the probability that one
word will follow a particular word or a particular pair, triplet, or
other combination of words.
In illustrating the statistical properties of sequences of letters or
words, we showed how material resembling English text can be
produced by a sequence of random choices among letters and
words, provided that the letters or words are chosen with due
regard for their probabilities or their probabilities of following a
preceding sequence of letters or words. In these examples, the
throw of a die or the picking of a letter out of a hat can serve to
“choose” the next symbol.
In writing or speaking, we exercise a similar choice as to what
we shall set down or say next. Sometimes we have no choice; Q
must be followed by U. We have more choice as to the next
symbol in beginning a word than in the middle of a word. However,
in any message source, living or mechanical, choice is continually
exercised. Otherwise, the messages produced by the source
would be predetermined and completely predictable.
Corresponding to the choice exercised by the message source in
producing the message, there is an uncertainty on the part of the
recipient of the message. This uncertainty is resolved when the
recipient examines the message. It is this resolution of uncertainty
which is the aim and outcome of communication.
80
Symbols, Signals and Noise
If the message source involved no choice, if, for instance, it
could produce only an endless string of ones or an endless string
of zeros, the recipient would not need to receive or examine the
message to know what it was; he could predict it in advance. Thus,
if we are to measure information in a rational way, we must have
a measure that increases with the amount of choice of the source
and, thus, with the uncertainty of the recipient as to what message
the source may produce and transmit.
Certainly, for any message source there are more long messages
than there are short messages. For instance, there are 2 possible
messages consisting of 1 binary digit, 4 consisting of 2 binary
digits, 16 consisting of 4 binary digits, 256 consisting of 8 binary
digits, and so on. Should we perhaps say that amount of information
should be measured by the number of such messages? Let us
consider the case of four telegraph lines used simultaneously in
transmitting binary digits between two points, all operating at the
same speed. Using the four lines, we can send 4 times as many
digits in a given period of time as we could using one line. It also
seems reasonable that we should be able to send 4 times as much
information by using four lines. If this is so, we should measure
information in terms of the number of binary digits rather than
in terms of the number of different messages that the binary digits
can form. This would mean that amount of information should be
measured, not by the number of possible messages, but by the
logarithm of this number.
The measure of amount of information which communication
theory provides does this and is reasonable in other ways as well.
This measure of amount of information is called entropy. If we want
to understand this entropy of communication theory, it is best first
to clear our minds of any ideas associated with the entropy of
physics. Once we understand entropy as it is used in communication
theory thoroughly, there is no harm in trying to relate it to
the entropy of physics, but the literature indicates that some
workers have never recovered from the confusion engendered by
an early admixture of ideas concerning the entropies of physics
and communication theory.
The entropy of communication theory is measured in bits. We
may say that the entropy of a message source is so many bits per
Entropy 81
letter, or per word, or per message. If the source produces symbols
at a constant rate, we can say that the source has an entropy of
so many bits per second.
Entropy increases as the number of messages among which the
source may choose increases. It also increases as the freedom of
choice (or the uncertainty to the recipient) increases and decreases
as the freedom of choice and the uncertainty are restricted. For
instance, a restriction that certain messages must be sent either very
frequently or very infrequently decreases choice at the source and
uncertainty for the recipient, and thus such a restriction must
decrease entropy.
It is best to illustrate entropy first in a simple case. The mathematical
theory of communication treats the message source as an
ergodic process, a process which produces a string of symbols that
are to a degree unpredictable. We must imagine the message source
as selecting a given message by some random, i.e., unpredictable
means, which, however, must be ergodic. Perhaps the simplest case
we can imagine is that in which there are only two possible symbols,
say, X and Y, between which the message source chooses
repeatedly, each choice uninfluenced by any previous choices. In
this case we can know only that X will be chosen with some
probability po and Y with some probability pu as in the outcomes
of the toss of a biased coin. The recipient can determine these
probabilities by examining a long string of characters (A’s, T’s)
produced by the source. The probabilities p0 and pi must not
change with time if the source is to be ergodic.
For this simplest of cases, the entropy H of the message source
is defined as
H = —(po log p0 + pi log pi) bits per symbol
Thus, the entropy is the negative of the sum of the probability po
that A" will be chosen (or will be received) times the logarithm of
po and the probability pi that Y will be chosen (or will be received)
times the logarithm of this probability.
Whatever plausible arguments one may give for the use of
entropy as defined in this and in more complicated cases, the real
and true reason is one that will become apparent only as we
proceed, and the justification of this formula for entropy will
82
Symbols, Signals and Noise
therefore be deferred. It is, however, well to note again that there
are different kinds of logarithms and that, in information theory,
we use logarithms to the base 2. Some facts about logarithms to
the base 2 are noted in Table X.
Table X
F raction p A nother W ay o f
W riting p
Still A nother W ay
o f W riting p L o g p
3 1 2“ -415 -.415
4 2-415
1 1 2_1 -1
2 21
3 1 2-1-415 -1.415
8 21-415
\ 1 2-2 -2
4 22
\ 1 2-3 -3
8 23
J_ 1 2-4 -4
16 2*
j_ 1 2-6 -6
64 26
1
256
1
28 2-s -8
The logarithm to the base 2 of a number is the power to which
2 must be raised to give the number.
Let us consider, for instance, a “message source” which consists
of the tossing of an honest coin. We can let X represent heads and
Y represent tails. The probability pi that the coin will turn up
heads is V 2 and the probability po that the coin will turn up tails
is also lh. Accordingly, from our expression for entropy and from
Table X we find that
H = - (V2 log Vi + V 2 log V 2 )
H = -[(VS)(-1) + (V4)(-l)]
H = 1 bit per toss
Entropy 83
If the message source is the sequence of heads and tails obtained
by tossing a coin, it takes one bit of information to convey whether
heads or tails has turned up.
Let us notice, now, that we can represent the outcome of successively
tossing a coin by a number of binary digits equal to the
number of tosses, letting 1 stand for heads and 0 stand for tails.
Hence, in this case at least, the entropy, one bit per toss, and the
number of binary digits which can represent the outcome, one
binary digit per toss, are equal. In this case at least, the number
of binary digits necessary to transmit the message generated by
the source (the succession of heads and tails) is equal to the entropy
of the source.
Suppose the message source produces a string of l’s and 0’s by
tossing a coin so weighted that it turns up heads 3A of the time and
tails only Vi of the time. Then
pi = 3A
po = lA
H — — ('A log ‘A + 3A log 3A)
H = -[(Vi) (-2) + (3/4) (-.415)1
H = .811 bit per toss
We feel that, in the case of a coin which turns up heads more
often than tails, we know more about the outcome than if heads
or tails were equally likely. Further, if we were constrained to
choose heads more often than tails we would have less choice than
if we could choose either with equal probability. We feel that this
must be so, for if the probability for heads were 1 and for tails 0,
we would have no choice at all. And, we see that the entropy for
the case above is only .811 bit per toss. We feel somehow that we
ought to be able to represent the outcome of a sequence of such
biased tosses by fewer than one binary digit per toss, but it is not
immediately clear how many binary digits we must use.
If we choose heads over tails with probability pi, the probability
po of choosing tails must of course be 1 — pi. Thus, if we know pi
we know p0 as well. We can compute H for various values of pi
and plot a graph of H vs. pi. Such a curve is shown in Figure V-l.
H has a maximum value of 1 when pi is 0.5 and is 0 when pi is 0
or 1, that is, when it is certain that the message source always
produces either one symbol or the other.
84 Symbols, Signals and Noise
Fig. V-l
Really, whether we call heads X and tails Y or heads Y and tails
X is immaterial, so the curve of H vs. pi must be the same as H
vs. po. Thus, the curve of Figure V-l is symmetrical about the
dashed center line at pi and p0 equal to 0.5.
A message source may produce successive choices among the
ten decimal digits, or among the twenty-six letters of the alphabet,
or among the many thousands of words of the English language.
Let us consider the case in which the message source produces one
among n symbols or words, with probabilites which are independent
of previous choices. In this case the entropy is defined as
n
H = — ^ pi log pi bits per symbol (5.1)
/ = 1
Here the sign 2 (sigma) means to sum or to add up various terms.
Entropy 85
Pi is the probability of the ith symbol being chosen. The i = 1 below
and n above the 2 mean to let / be 1, 2, 3, etc. up to n, so the equation
says that the entropy will be given by adding p\ log p\ and
P2 logp2 and so on, including all symbols. We see that when n = 2
we have the simple case which we considered earlier.
Let us take an example. Suppose, for instance, that we toss two
coins simultaneously. Then there are four possible outcomes, which
we can label with the numbers 1 through 4:
H H or 1
H Tori
T H or 3
T Tor 4
If the coins are honest, the probability of each outcome is V a and
the entropy is
H = — ('A log V a + V a log V a + V a log Y a + V a log V a)
H = - ( - V i — Vi - Vi - Vi)
H = 2 bits per pair tossed
It takes 2 bits of information to describe or convey the outcome
of tossing a pair of honest coins simultaneously. As in the case of
tossing one coin which has equal probabilities of landing heads or
tails, we can in this case see that we can use 2 binary digits to
describe the outcome of a toss: we can use 1 binary digit for each
coin. Thus, in this case too, we can transmit the message generated
by the process (of tossing two coins) by using a number of binary
digits equal to the entropy.
If we have some number n of symbols all of which are equally
probable, the probability of any particular one turning up is \/n,
so we have n terms, each of which is l/n log \/n. Thus, the entropy
is in this case
H = —log l/n bits per symbol
For instance, an honest die when rolled has equal probabilities of
turning up any number from 1 to 6. Hence, the entropy of the
sequence of numbers so produced must be — log V6, or 2.58 bits
per throw.
More generally, suppose that we choose each time with equal
86 Symbols, Signals and Noise
likelihood among all binary numbers with N digits. There are 2N
such numbers, so
n = 2N
From Table X we easily see that
log l/n = log 2~N = —N
Thus, for a source which produces at each choice with equal likelihood
some TV-digit binary number, the entropy is N bits per number.
Here the message produced by the source is a binary number
which can certainly be represented by binary digits. And, again,
the message can be represented by a number of binary digits equal
to the entropy of the message, measured in bits. This example
illustrates graphically how the logarithm must be the correct
mathematical function in the entropy.
Ordinarily the probability that the message source will produce
a particular symbol is different for different symbols. Let us take
as an example a message source which produces English words
independently of what has gone before but with the probabilities
characteristic of English prose. This corresponds to the first-order
word approximation given in Chapter III.
In the case of English prose, we find as an empirical fact that if
we order the words according to frequency of usage, so that the
most frequently used, the most probable word (the, in fact) is word
number 1, the next most probable word (of) is number 2, and so
on, then the probability for the rth word is very nearly (if r is not
too large)
pt = .l/r (5.2)
If equation 5.2 were strictly true, the points in Figure V-2, in which
word probability or frequency pr is plotted against word order or
rank r, would fall on the solid line which extends from upper left
to lower right. We see that this is very nearly so. This empirical
inverse relation between word probability and word rank is known
as Zipf’s law. We will discuss Zipf’s law in Chapter XII; here, we
propose merely to use it.
We can show that this equation (5.2) cannot hold for all words.
To see this, let us consider tossing a coin. If the probability of heads
Entropy 87
Fig. V-2
turning up is xk and the probability of tails turning up is x/i, then
there is no other possible outcome: lA + lA = \. If there were an
additional probability of Vio that the coin would stand on edge, we
would have to conclude that in a hundred tosses we would expect
110 outcomes: heads 50 times, tails 50 times, and standing on edge
10 times. This is patently absurd. The probabilities of all outcomes
must add up to unity. Now, let us note that if we add up successively
p\ plus p2, etc., as given by equation 5.2, we find that by the
time we came to psi2 i the sum of the successive probabilities has
become unity. If we took this literally, we would conclude that no
additional word could ever occur. Equation 5.1 must be a little in
error.
Nonetheless, the error is not great, and Shannon used equation
88
Symbols, Signals and Noise
5.2 in computing the entropy of a message source which produces
words independently but with the probability of their occurring in
English text. In order to make the sum of the probabilities of all
words unity, he included only the 8,727 most frequently used words.
He found the entropy to be 9.14 bits per word.
In Chapter IV, we saw that English text can be encoded letter
by letter by using 5 binary digits per character or 27.5 binary digits
per word. We also saw that by providing different sequences of
binary digits for each of 16,357 words and 27 characters, we could
encode English text by using about 14 binary digits per word. We
are now beginning to suspect that the number of binary digits
actually required is given by the entropy, and, as we have seen,
Shannon’s estimate, based on the relative probabilities of English
words, would be 9.14 binary digits per word,
As a next step in exploring this matter of the number of binary
digits required to encode the message produced by a message
source, we will consider a startling theorem which Shannon proved
concerning the “messages” produced by an ergodic source which
selects a sequence of letters or words independently with certain
probabilities.
Let us consider all of the messages the source can produce which
consist of some particular large number of characters. For example,
we might consider all messages which are 100,000 symbols
(letters, words, characters) long. More generally, let us consider
messages having a number M of characters. Some of these messages
are more probable than others. In the probable messages, symbol
1 occurs about Mp\ times, symbol 2 occurs about Mp2 times, etc.
Thus, in these probable messages each symbol occurs with about
the frequency characteristic of the source. The source might produce
other sorts of messages, for instance, a message consisting of
one symbol endlessly repeated or merely a message in which the
numbers of the various symbols differed markedly from M times
their probabilities, but it seldom does.
The remarkable fact is that, if H is the entropy of the source per
symbol, there are just about 2MH probable messages, and the rest
of the messages all have vanishingly small probabilities of ever
occurring. In other words, if we ranked the messages from most
probable to least probable, and assigned binary numbers of MH
Entropy 89
digits to the 2m most probable messages, we would be almost
certain to have a number corresponding to any M-symbol message
that the source actually produced.
Let us illustrate this in particular simple cases. Suppose that the
symbols produced are 1 or 0. If these are produced with equal
probabilities, a probability lA that for 1 and a probability 'A that
for 0 the entropy H is, as we have seen, 1 bit per symbol. Let us
let the source produce messages M digits long. Then MH = 1,000,
and, according to Shannon’s theorem, there must be 21000 different
probable messages.
Now, by using 1,000 binary digits we can write just 21000 different
binary numbers. Thus, in order to assign a different binary number
to each probable message, we must use binary numbers 1,000
digits long. This is just what we would expect. In order to designate
to the message destination which 1,000 digit binary number
the message source produces, we must send a message 1,000 binary
digits long.
But, suppose that the digits constituting the messages produced
by the message source are obtained by tossing a coin which turns
up heads, designating 1, % of the time and tails, designating 0, V a
of the time. The typical messages so produced will contain more
Ts than 0’s, but that is not all. We have seen that in this case the
entropy H is only .811 bit per toss. If M, the length of the message,
is again taken as 1,000 binary digits, MH is only 811. Thus, while
as before there are 21000 possible messages, there are only 2811
probable messages.
Now, by using 811 binary digits we can write 2811 different
binary numbers, and we can assign one of these to each of the
1,000-digit probable messages, leaving the other improbable 1,000-
digit messages unnumbered. .Thus, we can send word to a message
destination which probable 1,000-digit message our message source
produces by sending only 811 binary digits. And the chance that
the message source will produce an improbable 1,000-digit message,
to which we have assigned no number, is negligible. Of
course, the scheme is not quite foolproof. The message source may
still very occasionally turn up a message for which we have no label
among all 2811 of our 811-digit binary labels. In this case we cannot
transmit the message—at least, not by using 811 binary digits.
90
Symbols, Signals and Noise
We see that again we have a strong indication that the number
of binary digits required to transmit a message is just the entropy
in bits per symbol times the number of symbols. And, we might
note that in this last illustration we achieved such an economical
transmission by block encoding—that is, by lumping 1,000 (or some
other large number) message digits together and representing each
probable combination of digits by its individual code (of 811 binary
digits).
How firmly and generally can this supposition be established?
So far we have considered only cases in which the message
source produces each symbol (number, letter, word) independently
of the symbols it has produced before. We know this is not true
for English text. Besides the constraints of word frequency, there
are constraints of word order, so that the writer has less choice as
to what the next word will be than he would if he could choose it
independently of what has gone before.
How are we to handle this situation? We have a clue in the
block coding which we discussed in Chapter IV, and which has been
brought to our mind again in the last example. In an ergodic
process the probability of the next letter may depend only on the
preceding 1,2, 3,4,5, or more letters but not on earlier letters. The
second and third order approximations to English given in Chapter
III illustrate text produced by such a process. Indeed, in any
ergodic process of which we are to make mathematical sense the
effect of the past on what symbol will be produced next must
decrease as the remoteness of that past is greater. This is reasonably
valid in the case of real English as well. While we can imagine
examples to the contrary (the consistent use of the same name for
a character in a novel), in general the word I write next does not
depend on just what word I wrote 10,000 words back.
Now, suppose that before we encode a message we divide it up
into very long blocks of symbols. If the blocks are long enough,
only the symbols near the beginning will depend on symbols in the
previous block, and, if we make the block long enough, these
symbols that do depend on symbols in the previous block will
form a negligible part of all the symbols in the block. This makes
it possible for us to compute the entropy per block of symbols by
means of equation 5.1. To keep matters straight, let us call the
Entropy 91
probability of a particular one of the multitudinous long blocks of
symbols, which we will call the ith block, P(Bi). Then the entropy
per block will be
H = —'^ JP(Bl) log P(Bi) bits per block
i
Any mathematician would object to calling this the entropy. He
would say, the quantity H given by the above equation approaches
the entropy as we make the block longer and longer, so that it
includes more and more symbols. Thus, we must assume that we
make the blocks very long indeed and get a very close approximation
to the entropy, With this proviso, we can obtain the entropy
per symbol by dividing the entropy per block by the number N of
symbols per block
H = — (\/N)S^ j P{Bi) log P(Bi) bits per symbol (5.3)
i
In general, an estimate of entropy is always high if it fails to take
into account some relations between symbols. Thus, as we make
N, the number of symbols per block, greater and greater, H as
given by 5.3 will decrease and approach the true entropy.
We have insisted from the start that amount of information must
be so defined that if separate messages are sent over several telegraph
wires, the total amount of information must be the sum of
the amounts of information sent over the separate wires. Thus, to
get the entropy of several message sources operating simultaneously,
we add the entropies of the separate sources. We can go
further and say that if a source operates intermittently we must
multiply its information rate or entropy by the fraction of the time
that it operates in order to get its average information rate.
Now, let us say that we have one message source when we have
just sent a particular sequence of letters such as TH. In this case
the probability that the next'letter will be E is very high. We have
another particular message source when we have just sent NQ. In
this case the probability that the next symbol will be U is unity.
We calculate the entropy for each of these message sources. We
multiply the entropy of a source which we label Bx by the probability
p(Bi) that this source will occur (that is, by the fraction of
92
Symbols, Signals and Noise
instances in which this source is in operation). We multiply the
entropy of each other source by the probability that that source
will occur, and so on. Then we add all the numbers we get in this
way in order to get the average entropy or rate of the over-all
source, which is a combination of the many different sources, each
of which operates only part time. As an example, consider a source
involving digram probabilities only, so that the whole effect of the
past is summed up in the letter last produced. One source will be
the source we have when this letter is E; this will occur in .13 of
the total instances. Another source will be the source we have when
the letter just produced is W; this will occur in .02 of the total
instances.
Putting this in formal mathematical terms, we say that if a
particular block of N symbols, which we designate by B\, has just
occurred, the probability that the next symbol will be symbol Sj is
PBi(Sj)
The entropy of this “source” which operates only when a particular
block of N symbols designated by Bi has just been produced is
logPBi(Sj)
j
But, in what fraction of instances does this particular message
source operate? The fraction of instances in which this source
operates is the fraction of instances in which we encounter block
Bi rather than some other block of symbols; we call this fraction
p(Bi)
Thus, taking into account all blocks of N symbols, we write the
sum of the entropies of all the separate sources (each separate
source defined by what particular block Bi of N symbols has
preceded the choice of the symbol Sj) as
Hn = -^PpW pBiiSj) logpBi(Sj) (5.4)
‘>j
The i,j under the summation sign mean to let i and j assume all
possible values and to add all the numbers we get in this way.
As we let the number N of symbols preceding symbol Sj become
very large, Hy approaches the entropy of the source. If there are
Entropy 93
no statistical influences extending over more than N symbols (this
will be true for a digram source for N = 1 and for a trigram
source for N = 2), then HN is the entropy.
Shannon writes equation 5.4 a little differently, The probability
p{B\, Sj) of encountering the block followed by the symbol Sj
is the probability p(Bi) of encountering the block Bx times the
probability pBj(Sj) that symbol Sj will follow block fi{. Hence, we
can write 5.4 as follows:
In Chapter III we consider a finite-state machine, such as that
shown in Figure III-3, as a source of text. We can, if we wish, base
our computation of entropy on such a machine. In this case, we
regard each state of the machine as a message source and compute
the entropy for that state. Then we multiply the entropy for that
state by the probability that the machine will be in that state and
sum (add up) all states in order to get the entropy.
Putting the matter symbolically, suppose that when the machine
is in a particular state i it has a probability pi(j) of producing a
particular symbol which we designate by j. For instance, in a state
labeled i = 10 it might have a probability of 0.03 of producing the
third letter of the alphabet, which we label j = 3. Then
The entropy />io(3) = .03
of state i is computed in accord with 5.1:
J
Now, we say that the machine has a probability Pi of being in the
Ith state. The entropy per symbol for the machine as a source of
symbols is then
H = Y j PiHi bits per symbol
We can write this as
" = - 2 Pipi(j)\o%pi(j) bits per symbol (5.5)
i>j
94
Symbols, Signals and Noise
Pi is the probability that the finite-state machine is in the i111 state,
and pi(J) is the probability that it produces the jth symbol when
it is in the ith state. The i and j under the 2 mean to allow both i
and j to assume all possible values and to add all the numbers so
obtained.
Thus, we have gone easily and reasonably from the entropy of
a source which produces symbols independently and to which
equation 5.1 applies to the more difficult case in which the probability
of a symbol occurring depends on what has gone before. And,
we have three alternative methods for computing or defining the
entropy of the message source. These three methods are equivalent
and rigorously correct for true ergodic sources. We should remember,
of course, that the source of English text is only approximately
ergodic.
Once having defined entropy per symbol in a perfectly general
way, the problem is to relate it unequivocally to the average
number of binary digits per symbol necessary to encode a message.
We have seen that if we divide the message into a block of letters
or words and treat each possible block as a symbol, we can compute
the entropy per block by the same formula we used per
independent symbol and get as close as we like to the source
entropy merely by making the blocks very long.
Thus, the problem is to find out how to encode efficiently in
binary digits a sequence of symbols chosen from a very large group
of symbols, each of which has a certain probability of being chosen.
Shannon and Fano both showed ways of doing this, and Huffman
found an even better way, which we shall consider here.
Let us for convenience list all the symbols vertically in order of
decreasing probability. Suppose the symbols are the eight words
the, man, to, runs, house, likes, horse, sells, which occur independently
with probabilities of their being chosen, or appearing, as
listed in Table XI.
We can compute the entropy per word by means of 5.1; it is 2.21
bits per word. However, if we merely assigned one of the eight
3-digit binary numbers to each word, we would need 3 digits to
transmit each word. How can we encode the words more efficiently?
Figure V-3 shows how to construct the most efficient code for
encoding such a message word by word. The words are listed to the
Entropy 95
Table XI
W ord Probability
the .50
man .15
to .12
runs .10
house .04
likes .04
horse .03
sells .02
left, and the probabilities are shown in parentheses. In constructing
the code, we first find the two lowest probabilities, .02 (sells)
and .03 (horse), and draw lines to the point marked .05, the probability
of either horse or sells. We then disregard the individual
probabilities connected by the lines and look for the two lowest
probabilities, which are .04 (like) and .04 (house). We draw lines
to the right to a point marked .08, which is the sum of .04 and .04.
The two lowest remaining probabilities are now .05 and .08, so we
draw a line to the right connecting them, to give a point marked
THE
MAN
TO
RUNS
HOUSE
LIKE
HORSE
SELLS
96 Symbols, Signals and Noise
.13. We proceed thus until paths run from each word to a common
point to the right, the point marked 1,00. We then label each upper
path going to the left from a point 1 and each lower path 0. The
code for a given word is then the sequence of digits encountered
going left from the common point 1.00 to the word in question.
The codes are listed in Table XII.
Table XII
W ord Probability p C ode N um ber o f D igits
in Code, N N p
the .50 1 1 .50
man .15 001 3 .45
to .12 Oil 3 .36
runs .10 010 3 .30
house .04 00011 5 .20
likes .04 00010 5 .20
horse .03 00001 5 .15
sells .02 00000 5 .10
2.26
In Table XII we have shown not only each word and its code
but also the probability of each code and the number of digits in
each code. The probability of a word times the number of digits
in the code gives the average number of digits per word in a long
message due to the use of that particular word. If we add the
products of the probabilities and the numbers of digits for all the
words, we get the average number of digits per word, which is 2.26.
This is a little larger than the entropy per word, which we found
to be 2.21 bits per word, but it is a smaller number of digits than
the 3 digits per word we would have used if we had merely assigned
a different 3-digit code to each word.
Not only can it be proved that this Huffman code is the most
efficient code for encoding a set of symbols having different probabilities,
it can be proved that it always calls for less than one
binary digit per symbol more than the entropy (in the above
example, it calls for only 0.05 extra binary digits per symbol).
Now suppose that we combine our symbols into blocks of 1,2,
3, or more symbols before encoding. Each of these blocks will have
Entropy 97
a probability (in the case of symbols chosen independently, the
probability of a sequence of symbols will be the product of the
probabilities of the symbols). We can find a Huffman code for these
blocks of symbols. As we make the blocks longer and longer, the
number of binary digits in the code for each block will increase.
Yet, our Huffman code will take less than one extra digit per block
above the entropy in bits per block! Thus, as the blocks and their
codes become very long, the less-than-one extra digit of the Huffman
code will become a negligible fraction of the total number of
digits, and, as closely as we like (by making the blocks longer), the
number of binary digits per block will equal the entropy in bits
per block.
Suppose we have a communication channel which can transmit
a number C of off-or-on pulses per second. Such a channel can
transmit C binary digits per second. Each binary digit is capable
of transmitting one bit of information. Hence we can say that the
information capacity of this communication channel is C bits per
second, If the entropy H of a message source, measured in bits per
second, is less than C, then, by encoding with a Huffman code, the
signals from the source can be transmitted over the channel.
Not all channels transmit binary digits. A channel, for instance,
might allow three amplitudes of pulses, or it might transmit different
pulses of different lengths, as in Morse code. We can imagine
connecting various different message sources to such a channel.
Each source will have some entropy or information rate. Some
source will give the highest entropy that can be transmitted over
the channel, and this highest possible entropy is called the channel
capacity C of the channel and is measured in bits per second.
By means of the Huffman code, the output of the channel when
it is transmitting a message of this greatest possible entropy can
be coded into some least number of binary digits per second, and,
when long stretches of message are encoded into long stretches of
binary digits, it must take very close to C binary digits per second
to represent the signals passing over the channel.
This encoding can, of course, be used in the reverse sense, and
C independent binary digits per second can be so encoded as to
be transmitted over the channel. Thus, a source of entropy H can
be encoded into H binary digits per second, and a general discrete
98 Symbols, Signals and Noise
channel of capacity C can be used to transmit C bits per second.
We are now in a position to appreciate one of the fundamental
theorems of information theory, Shannon calls this the fundamental
theorem of the noiseless channel. He states it as follows:
Let a source have entropy H (bits per symbol) and a channel have a
capacity [to transmit] C bits per second. Then it is possible to encode the
ousput of the source in such a way as to transmit at the average rate
(C/H) — e symbols per second over the channel, where e is arbitrarily
small. It is not possible to transmit at an average rate greater than C/H.
Let us restate this without mathematical niceties. Any discrete
channel that we may specify, whether it transmits binary digits,
letters and numbers, or dots, dashes, and spaces of certain distinct
lengths has some particular unique channel capacity C. Any
ergodic message source has some particular entropy H. If H is less
than or equal to C, we can transmit the messages generated by the
source over the channel. If H is greater than C, we had better not
try to do so, because we just plain can’t.
We have indicated above how the first part of this theorem can
be proved. We have not shown that a source of entropy H cannot
be encoded in less than H binary digits per symbol, but this also
can be proved,
We have now firmly arrived at the fact that the entropy of a
message source measured in bits tells us how many binary digits
(or off-or-on pulses, or yeses-or-noes) are required, per character,
or per letter, or per word, or per second in order to transmit
messages produced by the source. This identification goes right
back to Shannon’s original paper. In fact, the word bit is merely
a contraction of binary digit and is generally used in place of
binary digit.
Here I have used bit in a particular sense, as a measure of
amount of information, and in other contexts I have used a different
expression, binary digit. I have done this in order to avoid a
confusion which might easily have arisen had I started out by using
bit to mean two different things.
After all, in practical situations the entropy in bits is usually
different from the number of binary digits involved. Suppose, for
instance, that a message source randomly produces the symbol 1
Entropy 99
with a probability V* and the symbol 0 with the probability % and
that it produces 10 symbols per second, Certainly such a source
produces binary digits at a rate of 10 per second, but the information
rate or entropy of the source is .811 bit per binary digit and
8.11 bits per second, We could encode the sequence of binary digits
produced by this source by using on the average only 8.11 binary
digits per second,
Similarly, suppose we have a communication channel which is
capable of transmitting 10,000 arbitrarily chosen off-or-on pulses
per second, Certainly, such a channel has a channel capacity of
10,000 bits per second. However, if the channel is used to transmit
a completely repetitive pattern of pulses, we must say that the
actual rate of transmission of information is 0 bits per second,
despite the fact that the channel is certainly transmitting 10,000
binary digits per second,
Here we have used bit only in the sense of a binary measure of
amount of information, as a measure of the entropy or information
rate of a message source in bits per symbol or in bits per second
or as a measure of the information transmission capabilities of a
channel in bits per symbol or bits per second. We can describe it
as an elementary binary choice or decision among two possibilities
which have equal probabilities. At the message source a bit represents
a certain amount of choice as to the message which will be
generated; in writing grammatical English we have on the average
a choice of about one bit per letter. At the destination a bit of
information resolves a certain amount of uncertainty; in receiving
English text there is on the average, about one bit of uncertainty
as to what the next letter will be.
When we are transmitting messages generated by an information
source by means of off-or-on pulses, we know how many binary
digits we are transmitting per second even when (as in most cases)
we don’t know the entropy of the source. (If we know the entropy
of the source in bits per second to be less than the binary digits
used per second, we would know that we could get along in principle
with fewer binary digits per second.) We know how to use the
binary digits to specify or determine one out of several possibilities,
either by means of a tree such as that of Figure IV-4 or by means
of a Huffman code such as that of Figure V-3. It is common in such
100
Symbols, Signals and Noise
a case to speak of the rate of transmission of binary digits as a bit
rate, but there is a certain danger that the inexperienced may
muddy their thinking if they do this.
All that I really ask of the reader is to remember that we have
used bit in one sense only, as a measure of information and have
called 0 or 1 a binary digit. If we can transmit 1,000 freely chosen
binary digits per second, we can transmit 1,000 bits of information
a second. It may be convenient to use bit to mean binary digit, but
when we do so we should be sure that we understand what we
are doing.
Let us now return for a moment to an entirely different matter,
the Huffman code given in Table XII and Figure V-3. When we
encode a message by using this code and get an uninterrupted
string of symbols, how do we tell whether we should take a particular
1 in the string of symbols as indicating the word the or as part
of the code for some other word?
We should note that of the codes in Table XII, none forms the
first part of another, This is called the prefix property. It has
important and, indeed, astonishing consequences, which are easily
illustrated. Suppose, for instance, that we encode the message: the
man sells the house to the man the horse runs to the man. The
encoded message is as follows:
the man sells the house
0 0 1 0 0 0 0 0 1 0 0 0
lii likes li man i
to the man the horse
1 1 1 0 0 1 1 0 0 0 0 1
to thei man , i the. horse
runs to [ the| man |
1 0 0 1 1 1 0 0 1
runs to 1 , 1 i thei man 1 l
Here the message words are written above the code groups.
Entropy 101
Now suppose we receive only the digits following the first vertical
dashed line below the digits. We start to decode by looking for the
shortest sequence of digits which constitutes a word in our code.
This is 00010, which corresponds to likes. We go on in this fashion.
The “decoded” words are written under the code, separated by
dashed lines.
We see that after a few errors the dashed lines correspond to the
solid lines, and from that point on the deciphered message is
correct. We don’t need to know where the message starts in order
to decode it as correctly as possible (unless all code words are of
equal length).
When we look back we can see that we have fulfilled the purpose
of this chapter. We have arrived at a measure of the amount of
information per symbol or per unit time of an ergodic source, and
we have shown how this is equal to the average number of binary
digits per symbol necessary to transmit the messages produced by
the source. We have noted that to attain transmission with negligibly
more bits than the entropy, we must encode the messages
produced by the source in long blocks, not symbol by symbol.
We might ask, however, how long do the blocks have to be? Here
we come back to another consideration. There are two reasons for
encoding in long blocks. One is, in order to make the average
number of binary digits per symbol used in the Huffman code
negligibly larger than the entropy per symbol. The other is, that
to encode such material as English text efficiently we must take
into account the influence of preceding symbols on the probability
that a given symbol will appear next. We have seen that we can
do this using equation 5.3 and taking very long blocks.
We return, then, to the question: how many symbols N must the
block of characters have so that (1) the Huffman code is very
efficient, (2) the entropy per block, disregarding interrelations
outside of the block, is very close to N times the entropy per
symbol? In the case of English text, condition 2 is governing,
Shannon has estimated the entropy per letter for English text
by measuring a person’s ability to guess the next letter of a message
after seeing 1, 2, 3, etc., preceding letters. In these texts the
“alphabet” used consisted of 26 letters plus the space.
Figure V-4 shows the upper and lower bounds on the entropy
of English plotted vs. the number of letters the person saw in
o to
Symbols, Signals and Noise
Fig. V-4
Entropy 103
making his prediction. While the curve seems to drop slowly as
the number of letters is increased from 10 to 15, it drops substantially
between 15 and 100. This would appear to indicate that we
might have to encode in blocks as large as 100 letters long in order
to encode English really efficiently,
From Figure V-4 it appears that the entropy of English text lies
somewhere between 0.6 and 1.3 bits per letter. Let us assume a
value of 1 bit per letter. Then it will take on the average 100 binary
digits to encode a block of 100 letters. This means that there are
2100 probable English sequences of 100 letters. In our usual decimal
notation, 2100 can be written as 1 followed by 30 zeroes, a fantastically
large number.
In endeavoring to find the probability in English text of all
meaningful blocks of letters 100 letters long, we would have to
count the relative frequency of occurrence of each such block.
Since there are 1030 highly likely blocks, this would be physically
impossible.
Further, this is impossible in principle. Most of these 1030
sequences of letters and spaces (which do not include all meaningful
sequences) have never been written down! Thus, it is impossible
to speak of their relative frequencies or probabilities of such long
blocks of letters as derived from English text.
Here we are really confronted with two questions: the accuracy
of the description of English text as the product of an ergodic
source and the most appropriate statistical description of that
source. One may believe that appropriate probabilities do exist in
some form in the human being even if they cannot be evaluated
by the examination of existing text. Or one may believe that the
probabilities exist and that they can be derived from data taken
in some way more appropriate than a naive computation of the
probabilities of sequences of letters. We may note, for instance,
that equations 5.4 and 5.5 also give the entropy of an ergodic
source. Equation 5.5 applies to a finite-state machine. We have
noted at the close of Chapter III that the idea of a human being
being in some particular state and in that state producing some
particular symbol or word is an appealing one.
Some linguists hold, however, that English grammar is inconsistent
with the output of a finite-state machine. Clearly, in trying
104
Symbols, Signals and Noise
to understand the structure and the entropy of actual English text
we would have to consider such text much more deeply than we
have up to this point.
It is safe if not subtle to apply an exact mathematical theory
blindly and mechanically to the ideal abstraction for which it holds.
We must be clever and wise in using even a good and appropriate
mathematical theory in connection with actual, nonideal problems.
We should seek a simple and realistic description of the laws governing
English text if we are to relate it with communication theory
as successfully as possible. Such a description must certainly
involve the grammar of the language, which we will discuss in the
next chapter.
In any event, we know that there are some valid statistics of
English text, such as letter and word frequencies, and the coding
theorems enable us to take advantage of such known statistics.
If we encode English letter by letter, disregarding the relative
frequencies of the letters, we require 4.76 binary digits per character
(including space). If we encode letter by letter, taking into account
the relative probabilities of various letters, we require 4.03 binary
digits per character. If we encode word by word, taking into
account relative frequencies of words, we require 1.66 binary digits
per character. And, by using an ingenious and appropriate means,
Shannon has estimated the entropy of English text to be between
.6 and 1.3 bits per letter, so that we may hope for even more
efficient encoding.
If, however, we mechanically push some particular procedure
for finding the entropy of English text to the limit, we can easily
engender not only difficulties but nonsense. Perhaps we can ascribe
this nonsense partly to differences between man as a source of
English text and our model of an ideal ergodic source, but partly
we should ascribe it to the use of an inappropriate approach. We
can surely say that the model of man as an ergodic source of text
is good and useful if not perfect, and we should regard it highly
for these qualities.
This chapter has been long and heavy going, and a summary
seems in order. Clearly, it is impossible to recapitulate briefly all
those matters which took so many pages to expound. We can only
re-emphasize the most vital points.
Entropy 105
In communication theory the entropy of a signal source in bits
per symbol or per second gives the average number of binary
digits, per symbol or per second, necessary to encode the messages
produced by the source.
We think of the message source as randomly, that is, unpre-
dictably, choosing one among many possible messages for transmission.
Thus, in connection with the message source we think of
entropy as a measure of choice, the amount of choice the source
excercises in selecting the one particular message that is actually
transmitted.
We think of the recipient of the message, prior to the receipt of
the message, as being uncertain as to which among the many
possible messages the message source will actually generate and
transmit to him. Thus, we think of the entropy of the message
source as measuring the uncertainty of the recipient as to which
message will be received, an uncertainty which is resolved on
receipt of the message.
If the message is one among n equally probable symbols or
messages, the entropy is log n. This is perfectly natural, for if we
have log n binary digits, we can use them to write out
21ogn _ n
different binary numbers, and one of these numbers can be used
as a label for each of the n messages.
More generally, if the symbols are not equally probable, the
entropy is given by equation 5.1. By regarding a very long block
of symbols, whose content is little dependent on preceding symbols,
as a sort of super symbol, equation 5.1 can be modified to give the
entropy per symbol for information sources in which the probability
that a symbol is chosen depends on what symbols have been
chosen previously. This gives us equation 5.3. Other general
expressions for entropj are given by equations 5.4 and 5.5.
By assuming that the symbols or blocks of symbols which a
source produces are encoded by a most efficient binary code called
a Huffman code, it is possible to prove that the entropy of an
ergodic source measured in bits is equal to the average number of
binary digits necessary to encode it.
An error-free communication channel may not transmit binary
106
Symbols, Signals and Noise
digits; it may transmit letters or other symbols. We can imagine
attaching different message sources to such a channel and seeking
(usually mathematically) the message source that causes the entropy
of the message transmitted over the channel to be as large
as possible. This largest possible entropy of a message transmitted
over an error-free channel is called the channel capacity. It can be
proved that, if the entropy of a source is less than the channel
capacity of the channel, messages from the source can be encoded
so that they can be transmitted over the channel. This is Shannon’s
fundamental theorem for the noiseless channel.
In principle, expressions such as equations 5.1, 5.3, 5.4, and 5.5
enable us to compute the entropy of a message source by statistical
analysis of messages produced by the source. Even for an ideal
ergodic source, this would often call for impractically long computations.
In the case of an actual source, such as English text, some
naive prescriptions for computing entropy can be meaningless.
An approximation to the entropy can be obtained by disregarding
the effect of some past symbols on the probability of the source
producing a particular symbol next. Such an approximation to the
entropy is always too large and calls for encoding by means of more
binary digits than are absolutely necessary. Thus, if we encode
English text letter by letter, disregarding even the relative probabilities
of letters, we require 4.76 binary digits per letter, while if
we encode word by word, taking into account the relative probability
of words, we require 1.66 binary digits per letter.
If we wanted to do even better we would have to take into
account other features of English such as the effect of the constraints
imposed by grammar on the probability that a message
source will produce a particular word.
While we do not know how to encode English text in a highly
efficient way, Shannon made an ingenious experiment which shows
that the entropy of English text must lie between .6 and 1.3 bits
per character. In this experiment a person guessed what letter
would follow the letters of a passage of text many letters long.
CHAPTER Language and
Meaning
The tw o great triumphs of information theory are establishing
the channel capacity and, in particular, the number of binary digits
required to transmit information from a particular source and
showing that a noisy communication channel has an information
rate in bits per character or bits per second up to which errorless
transmission is possible despite the noise. In each case, the results
must be demonstrated for discrete and for continuous sources and
channels.
After four chapters of by no means easy preparation, we were
finally ready to essay in the previous chapter the problem of the
number of binary digits required to transmit the information generated
by a truly ergodic discrete source. Were this book a text on
information theory, we would proceed to the next logical step, the
noisy discrete channel, and then on to the ergodic continuous
channel.
At the end of such a logical progress, however, our thoughts
would necessarily be drawn back to a consideration of the message
sources of the real world, which are only approximately ergodic,
and to the estimation of their entropy and the efficient encoding
of the messages they produce.
Rather than proceeding further with the strictly mathematical
aspects of communication theory at this point, is it not more
attractive to pause and consider that chief form of communication,
107
108
Symbols, Signals and Noise
language, in the light of communication theory? And, in doing so,
why should we not let our thoughts stray a little in viewing an important
part of our world from the small eminence we have
attained? Why should we not see whether even the broad problems
of language and meaning seem different to us in the light of what
we have learned?
In following such a course the reader should heed a word of
caution. So far the main emphasis has been on what we know. What
we know is the hard core of science. However, scientists find it very
difficult to share the things that they know with laymen. To understand
the sure and the reasonably sure knowledge of science takes
the sort of hard thought which I am afraid was required of the
reader in the last few chapters.
There is, however, another and easier though not entirely frivolous
side to science. This is a peculiar type of informed ignorance.
The scientist’s ignorance is rather different from the layman’s
ignorance, because the background of established fact and theory
on which the scientist bases his peculiar brand of ignorance excludes
a wide range of nonsense from his speculations. In the higher
and hazier reaches of the scientist’s ignorance, we have scientifically
informed ignorance about the origin of the universe, the ultimate
basis of knowledge, and the relation of our present scientific knowledge
to politics, free will, and morality. In this particular chapter
we will dabble in what I hope to be scientifically informed ignorance
about language.
The warning is, of course, that much of what will be put forward
here about language is no more than informed ignorance. The
warning seems necessary because it is very hard for laymen to tell
scientific ignorance from scientific fact. Because the ignorance is
necessarily expressed in broader, sketchier, and less qualified terms
than is the fact, it is easier to assimilate. Because it deals with grand
and unsolved problems, it is more romantic. Generally, it has a
wider currency and is held in higher esteem than is scientific fact.
However hazardous such ignorance may be to the layman, it is
valuable to the scientist. It is this vision of unattained lands, of
unsealed heights, which rescues him from complacency and spurs
him beyond mere plodding. But when the scientist is airing his
ignorance he usually knows what he is doing, while the unwarned
Language and Meaning 109
layman apparently often does not and is left scrambling about on
cloud mountains without ever having set foot on the continents of
knowledge.
With this caution in mind, let us return to what we have already
encountered concerning language and proceed thence.
In what follows we will confine ourselves to a discussion of
grammatical English. We all know (and especially those who have
had the misfortune of listening to a transcription of a seemingly
intelligible conversation or technical talk) that much spoken English
appears to be agrammatical, as, indeed, much of Gertrude
Stein is. So are many conventions and cliches. “Me heap big
chief” is perfectly intelligible anywhere in the country, yet it is
certainly not grammatical. Purists do not consider the inverted
word order which is so characteristic of second-rate poetry as being
grammatical.
Thus, a discussion of grammatical English by no means covers
the field of spoken and written communication, but it charts a
course which we can follow with some sense of order and interest.
We have noted before that, if we are to write what will be
accepted as English text, certain constraints must be obeyed. We
cannot simply set down any word following any other. A complete
grammar of a language would have to express all of these constraints
fully. It should allow within its rules the construction of
any sequence of English words which will be accepted, at some
particular time and according to some particular standard, as
grammatical.
The matter of acceptance of constructions as grammatical is a
difficult and hazy one. The translators who produced the King
James Bible were free to say “fear not,” “sin not,” and “speak not”
as well as “think not,” “do not,” or “have not,” and we frequently
repeat the aphorism “want not, waste not.” Yet in our everyday
speech or writing we would be constrained to say “do not fear,”
“do not sin,” or “do not speak,” and we might perhaps say, “If
you are not to want, you should not waste.” What is grammatical
certainly changes with time. Here we can merely notice this and
pass on to other matters.
Certainly, a satisfactory grammar must prescribe certain rules
which allow the construction of all possible grammatical utterances
110
Symbols, Signals and Noise
and of grammatical utterances only. Besides doing this, satisfactory
rules of grammar should allow us to analyze a sentence so as to
distinguish the features which were determined merely by the rules
of grammar from any other features.
If we once had such rules, we would be able to make a new estimate
of the entropy of English text, for we could see what part of
sentence structure is a mere mechanical following of rules and what
part involves choice or uncertainty and hence contributes to entropy.
Further, we could transmit English efficiently by transmitting
as a message only data concerning the choices exercised in
constructing sentences; at the receiver, we could let a grammar
machine build grammatical sentences embodying the choices specified
by the received message.
Even grammar, of course, is not the whole of language, for a
sentence can be very odd even if it is grammatical. We can imagine
that, if a machine capable of producing only grammatical sentences
made its choices at random, it might perhaps produce such a sentence
as “The chartreuse semiquaver skinned the feelings of the
manifold.” A man presumably makes his choices in some other
way if he says, “The blue note flayed the emotions of the multitude.”
The difference lies in what choices one makes while following
grammatical rules, not in the rules themselves. An understanding
of grammar would not unlock to us all of the secrets of
language, but it would take us a long step forward.
What sort of rules will result in the production of grammatical
sentences only and of all grammatical sentences, even when choices
are made at random? In Chapter III we saw that English-like
sequences of words can be produced by choosing a word at random
according to its probability of succeeding a preceding sequence
of words some M words long. An example of a second-order
word approximation, in which a word is chosen on the basis of its
succeeding the previous word, was given.
One can construct higher-order word approximations by using
the knowledge of English which is stored in our heads. One can,
for instance, obtain a fourth-order word approximation by simply
showing a sequence of three connected words to a person and asking
him to think up a sentence in which the sequence of words
occurs and to add the next word. By going from person to person
a long string of words can be constructed, for instance:
Language and Meaning
Ill
1. When morning broke after an orgy of wild abandon he said
her head shook vertically aligned in a sequence of words signifying
what.
2. It happened one frosty look of trees waving gracefully against
the wall.
3. When cooked asparagus has a delicious flavor suggesting
apples.
4. The last time I saw him when he lived.
These “sentences” are as sensible as they are because selections
of words were not made at random but by thinking beings. The
point to be noted is how astonishingly grammatical the sentences
are, despite the fact that rules of grammar (and sense) were applied
to only four words at a time (the three shown to each person
and the one he added). Still, example 4 is perhaps dubiously
grammatical.
If Shannon is right and there is in English text a choice of about
1 bit per symbol, then choosing among a group of 4 words could
involve about 22 binary choices, or a choice among some 10 million
4-word combinations. In principle, a computer could be made
to add words by using such a list of combinations, but the result
would not be assuredly grammatical, nor could we be sure that
this cumbersome procedure would produce all possible grammatical
sequences of words. There probably are sequences of words
which could form a part of a grammatical sentence in one case
and could not in another case. If we included such a sequence, we
would produce some nongrammatical sentences, and, if we excluded
it, we would fail to produce all grammatical sentences.
If we go to combinations of more than four words, we will favor
grammar over completeness. If we go to fewer than four words,
we will favor completeness over grammar. We can’t have both.
The idea of a finite-state machine recurs at this point. Perhaps
at each point in a sentence a sentence-producing machine should
be in a particular state, which allows it certain choices as to what
state it will go to next. Moreover, perhaps such a machine can deal
with certain classes or subclasses of words, such as singular nouns,
plural nouns, adjectives, adverbs, verbs of various tense and number,
and so on, so as to produce grammatical structures into which
words can he fitted rather than sequences of particular words.
The idea of grammar as a finite-state machine is particularly
112
Symbols, Signals and Noise
appealing because a mechanist would assert that man must be a
finite-state machine, because he consists of only a finite number
of cells, or of atoms if we push the matter further.
Noam Chomsky, a brilliant and highly regarded modem linguist,
rejects the finite-state machine as either a possible or a proper
model of grammatical structure. Chomsky points out that there
are many rules for constructing sequences of characters which cannot
be embodied in a finite-state machine. For instance, the rule
might be, choose letters at random and write them down until the
letter Z shows up, then repeat all the letters since the preceding Z
in reverse order, and then go on with a new set of letters, and so
on. This process will produce a sequence of letters showing clear
evidence of long-range order. Further, there is no limit to the possible
length of the sequence between Z’s. No finite-state machine
can simulate this process and this result.
Chomsky points out that there is no limit to the possible length
of grammatical sentences in English and argues that English sentences
are organized in such a way that this is sufficient to rule
out a finite-state machine as a source of all possible English text.
But, can we really regard a sentence miles long as grammatical
when we know darned well that no one ever has or will produce
such a sentence and that no one could understand it if it existed?
To decide such a question, we must have a standard of being
grammatical. While Chomsky seems to refer being or not being
grammatical, and some questions of punctuation and meaning as
well, to spoken English, I think that his real criterion is: a sentence
is grammatical if, in reading or saying it aloud with a natural
expression and thoughtfully but ingenuously, it is deemed grammatical
by a person who speaks it, or perhaps by a person who
hears it. Some problems which might plague others may not bother
Chomsky because he speaks remarkably well-connected and grammatical
English.
Whether or not the rules of grammar can be embodied in a
finite-state machine, Chomsky offers persuasive evidence that it is
wrong and cumbersome to try to generate a sentence by basing
the choice of the next word entirely and solely on words already
written down. Rather, Chomsky considers the course of sentence
generation to be something of this sort:
Language and Meaning 113
We start with one or another of several general forms the sentence
might take; for example, a noun phrase followed by a verb
phrase. Chomsky calls such a particular form of sentence a kernel
sentence. We then invoke rules for expanding each of the parts of
the kernel sentence. In the case of a noun phrase we may first describe
it as an article plus a noun and finally as “the man.” In the
case of a verb phrase we may describe it as a verb plus an object,
the object as an article plus a noun, and, in choosing particular
words, as “hit the ball.” Proceeding in this way from the kernel
sentence, noun phrase plus verb phrase, we arrive at the sentence,
“The man hit the ball.” At any stage we could have made other
choices. By making other choices at the final stages we might have
arrived at “A girl caught a cat.”
Here we see that the element of choice is not exercised sequentially
along the sentence from beginning to end. Rather, we choose
an over-all skeletal plan or scheme for the whole final sentence at
the start. That scheme or plan is the kernel sentence. Once the
kernel sentence has been chosen, we pass on to parts of the kernel
sentence. From each part we proceed to the constituent elements
of that part and from the constituent elements to the choice of
particular words. At each branch of this treelike structure growing
from the kernel sentence, we exercise choice in arriving at the
particular final sentence, and, of course, we chose the kernel sentence
to start with.
Here I have indicated Chomsky’s ideas very incompletely and
very sketchily. For instance, in dealing with irregular forms of
words Chomsky will first indicate the root word and its particular
grammatical form, and then he will apply certain obligatory rules
in arriving at the correct English form. Thus, in the branching construction
of a sentence, use is made both of optional rules, which
allow choice, and of purely mechanical, deterministic obligatory
rules, which do not.
To understand this approach further and to judge its merit, one
must refer to Chomsky’s book,1 and to the references he gives.
Chomsky must, of course, deal with the problem of ambiguous
sentences, such as, “The lady scientist made the robot fast while
she ate.” The author of this sentence, a learned information theo-
1Noam Chomsky, Syntactic Structures, Mouton and Co., ’s-Gravenhage, 1957.
114 Symbols, Signals and Noise
rist, tells me that, allowing for the vernacular, it has at least four
different meanings. It is perhaps too complicated to serve as an
example for detailed analysis.
We might think that ambiguity arises only when one or more
words can assume different meanings in what is essentially the same
grammatical structure. This is the case in “he was mad” (either
angry or insane) or “the pilot was high” (in the sky or in his cups).
Chomsky, however, gives a simple example of a phrase in which
the confusion is clearly grammatical. In “the shooting of the
hunters,” the noun hunters may be either the subject, as in “the
growling of lions” or the object, as in “the growing of flowers.”
Chomsky points out that different rules of transformation applied
to different kernel sentences can lead to the same sequence of
grammatical elements. Thus, “the picture was painted by a real
artist” and “the picture was painted by a new technique” seem to
correspond grammatically word for word, yet the first sentence
could have arisen as a transformation of “a real artist painted the
picture” while the second could not have arisen as a transformation
of a sentence having this form. When the final words as well
as the final grammatical elements are the same, the sentence is
ambiguous.
Chomsky also faces the problem that the distinction between
the provinces of grammar and meaning is not clear. Shall we say
that grammar allows adjectives but not adverbs to modify nouns?
This allows “colorless green.” Or should grammar forbid the association
of some adjectives with some nouns, of some nouns with
some verbs, and so on? With one choice, certain constructions are
grammatical but meaningless; with the other they are ungrammatical.
We see that Chomsky has laid out a plan for a grammar of
English which involves at each point in the synthesis of a sentence
certain steps which are either obligatory or optional. The processes
allowed in this grammar cannot be carried out by a finite-state
machine, but they can be carried out by a more general machine
called a Turing machine, which is a finite-state machine plus an
infinitely long tape on which symbols can be written and from
which symbols can be read or erased. The relation of Chomsky’s
grammar to such machines is a proper study for those interested
in automata.
Language and Meaning 115
We should note, however, that if we arbitrarily impose some
bound on the length of a sentence, even if we limit the length to
1,000 or 1 million words, then Chomsky’s grammar does correspond
to a finite-state machine. The imposition of such a limit on sentence
length seems very reasonable in a practical way.
Once a general specification or model of a grammar of the sort
Chomsky proposes is set up, we may ask under what circumstances
and how can an entropy be derived which will measure the choice
or uncertainty of a message source that produces text according
to the rules of the grammar? This is a question for the mathematically
skilled information theorist.
Much more important is the production of a plausible and
workable grammar. This might be a phrase-structure grammar, as
Chomsky proposes, or it might take some other form. Such a
grammar might be incomplete in that it failed to produce or analyze
some constructions to be found in grammatical English. It
seems more important that its operation should correspond to what
we know of the production of English by human beings. Further,
it should be simple enough to allow the generation and analysis
of text by means of an electronic computer. I believe that computers
must be used in attacking problems of the structure and
statistics of English text.
While a great many people are convinced that Chomsky’s
phrase-structure approach is a very important aspect of grammar,
some feel that his picture of the generation of sentences should be
modified or narrowed if it is to be used to describe the actual generation
of sentences by human beings. Subjectively, in speaking
or listening to a speaker one has a strong impression that sentences
are generated largely from beginning to end. One also gets the
impression that the person generating a sentence doesn’t have a
very elaborate pattern in his head at any one time but that he
elaborates the pattern as he goes along.
I suspect that studies of the form of grammars and of the statistics
of their use as revealed by language will in the not distant
future tell us many new things about the nature of language and
about the nature of men as well. But, to say something more particular
than this, I would have to outreach present knowledge—
mine and others.
A grammar must specify not only rules for putting different types
116
Symbols, Signals and Noise
of words together to make grammatical structures; it must divide
the actual words of English into classes on the basis of the places
in which they can appear in grammatical structures. Linguists make
such a division purely on the basis of grammatical function without
invoking any idea of meaning. Thus, all we can expect of a
grammar is the generation of grammatical sentences, and this includes
the example given earlier: “The chartreuse semiquaver
skinned the feelings of the manifold.” Certainly the division of
words into grammatical categories such as nouns, adjectives, and
verbs is not our sole guide concerning the use of words in producing
English text.
What does influence the choice among words when the words
used in constructing grammatical sentences are chosen, not at
random by a machine, but rather by a live human being who,
through long training, speaks or writes English according to the
rules of the grammar? This question is not to be answered by a
vague appeal to the word meaning. Our criteria in producing English
sentences can be very complicated indeed. Philosophers and
psychologists have speculated about and studied the use of words
and language for generations, and it is as hard to say anything entirely
new about this as it is to say anything entirely true. In particular,
what Bishop Berkeley wrote in the eighteenth century
concerning the use of language is so sensible that one can scarcely
make a reasonable comment without owing him credit.
Let us suppose that a poet of the scanning, rhyming school sets
out to write a grammatical poem. Much of his choice will be exercised
in selecting words which fit into the chosen rhythmic pattern,
which rhyme, and which have alliteration and certain consistent
or agreeable sound values. This is particularly notable in Poe’s
“The Bells,” “Ulalume,” and “The Raven.”
Further, the poet will wish to bring together words which through
their sound as well as their sense arouse related emotions or impressions
in the reader or hearer. The different sections of Poe’s
“The Bells” illustrate this admirably. There is a marked contrast
between:
How they tinkle, tinkle, tinkle,
In the icy air of night!
While the stars that oversprinkle
Language and Meaning
All the heavens, seem to twinkle
In a crystalline delight;...
117
Through the balmy air of night
How they ring out their delight!
From the molten-golden notes,
And all in tune,
What a liquid ditty floats ...
Sometimes, the picture may be harmonious, congruous, and
moving without even the trivial literal meaning of this verse of
Poe’s, as in Blake’s two lines:
Tyger, Tyger, burning bright
In the forests of the night...
In instances other than poetry, words may be chosen for euphony,
but they are perhaps more often chosen for their associations with
and ability to excite passions such as those listed by Berkeley: fear,
love, hatred, admiration, disdain. Particular words or expressions
move each of us to such feelings. In a given culture, certain words
and phrases will have a strong and common effect on the majority
of hearers, just as the sights, sounds or events with which they are
associated do. The words of a hymn or psalm can induce a strong
religious emotion; political or racial epithets, a sense of alarm or
contempt, and the words and phrases of dirty jokes, sexual
excitement.
One emotion which Berkeley does not mention is a sense of
understanding. By mouthing commonplace and familiar patterns
of words in connection with ill-understood matters, we can associate
some of our emotions of familiarity and insight with our perplexity
about history, life, the nature of knowledge, consciousness,
death, and Providence. Perhaps such philosophy as makes use of
common words should be considered in terms of assertion of a
reassurance concerning the importance of man’s feelings rather
than in terms of meaning.
One could spend days on end examining examples of motivation
in the choice of words, but we do continually get back to the matter
of meaning. Whatever meaning may be, all else seems lost without
118
Symbols, Signals and Noise
it. A Chinese poem, hymn, deprecation, or joke will have little effect
on me unless I understand Chinese in whatever sense those who
know a language understand it.
Though Colin Cherry, a well-known information theorist, appears
to object, I think that it is fair to regard meaningful language
as a sort of code of communication. It certainly isn’t a simple code
in which one mechanically substitutes a word for a deed. It’s more
like those elaborate codes of early cryptography, in which many
alternative code words were listed for each common letter or word
(in order to suppress frequencies). But in language, the listings may
overlap. And one person’s code book may have different entries
from another’s, which is sure to cause confusion.
If we regard language as an imperfect code of communication,
we must ultimately refer meaning back to the intent of the user.
It is for this reason that I ask, “What do you mean?” even when I
have heard your words. Scholars seek the intent of authors long
dead, and the Supreme Court seeks to establish the intent of Congress
in applying the letter of the law.
Further, if I become convinced that a man is lying, I interpret
his words as meaning that he intends to flatter or deceive me. If I
find that a sentence has been produced by a computer, I interpret
it to mean that the computer is functioning very cleverly.
I don’t think that such matters are quibbles; it seems that we
are driven to such considerations in connection with meaning if
we do regard language as an imperfect code of communication,
and as one which is sometimes exploited in devious ways. We are
certainly far from any adequate treatment of such problems.
Grammatical sentences do, however, have what might be called
a formal meaning, regardless of intent. If we had a satisfactory
grammar, a machine should be able to establish the relations between
the words of a sentence, indicating subject, verb, object, and
what modifying phrases or clauses apply to what other words. The
next problem beyond this in seeking such formal meaning in sentences
is the problem of associating words with objects, qualities,
actions, or relations in the world about us, including the world of
man’s society and of his organized knowledge.
In the simple communications of everyday life, we don’t have
much trouble in associating the words that are used with the proper
Language
and Meaning
119
objects, qualities, actions, and relations. No one has trouble with
“close the east window” or “Henry is dead,” when he hears such
a simple sentence in simple, unambiguous surroundings. In a
familiar American room, anyone can point out the window; we
have closed windows repeatedly, and we know what direction east
is. Also, we know Henry (if we don’t get Henry Smith mixed up
with Henry Jones), and we have seen dead people. If the sentence
is misheard or misunderstood, a second try is almost sure to
succeed.
Think, however, how puzzling the sentence about the window
would be, even in translation, to a shelterless savage. And we can
get pretty puzzled ourselves concerning such a question as, is a
virus living or dead?
It appears that much of the confusion and puzzlement about the
associations of words with things of the world arose through an
effort by philosophers from Plato to Locke to give meaning to such
ideas as window, cat, or dead by associating them with general ideas
or ideal examples. Thus, we are presumed to identify a window by
its resemblance to a general idea of a window, to an ideal window,
in fact, and a cat by its resemblance to an ideal cat which embodies
all the attributes of cattiness. As Berkeley points out, the abstract
idea of a (or the ideal) triangle must at once be “neither oblique,
rectangle, equilateral, equicrural nor scaleron, but all and none of
these at once.”
Actually, when a doctor pronounces a man dead he does so on
the basis of certain observed signs which he would be at a loss to
identify in a virus. Further, when a doctor makes a diagnosis, he
does not start out by making an over-all comparison of the patient’s
condition with an ideal picture of a disease. He first looks for such
signs as appearance, temperature, pulse, lesions of the skin, inflammation
of the throat, and so on, and he also notes such symptoms
as the patient can describe to him. Particular combinations of signs
and symptoms indicate certain diseases, and in differential diagnoses
further tests may be used to distinguish among diseases producing
similar signs and symptoms.
In a similar manner, a botanist identifies a plant, familiar or
unfamiliar, by the presence or absence of certain qualities of size,
color, leaf shape and disposition, and so on. Some of these quali
120
Symbols, Signals and Noise
ties, such as the distinction between the leaves of monocotyledon-
ous and dicotyledonous plants, can be decisive; others, such as size,
can be merely indicative. In the end, one is either sure he is right
or perhaps willing to believe that he is right; or the plant may be
a new species.
Thus, in the workaday worlds of medicine and botany, the ideal
disease or plant is conspicuous by its absence as any actual useful
criterion. Instead, we have lists of qualities, some decisive and some
merely indicative.
The value of this observation has been confirmed strongly in
recent work toward enabling machines to carry out tasks of recognition
or classification. Early workers, perhaps misled by early
philosophers, conceived the idea of matching a letter to an ideal
pattern of a letter or the spectrogram of a sound to an ideal spectrogram
of the sound. The results were terrible. Audrey, a pattern-
matching machine with the bulk of a hippo and brains beneath
contempt, could recognize digits spoken by one voice or a selected
group of voices, but Audrey was sadly fallible. We should, I think,
conclude that human recognition works this way in very simple
cases only, if at all.
Later and more sophisticated workers in the field of recognition
look for significant features. Thus, as a very simple example, rather
than having an ideal pattern of a capital Q, one might describe Q
as a closed curve without corners or reversals of curvature and with
something attached between four and six o’clock.
In 1959, L. D. Harmon built at the Bell Laboratories a simple
device weighing a few pounds which almost infallibly recognizes
the digits from one to zero written out as words in longhand. Does
this gadget match the handwriting against patterns? You bet it
doesn’t! Instead, it asks such questions as, how many times did
the stylus go above or below certain lines? Were I’s dotted or T’s
crossed?
Certainly, no one doubts that words refer to classes of objects,
actions, and so on. We are surrounded by and involved with a large
number of classes and subclasses of objects and actions which we'
can usefully associate with words. These include such objects as
plants (peas, sunflowers ...), animals (cats, dogs ...), machines
(autos, radios ...), buildings (houses, towers ...), clothing (skirts,
Language and Meaning
121
socks...), and so on. They include such very complicated sequences
of actions as dressing and undressing (the absent-minded, including
myself, repeatedly demonstrate that they can do this unconsciously);
tying one’s shoes (an act which children have considerable
difficulty in learning), eating, driving a car, reading, writing, adding
figures, playing golf or tennis (activities involving a host of distinct
subsidiary skills), listening to music, making love, and so on and
on and on.
It seems to me that what delimits a particular class of objects,
qualities, actions, or relations is not some sort of ideal example.
Rather, it is a list of qualities. Further, the list of qualities cannot
be expected to enable us to divide experience up into a set of logical,
sharply delimited, and all-embracing categories. The language
of science may approach this in dealing with a narrow range of
experience, but the language of everyday life makes arbitrary,
overlapping, and less than all-inclusive divisions of experience. Yet,
I believe that it is by means of such lists of qualities that we identify
doors, windows, cats, dogs, men, monkeys, and other objects
of daily life. I feel also that this is the way in which we identify
common actions such as running, skipping, jumping, and tying,
and such symbols as words, written and spoken, as well.
I think that it is only through such an approach that we can hope
to make a machine classify objects and experience in terms of
language, or recognize and interpret language in terms of other
language or of action. Further, I believe that when a word cannot
offer a table of qualities or signs whose elements can be traced back
to common and familiar experiences, we have a right to be wary
of the word.
If we are to understand language in such a way that we can hope
some day to make a machine which will use language successfully,
we must have a grammar and we must have a way of relating words
to the world about us, but this is of course not enough. If we are to
regard sentences as meaningful, they must in some way correspond
to life as we live it.
Our lives do not present fresh objects and fresh actions each day.
They are made up of familiar objects and familiar though complicated
sequences of actions presented in different groupings and
orders. Sometimes we learn by adding new objects, or actions, or
122
Symbols, Signals and Noise
combinations of objects or sequences of actions to our stock, and
so we enrich or change our lives. Sometimes we forget objects and
actions.
Our particular actions depend on the objects and events about
us. We dodge a car (a complicated sequence of actions). When
thirsty, we stop at the fountain and drink (another complicated but
recurrent sequence). In a packed crowd we may shoulder someone
out of the way as we have done before. But our information about
the world does not all come from direct observation, and our influence
on others is happily not confined to pushing and shoving.
We have a powerful tool for such purposes: language and words.
We use words to learn about relations among objects and activities
and to remember them, to instruct others or to receive instruction
from them, to influence people in one way or another. For the
words to be useful, the hearer must understand them in the same
sense that the speaker means them, that is, insofar as he associates
them with nearly enough the same objects or skills. It’s no use,
however, to tell a man to read or to add a column of figures if he
has never carried out these actions before, so that he doesn’t have
these skills. It is no use to tell him to shoot the aardvark and not
the gnu if he has never seen either.
Further, for the sequences of words to be useful, they must refer
to real or possible sequences of events. It’s of no use to advise a
man to walk from London to New York in the forenoon immediately
after having eaten a seven o’clock dinner.
Thus, in some way the meaningfulness of language depends not
only on grammatical order and on a workable way of associating
words with collections of objects, qualities, and so on; it also depends
on the stmcture of the world around us. Here we encounter
a real and an extremely serious difficulty with the idea that we can
in some way translate sentences from one language into another
and accurately preserve the “meaning.”
One obvious difficulty in trying to do this arises from differences
in classification. We can refer to either the foot or the lower leg;
the Russians have one word for the foot plus the lower leg. Hungarians
have twenty fingers (or toes), for the word is the same for
either appendage. To most of us today, a dog is a dog, male or
female, but men of an earlier era distinguished sharply between a
Language and Meaning
123
dog and a bitch. Eskimos make, it is said, many distinctions among
snow which in our language would call for descriptions, and for
us even these descriptions would have little real content of importance
or feeling, because in our lives the distinctions have not been
important. Thus, the parts of the world which are common and
meaningful to those speaking different languages are often divided
into somewhat different classes. It may be impossible to write down
in different languages words or simple sentences that specify exactly
the same range of experience.
There is a graver problem than this, however. The range of
experience to which various words refer is not common among all
cultures. What is one to do when faced with the problem of translating
a novel containing the phrase, “tying one’s shoelace,” which
as we have noted describes a complicated action, into the language
of a shoeless people? An elaborate description wouldn’t call up the
right thing at all. Perhaps some cultural equivalent (?) could be
found. And how should one deal with the fact that “he built a
house” means personal tree cutting and adzing in a pioneer novel,
while it refers to the employment of an architect and a contractor
in a contemporary story?
It is possible to make some sort of translation between closely
related languages on a word-for-word or at least phrase-for-phrase
basis, though this is said to have led from “out of sight, out of
mind” to “blind idiot.” When the languages and cultures differ in
major respects, the translator has to think what the words mean
in terms of objects, actions, or emotions and then express this
meaning in the other language. It may be, of course, that the culture
with which the language is associated has no close equivalents
to the objects or actions described in the passage to be translated.
Then the translator is really stuck.
How, oh how is the man who sets out to build a translating
machine to cope with a problem such as this? He certainly cannot
do so without in some way enabling the machine to deal effectively
with what we refer to as understanding. In fact, we see understanding
at work even in situations which do not involve translation
from one language into another. A screen writer who can quite
accurately transfer the essentials of a scene involving a dying uncle
in Omsk to one involving a dying father in Dubuque will repeatedly
124 Symbols, Signals and Noise
make complete nonsense in trying to rephrase a simple technical
statement. This is clearly because he understands grief but not
science.
Having grappled painfully with the word meaning, we are now
faced with the word understanding. This seems to have two sides.
If we understand algebra or calculus, we can use their manipulations
to solve problems we haven’t encountered before or to supply
proofs of theorems we haven’t seen proved. In this sense, understanding
is manifested by a power to do, to create, not merely to
repeat. To some degree, an electronic computer which proves
theorems in mathematical logic which it has not encountered before
(as computers can be programmed to do) could perhaps be
said to understand the subject. But there is an emotional side to
understanding, too. When we can prove a theorem in several ways
and fit it together with other theorems or facts in various manners,
when we can view a field from many aspects and see how it all fits
together, we say that we understand the subject deeply. We attain
a warm and confident feeling about our ability to cope with it. Of
course, at one time or another most of us have felt the warmth
without manifesting the ability. And how disillusioned we were at
the critical test!
In discussing language from the point of view of information
theory, we have drifted along a tide of words, through the imperfectly
charted channels of grammar and on into the obscurities of
meaning and understanding. This shows us how far ignorance can
take one. It would be absurd to assert that information theory, or
anything else, has enabled us to solve the problems of linguistics,
of meaning, of understanding, of philosophy, of life. At best, we
can perhaps say that we are pushing a little beyond the mechanical
constraints of language and getting at the amount of choice
that language affords. This idea suggests views concerning the use
and function of language, but it does not establish them. The
reader may share my freely offered ignorance concerning these
matters, or he may prefer his own sort of ignorance.
CHAPTER VII Efficient Encoding
We will never again understand nature as well as Greek
philosophers did. A general explanation of common phenomena
in terms of a few all-embracing principles no longer satisfies us.
We know too much. We must explain many things of which the
Greeks were unaware. And, we require that our theories harmonize
in detail with the very wide range of phenomena which they seek
to explain. We insist that they provide us with useful guidance
rather than with rationalizations. The glory of Newtonian mechanics
is that it has enabled men to predict the positions of planets
and satellites and to understand many other natural phenomena
as well; it is surely not that Newtonian mechanics once inspired
and supported a simple mechanistic view of the universe at large,
including life.
Present-day physicists are gratified by the conviction that all
(non-nuclear) physical, chemical, and biological properties of matter
can in principle be completely and precisely explained in all
their detail by known quantum la,ws, assuming only the existence
of electrons and of atomic nuclei of various masses and charges.
It is somewhat embarrassing, however, that the only physical system
all of whose properties actually have been calculated exactly
is the isolated hydrogen atom.
Physicists are able to predict and explain some other physical
phenomena quite accurately and many more semiquantitatively.
However, a basic and accurate theoretical treatment, founded on
electrons, nuclei, and quantum laws only, without recourse to
125
126
Symbols, Signals and Noise
other experimental data, is lacking for most common thermal,
mechanical, electrical, magnetic, and chemical phenomena. Tracing
complicated biological phenomena directly back to quantum
first principles seems so difficult as to be scarcely relevant to the
real problems of biology. It is almost as if we knew the axioms of
an important field of mathematics but could prove only a few
simple theorems.
Thus, we are surrounded in our world by a host of intriguing
problems and phenomena which we cannot hope to relate through
one universal theory, however true that theory may be in principle.
Until recently the problems of science which we commonly associate
with the field of physics have seemed to many to be the most
interesting of all the aspects of nature which still puzzle us. Today,
it is hard to find problems more exciting than those of biochemistry
and physiology.
I believe, however, that many of the problems raised by recent
advances in our technology are as challenging as any that face us.
What could be more exciting than to explore the potentialities of
electronic computers in proving theorems or in simulating other
behavior we have always thought of as “human”? The problems
raised by electrical communication are just as challenging. Accurate
measurements made by electrical means have revolutionized
physical acoustics. Studies carried out in connection with telephone
transmission have inaugurated a new era in the study of
speech and hearing, in which previously accepted ideas of physiology,
phonetics, and liguistics have proved to be inadequate.
And, it is this chaotic and intriguing field of much new ignorance
and of a little new knowledge to which communication theory
most directly applies.
If communication theory, like Newton’s laws of motion, is to be
taken seriously, it must give us useful guidance in connection with
problems of communication. It must demonstrate that it has a
real and enduring substance of understanding and power. As the
name implies, this substance should be sought in the efficient and
accurate transmission of information. The substance indeed exists.
As we have seen, it existed in an incompletely understood form
even before Shannon’s work unified it and made it intelligible.
Efficient Encoding 127
To deal with the matter of accurate transmission of information
we need new basic understanding, and this matter will be tackled
in the next chapter. The foregoing chapters have, however, put us
in a position to discuss some challenging aspects of the efficient
transmission of information.
We have seen that in the entropy of an information source
measured in bits per symbol or per second we have a measure of
the number of binary digits, of off-or-on pulses, per symbol or per
second which are necessary to transmit a message. Knowing this
number of binary digits required for encoding and transmission, we
naturally want a means of actually encoding messages with, at the
most, not many more binary digits than this minimum number.
Novices in mathematics, science, or engineering are forever demanding
infallible, universal, mechanical methods for solving
problems. Such methods are valuable in proving that problems
can be solved, but in the case of difficult problems they are seldom
practical, and they may sometimes be completely unfeasible.
As an example, we may note that an explicit solution of the general
cubic equation exists, but no one ever uses it in a practical
problem. Instead, some approximate method suited to the type or
class of cubics actually to be solved is resorted to.
The person who isn’t a novice thinks hard about a specific problem
in order to see if there isn’t some better approach than a
machine-like application of what he has been taught. Let us see
how this applies in the case of information theory. We will first
consider the case of a discrete source which produces a string of
symbols or characters.
In Chapter V, we saw that the entropy of a source can be computed
by examining the relative probabilities of occurrence of
various long blocks of characters. As the length of the block is
increased, the approximation to the entropy gets closer and closer.
In a particular case, perhaps blocks 5, or 10, or 100 characters in
length might be required to give a very good approximation to
the entropy.
We also saw that by dividing the message into successive blocks
of characters, to each of which a probability of occurrence can be
attached, and by encoding these blocks into binary digits by means
128 Symbols, Signals and Noise
of the Huffman code, the number of digits used per character
approaches the entropy as the blocks of characters are made longer
and longer.
Here indeed is our foolproof mechanical scheme. Why don’t we
simply use it in all cases?
To see one reason, let us examine a very simple case. Suppose
that an information source produces a binary digit, a 1 or a 0,
randomly and with equal probability and then follows it with the
same digit twice again before producing independently another
digit. The message produced by such a source might be:
0 0 0 1 1 1 0 0 0 1 1 1 1 1 1 0 0 0 0 0 0 1 1 1
Would anyone be foolish enough to divide such a message
successively into blocks of 1, 2, 3, 4, 5, etc., characters, compute
the probabilities of the blocks, encode them with a Huffman code,
and note the improvement in the number of binary digits required
for transmission? I don’t know; it sometimes seems to me that there
are no limits to human folly.
Clearly, a much simpler procedure is not only adequate but
absolutely perfect. Because of the repetition, the entropy is clearly
the same as for a succession of a third as many binary digits chosen
randomly and independently with equal probability of 1 or 0. That
is, it is Vs binary digit per character of the repetitious message. And,
we can transmit the message perfectly efficiently simply by sending
every third character and telling the recipient to write down each
received character three times.
This example is simple but important. It illustrates the fact that
we should look for natural structure in a message source, for salient
features of which we can take advantage.
The discussion of English text in Chapter IV illustrates this. We
might, for instance, transmit text merely as a picture by television
or facsimile. This would take many binary digits per character. We
would be providing a transmission system capable of sending not
only English text, but Cyrillic, Greek, Sanskrit, Chinese, and other
text, and pictures of landscapes, storms, earthquakes, and Marilyn
Monroe as well. We would not be taking advantage of the elementary
and all-important fact that English text is made up of letters.
If we encode English text letter by letter, taking no account of
Efficient Encoding 129
the different probabilities of various letters (and excluding the
space), we need 4.7 binary digits per letter. If we take into account
the relative probabilities of letters, as Morse did, we need 4.14
binary digits per letter.
If we proceeded mechanically to encode English text more
efficiently, we might go on to encoding pairs of letters, sequences
of three letters, and so on. This, however, would provide for
encoding many sequences of letters which aren’t English words. It
seems much more sensible to go on to the next larger unit of
English text, the word. We have seen in Chapter IV that we would
expect to use only about 9 binary digits per word or 1.7 binary
digits per character in so encoding English text.
If we want to proceed further, the next logical step would be to
consider the structure of phrases or sentences; that is, to take
advantage of the rules of grammar. The trouble is that we don’t
know the rules of grammar completely enough to help us, and if
we did, a communication system which made use of these rules
would probably be impractically complicated. Indeed, in practical
cases it still seems best to encode the letters of English text independently,
using at least 5 binary digits per character.
It is, however, important to get some idea of what could be
accomplished in transmitting English text. To this end, Shannon
considered the following communication situation. Suppose we ask
a man, using all his knowledge of English, to guess what the next
character in some English text is. If he is right we tell him so, and
he writes the character down. If he is wrong, we may either tell
him what the character actually is or let him make further guesses
until he guesses the right character.
Now, suppose that we regard this process as taking place at the
transmitter, and say that we have an absolutely identical twin to
guess for us at the receiver, a twin who makes just the same mistakes
that the man at the transmitter does. Then, to transmit the
text, we let the man at the receiver guess. When the man at the
transmitter guesses right, so will the man at the receiver. Thus, we
need send information to the man at the receiver only when the
man at the transmitter guesses wrong and then only enough information
to enable the men at the transmitter and the receiver to
write down the right character.
130 Symbols, Signals and Noise
Shannon has drawn a diagram of such a communication system,
which is shown in Figure VII-1. A predictor acts on the original
text. The prediction of the next letter is compared with the actual
letter. If an error is noted, some information is transmitted. At the
receiver, a prediction of the next character is made from the already
reconstructed text. A comparison involving the received signal is
carried out. If no error has been made, the predicted character is
used; if an error has been made, the “reduced text” information
coming in will make it possible to correct the error.
Of course, we don’t have such identical twins or any other highly
effective identical predictors. Nonetheless, a much simpler but
purely mechanical system based on this diagram has been used in
transmitting pictures. Shannon’s purpose was different, however.
By using just one person, and not twins, he was able to find what
transmission rate would be required in such a system merely by
examining the errors made by the one man in the transmitter
situation. The results are summed up in Figure V-4 of Chapter V.
A better prediction is made on the basis of the 100 preceding
letters than on the basis of the preceding 10 or 15. To correct the
errors in prediction, something between 0.6 and 1.3 binary digits
per character is required. This tells us that, insofar as this result
is correct, the entropy of English text must lie between .6 and 1.3
bits per letter.
A discrete source of information provides a good example for
discussion but not an example of much practical importance in
communication. The reason is that, by modern standards of electrical
communication, it takes very few binary digits or off-or-on
pulses to send English text. We have to hurry to speak a few
hundred words a minute, yet it is easy to send over a thousand
words of text over a telephone connection in a minute or to send
10 million words a minute over a TV channel, and, in principle if
not in practice, we could transmit some 50,000 words a minute over
ORIGINAL COMPARISON COMPARISON ORIGINAL
REDUCED TEXT
Fig. V II-i
Efficient Encoding 131
a telephone channel and some 50 million words a minute over a
TV channel. As a matter of fact, in practical cases we have even
retreated from Morse’s ingenious code which sends an E faster than
a Z. A teletype system uses the same length of signal for any letter.
Efficient encoding is thus potentially more important for voice
transmission than for transmission of text, for voice takes more
binary digits per word than does text. Further, efficient encoding
is potentially more important for TV than for voice.
Now, a voice or a TV signal is inherently continuous as opposed
to English text, numbers, or binary digits, which are discrete.
Disregarding capitalization and punctuation, an English character
may be any one of the letters or the space. At a given moment, the
sound wave or the human voice may have any pressure at all lying
within some range of pressures. We have noted in Chapter IV that
if the frequencies of such a continuous signal are limited to some
bandwidth B, the signal can be accurately represented by 2B
samples or measurements of amplitude per second.
We remember, however, that the entropy per character depends
on how many values the character can assume. Since a continuous
signal can assume an infinite number of different values at a sample
point, we are led to assume that a continuous signal must have an
entropy of an infinite number of bits per sample.
This would be true if we required an absolutely accurate reproduction
of the continuous signal. However, signals are transmitted
to be heard or seen. Only a certain degree of fidelity of reproduction
is required. Thus, in dealing with the samples which specify
continuous signals, Shannon introduces a fidelity criterion. To
reproduce the signal in a way meeting the fidelity criterion requires
only a finite number of binary digits per sample or per second, and
hence we can say that, within the accuracy imposed by a particular
fidelity criterion, the entropy of a continuous source has a particular
value in bits per sample or bits per second.
It is extremely important to realize that the fidelity criterion
should be associated with long stretches of the signal, not with
individual samples. For instance, in transmitting a sound, if we
make each sample 10 per cent larger, we will merely make the
sound louder, and no damage will be done to its quality. If we make
a random error of 10 per cent in each sample, the recovered signal
132
Symbols, Signals and Noise
will be very noisy. Similarly, in picture transmission an error in
brightness or contrast which changes smoothly and gradually
across the picture will pass unnoticed, but an equal but random
error differing from point to point will be intolerable.
We have seen that we can send a continuous signal by quantizing
each sample, that is, by allowing it to assume only certain preassigned
values. It appears that 128 values are sufficient for the
transmission of telephone-quality speech or of pictures. We must
realize, however, that, in quantizing a speech signal or a picture
signal sample by sample, we are proceeding in a very unsophisticated
manner, just as we are if we encode text letter by letter rather
than word by word.
The name hyperquantization has been given to the quantization
of continuous signals of more than one sample at a time. This is
undoubtedly the true road to efficient encoding of continuous
signals. One can easily ruin his chances of efficient encoding completely
by quantizing the samples at the start. Yet, to hyperquantize
a continuous signal is not easy. Samples are quantized independently
in present pulse code modulation systems that carry telephone
conversations from telephone office to telephone office and from
town to town, and in the digital switching systems that provide
much long distance switching. Samples are quantized independently
in sending pictures back from Mars, Jupiter and farther
planets.
In pulse code modulation, the nearest of one of a number of
standard levels or amplitudes is assigned to each sample. As an
example, if eight levels were used, they might be equally spaced
as in a of Figure VII-2. The level representing the sample is then
transmitted by sending the binary number written to the right of it.
Some subtlety of encoding can be used even in such a system.
Instead of the equally spaced amplitudes of Figure VII-2u, we can
use quantization levels which are close together for small signals
and farther apart for large signals, as shown in Figure VII-2b. The
reason for doing this is, of course, that our ears are sensitive to a
fractional error in signal amplitude rather than to an error of so
many dynes below or above average pressure or so many volts
positive or negative, in the signal. By such companding (compressing
the high amplitudes at the transmitter and expanding them again
Efficient Encoding 133
111
UJ
o
o
2
<
t
1
11 1
11 0
101
1 00
011
01 0
001
000
1 1 0
1 01
1 oo
ZERO
___
011 AMPLITUDE
01 0
001
000
(a) (b)
Fig. VII-2
at the receiver), 7 binary digits per sample can give a signal almost
as good as 11 binary digits would if the signal levels transmitted
were separated by equal differences in amplitude.
To send speech more efficiently than this, we need to examine
the characteristics both of speech and of hearing. After all, we
require only enough accuracy of transmission to convince the
hearer that transmission is good enough.
Efficiency is not everything. A vocoder can transmit only one
voice, not two or more at a time. Also, vocoders behave badly
when one speaks in the presence of loud noise. Trying to transmit
the actual speech waveform more efficiently, or waveform decoding,
avoids these problems, but 15,000-20,000 binary digits per second
are required for acceptable speech.
Figure VII-3 shows the wave forms of several speech sounds,
that is, how the pressure of the sound wave or the voltage representing
it in a communication system varies with time. We see that
many of the wave forms, and especially those for the vowels (a
through d), repeat over and over almost exactly. Couldn’t we
perhaps transmit just one complete period of variation and use it
134 Symbols, Signals and Noise
o 0.01 0.02
TIME IN SECONDS
Fig. VII-3
Efficient Encoding 135
to replace several succeeding periods? This is very difficult, for it
is hard for a machine to determine just how long a period is in
actual speech. It has been tried. The speech reproduced is intelligible
but seriously distorted.
If speech is to be encoded efficiently, a much more fundamental
approach is required. We must know how great a variety of speech
sounds must be transmitted and how effective our sense of hearing
is in distinguishing among speech sounds.
The fluctuations of air pressure which constitute the sounds of
speech are very rapid indeed, of the order of thousands per second.
Our voluntary control over our vocal tracts is exercised at a much
lower rate. At the most, we change the manner of production of
sounds a few tens of times a second. Thus, speech may well be
(and is) simpler than we might conclude by examining the rapidly
fluctuating sound waves of speech.
What control do we exercise over our vocal organs? First of all,
we control the production of voiced sounds by our control over our
vocal cords. These are two lips or folds of muscular tissue attached
to a cartilaginous box called the larynx, which is prominent in man
as the Adam’s apple. When we are not giving voice to sound, these
are wide open. They can be drawn together more or less tightly,
so that when air from the lungs is forced through them they emit
a sound something like a Bronx cheer. If they are held very tight,
the sound has a high pitch; if they are more relaxed, the sound has
a lower pitch.
The pulses of air passing the vocal cords contain many frequencies.
The mouth and lips act as a complex resonator which emphasizes
certain frequencies more than others. What frequencies are
emphasized depends on how much and at what position the tongue
is raised or humped in the mouth, on whether the soft palate opens
the nasal cavities to the mouth and throat, and on the opening of
the jaws and the position of the lips.
Particular sounds of voiced speech, which includes vowels and
other continuants, such as m and r, are formed by exciting the vocal
cords and giving particular characteristic shapes to the mouth.
Stop consonants, or plosives, such as p, b, g, t, are formed by
stopping off the vocal passage at various points with the tongue
or lips, creating an air pressure, and suddenly releasing it. The vocal
136
Symbols, Signals and Noise
cords are used in producing some of these sounds (b, for instance)
and not in producing others (p, for instance).
Fricatives, such as s and sh, are produced by the passage of air
through various constrictions. Sometimes the vocal cords are used
as well (in a zh sound, as in azure).
A specification of the movements of the vocal organs would be
much more slowly changing than a description of the sound produced.
May this not be a clue to efficient encoding of speech?
In the early thirties, long before Shannon’s work on information
theory, Homer Dudley of the Bell Laboratories invented such a
form of speech transmission, which he called the vocoder (from
voice coder). The transmitting (analyzer) and receiving (synthesizer)
units of a vocoder are illustrated in Figure VII-4.
In the analyzer, an electrical replica of the speech is fed to 16
filters, each of which determines the strength of the speech signal
in a particular band of frequencies and transmits a signal to the
synthesizer which gives this information. In addition, an analysis
is made to determine whether the sound is voiceless (s, f) or voiced
(o, u) and, if voiced, what the pitch is.
At the synthesizer, if the sound is voiceless, a hissing noise is
produced; if the sound is voiced a sequence of electrical pulses is
produced at the proper rate, corresponding to the puffs of air
passing the vocal cords of the speaker.
The hiss or pulses are fed to an array of filters, each passing a
band of frequencies corresponding to a particular filter in the
analyzer. The amount of sound passing through a particular filter
in the synthesizer is controlled by the output of the corresponding
analyzer filter so as to be the same as that which the analyzer filter
indicates to be present in the voice in that frequency range.
This process results in the reproduction of intelligible speech.
In effect, the analyzer listens to and analyzes speech, and then
instructs the synthesizer, which is an artificial speaking machine,
how to say the words all over again with the very pitch and accent
of the speaker.
Most vocoders have a strong and unpleasant electrical accent.
The study of this has led to new and important ideas concerning
what determines and influences speech quality; we cannot afford
time to go into this matter here. Even imperfect vocoders can be
ANALYZER
BAND FILTERS AND
ENERGY-MEASURING DEVICES
SYNTHESIZER
CONTROL OF ENERGY- EXCITATION
DISTRIBUTION SIGNALS , SIGNAL
COMBINING
EQUIPMENT
Efficient Encoding
Fig VII-4
u> ~~l
138
Symbols, Signals and Noise
very useful. For instance, it is sometimes necessary to resort to
enciphered speech transmission. If one merely directly reduces
speech to binary digits by pulse code modulation, 30,000 to 60,000
binary digits per second must be sent. By using a vocoder, speech
can be sent with about 2,400 binary digits per second.
The channel vocoder of Figure VII-4 is only one example of a
large class of devices (we may call them all vocoders, if we wish)
that analyze speech and transmit signals which drive a speaking
machine. In linear predictive encoding the analysis finds slowly
varying coefficients that predict the next speech sample as a
weighted sum of several past samples. An error signal can be sent
as well, which is used to correct the output of the speaking machine.
Linear predictive coding gives very good speech if 9,600 binary
digits per second are transmitted, intelligible speech at 2,400 binary
digits per second, and barely intelligible speech at 600 binary
digits per second.
Various other parameters of speech can be derived from the
linear predictive coefficients. The channel signals characteristic
of the channel vocoder of Figure VII-4 can be derived from the
linear predictive coefficients. So can the resonant frequencies of
the vocal tract characteristic of various speech sounds. These
resonant frequencies are called formants. When we transmit these
resonant frequencies and use them to reconstruct speech we say
we have a formant tracking vocoder. It has been proposed to
derive parameters describing the shape of the vocal tract and to
transmit these. If, only if, we could use the coefficients to recognize
speech sounds, or phonemes, and merely transmit their labels, we
would have a phoneme vocoder that would transmit speech with
the efficiency of text.
Let us consider the vocoder for a moment before leaving it.
We note that transmission of voice using even the most economical
of vocoders takes many more binary digits per word than
transmission of English text. Partly, this is because of the technical
difficulties of analyzing and encoding speech as opposed to print.
Partly, it is because, in the case of speech, we are actually transmitting
information about speech quality, pitch, and stress, and
accent as well as such information as there is in text. In other
Efficient Encoding 139
words, the entropy of speech is somewhat greater per word than
the entropy of text.
That the vocoder does encode speech more efficiently than other
methods depends on the fact that the configuration of the vocal
tract changes less rapidly than the fluctuations of the sound waves
which the vocal tract produces. Its effectiveness also depends on
limitations of the human sense of hearing.
From an electrical point of view, the most complicated speech
sounds are the hissing fricatives, such as sh (f of Figure VII-3) and
s (g of Figure VII-3). Furthermore, the wave forms of two s’s
uttered successively may have quite a different sequence of ups and
downs. It would take many binary digits per second 10 transmit
each in full detail. But, to the ear, one s sounds just like another
if it has in a broad way the same frequency content. Thus, the
vocoder doesn’t have to reproduce the s sound the speaker uttered;
it has merely to reproduce an s sound that has roughly the same
frequency content and hence sounds the same.
We see that, in transmitting speech, the royal road to efficient
encoding appears to be the detection of certain simple and important
patterns and their recreation at the receiving end. Because of
the greater channel capacity required, efficient encoding is even
more important in TV transmission than in speech transmission.
Can we perhaps apply a similar principle in TV?
The TV problem is much more difficult than the speech transmission
problem. Partly, this is because the sense of sight is inherently
more detailed and discriminating than the sense of hearing.
Partly, though, it is because many sorts of pictures from many
sources are transmitted by TV, while speech is all produced by the
same sort of vocal apparatus.
In the face of these facts, is some vocoder-like way of transmitting
pictures possible if we confine ourselves to one sort of
picture source, for instance, the human face?
One can conceive of such a thing. Imagine that we had at the
receiver a sort of rubbery model of a human face. Or we might have
a description of such a model stored in the memory of a huge
electronic computer. First, the transmitter would have to look at
the face to be transmitted and “make up” the model at the receiver
in shape and tint. The transmitter would also have to note the
140
Symbols, Signals and Noise
sources of light and reproduce these in intensity and direction at
the receiver. Then, as the person before the transmitter talked, the
transmitter would have to follow the movements of his eyes, lips
and jaws, and other muscular movements and transmit these so
that the model at the receiver could do likewise. Such a scheme
might be very effective, and it could become an important invention
if anyone could specify a useful way of carrying out the
operations I have described. Alas, how much easier it is to say what
one would like to do (whether it be making such an invention,
composing Beethoven’s tenth symphony, or painting a masterpiece
on an assigned subject) than it is to do it.
In our day of unlimited science and technology, people’s unfulfilled
aspirations have become so important to them that a special
word, popular in the press, has been coined to denote such dreams.
That word is breakthrough. More rarely, it may also be used to
describe something, usually trivial, which has actually been
accomplished.
If we turn from such dreams of the future, we find that all actual
picture-transmission systems follow a common pattern. The picture
or image to be transmitted is scanned to discover the brightness at
successive points. The scanning is carried out along a sequence of
closely spaced lines. In color TV, three images of different colors
are scanned simultaneously. Then, at the receiver, a point of light
whose intensity varies in accord with the signal from the transmitter
paints out the picture in light and shade, following the same line
pattern. So far all practical attempts at efficient encoding have
started out with the signal generated by such a scanning process.
The outstanding efficient encoding scheme is that used in color
TV. The brightness of a color TV picture has very fine detail; the
pattern of color has very much less detail. Thus, color TV of almost
the same detail as monochrome TV can be sent over the same
channel as is used for monochrome. Of course, color TV uses an
analog signal; the picture is not reduced to discrete on-or-off pulses.
Increasingly, pulse code modulation will be used to transmit all
sorts of signals, including television signals. The picture to be transmitted
will be scanned in a conventional way, but its brightness
will be encoded as a succession of binary numbers that specify
the brightnesses of a succession of discrete picture elements or
Efficient Encoding 141
pixels that lie along each scanning line. This is how pictures were
sent back from Mars by the Mariner lander, and from Jupiter and
its moons by the Voyager spacecraft.
All recent work aimed at encoding television efficiently is digital.
It deals with successions of binary numbers that represent successive
pixel brightnesses.
In large parts of a TV picture the brightness changes gradually
and smoothly from pixel to pixel. In such areas of the picture, a
good prediction can be made of the brightness of the next pixel
from the brightness of preceding pixels in the same line, and
perhaps in the preceding line. At the receiver we need know only
the error in such a prediction, so we need transmit only the small
difference between the true brightness and a brightness which we
predict at the receiver as well as at the transmitter. Of course, in
“busy” portions of the picture, prediction will be poor, and the
brightness difference that must be sent will be great.
We can transmit brightness differences most efficiently by using
a Huffman code, with short code words for more frequently occurring
small brightness differences and long code words for less
frequently occurring, large brightness differences. If we do this,
the binary digits of the coded differences will be generated at an
uneven rate, at a slow rate when smooth portions of the picture are
scanned and at a faster rate when busy portions of the picture are
scanned. In order to transmit the binary digits at a constant rate,
the digits must be fed into a buffer, which stores the incoming
digits and feeds them out at a constant rate equal to the average
rate at which they come in. A similar buffer must be used at the
receiving end.
By means of such intraframe encoding, the number of binary
digits per second needed to transmit a good TV picture can be
reduced to Vi to V3 of the number of binary digits used in initially
encoding the pixel brightnesses.
Much greater gains can be made through interframe encoding, in
which the pixel brightnesses of the whole previous TV picture are
stored and used in predicting the brightness of the next pixel to
be sent. This is particularly effective in transmitting pictures of
people against a fixed background, for the brightnesses of pixels in
the background don’t change from frame to frame.
142
Symbols, Signals and Noise
Even more elaborate experimental schemes make use of the fact
that when a figure in front of a background moves, it moves as a
whole. Thus, the brightnesses of the pixels in the moving figure can
be predicted from the brightnesses of pixels which are a constant
distance away in the previous frame.
If each pixel of a TV picture is represented by 8 binary digits (a
very good picture), the picture can be transmitted by sending
around 100 million binary digits per second. By intraframe encoding
this can be reduced to perhaps 32 million. With interframe
coding this has been reduced to as little as 6 million. A reduction
to 1.5 million seems conceivable for such pictures as the head of a
person against a fixed background.
The transform method is another approach to the efficient transmission
of TV pictures. In the transform method, the pattern of
pixel brightnesses that make up the TV picture, or some portion of
it, is represented as the sum of a chosen set of standardized patterns
whose amplitudes are transmitted with chosen accuracies.
Reviewing what has been said, we see that there are three important
principles in encoding signals efficiently: (1) Don’t encode
the signal one sample or one character at a time; encode a considerable
stretch of a signal at a time (hyperquantization); (2) take
into account the limitations on the source of the signal; (3) take
into account any inabilities of the eye or the ear to detect errors
in a reconstruction of the signal.
The vocoder illustrates these principles excellently. The fine
temporal structure of the speech wave is not examined in detail.
Instead, a description specifying the average intensities over certain
ranges of frequencies is transmitted, together with a signal which
tells whether the speech is voiced or unvoiced and, if it is voiced,
what its pitch is. This description of a signal is efficient because the
vocal organs don’t change position rapidly in producing speech.
At the receiver, the vocoder generates a speech signal which doesn’t
resemble the original speech signal in fine detail but sounds like
the original speech signal, because of the natural limitations of
our hearing.
The vocoder is a sort of paragon of efficient transmission
devices. Next perhaps comes color TV, in which the variations of
Efficient Encoding 143
color over the picture are defined much less sharply than variations
of intensity are. This takes advantage of the eyes’ inability to see
fine detail in color patterns.
Beyond this, the present art of communication has had to make
use of means which, because they do not encode long stretches of
signal at a time, must, according to communication theory, be
rather inefficient.
Still, efficient encoding is potentially important. This is especially
so in the case of the transmission of relatively broad-band signals
(TV or even voice signals) over very expensive circuits, such as
transoceanic telephone cables.
No doubt much ingenuity will be spent in efficient encoding in
the future, and many startling results will be attained. But we
should perhaps beware of going too far.
Imagine, for instance, that we send English text letter by letter.
If we make an error in sending a few letters we can still make some
sense out of the text:
Hore I hove reploced a few vowols by o.
We can even replace the vowels by x’s and read with some
facility:
Hxrx X hxvx rxplxcxd thx vxwxls bx x.
It is more efficient to encode English text word by word. In this
case, if an error is made in transmission, we are not tipped off by
finding a misspelled word. Instead, one word is replaced by
another. This might have embarrassing results. Suppose it changed
“The President is a good Republican” to “The President is a good
Communist” (or donkey, or poltroon, or many other nouns).
We might still detect an error by the fact that the word was
inappropriate. But suppose we used a more refined encoding
scheme that could reproduce grammatical utterances only. Then
we would have little chance of detecting an error in transmission.
English text, and most other information sources are redundant
in that the messages they produce give many clues to the recipient.
A few errors caused by replacing one letter by another don’t
destroy the message because we can infer it from other letters
which are transmitted correctly. Indeed, it is only because of this
redundancy that anyone can read my handwriting. When a continuous
signal is sent a sample at a time, a few errors in sample
144
Symbols, Signals and Noise
amplitude result in a few clicks in sound transmission or in a few
specks in picture transmission.
Our ideal so far has been to remove this redundancy, so that we
transmit the absolutely minimum number of clues by means of
which the message can be reconstructed. But we see that if we do
this with perfect success, any error in transmission will send, not
a distorted message, but a false and misleading message. If we fall
a little short of the ideal, an error may produce merely a terrible
garble.
We all know that there is some noise in electrical communication
—a hiss in the background on radio and a little snow at least in
TV. That such noise is an inevitable fact of nature we must accept.
Is this going to vitiate in principle our grand plan to encode die
messages from a signal source into scarcely more binary digits than
the entropy of the source?
This is the subject that we will consider in the next chapter.
CHAPTER VIII The Noisy
Channel
It is hard to put oneself in the place of another, and,
especially, it is hard to put oneself in the place of a person of an
earlier day. What would a Victorian have thought of present-day
dress? Were Newton’s laws of motion and of gravitation as astonishing
and disturbing to his contemporaries as Einstein’s theory
of relativity appears to have been to his? And what is disturbing
about relativity? Present-day students accept it, not only without
a murmur, but with a feeling of inevitability, as if any other idea
must be very odd, surprising, and inexplicable.
Partly, this is because our attitudes are bred of our times and
surroundings. Partly, in the case of science at least, it is because
ideas come into being as a response to new or better-phrased
questions. We remember that according to Plato, Socrates drew a
geometrical proof from a slave simply by means of an ingenious
sequence of questions. Those who have not seriously asked themselves
a particular question are not likely to have come upon the
proper answer, and, sometimes, when the question is phrased with
the answer in mind, the answer appears to be obvious.
Those interested in communication have been aware from the
very beginning that communication circuits or channels are imperfect.
In telephony and radio, we hear the desired signal against
a background of noise, which may be strong or faint and which
may vary in quality from the crackling of static to a steady hiss.
145
146
Symbols, Signals and Noise
In TV, the picture is overlaid faintly or strongly with an ever-
changing granular “snow.” In teletypewriter transmission, the
received character may occasionally differ from that transmitted.
Suppose that one had questioned a communication engineer
about this general problem of “noise” in 1945. One might have
asked, “What can one do about noise?” The engineer might have
answered, “You can increase the transmitter power or make the
receiver less noisy. And be sure that the receiver is insensitive to
disturbances with frequencies other than the signal frequencies.”
One might have persisted, “Can’t one do anything else?” The
engineer might have answered, “Well, by using frequency modulation,
which takes a very large band width, one can reduce the
effect of noise.”
Suppose, however, that one had asked, “In teletypewriter systems,
noise may cause some received characters to be wrong; how
can one guard against this?” The engineer could and might perhaps
have answered, “I know that if I use five off-or-on pulses to represent
a decimal digit and assign to the decimal digits only such
sequences as all have two ons and three offs, I can often tell when
an error has been made in transmission, for when errors are made
the received sequence may have other than 2 ons.”
One might have pursued the matter further with, “If the teletypewriter
circuit does cause errors is there any way that one can get
the correct message to the destination?” The engineer might have
answered, “I suppose you can if you repeat it enough times, but
that’s very wasteful. You’d better fix the circuit.”
Here we are getting pretty close to questions that just hadn’t
been asked before Shannon asked them. Nonetheless, let us go on
and imagine that one had said, “Suppose that I told you that by
properly encoding my message, I can send it over even a noisy
channel with a completely negligible fraction of errors, a fraction
smaller than any assignable value. Suppose that I told you that, if
the sort of noise in the channel is known and if its magnitude is
known, I can calculate just how many characters I can send over
the channel per second and that, if I send any number fewer than
this, I can do so virtually without error, while if I try to send more,
I will be bound to make errors.”
The engineer might well have answered, “You’d sure have to
The Noisy Charnel
147
show me. I never thought of things in quite that way before, but
what you say seems extremely improbable. Why, every time the
noise increases, the error rate increases. Of course, repeating a
message several times does work better when there aren’t too many
errors. But, it is always very costly. Maybe there’s something in
what you say, but I’d be awfully surprised if there was. Still, the
way you put it...”
Whatever we may imagine concerning an engineer benighted in
the days of error, mathematicians and engineers who have survived
the transition all feel that Shannon’s results concerning the transmission
of information over a noisy channel were and still are very
surprising. Yet I have known an intelligent layman to see nothing
remarkable in Shannon’s results. What is one to think of this?
Perhaps the best course is merely to describe and explain the
problem of the noisy channel as we now understand it, raising and
answering questions that, however natural and inevitable they now
seem, belong in their trend and content to the post-Shannon era.
The reader can be surprised or not as he chooses.
So far we have discussed both simple and complex means for
encoding text and numbers for efficient transmission. We have
noted further that any electrical signal of limited band width W
can be represented by 2W amplitudes or samples per second,
measured or taken at intervals 1 /2 W seconds apart. We have seen
that, by means of pulse code modulation, we can use some number,
around 7, of binary digits to represent adequately the amplitude
of any sample. Thus, by using pulse code modulation or some
more complicated and more efficient scheme, we can transmit
speech or picture signals by means of a sequence of binary digits
or off-or-on or positive-or-negative pulses of current.
All of this works perfectly if the recipient of the message receives
the same signal that the sender transmits. The actual facts are different.
Sometimes he receives a 0 when a 1 is transmitted, and
sometimes he receives a 1 when a 0 is transmitted. This can happen
through the malfunction of electrical relays in a slow-speed
telegraph circuit or through the malfunction of vacuum tubes or
transistors in a higher speed circuit. It can also happen because of
interfering signals or noise, either noise from man-made apparatus,
or noise from magnetic storms.
148
Symbols, Signals and Noise
We can easily see in a simple case how errors can occur because
of the admixture of noise with a signal. Imagine that we want to
send a large number of binary digits, 0 or 1, per second over a wire
by means of an electrical signal. We may represent the signal conveying
these digits by the succession of samples s of Figure VIII-1,
each of which will be + 1 or — 1. Here we have a succession of
positive and negative voltages which represent the digits 10 11
10 0 10.
Now suppose a random noise voltage, which may be either
positive or negative, is added to the signal. We can represent this
also by a number of noise samples n of Figure VIII-1 taken simultaneously
with the signal samples. The signal plus the noise is
obtained by adding the signal and the noise samples and is shown
as $ + n in Figure VIII-1.
If we interpret a positive signal-plus-noise in the received message
as a 1 and a negative signal-plus-noise as a 0, then the received
1 0 1 1 1 0 0 1 0
I ' " I I ' I
s+n
r
ERRORS
POSITION
1
X
2
O 1 1
X
3 4 5
Fig. V lll-1
0 1 1 0
X
6 7 8 9
The Noisy Channel
149
message will be represented by the digits r of Figure VIII-1. Thus,
errors in transmission, as indicated, occur in positions 2, 3, and 7.
The effect of such errors in transmission can range from annoying
to dangerous. In speech or picture transmission by means of
simple coding schemes, they result in clicks, hissing noises, or
“snow.” If more efficient, block encoding schemes are used (hyperquantization)
the effects of errors will be more pronounced. In
general, however, we may expect the most dangerous effects of
errors in the transmission of text.
In the transmission of English text by conventional means, errors
merely put a wrong letter in here and there. The text is so redundant
that we catch such errors by eye. However, when type is set
remotely by teletypewriter signals, as it is, for instance, in the
simultaneous printing of news magazines in several parts of the
country, even errors of this sort can be costly.
When numbers are sent errors are much more serious. An error
might change $1,000 into $9,000. If the error occurred in a program
intended to make an electronic computer carry out a complicated
calculation, the error could easily cause the whole calculation
to be meaningless.
Further, we have seen that, if we encode English text or any other
signal very efficiently, so as largely to remove the redundancy, an
error can cause a gross change in the meaning of the received
signal.
When errors are very important to us, how indeed may we guard
against them? One way would be to send every letter twice or to
send every binary digit used in transmitting a letter or a number
twice. Thus, in transmitting the binary sequence 10 10 0 1 10 1,
we might send and receive as follows:
sent 1 1 0 0 1 1 0 0 0 0 1 1 1 1 0 0 1 1
received 1 1 0 0 1 1 0 0 0 1 1 1 1 1 0 0 1 1
X
error
For a given rate of sending binary digits, this will cut our rate of
transmitting information in half, for we have to pause and retransmit
every digit. However, we can now see from the received signal
than an error has occurred at the marked point, because instead
150
Symbols, Signals and Noise
of a pair of like digits, 0 0 or 1 1, we have received a pair of unlike
digits, 0 1. We don’t know whether the correct, transmitted pair
was 0 0 or 1 1. We have detected the error, but we have not
corrected it.
If errors aren’t too frequent, that is, if the chance of two errors
occurring in the transmission of three successive digits is negligible,
we can correct as well as detect an error by transmitting each digit
three times, as follows:
sent 1 1 1 0 0 0 1 1 1 0 0 0 0 0 0 1 1 1 1 1 1
received 1 1 1 0 0 0 1 0 1 0 0 0 0 0 0 1 1 1 1 1 1
A
error
We have now cut our rate of transmission to one-third, because we
have to pause and retransmit each digit twice. However, we can
now correct the error indicated by the fact that the digits in the
indicated group 10 1 are not all the same. If we assume that there
was only one error in the transmission of this group of digits, then
the transmitted group must have been 111, representing 1, rather
than 0 0 0, representing 0.
We see that a very simple scheme of repeating transmitted digits
can detect or even correct infrequent errors of transmission. But
how costly it is! If we use this means of error correction or detection,
even when almost all of the transmitted digits are correct we
have to cut our rate of transmission in half by repeating digits in
order just to detect errors, and we have to cut our rate of transmission
to one-third by transmitting each digit three times in order
to get error correction. Moreover, these schemes won’t work if
errors are frequent enough so that more than one will sometimes
occur in the transmission of two or three digits.
Clearly, this simple approach will never lead to a sound understanding
of the possibility of error correction. What is required is
a deep and powerful mathematical attack. This is just what Shannon
provided in discovering and proving his fundamental theorem
for the noisy channel. It is the course of his reasoning that we are
about to follow.
In formulating an abstract and general model of noise or errors,
we will deal with the case of a discrete communication system
The Noisy Channel
151
which transmits some group of characters, such as the digits from
0 to 9 or the letters of the alphabet. For convenience, let us consider
a system for transmitting the digits 0 through 9. This is illustrated
in Figure VIII-2. At the left we have a number of little circles
labeled with the digits; we may regard these little circles as pushbuttons.
To ‘he right we have a number of little circles, again
labeled with the digits. We may regard these as lights. When we
push a digit button at the transmitter to the left, some digit light
lights up at the receiver to the right.
If our communication system were noiseless, pushing the 0
button would always light the 0 light, pushing the 1 button would
always light the 1 light, and so on. However, in an imperfect or
noisy communication system, pushing the 4 button, for instance,
may light the 0 light, or the 1 light, or the 2 light, or any other light,
as shown by the lines radiating from the 4 button in Figure VIII-2.
In a simple, noisy communication system, we can say that when
we press a button the light which lights is a matter of chance,
0
X
2
3
4
5
6
7
8
9
Fig. VIII-2
152
Symbols, Signals and Noise
independent of what has gone before and that, if the 4 button is
pressed, there is some probability /?4(6) that the 6 light will light,
and so on.
If the sender can’t be sure which light will light when he presses
a particular button, then the recipient of the message can’t be sure
which button was pressed when a particular light lights. This is
indicated by the arrows from light 6 to various buttons on the left.
If, for instance, light 6 lights, there is some probability p§ (4) that
button 4 was pressed, and so on. Only for a noiseless system will
p§((>) be unity and />6(4), pe(9), etc., be zero.
The diagram of Figure VIII-2 would be too complicated if all
possible arrows were put in, and the number of probabilities is too
great to list, but I believe that the general idea of the degree and
nature of uncertainty of the character received when the sender
tries to send a particular character and the uncertainty of the
character sent when the recipient receives a particular character,
have been illustrated. Let us now consider this noisy communication
channel in a rather general way. In doing so we will represent
by x all of the characters sent and by 7 all of the characters received.
The characters x are just the characters generated by the message
source from which the message comes. If there are m of these
characters and if they occur independently with probabilities p(x),
then we know from Chapter V that the entropy H(x) of the message
source, the rate at which the message source generates information,
must be m
H(x) = -/>(*) logp(x) (8.1)
x= 1
We can regard the output of the device, which we designate by
y, as another message source. The number of lights need not be
equal to the number of buttons, but we will assume that it is, so
that there are m lights. The entropy of the output will be
m
H (y) = 2 - p W 1 o § p(y) (8-2)
7=1
We note that while H(x) depends only on the input to the communication
channel, H(y) depends both on the input to the channel
and on the errors made in transmission. Thus, the probability of
The Noisy Channel
153
receiving a 4 if nothing but a 4 is ever sent is different from the
probability of receiving a 4 if transmitting buttons are pressed
at random.
If we imagine that we can see both the transmitter and the
receiver, we can observe how often certain combinations of x and
y occur; say, how often 4 is sent and 6 is received. Or, knowing
the statistics of the message source and the statistics of the noisy
channel, we can compute such probabilities. From these we can
compute another entropy.
m m
H(x,y) = 2 2 - p(x,y) log p(x,y) (8.3)
x = 1 x = 1
This is the uncertainty of the combination of x and y.
Further, we can say, suppose that we know x (that is, we know
what key was pressed). What are the probabilities of various lights
lighting (as illustrated by the arrows to the right in Figure VIII-2)?
This leads to an entropy,
m m
Hx(y) = 2 P(x)P*(y) logM /) (8-4)
x = ly = 1
This is a conditional entropy of uncertainty. Its form is reminiscent
of the entropy of a finite-state machine. As in that case, we
multiply the uncertainty for a given condition (state, value of x)
by the probability that that condition (state, value of x) will occur
and sum over all conditions (states, values of x).
Finally, suppose we know what light lights. We can say what the
probabilities are that various buttons were pressed. This leads to
another conditional entropy
m m
h ,(x ) = 2 2 -p{y)Ptix)\o%pt(x) (8.5)
y= \x = \
This is the sum overly of the probability thaty is received times
the uncertainty that x is sent when y is received.
These conditional entropies depend on the statistics of the
message source, because they depend on how often x is transmitted
or how often y is received, as well as on the errors made in
transmission.
154 Symbols, Signals and Noise
The entropies listed above are best interpreted as uncertainties
involving the characters generated by the message source and the
characters received by the recipient. Thus:
H(x) is the uncertainty as to x, that is, as to which character will
be transmitted.
H(y) is the uncertainty as to which character will be received
in the case of a given message source and a given communication
channel.
H(x, y) is the uncertainty as to when x will be transmitted and
y received.
Hx(y) is the uncertainty of receivings when x is transmitted. It
is the average uncertainty of the sender as to what will be received.
Hy(x) is the uncertainty that x was transmitted when y is
received. It is the average uncertainty of the message recipient as
to what was actually sent.
There are relations among these quantities:
H{x, y) = H(x) + Hx(y) (8.6)
That is, the uncertainty of sending x and receiving y is the
uncertainty of sending x plus the uncertainty of receivings when
x is sent.
H(x,y) = H(y) + Hy(x) (8.7)
That is, the uncertainty of receiving y and sending x is the
uncertainty of receivings plus the uncertainty that x was sent when
S was received.
We see that when Hx(y) is zero, Hy(x) must be zero, and H(y)
is then just H(x). This is the case of the noiseless channel, for
which the entropy of the received signal is just the same as the
entropy of the transmitted signal. The sender knows just what will
be received, and the recipient of the message knows just what
was sent.
The uncertainty as to which symbol was transmitted when a
given symbol is received, that is, Hy(x) seems a natural measure
of the information lost in transmission. Indeed, this proves to be
the case, and the quantity Hy{x) has been given a special name;
it is called the equivocation of the communication channel. If we
The Noisy Channel
155
take H{x) and Hv{x) as entropies in bits per second, the rate R of
transmission of information over the channel can be shown to be,
in bits per second,
R = H(x) - Hv(x) (8.8)
That is, the rate of transmission of information is the source rate
or entropy less the equivocation. It is the entropy of the message
as sent less the uncertainty of the recipient as to what message
was sent.
The rate is also given by
R = H(y) - Hx(y) (8.9)
That is, the rate is the entropy of the received signal y less the
uncertainty that y was received when x was sent. It is the entropy
of the message as received less the sender’s uncertainty as to what
will be received.
The rate is also given by
R = H(x) + H(y) - H(x, y) (8.10)
The rate is the entropy of x plus the entropy ofy less the uncertainty
of occurrence of the combination x and y. We will note from
8.3 that for a noiseless channel, since p(x, y) is zero except when
x = y, and H{x,y) = H(x) = H(y). The information rate is just
the entropy of the information source, H(x).
Shannon makes expression 8.8 for the rate plausible by means
of the sketch shown in Figure VIII-3. Here we assume a system in
which an observer compares transmitted and received signals and
then sends correction data by means of which the erroneous
received signal is corrected. Shannon is able to show that in order
to correct the message, the entropy of the correction signal must
be equal to the equivocation.
We see that the rate R of relation 8.8 depends both on the
channel and on the message source. How can we describe the
capacity of a noisy or imperfect channel for transmitting information?
We can choose the message source so as to make the rate R
as large as possible for a given channel. This maximum possible
rate of transmission for the channel is called the channel capacity
156 Symbols, Signals and Noise
CORRECTION DATA
Fig. VIII-3
C. Shannon’s fundamental theorem for a noisy channel involves
the channel capacity C. It says:
Let a discrete channel have a capacity C and a discrete source the
entropy per second H. If H < C there exists a coding system such that the
output of the source can be transmitted over the channel with an arbitrarily
small frequency of errors (or an arbitrarily small equivocation). If H > C
it is possible to encode the source so that the equivocation is less than
H — C + e, where e is arbitrarily small. There is no method of encoding
which gives an equivocation less than H — C.
This is a precise statement of the result which so astonished
engineers and mathematicians. As errors in transmission become
more probable, that is, as they occur more frequently, the channel
capacity as defined by Shannon gradually goes down. For instance,
if our system transmits binary digits and if some are in error, the
channel capacity C, that is, number of bits of information we can
send per binary digit transmitted, decreases. But the channel
capacity decreases gradually as the errors in transmission of digits
become more frequent. To achieve transmission with as few errors
as we may care to specify, we have to reduce our rate of transmission
so that it is equal to or less than the channel capacity.
How are we to achieve this result? We remember that in efficiently
encoding an information source, it is necessary to lump
many characters together and so to encode the message a long
block of characters at a time. In making very efficient use of a noisy
channel, it is also necessary to deal with sequences of received
The Noisy Channel
157
characters, each many characters long. Among such blocks, only
certain transmitted and received sequences of characters will occur
with other than a vanishing probability.
In proving the fundamental theorem for a noisy channel, Shannon
finds the average frequency of error for all possible codes (for
all associations of particular input blocks of characters with particular
output blocks of characters), when the codes are chosen at
random, and he then shows that when the channel capacity is
greater than the entropy of the source, the error rate averaged over
all of these encoding schemes goes to zero as the block length is
made very long. If we get this good a result by averaging over all
codes chosen at random, then there must be some one of the codes
which gives this good a result. One information theorist has characterized
this mode of proof as weird. It is certainly not the sort
of attack that would occur to an uninspired mathematician. The
problem isn’t one which would have occurred to an uninspired
mathematician, either.
The foregoing work is entirely general, and hence it applies to
all problems. I think it is illuminating, however, to return to the
example of the binary channel with errors, which we discussed
early in this chapter and which is illustrated in Figure VIII-1, and
see what Shannon’s theorem has to say about this simple and
common case.
Suppose that the probability that over this noisy channel a0 will
be received as a 0 is equal to the probability p that a 1 will be
received as a 1. Then the probability that a 1 will be received as a
0 or a 0 as a 1 must be (1 — p). Suppose further that these probabilities
do not depend on past history and do not change with
time. Then, the proper abstract representation of this situation is
a symmetric binary channel (in the manner of Figure VIII-2) as
shown in Figure VIII-4.
Because of the symmetry of this channel, the maximum information
rate, that is, the channel capacity, will be attained for a
message source such that the probability of sending a 1 is equal
to the probability of sending a zero. Thus, in the case of x (and,
because the channel is symmetrical, in the case ofy also)
p(\) = p( 0) = Vi
158
Symbols, Signals and Noise
We already know that under these circumstances
H(x) = H(y)
= - ('A log Vi + Vi log Vi)
= 1 bit per symbol
What about the conditional probabilities? What about the
equivocation, for instance, as given by 8.5? Four terms will contribute
to this conditional entropy. The sources and contributions
are:
The probability that 1 is received is Vi. When 1 is received, the
probability that 1 was sent is p and the probability that 0 was
sent is (1 — p). The contribution to the equivocation from these
events is:
Vi(-p\ogp - (1 - p ) log (1 - p) )
There is a probability of Vi that 0 is received. When 0 is received,
the probability that 0 was sent is p and the probability that 1 was
sent is (1 — p). The contribution to the equivocation from these
events is:
Vi{-p\ogp - (1 - p) log (1 - p) )
Accordingly, we see that, for the symmetrical binary channel, the
equivocation, the sum of these terms, is
Hv(x) = p logp - (1 - p ) log (1 - p)
Thus the channel capacity C of the symmetrical binary channel
is, from 8.8,
C = \ + p\ogp + (1 - p) log (1 - p)
o P
Fig. VIII-4
0
The Noisy Channel
159
We should note that this channel capacity C is just unity less the
function plotted against p in Figure V-l. We see that ifp is Vi, the
channel capacity is 0. This is natural, for in this case, if we receive
a 1, it is equally likely that a 1 or a 0 was transmitted, and the
received message does nothing to resolve our uncertainty as to
what digit the sender sent. We should also note that the channel
capacity is the same for p = 0 as for p = 1. If we consistently
receive a 0 when we transmit a 1 and a 1 when we transmit a 0,
we are just as sure of the sender’s intentions as if we always get a
1 for a 1 and a 0 for a 0.
If, on the average, 1 digit in 10 is in error, the channel capacity
is reduced to .53 of its value for errorless transmission, and for one
error in 100 digits, the channel capacity is reduced to .92 merely.
The writer would like to testify at this point that the simplicity
of the result we have obtained for the symmetrical binary channel
is in a sense misleading (it was misleading to the writer at least).
The expression for the optimum rate (channel capacity) of an
unsymmetrical binary channel in which the probability that a 1 is
received as a 1 isp and the probability that a 0 is received as a 0
is a different number q is a mess, and more complicated channels
must offer almost intractable problems.
Perhaps for this reason as well as for its practical importance,
much consideration has been given to transmission over the symmetrical
binary channel. What sort of codes are we to use in order
to attain errorless transmission over such a channel? Examples
devised by R. W. Hamming were mentioned by Shannon in his
original paper. Later, Marcel J. E. Golay published concerning
error-correcting codes in 1949, and Hamming published his work
in 1950. We should note that these codes were devised subsequent
to Shannon’s work. They might, I suppose, have been devised
before, but it was only when Shannon showed error-free transmission
to be possible that people asked, “How can we achieve it?”
We have noted that to get an efficient correction of errors, the
encoder must deal with a long sequence of message digits. As a
simple example, suppose we encode our message digits in blocks of
16 and add after each block a sequence of check digits which enable
us to detect a single error in any one of the digits, message digits
or check digits. As a particular example, consider the sequence of
160
Symbols, Signals and Noise
message digits 1 1 0 1 0 0 1 101 0 1 100 0. To find the
appropriate check digits, we write the 0’s and l’s constituting the
message digits in the 4 by 4 grid shown in Figure VIII-5. Associated
with each row and each column is a circle. In each circle is
a 0 or a 1 chosen so as to make the total number of l’s in the
column or row (including the circle as well as the squares) even.
Such added digits are called check digits. For the particular assortment
of message digits used as an example, together with the
appropriately chosen check digits, the numbers of l’s in successive
columns (left to right) and 2, 2, 2, 4, all being even numbers, and
the numbers of l’s in successive rows (top to bottom) are 4, 2,2,2,
which are again all even.
What happens if a single error is made in the transmission of a
message digit among the 16? There will be an odd number of ones
in a row and in a column. This tells us to change the message digit
where the row and column intersect.
What happens if a single error is made in a check digit? In this
case there will be an odd number of ones in a row or in a column.
We have detected an error, but we see that it was not among the
message digits.
The total number of digits transmitted for 16 message digits is
16 + 8, or 24; we have increased the number of digits needed in
the ratio 24/16, or 1.5. If we had started out with 400 message
digits, we would have needed 40 check digits and we would have
increased the number of digits needed only in the ratio of440/400,
Fig. VIII-5
The Noisy Channel
161
or 1.1. Of course, we would have been able to correct only one
error in 440 rather than one error in 24.
Codes can be devised which can be used to correct larger numbers
of errors in a block of transmitted characters. Of course, more
check digits are needed to correct more errors. A final code, however
we may devise it, will consist of some set of 2M blocks of 0’s
and l’s representing all of the blocks of digits M digits long which
we wish to transmit. If the code were not error correcting, we
could use a block just M digits long to represent each block of M
digits which we wish to transmit. We will need more digits per
block because of the error-correcting feature.
When we receive a given block of digits, we must be able to
deduce from it which block was sent despite some number n of
errors in transmission (changes of 0 to 1 or 1 to 0). A mathematician
would say that this is possible if the distance between any
two blocks of the code is at least 2n + 1.
Here distance is used in a queer sense indeed, as defined by the
mathematician for his particular purpose. In this sense, the distance
between two sequences of binary digits is the number of 0’s
or l’s that must be changed in order to convert one sequence into
the other. For instance, the distance between 0 0 10 and 1111
is 3, because we can convert one sequence into the other only by
changing three digits in one sequence or in the other.
When we make n errors in transmission, the block of digits we
receive is a distance n from the code word we sent. It may be a
distance n digits closer to some other code word. If we want to be
sure that the received block will always be nearer to the correct
code word, the one that was sent, than to any other code word,
then the distance from any code word to any other code word must
be at least In + 1.
Thus, one problem of block coding is to find 2M equal length
code words (longer than M binary digits) that are all at least a
distance 2n + 1 from one another. The code words must be as
short as possible. The codes of Hamming and Golay are efficient,
and other efficient codes have been found.
Another problem of block coding is to provide a feasible scheme
for encoding and, especially, for decoding. Simply listing code
words won’t do. The list would be too long. Encoding blocks of 20
binary digits (M = 20) requires around a million code words. And,
162
Symbols, Signals and Noise
finding the code word nearest to some received block of digits
would take far too long.
Algebraic coding theory provides means for coding and decoding
with the correction of many errors. Slepian was a pioneer in this field
and important contributors can be identified by the names of types
of algebraic codes: Reed-Solomon codes and Bose-Chaudhuri-
Hocquenghem codes provide examples. Elwin Berlekamp contributed
greatly to mathematical techniques for calculating the
nearest code word more simply.
Convolutional codes are another means of error correction. In
convolution coding, the latest M digits of the binary stream to be
sent are stored in what is called a shift register. Every time a new
binary digit comes in, 2 (or 3, or 4) are sent out by the coder. The
digits sent out are produced by what is called modulo 2 addition of
various digits stored in the shift register. (In modulo 2 addition
of binary numbers one doesn’t “carry.”)
Convolutional encoding has been traced to early ideas of Elias,
but the earliest coding and decoding scheme published is that in a
patent of D. W. Hagelbarger, filed in 1958. Convolutional decoding
really took off in 1967 when Andrew J. Viterbi invented an
optimum and simple decoding scheme called maximum likelihood
decoding.
Today, convolutional decoding is used in such valuable, noisy
communication channels as in sending pictures of Jupiter and its
satellites back from the Voyager spacecraft. Convolutional coding
is particularly valuable in such applications because Viterbi’s
maximum likelihood decoding can take advantage of the strength
as well as the sign of a received pulse.
If we receive a very small positive pulse, it is almost as likely
to be a negative pulse plus noise as it is to be a positive pulse plus
noise. But, if we receive a large positive pulse, it is much likelier
to be a positive pulse plus noise than a negative pulse plus noise.
Viterbi,decoding can take advantage of this.
Block coding is used in protecting the computer storage of vital
information. It can also be used in the transmission of binary information
over inherently low-noise data circuits.
Many existing circuits that are used to transmit data are subject
to long bursts of noise. When this is so, the most effective form of
The Noisy Channel
163
error correction is to divide the message up into long blocks of
digits and to provide foolproof error detection. If an error is detected
in a received block, retransmission of the block is requested.
Mathematicians are fascinated by the intricacies and challenges
of block coding. In the eyes of some, information theory has become
essentially algebraic coding theory. Coding theory is important
to information theory. But, in its inception, in Shannon’s
work, information theory was, as we have seen, much broader. And
even in coding itself, we must consider source coding as well as
channel coding.
In Chapter VII, we discussed ways of removing redundancy
from a message so that it could be transmitted by means of fewer
binary digits. In this chapter, we have considered the matter of
adding redundancy to a nonredundant message in order to attain
virtually error-free transmission over a noisy channel. The fact that
such error-free transmission can be attained using a noisy channel
was and is surprising to communication engineers and mathematicians,
but Shannon has proved that it is necessarily so.
Prior to receiving a message over an error-free channel, the
recipient is uncertain as to what particular message out of many
possible messages the sender will actually transmit. The amount
of the recipient’s uncertainty is the entropy or information rate of
the message source, measured in bits per symbol or per second.
The recipient’s uncertainty as to what message the message source
will send is completely resolved if he receives an exact replica of
the message transmitted.
A message may be transmitted by means of positive and negative
pulses of current. If a strong enough noise consisting of random
positive and negative pulses is added to the signal, a positive
signal pulse may be changed into a negative pulses or a negative
signal pulse may be changed into a positive pulse. When such a
noisy channel is used to transmit the message, if the sender sends
any particular symbol there is some uncertainty as to what symbol
will be received by the recipient of the message.
When the recipient receives a message over a noisy channel, he
knows what message he has received, but he cannot ordinarily be
sure what message was transmitted. Thus, his uncertainty as to
what message the sender chose is not completely resolved even on
164
Symbols, Signals and Noisethe receipt of a message. The remaining uncertainty depends on
the probability that a received symbol will be other than the
symbol transmitted.
From the sender’s point of view, the uncertainty of the recipient
as to the true message is the uncertainty, or entropy, of the message
source plus the uncertainty of the recipient as to what message was
transmitted when he knows what message was received. The measure
which Shannon provides of this latter uncertainty is the equivocation,
and he defines the rate of transmission of information as the
entropy of the message source less the equivocation.
The rate of transmission of information depends both on the
amount of noise or uncertainty in the channel and on what message
source is connected to the channel at the transmitting end. Let us
suppose that we choose a message source such that this rate of
transmission which we have defined is as great as it is possible to
make it. This greatest possible rate of transmission is called the
channel capacity for a noisy channel. The channel capacity is
measured in bits per symbol or per second.
So far, the channel capacity is merely a mathematically defined
quantity which we can compute if we know the probabilities of
various sorts of errors in the transmission of symbols. The channel
capacity is important, because Shannon proves, as his fundamental
theorem for the noisy channel, that when the entropy or information
rate of a message source is less than this channel capacity, the
messages produced by the source can be so encoded that they can
be transmitted over the noisy channel with an error less than any
specified amount.
In order to encode messages for error-free transmission over
noisy channels, long sequences of symbols must be lumped together
and encoded as one supersymboL This is the sort of block encoding
that we have encountered earlier. Here we are using it for a new
purpose. We are not using it to remove the redundancy of the
messages produced by a message source. Instead, we are using it
to add redundancy to nonredundant messages so that they can be
transmitted without error over a noisy channel. Indeed, the whole
problem of efficient and error-free communication turns out to be
that of removing from messages the somewhat inefficient redundancy
which they have and then adding redundancy of the right
The Noisy Channel165
sort in order to allow correction of errors made in transmission.
The redundant digits we must use in encoding messages for
error-free transmission, of course, slow the speed of transmission.
We have seen that in using a binary symmetric channel in which
1 transmitted digit in 100 is erroneously received, we can send only
92 correct nonredundant message digits for each 100 digits we feed
into the noisy channel. This means that on the average, we must
use a redundant code in which, for each 92 nonredundant message
digits, we must include in some way 8 extra check digits thus
making the over-all stream of digits redundant.
Shannon’s very general work tells us in principle how to proceed.
But, the mathematical difficulties of treating complicated channels
are great. Even in the case of the simple, symmetric, off-on binary
channel, the problem of finding efficient codes is formidable,
although mathematicians have found a large number of best codes.
Alas, even these seem to be too complicated to use!
Is this a discouraging picture? How much wiser we are than in
the days before information theory! We know what the problem
is. We know in principle how well we can do, and the result has
astonished engineers and mathematicians. Further, we do have
effective error-correcting codes that are used in a variety of applications,
including the transmission back to earth of glamorous
pictures of far planets.
CHAPTER I A Many Dimensions
Years and years ago (over thirty) I found in the public library
of St. Paul a little book which introduced me to the mysteries of
the fourth dimension. It was Flatland, by Abbott. It describes a
two-dimensional world without thickness. Such a world and all its
people could be drawn in complete detail, inside and out, on a
sheet of paper.
What I now most remember and admire about the book are the
descriptions of Flatland society. The inhabitants are polygonal,
and sidedness determines social status. The most exalted of the
multisided creatures hold the honorary status of circles. The lowest
order is isosceles triangles. Equilateral triangles are a step higher,
for regularity is admired and required. Indeed, irregular children
are cracked and reset to attain regularity, an operation which is
frequently fatal. Women are extremely narrow, needle-like creatures
and are greatly admired for their swaying motion. The author
of record, A. Square, accords well with all we have come to
associate with the word.
Flatland has a mathematical moral as well. The protagonist is
astonished when a circle of varying size suddenly appears in his
world. The circle is, of course, the intersection of a three-dimensional
creature, a sphere, with the plane of Flatland. The sphere
explains the mysteries of three dimensions to A. Square, who in
turn preaches the strange doctrine. The reader is left with the
thought that he himself may someday encounter a fluctuating and
disappearing entity, the three-dimensional intersection of a fourdimensional
creature with our world.
166
Many Dimensions
167
Four-dimensional cubes or tesseracts, hyperspheres, and other
hypergeometric forms are old stuff both to mathematicians and to
science fiction writers. Supposing a fourth dimension like unto the
three which we know, we can imagine many three-dimensional
worlds existing as close to one another as the pages of a manuscript,
each imprinted with different and distinct characters and
each separate from every other. We can imagine traveling through
the fourth dimension from one world to another or reaching
through the fourth dimension into a safe to steal the bonds or into
the abdomen to snatch an appendix.
Most of us have heard also that Einstein used time as a fourth
dimension, and some may have heard of the many-dimensional
phase spaces of physics, in which the three coordinates and three
velocity components of each of many particles are all regarded
as dimensions.
Clearly, this sort of thing is different from the classical idea of
a fourth spatial dimension which is just like the three dimensions
of up and down, back and forth, and left and right, those we all
know so well. The truth of the matter is that nineteenth-century
mathematicians succeeded in generalizing geometry to include any
number of dimensions or even an infinity of dimensions.
These dimensions are for the pure mathematician merely mental
constructs. He starts out with a line called the x direction or x axis,
as shown in a of Figure IX-1. Some point p lies a distance xp to
the right of the origin O on the x axis. This coordinate xp in fact
describes the location of the point p.
The mathematician can then add a y axis perpendicular to the
x axis, as shown in b of Figure IX-1. He can specify the location
of a point p in the two-dimensional space or plane in which these
axes lie by means of two numbers or coordinates: the distance from
the origin O in the y direction, that is, the height yp, and the
distance from the origin O in the x direction xp, that is, how far p
is to the right of the origin 0.
In c of Figure IX-1 the x, y, and z axes are supposed to be all
perpendicular to one another, like the edges of a cube. These axes
represent the directions of the three-dimensional space with which
we are familiar. The location of the point p is given by its height
yp above the origin O, its distance xp to the right of the origin O,
and its distance zp behind the origin O.
168
Symbols, Signals and Noise
y
T
l
0 p
ii
i
t
i
i
i
i
i
i
i
ii i
i
I
1*— Xp — *1 •— Xp “ *|
(a ) (t>)
Of course, in the drawing c of Figure IX-1 the x, y, and z axes
aren’t really all perpendicular to one another. We have here merely
a two-dimensional perspective sketch of an actual three-dimensional
situation in which the axes are all perpendicular to one
another. In d of Figure IX-1, we similarly have a two-dimensional
perspective sketch of axes in a five-dimensional space. Since we
come to the end of the alphabet in going from x to z, we have
merely labeled these directions x\, x2, x3, x4, x5, according to the
practice of mathematicians.
Of course these five axes of d of Figure IX-1 are not all perpen
Many Dimensions
169
dicular to one another in the drawing, but neither are the three
axes of c. We can’t lay out five mutually perpendicular lines in our
three-dimensional space, but the mathematician can deal logically
with a “space” in which five or more axes are mutually perpendicular.
He can reason out the properties of various geometrical
figures in a five-dimensional space, in which the position of a point
p is described by five coordinates Xip, x2p, x%p, x4p, Xsp. To make
the space like ordinary space (a Euclidean space) the mathematician
says that the square of the distance d of the pointp from the origin
shall be given by
d2 = *ip2 + x2p2 -I- x3p2 + x4p2 + x5p2 (9.1)
In dealing with multidimensional spaces, mathematicians define
the “volume” of a “cubical” figure as the product of the lengths
of its sides. Thus, in a two-dimensional space the figure is a square,
and, if the length of each side is L, the “volume” is the area of the
square, which is L2. In three-dimensional space the volume of a
cube of width, height, and thickness L is L3. In five-dimensional
space the volume of a hypercube of extent L in each direction is
L5, and a ninety-nine dimensional cube L on a side would have a
volume L".
Some of the properties of figures in multidimensional space are
simple to understand and startling to consider. For instance, consider
a circle of radius 1 and a concentric circle of radius Vi inside
of it, as shown in Figure IX-2. The area (“volume”) of a circle is
nr2, so the area of the outer circle is n and the area of the inner
Fig. IX -2
170
Symbols, Signals and Noise
circle is tt Q/i ) 2 = Thus, a quarter of the area of the whole
circle lies within a circle of half the diameter.
Suppose, however, that we regard Figure IX-2 as representing
spheres. The volume of a sphere is (%)wr3, and we find that 14 of
the volume of a sphere lies within a sphere of Vi diameter. In a
similar way, the volume of a hypersphere of n dimensions is proportional
to r™, and as a consequence the fraction of the volume
which lies in a hypersphere of half the radius is Vin. For instance,
for n = 7 this is a fraction 1/128.
We could go through a similar argument concerning the fraction
of the volume of a hypersphere of radius r that lies within a sphere
of radius 0.99r. For a 1,000-dimension hypersphere we find that a
fraction 0.00004 of the volume lies in a sphere of 0.99 the radius.
The conclusion is inescapable that in the case of a hypersphere of
a very high dimensionality, essentially all of the volume lies very
near to the surface!
Are such ideas anything but pure mathematics of the most
esoteric sort? They are pure and esoteric mathematics unless we
attach them to some problem pertaining to the physical world.
Imaginary numbers, such as \/— 1, once had no practical physical
meaning. However, imaginary numbers have been assigned meanings
in electrical engineering and physics. Can we perhaps find a
physical situation which can be represented accurately by the
mathematical properties of hyperspace? We certainly can, right in
the field of communication theory. Shannon has used the geometry
of multidimensional space to prove an important theorem concerning
the transmission of continuous, band-limited signals in the
presence of noise.
Shannon’s work provides a wonderful example of the use of a
new point of view and of an existing but hitherto unexploited
branch of mathematics (in this case, the geometry of multidimensional
spaces) in solving a problem of great practical interest.
Because it seems to me so excellent an example of applied mathematics,
I propose to go through a good deal of Shannon’s reasoning.
I believe that the course of this reasoning is more unfamiliar
than difficult, but the reader will have to embark on it at his
own peril.
Many Dimensions
171
In order to discuss this problem of transmission of continuous
signals in the presence of noise, we must have some common
measure of the strength of the signal and of the noise. Power turns
out to be an appropriate and useful measure.
When we exert a force of 1 lb over a distance of 1 ft in raising
a 1 lb weight to the height of 1 ft we do work. The amount of work
done is 1 foot-pound (ft-lb). The weight has, by virtue of its height,
an energy of 1 ft-lb. In falling, the weight can do an amount of
work (as in driving a clock) equal to this energy.
Power is rate of doing work. A machine which expends 33,000
ft-lb of energy and does 33,000 ft-lb of work in a minute has by
definition a power of 1 horsepower (hp).
In electrical calculations, we reckon energy and work in terms
of a unit called the joule and power in terms of a unit called a watt.
A watt is one joule per second.
If we double the voltage of a signal, we increase its energy and
power by a factor of 4. Energy and power are proportional to the
square of the voltage of a signal.
We have seen as far back as Chapter IV that a continuous
signal of band width W can be represented completely by its
amplitude at 2 IT sample points per second. Conversely, we can
construct a band-limited signal which passes through any 2W
sample points per second which we may choose. We can specify
each sample arbitrarily and change it without changing any other
sample. When we so change any sample we change the corresponding
band-limited signal.
We can measure the amplitudes of the samples in volts. Each
sample represents an energy proportional to the square of its
voltage.
Thus, we can express the squares of the amplitudes of the
samples in terms of energy. By using rather special units to measure
energy, we can let the energy be equal to the square of the sample
amplitude, and this won’t lead to any troubles.
Let us, then, designate the amplitudes of successive and correctly
chosen samples of a band-limited signal, measured perhaps
in volts, by the letters xi, X2 , x3, etc. The parts of the signal energy
represented by the samples will be Xi2, xff, x32, etc. The total
172 Symbols, Signals and Noise
energy of the signal, which we shall call E, will be the sum of
these energies:
E = x i2 + X2 2 + X3 2 + etc. (9.2)
But we see that in geometrical terms E is just the square of the
distance from the origin, as given by 9.1, if x\, x2, x3, etc., are the
coordinates of a point in multidimensional space!
Thus, if we let the amplitudes of the samples of a band-limited
signal be the coordinates of a point in hyperspace, the point itself
represents the complete signal, that is, all the samples taken
together, and the square of the distance of the point from the origin
represents the energy of the complete signal.
Why should we want to represent a signal in this geometrical
fashion? The reason that Shannon did so was to prove an important
theorem of communication theory concerning the effect of
noise on signal transmission.
In order to see how this can be done, we should recall the
mathematical model of a signal source which we adopted in
Chapter III. We there assumed that the source is both stationary
and ergodic. These assumptions must extend to the noise we consider
and to the combined “source” of signal plus noise.
It is not actually impossible that such a source might produce a
signal or a noise consisting of a very long succession of very high-
energy samples or of very low-energy samples, any more than it is
impossible that an ergodic source of letters might produce an
extremely long run of E’s. It is merely very unlikely. Here we are
dealing with the theorem we encountered first in Chapter V. An
ergodic source can produce a class of messages which are probable
and a class which are so very improbable that we can disregard
them. In this case, the improbable messages are those for which
the average power of the samples produced departs significantly
from the time average (and the ensemble average) characteristic
of the ergodic source.
Thus, for all the long messages that we need to consider, there
is a meaningful average power of the signal which does not change
appreciably with time. We can measure this average power by
adding the energies of a large number of successive samples and
dividing by the time T during which the samples are sent. As we
Many Dimensions
173
make the time T longer and longer and the number of samples
larger and larger, we will get a more and more accurate value for
the average power. Because the source is stationary, this average
power will be the same no matter what succession of samples
we use.
We can say this in a different way. Except in cases so unlikely
that we need not consider them, the total energy of a large number
of successive samples produced by a stationary source will be
nearly the same (to a small fractional difference) regardless of what
particular succession of samples we choose.
Because the signal source is ergodic as well as stationary, we can
say more. For each signal the source produces, regardless of what
the particular signal is, it is practically certain that the energy of
the same large number of successive samples will be nearly the
same, and the fractional differences among energies get smaller
and smaller as the number of samples is made larger and larger.
Let us represent the signals from such a source by points in
hyperspace. A signal of band width W and duration T can be
represented by 2 WT samples, and the amplitude of each of these
samples is the distance along one coordinate axis of hyperspace.
If the average energy per sample is P, the total energy of the 2 WT
samples will be very close to 2WTP if 2 IFF is a very large number
of samples. We have seen that this total energy tells how far from
the origin the point which represents the signal is. Thus, as the
number of samples is made larger and larger, the points representing
different signals of the same duration produced by the source
lie within a smaller and smaller distance (measured as a fraction
of the radius) from the surface of a hypersphere of radius yflWTP.
The fact that the points representing the different signals all lie so
close to the surface is not surprising if we remember that for a
hypersphere of high dimensionality almost all of the volume is very
close to the surface.
We receive, not the signal itself, but the signal with noise added.
The noise which Shannon considers is called white Gaussian noise.
The word white implies that the noise contains all frequencies
equally, and we assume that the noise contains all frequencies
equally up to a frequency of W cycles per second and no higher
frequencies. The word Gaussian refers to a law for the probability
174
Symbols,
Signals and Noise
of samples of various amplitudes, a law which holds for many
natural sources of noise. For such Gaussian noise, each of the 2W
samples per second which represent it is uncorrelated and independent.
If we know the average energy of the samples which we
will call N, knowing the energy of some samples doesn’t help to
predict the energy of others. The total energy of 2WT samples will
be very nearly 2WTN i£2WT is a large number of samples, and
the energy will be almost the same for any succession of noise
samples that are added to the signal samples.
We have seen that a particular succession of signal samples is
represented by some point in hyperspace a distance \/2 WTP from
the origin. The sum of a signal plus noise is represented by some
point a little distance away from the point representing the signal
alone. In fact, we see that the distance from the point representing
the signal alone to the point representing the signal plus the noise
is \/2 WTN. Thus, the signal plus the noise lies in a little hypersphere
of radius y/2 WTN centered on the point representing the
signal alone.
Now, we don’t receive the signal alone. We receive a signal of
average energy P per sample plus Gaussian noise of average
energy N per sample. In a time T, the total received energy is
2WT(P + N) and the point representing whatever signal was
sent plus whatever noise was added to it lies within a hypersphere
of radius yJ2WT{P + N).
After we have received a signal plus noise for T seconds we can
find the location of the point representing the signal plus noise.
But how are we to find the signal? We only know that the signal
lies within a distance \/2 WTN of the point representing the signal
plus noise.
How can we be sure of deducing what signal was sent? Suppose
that we put into the hypersphere of radius \/2 WT( P + N), in
which points representing a signal plus noise must lie, a large
number of little nonoverlapping hyperspheres of radius a bare
shade larger than \/2 WTN. Let us then send only signals represented
by the center points of these little spheres.
When we receive the 2WT samples of any particular one of these
signals plus any noise samples, the corresponding point in hyper-
space can only lie within the particular little hypersphere surround
Many Dimensions
175
ing that signal point and not within any other. This is so because,
as we have noted, the points representing long sequences of samples
produced by an ergodic noise source must be almost at the surface
of a sphere of radius \/2 WTN. Thus, the signal sent can be identified
infallibly despite the presence of the noise.
How many such nonoverlapping hyperspheres of radius
^IWTN can be placed in a hypersphere of radius \/2WT(P + N)1
The number certainly cannot be larger than the ratio of the volume
of the larger sphere to that of the smaller sphere.
The number n of dimensions in the space is equal to the number
of signal (and noise) samples 2 WT. The volume of a hypersphere
in a space of n dimensions is proportional to r™. Hence, the ratio
of the volume of the large signal-plus-noise sphere to the volume
of the little noise sphere is
This is a limit to the number of distinguishable messages we can
transmit in a time T. The logarithm of this number is the number
of bits which we can transmit in the time T. It is
As the message is. T seconds long, the corresponding number of bits
per second C is
C= IT log (1 + P/N) (9.3)Having got to this point, we can note that the ratio of average
energy per signal sample to average energy per noise sample must
be equal to the ratio of average signal power to average noise
power, and we can, in 9.3, regard P/N as the ratio of signal power
to noise power instead of as the’ratio of average signal-sample
energy to average noise-sample energy.
The foregoing argument, which led to 9.3, has merely shown that
no more than C bits per second can be sent with a band width of
W cycles per second using a signal of power P mixed with a
Gaussian noise of power N. However, by a further geometrical
argument, in which he makes use of the fact that the volume of a
176
Symbols, Signals and Noise
hypersphere of high dimensionality is almost all very close to the
surface, Shannon shows that the signaling rate can approach as
close as one likes to C as given by 9.3 with as small a number of
errors as one likes. Hence, C, as given by 9.3, is the channel
capacity for a continuous channel in which a Gaussian noise is
added to the signal.
It is perhaps of some interest to compare equation 9.3 with the
expressions for speed of transmission and for information which
Nyquist and Hartley proposed in 1928 and which we discussed in
Chapter II. Nyquist and Hartley’s results both say that the number
of binary digits which can be transmitted per second is
n log m
Here m is the number of different symbols, and n is the number
of symbols which are transmitted per second.
One sort of symbol we can consider is a particular value of
voltage, as, +3, +1, - 1, or - 3. Nyquist knew, as we do, that the
number of independent samples or values of voltage which can be
transmitted per second is 2W. By using this fact, we can rewrite
equation 9.3 in the form
C = (n/2) log (1 + P/N)
C = «logV(l + P / N )
Here we are really merely retracing the steps which led us to 9.3.
We see that in equation 9.3 we have got at the average number
m of different symbols we can send per sample, in terms of the ratio
of signal power to noise power. If the signal power becomes very
small or the noise power becomes very large, so that P/N is
nearly 0, then the average number of different symbols we can
transmit per sample goes to
log 1 = 0
Thus, the average number of symbols we can transmit per sample
and the channel capacity go to 0 as the ratio of signal power to
noise power goes to 0. Of course, the number of symbols we can
transmit per sample and the channel capacity become large as we
make the ratio of signal power to noise power large.
Our understanding of how to send a large average number of
Many Dimensions
177
independent symbols per sample has, however, gone far beyond
anything which Nyquist or Hartley told us. We know that if we
are to do this most efficiently we must, in general, not try to
encode a symbol for transmission as a particular sample voltage
to be sent by itself. Instead, we must, in general, resort to the
now-familiar procedure of block encoding and encode a long
sequence of symbols by means of a large number of successive
samples. Thus, if the ratio of signal power to noise power is 24,
we can on the average transmit with negligible error yl+24 =
\/25 = 5 different symbols per sample, but we can’t transmit any
of 5 different symbols by means of one particular sample.
In Figure VIII-1 of Chapter VIII, we considered sending binary
digits one at a time in the presence of noise by using a signal which
was either a positive or a negative pulse of a particular amplitude
and calling the received signal a 1 if the signal plus noise was
positive and a 0 if the received signal plus noise was negative.
Suppose that in this case we make the signal powerful enough
compared with the noise, which we assume to be Gaussian, so that
only 1 received digit in 100,000 will be in error. Calculations show
that this calls for about six times the signal power which equation
9.3 says we will need for the same band width and noise power.
The extra power is needed because we use as a signal either a short
positive or negative pulse specifying one binary digit, rather than
using one of many long signals consisting of many different samples
of various amplitudes to represent many successive binary digits.
One very special way of approaching the ideal signaling rate or
channel capacity for a small, average signal power in a large noise
power is to concentrate the signal power in a single short but
powerful pulse and to send this pulse in one of many possible time
positions, each of which represents a different symbol. In this very
special and unusual case we can efficiently transmit symbols one
at a time.
If we wish to approach Shannon’s limit for a chosen bandwidth
we must use as elements of the code long, complicated signal waves
that resemble gaussian noise.
We can if we wish look on relation 9.3 not narrowly as telling
us how many bits per second we can send over a particular communication
channel but, rather, as telling us something about the
178
Symbols, Signals and Noise
possibilities of transmitting a signal of a specified band width with
some required signal-to-noise ratio over a communication channel
of some other band width and signal-to-noise ratio. For instance,
suppose we must send a signal with a band width of 4 megacycles
per second and attain a ratio of signal power to noise power P/N
of 1,000. Relation 9.3 tells us that the corresponding channel
capacity is
C = 40,000,000 bits/second
But the same channel capacity can be attained with the combinations
shown in Table XIII.
Table XIII
Com binations o f W a n d P /N Which Give
Sam e Channel C apacity
w P/N
4,000,000 1,000
8,000,000 30.6
2,000,000 1,000,000
We see from Table XIII that, in attaining a given channel
capacity, we can use a broader band width and a lower ratio of
signal to noise or a narrower band width and a higher ratio of
signal to noise.
Early workers in the field of information theory were intrigued
with the idea of cutting down the band width required by increasing
the power used. This calls for lots of power. Experience has
shown that it is much more useful and practical to increase the
band width so as to get a good signal-to-noise ratio with less power
than would otherwise be required.
This is just what is done in FM transmission, as an example. In
FM transmission, a particular amplitude of the message signal to
be transmitted, which may, for instance, be music, is encoded as
a radio signal of a particular frequency. As the amplitude of the
message signal rises and falls, the frequency of the FM signal
which represents it changes greatly, so that in sending high fidelity
music which has a band width of 15,000 cps, the FM radio signal
can range over a band width of 150,000 cps. Because FM trans
Many Dimensions
179
mission makes use of a band width much larger than that of the
music of which it is an encoding, the signal-to-noise ratio of the
received music can be much higher than the ratio of signal power
to noise power in the FM signal that the radio receiver receives.
FM is not, however, an ideally efficient system; it does not work
the improvement which we might expect from 9.3.
Ingenious inventors are ever devising improved systems of modulation.
Twice in my experience someone has proposed to me a
system which purported to do better than equation 9.3, for the ideal
channel capacity, allows. The suggestions were plausible, but I
knew, just as in the case of perpetual motion machines, that
something had to be wrong with them. Careful analysis showed
where the error lay. Thus, communication theory can be valuable
in telling us what can’t be accomplished as well as in suggesting
what can be.
One thing that can’t be accomplished in improving the signal-
to-noise ratio by increasing the band width is to make a system
which will behave in an orderly and happy way for all ratios of
signal power to noise power.
According to the view put forward in this chapter, we look on
a signal as a point in a multidimensional space, where the number
of dimensions is equal to the number of samples. To send a narrow-
band signal of a few samples by means of a broad-band signal
having more samples, we must in some way map points in a space
of few dimensions into points in a space of more dimensions in a
one-to-one fashion.
Way back in Chapter I, we proved a theorem concerning the
mapping of points of a space of two dimensions (a plane) onto
points of a space of one dimension (a line). We proved that if we
map each point of the plane in a one-to-one fashion into a single
corresponding point on the line, the mapping cannot be continuous.
That is, if we move smoothly along a path in the plane from
point to nearby point, the corresponding positions on the line must
jump back and forth discontinuously. A similar theorem is true
for the mapping of the points of any space onto a space of different
dimensionality. This bodes trouble for transmission schemes
in which few message samples are represented by many signal
samples.
180 Symbols, Signals and Noise
Shannon gives a simple example of this sort of trouble, which
is illustrated in Figure IX-3. Suppose that we use two sample
amplitudes V2 and Vi to represent a single sample amplitude u. We
regard V2 and Vi as the distance up from and to the right of the
lower left hand corner of a square. In the square, we draw a snaky
line which starts near the lower left-hand corner and goes back
and forth across the square, gradually progressing upward. We let
distance along this line, measured from its origin near the lower
left-hand corner to some specified point along the line, be u, the
voltage or amplitude of the signal to be transmitted.
Certainly, any value of u is represented by particular values of
Vi and V2- We see that the range of vi or V2 is less than the range
of u. We can transmit vi and V2 and then reconstruct u with great
accuracy. Or can we?
Suppose a little noise gets into vi and v2, so that, when we try
to find the corresponding value of u at the receiver, we land
somewhere in a circle of uncertainty due to noise. As long as the
diameter of the circle is less than the distance between the loops
of the snaky path, we can tell what the correct value of u is to a
fractional error much smaller than the fractional error of vi or v2.
Fig. IX-3
Many Dimensions 181
But if the noise is larger, we can’t be sure which loop of the snaky
path was intended, and we frequently make a larger error in u.
This sort of behavior is inevitable in systems, such as FM, which
use a large band width in order to get a better signal-to-noise ratio.
As the noise added in transmission is increased, the noise in the
received (demodulated) signal at first increases gradually and then
increases catastrophically. The system is said to “break” at this
level of signal to noise. Here we have an instance in which a
seemingly abstract theorem of mathematics tells us that a certain
type of behavior cannot be avoided in electrical communication
systems of a certain general type.
The approach in this chapter has been essentially geometrical.
This is only one way of dealing with the problems of continuous
signals. Indeed, Shannon gives another in his book on communication
theory, an approach which is applicable to all types of
signals and noise. The geometrical approach is interesting, however,
because it is proving illuminating and fruitful in many problems
concerning electric signals which are not directly related to
communication theory.
Here we have arrived at a geometry of band-limited signals by
sampling the signals and then letting the amplitudes of the samples
be the coordinates of a point in a multidimensional space. It is
possible, however, to geometrize band-limited signals without
speaking in terms of samples, and mathematicians interested in
problems of signal transmission have done this. In fact, it is becoming
increasingly common to represent band-limited signals as
points in a multidimensional “signal space” or “function space”
and to prove theorems about signals by the methods of geometry.
The idea of signals as points in a multidimensional signal space
or function space is important, because it enables mathematicians
to think about and to make statements which are true about all
band-limited signals, or about large classes of band-limited signals,
without considering the confusing details of particular signals,
just as mathematicians can make statements about all triangles or
all right triangles. Signal space is a powerful tool in the hands or,
rather, in the minds of competent mathematicians. We can only
wonder and admire.
182
Symbols, Signals and Noise
From the point of view of communication theory, our chief
concern in this chapter has been to prove an important theorem
concerning a noisy continuous channel. The result is embodied in
equation 9.3, which gives the rate at which we can transmit binary
digits with negligible error over a continuous channel in which a
signal of band width W and power P is mixed with a white
Gaussian noise of band width W and power N.
Nyquist knew, in 1928, that one can send 2 W independent
symbols per second over a channel of band width 2 W, but he didn’t
know how many different symbols could be sent per second for a
given ratio of signal power to noise power. We have found this out
for the case of a particular, common type of noise. We also know
that even if we can transmit some average number m of symbols
per sample, in general, we can’t do this by trying to encode successive
symbols independently as particular voltages. Instead, we
must use block encoding, and encode a large number of successive
symbols together.
Equation 9.3 shows that we can use a signal of large band width
and low ratio of signal power to noise power in transmitting a
message which has a small band width and a large ratio of signal
power to noise power. FM is an example of this. Such considerations
will be pursued further in Chapter X.
This chapter has had another aspect. In it we have illustrated
the use of a novel viewpoint and the application of a powerful field
of mathematics in attacking a problem of communication theory.
Equation 9.3 was arrived at by the by-no-means-obvious expedient
of representing long electrical signals and the noises added to them
by points in a multidimensional space. The square of the distance
of a point from the origin was interpreted as the energy of the
signal represented by the point.
Thus, a problem in communication theory was made to correspond
to a problem in geometry, and the desired result was arrived
at by geometrical arguments. We noted that the geometrical representation
of signals has become a powerful mathematical tool in
studying the transmission and properties of signals.
The geometrization of signal problems is of interest in itself, but
it is also of interest as an example of the value of seeking new
Many Dimensions
183
mathematical tools in attacking the problems raised by our increasingly
complex technology. It is only by applying this order of
thought that we can hope to deal with the increasingly difficult
problems of engineering.
CHAPTER A Information Theory
and Physics
I have given something of the historical background of communication
theory in Chapter II. From this we can see that
communication theory is an outgrowth of electrical communication,
and we know that the behavior of electric currents and
electric and magnetic fields is a part of physics.
To Morse and to other early telegraphists, electricity provided
a very limited means of communication compared with the human
voice or the pen in hand. These men had to devise codes by means
of which the letters of the alphabet could be represented by turning
an electric current successively on and off. This same problem of
the representation of material to be communicated by various sorts
of electrical signals has led to the very general ideas concerning
encoding which are so important in communication theory. In this
relation of encoding to particular physical phenomena, we see one
link between communication theory and physics.
We have also noted that when we transmit signals by means of
wire or radio, we receive them inevitably admixed with a certain
amount of interfering disturbances which we call noise. To some
degree, we can avoid such noise. The noise which is generated in
our receiving apparatus we can reduce by careful design and by
ingenious invention. In receiving radio signals, we can use an
antenna which receives signals most effectively from the direction
of the transmitter and which is less sensitive to signals coming from
184
Information Theory and Physics 185
other directions. Further, we can make sure that our receiver
responds only to the frequencies we mean to use and rejects interfering
signals and noise of other frequencies.
Still, when all this is done, some noise will inevitably remain,
mixed with the signals that we receive. Some of this noise may
come from the ignition systems of automobiles. Far away from
man-made sources, some may come from lightning flashes. But
even if lightning were abolished, some noise would persist, as
surely as there is heat in the universe.
Many years ago an English biologist named Brown saw small
pollen particles, suspended in a liquid, dance about erratically in
the field of his microscope. The particles moved sometimes this way
and sometimes that, sometimes swiftly and sometimes slowly. This
we call Brownian motion. Brownian motion is caused by the impact
on the particles of surrounding molecules, which themselves
execute even a wilder dance. One of Einstein’s first major works
was a mathematical analysis of Brownian motion.
The pollen grains which Brown observed would have remained
at rest had the molecules about them been at rest, but molecules
are always in random motion. It is this motion which constitutes
heat. In a gas, a molecule moves in a disorganized way. It moves
swiftly or slowly in straight lines between frequent collisions. In
a liquid, the molecules jostle about in close proximity to one
another but continually changing place, sometimes moving swiftly
and sometimes slowly. In a solid, the molecules vibrate about their
mean positions, sometimes with a large amplitude and sometimes
with a small amplitude, but never moving much with respect to
their nearest neighbors. Always, however, in gas, liquid, or solid,
the molecules move, with an average energy due to heat which is
proportional to the temperature above absolute zero, however
erratically the speed and energy may vary from time to time and
from molecule to molecule.
Energy of mechanical motion is not the only energy in our
universe. The electromagnetic waves of radio and light also have
energy. Electromagnetic waves are generated by changing currents
of electricty. Atoms are positively charged nuclei surrounded by
negative electrons, and molecules are made up of atoms. When
the molecules of a substance vibrate with the energy of heat,
186
Symbols,
Signals and Noise
relative motions of the charges in them can generate electromagnetic
waves, and these waves have frequencies which include those
of what we call radio, heat, and light waves. A hot body is said
to radiate electromagnetic waves, and the electromagnetic waves
that it emits are called radiation.
The rate at which a body which is held at a given temperature
radiates radio, heat, and light waves is not the same for all substances.
Dark substances emit more radiation than shiny substances.
Thus, silver, which is called shiny because it reflects most
of any waves of radio, heat, or light falling on it, is a poor radiator,
while the carbon particles of black ink constitute a good radiator.
When radiation falls on a substance, the fraction that is reflected
rather than absorbed is different for radiation of different frequencies,
such as radio waves and light waves. There is a very general
rule, however, that for radiation of a given frequency, the amount
of radiation a substance emits at a given temperature is directly
proportional to the fraction of any radiation falling on it which is
absorbed rather than reflected. It is as if there were a skin around
each substance which allowed a certain fraction of any radiation
falling on it to pass through and reflected the rest, and as if the
fraction that passed through the skin were the same for radiation
either entering or leaving the substance.
If this were not so, we might expect a curious and unnatural (as
we know the laws of nature) phenomenon. Let us imagine a completely
closed box or furnace held at a constant temperature. Let
us imagine that we suspend two bodies inside the furnace. Suppose
(contrary to fact) that the first of these bodies reflected radiation
well, absorbing little, and that it also emitted radiation strongly,
while the second absorbed radiation well, reflecting little, but
emitted radiation poorly. Suppose that both bodies started out at
the same temperature. The first would absorb less radiation and
emit more radiation than the second, while the second would
absorb more radiation and emit less radiation than the first. If this
were so, the second body would become hotter than the first.
This is not the case, however; all bodies in a closed box or
furnace whose walls are held at a constant, uniform temperature
attain just exactly the same temperature as the walls of the furnace,
whether the bodies are shiny, reflecting little radiation and absorb
Information Theory and Physics 187
ing much, or whether they are dark, reflecting little radiation and
absorbing much. This can be so only if the ability to absorb rather
than reflect radiation and the ability to emit radiation go hand in
hand, as they always do in nature.
Not only do all bodies inside such a closed furnace attain the
same temperature as the furnace; there is also a characteristic
intensity of radiation in such an enclosure. Imagine a part of the
radiation inside the enclosure to strike one of the walls. Some will
be reflected back to remain radiation in the enclosure. Some will
be absorbed by the walls. In turn, the walls will emit some radiation,
which will be added to that reflected away from the walls.
Thus, there is a continual interchange of radiation between the
interior of the enclosure and the walls.
If the radiation in the interior were very weak, the walls would
emit more radiation than the radiation which struck and was
absorbed by them. If the radiation in the interior were very strong,
the walls would receive and absorb more radiation than they
emitted. When the electromagnetic radiation lost to the walls is
just equal to that supplied by the walls, the radiation is said to be
in equilibrium with its material surroundings. It has an energy
which increases with temperature, just as the energy of motion
of the molecules of a gas, a liquid, or a solid increases with
temperature.
The intensity of radiation in an enclosure does not depend on
how absorbing or reflecting the walls of the enclosure are; it
depends only on the temperature of the walls. If this were not so
and we made a little hole joining the interior of a shiny, reflecting
enclosure with the interior of a dull, absorbing enclosure at the
same temperature, there would have to be a net flow of radiation
through the hole from one enclosure to another at the same temperature.
This never happens.
We thus see that there is a particular intensity of electromagnetic
radiation, such as light, heat, and radio waves, which is characteristic
of a particular temperature. Now, while eletromagnetic waves
travel through vacuum, air, or insulating substances such as glass,
they can be guided by wires. Indeed, we can think of the signal
sent along a pair of telephone wires either in terms of the voltage
between the wires and the current of electrons which flows in the
188
Symbols, Signals and Noise
wires, or in terms of a wave made up of electric and magnetic fields
between and around the wires, a wave which moves along with the
current. As we can identify electrical signals on wires with electromagnetic
waves, and as hot bodies radiate electromagnetic waves,
we should expect heat to generate some sort of electrical signals.
J. B. Johnson, who discovered the electrical fluctuations caused
by heat, described them, not in terms of electromagnetic waves but
in terms of a fluctuating voltage produced across a resistor.
Once Johnson had found and measured these fluctuations,
another physicist was able to find a correct theoretical expression
for their magnitude by applying the principles of statistical mechanics.
This second physicist was none other than H. Nyquist,
who, as we saw in Chapter II, also contributed substantially to the
early foundations of information theory.
Nyquist’s expression for what is now called either Johnson noise
or thermal noise is
V2 = 4kTRW (10.1)
Here V2 is the mean square noise voltage, that is, the average
value of the square of the noise voltage, across the resistor, k is
Boltzmann’s constant:
k = 1.37 x 10"23 joule/degree
T is the temperature of the resistor in degrees Kelvin, which is the
number of Celsius or centigrade degrees (which are % as large as
Fahrenheit degrees) above absolute zero. Absolute zero is —273°
centigrade or —459° Fahrenheit. R is the resistance of the resistor
measured in ohms. W is the band width of the noise in cycles
per second.
Obviously, the band width W depends only on the properties of
our measuring device. If we amplify the noise with a broad-band
amplifier we get more noise than if we use a narrow-band amplifier
of the same gain. Hence, we would expect more noise in a television
receiver, which amplifies signals over a band width of several
million cycles per second, than in a radio receiver, which amplifies
signals having a band width of several thousand cycles per second.
We have seen that a hot resistor produces a noise voltage. If we
connect another resistor to the hot resistor, electric power will flow
Information Theory and Physics
189
to this second resistor. If the second resistor is cold, the power will
heat it. Thus, a hot resistor is a potential source of noise power.
What is the most noise power N that it can supply? The power is
N = \tTW (10.2)
In some ways, 10.2 is more satisfactory than 10.1. For one thing,
it has fewer terms; the resistance R no longer appears. For another
thing, its form is suitable for application to somewhat different
situations.
For instance, suppose that we have a radio telescope, a big
parabolic reflector which focuses radio waves into a sensitive radio
receiver. I have indicated such a radio telescope in Figure X-l.
Suppose we point the radio telescope at different celestial or terrestrial
objects, so as to receive the electromagnetic noise which
they radiate because of their temperature.
We find that the radio noise power received is given by 10.2,
where T is the temperature of the object at which the radio
telescope points.
If we point the telescope down at water or at smooth ground,
what it actually sees is the reflection of the sky, but if we point it
at things which don’t reflect radio waves well, such as leafy trees
or bushes, we get a noise corresponding to a temperature around
290° Kelvin (about 62° Fahrenheit), the temperature of the trees.
If we point the radio telescope at the moon and if the telescope
Fig. X -J
190
Symbols, Signals and Noise
is directive enough to see just the moon and not the sky around
it, we get about the same noise, which corresponds not to the
temperature of the very surface of the moon but to the temperature
a fraction of an inch down, for the substance of the moon is somewhat
transparent to radio waves.
If we point the telescope at the sun, the amount of noise we
obtain depends on the frequency to which we tune the radio
receiver. If we tune the receiver to a frequency around 10 million
cycles per second (a wave length of 30 meters), we get noise
corresponding to a temperature of around a million degrees Kelvin;
this is the temperature of the tenuous outer corona of the sun. The
corona is transparent to radio waves of shorter wave lengths, just
as the air of the earth is. Thus, if we tune the radio receiver to a
frequency of around 10 billion cycles per second, we receive radiation
corresponding to the temperature of around 8,000° Kelvin,
the temperature a little above the visible surface. Just why the
corona is so much hotter than the visible surface which lies below
it is not known.
The radio noise from the sky is also different at different frequencies.
At frequencies above a few billion cycles per second the
noise corresponds to a temperature of about 3.5° Kelvin. At lower
frequencies the noise is greater and increases steadily as the frequency
is lowered. The Milky Way, particular stars, and island
universes or galaxies in collision all emit large amounts of radio
noise. The heavens are telling the age of the universe through their
3.5 degree microwave radiation, but other signals come from other
places at other frequencies.
Nonetheless, Johnson or thermal noise constitutes a minimum
noise which we must accept, and additional noise sources only
make the situation worse. The fundamental nature of Johnson
noise has led to its being used as a standard in the measurement of
the performance of radio receivers.
As we have noted, a radio receiver adds a certain noise to the
signals it receives. It also amplifies any noise that it receives. We
can ask, how much amplified Johnson noise would just equal the
noise the receiver adds? We can specify this noise by means of an
equivalent noise temperature Tn. This equivalent noise temperature
Tn is a measure of the noisiness of the radio receiver. The smaller
Tn is the better the receiver is.
Information Theory and Physics 191
We can interpret the noise temperature Tn in the following way.
If we had an ideal noiseless receiver with just the same gain and
band width as the actual receiver and if we added Johnson noise
corresponding to the temperature Tn to the signal it received, then
the ratio of signal power to noise power would be the same for the
ideal receiver with the Johnson noise added to the signal as for
the actual receiver.
Thus, the noise temperature Tn is a just measure of the noisiness
of the receiver. Sometimes another measure based on Tn is used;
this is called the noise figure NF. In terms of Tn, the noise figure is
293
(10.3)
The noise figure was defined for use here on earth, where every
signal has mixed with it noise corresponding to a temperature of
around 293 ° Kelvin. The noise figure is the ratio of the total output
noise, including noise due to Johnson noise for a temperature of
293 ° Kelvin at the input and noise produced in the receiver, to the
amplified Johnson noise alone.
Of course, the equivalent noise temperature Tn of a radio receiver
depends on the nature and perfection of the radio receiver, and the
lowest attainable noise figure depends on the frequency of operation.
However, Table XIV below gives some rough figures for
various sorts of receivers.
The effective temperatures of radio receivers and the tempera-
Table XIV
Type of Receiver
Equivalent
Noise Temperature,
Degrees Kelvin
1500°
20°
Good Radio or TV receiver Maser receiving station for space missions Parametric amplifier receiver 50°
192
Symbols,
Signals and Noise
tures of the objects at which their antennas are directed are very
important in connection with communication theory, because
noise determines the power required to send messages. Johnson
noise is Gaussian noise, to which equation 9.3 applies. Thus,
ideally, in order to transmit C bits per second, we must have a
signal power P related to the noise power N by a relation that was
derived in the preceding chapter:
C = uq°g (L ±-P) (9.3)
If we use expression 10.2 for noise, this becomes
c=Wtoi(aw) <ia4)
Let us assume a given signal power P. If we make W very
small, C will become very small. However, if we make W larger
and larger, C does not become larger and larger without limit, but
rather it approaches a limiting value. When P/kTW becomes very
small compared with unity, 10.4 becomes
C = (10.5)
We can also write this
P = 0.693 kTC (10.6)
Relation 10.6 says that, even when we use a very wide band
width, we need at least a power 0.693 kTjoule per second to send
one bit per second, so that on the average we must use an energy
of 0.693 kT joule for each bit of information we transmit. We
should remember, however, that equation 9.3 holds only for an
ideal sort of encoding in which many characters representing many
bits of information are encoded together into a long stretch of
signal. Most practical communication systems require much more
energy per bit, as we noted in Chapter IX.
But, haven’t we forgotten something? What about quantum
effects? These may not be important in radio, but they are certainly
important in communication by light. And, light has found a wide
range of uses. Tiny optical fibers carry voice and other com
Information Theory and Physics 193
munication traffic. Pulses of light reflected from “corner reflectors”
left by astronauts on the moon make it possible to track the earth-
moon distance with an accuracy of about a tenth of a meter.
Harry Nyquist was a very forward-looking man. The derivation
he gave for an expression for Johnson noise applies to all frequencies,
including those of light. His expression for Johnson noise
power in a bandwidth W was
(10.7)
Here / is frequency in cycles per second and
h = 6.63 X 10~34 joule/sec.
is Planck’s constant. We commonly associate Planck’s constant
with the energy of a single photon of light, an energy E which is
E = hf ( 1 0 .8 )
Quantum effects become important when h/ is comparable to or
larger than kT. Thus, the frequency above which the exact expression
for Johnson noise, 10.7, departs seriously from the expression
valid at low frequencies, 10.2, will be about
/ = kr/h = 2.07 X 1010T (10.9)
When we take quantum effects into account, we find less not
more noise at higher frequencies, and very much less noise at the
frequency of light. But, there are quantum limitations other than
those imposed by Johnson noise. It turns out that, ideally, 0.693
kT joules per bit is still the limit, even at the frequency of light.
Practically, it is impossible to approach this limit closely. And,
there is a common but wrong way of communicating which is very
much worse. This is to amplify a weak received signal with the best
sort of amplifier that is theoretically possible. Why is this bad?
At low frequencies, when we amplify a weak pulse we simply
194 Symbols, Signals and Noise
obtain a pulse of greater power, and we can measure both the time
that this stronger pulse peaks and the frequency spectrum of the
pulse. But, the quantum mechanical uncertainty principle says that
we cannot measure both time and frequency with perfect accuracy.
In fact, if At is the error in measurement of time and A/ is the
error in measurement of frequency, the best we can possibly do is
AtAf = 1
At = l/Af ( 1 0.1 0 )
The implication of 10.10 is that if we define the frequency of a
pulse very precisely by making A/ small, we cannot measure the
time of arrival of the pulse very accurately. In more mundane
terms, we can’t measure the time of arrival of a long narrow-band
pulse as accurately as we can measure the time of arrival of a short
broad-band pulse. But, just how are we frustrated in trying to make
such a measurement?
Suppose that we do amplify a weak pulse with the best possible
amplifier, and also shift all frequencies of the pulse down to some
low range of frequencies at which quantum effects are negligible.
We find that, mixed with the amplified signal, there will be a noise
power N
N = GhfW ( 1 0.1 1 )
Here / is the frequency of the original high-frequency signal, G is
the power gain of the amplifying and frequency shifting system and
W is the bandwidth. This noise is just enough to assure that we
don’t make measurements more accurately than allowed by the uncertainty
principle as expressed in 10.10.
In order to increase the accuracy of a time measurement, we
must increase the bandwidth W of the pulse. But, the added noise
due to increasing the bandwidth, as expressed by 10.1 1 , just undoes
any gain in accuracy of measurement of time.
From 10.11 we can make the same argument that we made on
page 192, and conclude that because of quantum effects we must
use an energy of at least 0.693 hf joules per bit in order to communicate
by means of a signal of frequency /. This argument is
Information Theory and Physics
195
valid only for communication systems in which we amplify the
received signal with the best possible amplifier—an amplifier
which adds to the signal only enough noise to keep us from violating
the uncertainty principle.
Is there an alternative to amplifying the weak received signal?
At optical frequencies there certainly is. Quanta of light can be
used to produce tiny pulses of electricity. Devices called photodiodes
and photomultipliers produce a short electrical output pulse
when a quantum of light reaches them—although they randomly
fail to respond to some quanta. The quantum efficiency of actual
devices is less than 100%.
Nonetheless, ideally we can detect the time of arrival of any
quantum of light by using that quantum to produce a tiny electrical
pulse. Isn’t this contrary to the uncertainty principle? No, because
when we measure the time of arrival of a quantum in this way we
can’t tell anything at all about the frequency of the quantum.
Photo detectors, commonly called photon counters, are used to
measure the time of arrival of pulses of light reflected from the
comer reflectors that astronauts left on the moon. They are also
used in lightwave communication via optical fibers. Of course they
aren’t used as effectively as is ideally possible. What is the ideal
limit? It turns out to be the same old limit we found on page 192,
that is, 0.693 kT joules per bit. Quantum effects don’t change that
limiting performance, but they do make it far more difficult to
attain. Why is this so?
The energy per photon is h/. Ideally, the energy per bit is 0.693
kT. Thus, ideally the bits per photon must be
h//(0.693 kT)
or
1.44 (h//kT) bits per photon
How can we transmit many bits of energy by measuring the time
of arrival of just a few photons, or one photon? We can proceed
in this way. At the transmitter, we send out a pulse of light in one
of M sub-intervals of some long time interval T. At the receiving
end, the particular time interval in which we receive a photon
conveys the message.
196 Symbols, Signals and Noise
At best, this will make it possible to convey log M bits of information
for each interval T. But, sometimes we will receive no
photons in any time interval. And, sometimes thermal photons—
Johnson noise photons—will arrive in a wrong time interval. This
is what limits the transmission to 1.44 (h//kT) bits per photon.
In reality, we can send far fewer bits per photon because it is
impractical to implement effective systems that will send very many
bits per photon.
We have seen through a combination of information theory and
physics that it takes at least 0.693 kT joules of energy to convey
one bit of information.
In most current radio receivers the noise actually present is
greater than the environmental noise because the amplifier adds
noise corresponding to some temperature higher than that of the
environment. Let us use as the noise temperature Tn, the equivalent
noise temperature corresponding to the noise actually added to the
signal. How does actual performance compare with the ideal expression
by equation (10.4)?
When we do not use error correction, but merely use enough
signal power so that errors in receiving bits of information are very
infrequent (around one error in 108 bits received), at best we must
use about 10 times as much power per bit as calculated from
expression 10.4.
The most sophisticated communication systems are those used
in sending data back from deep-space missions. These are extremely
low-noise maser receivers, and they make use of sophisticated
error correction, including convolutional coding and decoding
using the Viterbi algorithm. In sending pictures of Jupiter and its
satellites back to earth, the Voyager spacecraft could transmit
115,200 binary digits per second with an error rate of one in 200
by using only 21.3 watts of power. The power is only 4.4 db more
than the ideal limit using infinite bandwidth.
Pluto is about 6 X 1012 meters from earth. Ideally, how fast
could we send data back from Pluto? We’ll assume noise from space
- only, and no atmosphere absorption.
If we use a transmitting antenna of effective area AT and a receiving
antenna of effective area A „, the ratio of received power
Information Theory and Physics 197
PB to transmitted power PT is given by Friis’s transmission formula
P it _ A b A t
PT ~ A2L2
( 1 0.1 2 )
where A is the wavelength used in communication and L is the
distance from transmitter to receiver, which in our case is 6 X 1012
meters.
Purely arbitrarily, we will assume a transmitter power of 10
watts. We will consider two cases. In the first, we use a wavelength
of 1 cm or .01 meters. At this wavelength the temperature of space
is indubitably 3.5° K, and we will assume that the transmitting
antenna is 3.16 meters square, with an area of 10 square meters, and
the receiving antenna is 31.6 meters square, with an area of 1,000
square meters. According to expression 10.12, if the transmitted
power is 10 watts the received power will be 2.8 X 10-17 watts or
joules per second. If we take the energy per bit as 0.693 kT, T as
3.5°K, and use the value of Boltzmann’s constant given on page
188, we conclude that our communication system can send us over
800,000 bits per second, a very useful amount of information.
What about an optical communication system? Let us assume a
wavelength of 6 X 10_T meters, corresponding to a frequency of
5 X 1014 cycles per second. This is visible light. We will assume
somewhat smaller antennas (lenses or mirrors), a transmitting
antenna one meter square, with an area of 1 square meter and a
receiving antenna 10 meters square, with an area of 100 square
meters. Again, we will assume a transmitted power of 10 watts.
The effective optical “temperature” of space, that is, the total
starlight, is a little uncertain; we will take it as 350°K. We calculate
a transmission capacity of 800 billion bits per second for our
optical link.
If we receive 800 billion bits per second, we must receive 100
bits per photon. It seems unlikely that this is possible. But even if
we received only one bit per photon we would receive 8 billion
bits/second. Optical communication may be the best way to communicate
over long distances in space.
198
Symbols, Signals and Noise
From the point of view of information theory, the most interesting
relation between physics and information theory lies in the
evaluation of the unavoidable limitations imposed by the laws of
physics on our ability to communicate. In a very fundamental
sense, this is concerned with the limitations imposed by Johnson
noise and quantum effects. It also, however, includes limitations
imposed by atmospheric turbulence and by fluctuations in the
ionosphere, which can distort a signal in a way quite different from
adding noise to it. Many other examples of this sort of relation of
physics to information theory could be unearthed.
Physicists have thought of a connection between physics and
communication theory which has nothing to do with the fundamental
problem that communication theory set out to solve, that
is, the possibilities of the limitations of efficient encoding in transmitting
information over a noisy channel. Physicists propose to use
the idea of the transmission of information in order to show the
impossibility of what is called a perpetual-motion machine of the
second kind. As a matter of fact, this idea preceded the invention
of communication theory in its present form, for L. Szilard put
forward such ideas in 1929.
Some perpetual-motion machines purport to create energy; this
violates the first law of thermodynamics, this is, the conservation
of energy.
Other perpetual-motion machines purport to convert the disorganized
energy of heat in matter or radiation which is all at the
same temperature into ordered energy, such as the rotation of a
flywheel. The rotating flywheel could, of course, be used to drive
a refrigerator which would cool some objects and heat others.
Thus, this sort of perpetual motion could, without the use of
additional organized energy, transfer the energy of heat from cold
material to hot material.
The second law of thermodynamics can be variously stated: that
heat will not flow from a cold body to a hot body without the
expenditure of organized energy or that the entropy of a system
never decreases. The second sort of perpetual-motion machine
violates the second law of thermodynamics.
One of the most famous perpetual-motion machines of this
second kind was invented by James Clerk Maxwell. It makes use
of a fictional character called Maxwell’s demon.
Information Theory and Physics
199
I have pictured Maxwell’s demon in Figure X-2. He inhabits a
divided box and operates a small door connecting the two chambers
of the box. When he sees a fast molecule heading toward the
door from the far side, he opens the door and lets it into his side.
When he sees a slow molecule heading toward the door from his
side he lets it through. He keeps slow molecules from entering his
side and fast molecules from leaving his side. Soon, the gas in his
side is made up of fast molecules. It is hot, while the gas on the
other side is made up of slow molecules and it is cool. Maxwell’s
demon makes heat flow from the cool chamber to the hot chamber.
I have shown him operating the door with one hand and thumbing
his nose at the second law of thermodynamics with his other hand.
Maxwell’s demon has been a real puzzler to those physicists who
have not merely shrugged him off. The best general objection we
can raise to him is that, since the demon’s environment is at thermal
equilibrium, the only light present is the random electromagnetic
radiation corresponding to thermal noise, and this is so chaotic
that the demon can’t use it to see what sort of molecules are coming
toward the door.
We can think of other versions of Maxwell’s demon. What about
putting a spring door between the two chambers, for instance? A
molecule hitting such a door from one side can open it and go
through; one hitting it from the other side can’t open it at all.
Won’t we end up with all the molecules and their energy on the
side into which the spring door opens?
Fig. X-2
200 Symbols, Signals and Noise
One objection which can be raised to the spring door is that, if
the spring is strong, a molecule can’t open the door, while, if the
spring is weak, thermal energy will keep the door continually
flapping, and it will be mostly open. Too, a molecule will give
energy to the door in opening it. Physicists are pretty well agreed
that such mechanical devices as spring doors or delicate ratchets
can’t be used to violate the second law of thermodynamics.
Arguing about what will and what won’t work is a delicate
business. An ingenious friend fooled me completely with his
machine until I remembered that any enclosure at thermal equilibrium
must contain random electromagnetic radiation as well as
molecules. However, there is one simple machine which, although
it is frictionless, ridiculous, and certainly inoperable in any practical
sense, is, I believe, not physically impossible in the very special
sense in which physicists use this expression. This machine is
illustrated in Figure X-3.
The machine makes use of a cylinder C and a frictionless piston
P. As the piston moves left or right, it raises one of the little pans
p and lowers the other. The piston has a door in it which can be
opened or closed. The cylinder contains just one molecule M. The
whole device is at a temperature T. The molecule will continually
gain and lose energy in its collisions with the walls, and it will have
an average energy proportional to the temperature.
When the door in the piston is open, no work will be done if we
move the piston slowly to the right or to the left. We start by
centering the piston with the door open. We clamp the piston in
D C
P 'M £
Fig. X-3
Information Theory and Physics
201
the center and close the door. We then observe which side of the
piston the molecule is on. When we have found out which side of
the piston the molecule is on, we put a little weight from low shelf
Si onto the pan on the same side as the molecule and unclamp
the piston. The repeated impact of the molecule on the piston will
eventually raise the weight to the higher shelf S2, and we take the
weight off and put it on this higher shelf. We then open the door
in the piston, center it, and repeat the process. Eventually, we will
have lifted an enormous number of little weights from the lower
shelves Si to the upper shelves S2. We have done organized work
by means of disorganized thermal energy!
How much work have we done? It is easily shown that the
average force F which the molecule exerts on the piston is
k T
L ( 10. 10)
Here L is the distance from the piston to the end of the cylinder
on the side containing the molecule. When we allow the molecule
to push against the piston and slowly drive it to the end of the
cylinder, so that the distance is doubled, the most work W that the
molecule can do is
W = 0.693 kT (10.11)
Actually, in lifting a constant weight the work done will be less,
but 10.11 represents the limit. Did we get this free?
Not quite! When we have centered the piston and closed the
door it is equally likely that we will find the molecule in either half
of the cylinder. In order to know which pan to put the weight on,
we need one bit of information, specifying which side the molecule
is on. To make the machine run we must receive this information
in a system which is at a temperature T. What is the very least
energy needed to transmit one bit of information at the temperature
T? We have already computed this; from equation 10.6 we
see that it is exactly 0.693 kT joule, just equal to the most energy
the machine can generate. In principle, this applies to the quantum
case, if we do the best that is possible. Thus, we use up all the
output of the machine in transmitting enough information to make
the machine run!
202
Symbols, Signals and Noise
It’s useless to argue about the actual, the attainable, as opposed
to the limiting efficiency of such a machine; the important thing
is that even at the very best we could only break even.
We have now seen in one simple case that the transmission of
information in the sense of communication theory can enable us
to convert thermal energy into mechanical energy. The bit which
measures amount of information used is the unit in terms of which
the entropy of a message source is measured in communication
theory. The entropy of thermodynamics determines what part of
existing thermal energy can be turned into mechanical work. It
seems natural to try to relate the entropy of thermodynamics and
statistical mechanics with the entropy of communication theory.
The entropy of communication theory is a measure of the uncertainty
as to what message, among many possible messages, a
message source will actually produce on a given occasion. If the
source chooses a message from among m equally probable messages,
the entropy in bits per message is the logarithm to the base
2 of m; in this case it is clear that such messages can be transmitted
by means of log m binary digits per message. More generally, the
importance of the entropy of communication theory is that it
measures directly the average number of binary digits required to
transmit messages produced by a message source.
The entropy of statistical mechanics is the uncertainty as to what
state a physical system is in. It is assumed in statistical mechanics
that all states of a given total energy are equally probable. The
entropy of statistical mechanics is Boltzmann’s constant times the
logarithm to the base e of the number of possible states. This
entropy has a wide importance in statistical mechanics. One matter
of importance is that the free energy, which we will call F.E., is
given by
F.E. = E - HT (10.12)
Here E is the total energy, H is the entropy, and T is the temperature.
The free energy is the part of the total energy which, ideally,
can be turned into organized energy, such as the energy of a
lifted weight.
In order to understand the entropy of statistical mechanics, we
have to say what a physical system is, and we will do this by citing
Information Theory and Physics 203
a few examples. A physical system can be a crystalline solid, a
closed vessel containing water and water vapor, a container filled
with gas, or any other substance or collection of substances. We
will consider such a system when it is at equilibrium, that is, when
it has settled down to a uniform temperature and when any physical
or chemical changes that may tend to occur at this temperature
have gone as far as they will go.
As a particular example of a physical system, we will consider
and idealized gas made up of a lot of little, infinitely small particles,
whizzing around every which way in a container.
The state of such a system is a complete description, or as complete
a description as the laws of physics allow, of the positions
and velocities of all of these particles. According to classical
mechanics (Newton’s laws of motion), each particle can have any
velocity and energy, so there is an uncountably infinite number of
states, as there is such an uncountable infinity of points in a line
or a square. According to quantum mechanics, there is an infinite
but countable number of states. Thus, the classical case is analogous
to the difficult communication theory of continuous signals,
while the more exact quantum case is analogous to the communication
theory of discrete signals which are made up of a countable
set of distinct, different symbols. We have dealt with the theory of
discrete signals at length in this book.
According to quantum mechanics, a particle of an idealized gas
can move with only certain energies. When it has one of these
allowed energies, it is said to occupy a particular energy /eve/. How
large will the entropy of such a gas be? If we increase the volume
of the gas, we increase the number of energy levels within a given
energy range. This increases the number of states the system can
be in at a given temperature, and hence it increases the entropy.
Such an increase in entropy occurs if a partition confining a gas
to a portion of a container is removed and the gas is allowed to
expand suddenly into the whole container.
If the temperature of a gas of constant volume is increased, the
particles can occupy energy levels of higher energy, so more combinations
of energy levels can be occupied; this increases the
number of states, and the entropy increases.
If a gas is allowed to expand against a slowly moving piston and
204
Symbols, Signals and Noise
no heat is added to the gas, the number of energy levels in a given
energy range increases, but the temperature of the gas decreases
just enough so as to keep the number of states and the entropy
the same.
We see that for a given temperature, a gas confined to a small
volume has less entropy than the same gas spread through a larger
volume. In the case of the one-molecule gas of Figure X-3, the
entropy is less when the door is closed and the molecule is confined
to the space on one side of the piston. At least, the entropy is less
if we know which side of the piston the molecule is on.
We can easily compute the decrease in entropy caused by halving
the volume of an ideal, one-molecule, classical gas at a given
temperature. In halving the volume we halve the number of states,
and the entropy changes by an amount
k log*. Vi = — 0.693 k
The corresponding change in free energy is the negative of T times
this change in entropy, that is,
0.693 kT
This is just the work that, according to 10.11, we can obtain by
halving the volume of the one-molecule gas and then letting it
expand against the piston until the volume is doubled again. Thus,
computing the change in free energy is one way of obtaining 10.11.
In reviewing our experience with the one-molecule heat engine
in this light, we see that we must transmit one bit of information
in order to specify on which side of the piston the molecule is. We
must transmit this information against a background of noise
corresponding to the uniform temperature 71 To do this takes
0.693 kT joule of energy.
Because we now know that the molecule is definitely on a particular
side of the piston, the entropy is 0.693 k less than it would
be if we were uncertain as to which side of the piston the molecule
was on. This reduction of entropy corresponds to an increase in free
energy of 0.693 kT joule. This free energy we can turn into work
by allowing the piston to move slowly to the unoccupied end of
the cylinder while the molecule pushes against it in repeated impacts.
At this point the entropy has risen to its original value, and
Information Theory and Physics
205
we have obtained from the system an amount of work which, alas,
is just equal to the minimum possible energy required to transmit
the information which told us on which side of the piston the
molecule was.
Let us now consider a more complicated case. Suppose that a
physical system has at a particular temperature a total of m states.
Suppose that we divide these states into n equal groups. The
number of states in each of these groups will be m/n.
Suppose that we regard the specification as to which one of the
n groups of states contains the state that the system is in as a
message source. As there are n equally likely groups of states, the
communication-theory entropy of the source is log n bits. This
means that it will take n binary digits to specify the particular
group of states which contains the state the system is actually in.
To transmit this information at a temperature T requires at least
.693 kTlog n = kTloge n
joule of energy. That is, the energy required to transmit the message
is proportional to the communication-theory entropy of the message
source.
If we know merely that the system is in one of the total of m
states, the entropy is
k loge m
If we are sure that the system is in one particular group of states
containing only m/n states (as we are after transmission of the
information as to which state the system is in), the entropy is
k log, ~ = k (log* m - loge «)
The change in entropy brought about by information concerning
which one of the n groups of states the system is in is thus
—k loge n
The corresponding increase in free energy is
k T loge n
But this is just equal to the least energy necessary to transmit the
206 Symbols, Signals and Noise
information as to which group of states contains the state the
system is in, the information that led to the decrease in entropy
and the increase in free energy.
We can regard any process which specifies something concerning
which state a system is in as a message source. This source
generates a message which reduces our uncertainty as to what state
the system is in. Such a source has a certain communication-theory
entropy per message. This entropy is equal to the number of binary
digits necessary to transmit a message generated by the source. It
takes a particular energy per binary digit to transmit the message
against a noise corresponding to the temperature T of the system.
The message reduces our uncertainty as to what state the system
is in, thus reducing the entropy (of statistical mechanics) of the
system. The reduction of entropy increases the free energy of the
system. But, the increase in free energy is just equal to the minimum
energy necessary to transmit the message which led to the
increase of free energy, an energy proportional to the entropy of
communication theory.
This, I believe, is the relation between the entropy of communication
theory and that of statistical mechanics. One pays a price
for information which leads to a reduction of the statistical-
mechanical entropy of a system. This price is proportional to the
communication-theory entropy of the message source which produces
the information. It is always just high enough so that a
perpetual motion machine of the second kind is impossible.
We should note, however, that a message source which generates
messages concerning the state of a physical system is one very
particular and peculiar kind of message source. Sources of English
text or of speech sounds are much more common. It seems irrelevant
to relate such entropies to the entropy of physics, except
perhaps through the energy required to transmit a bit of information
under highly idealized conditions.
There is one very odd implication of what we have just covered.
Clearly, the energy needed to transmit information about the state
of a physical system keeps us from ever knowing the past in complete
detail. If we can never know the past fully, can we declare
that the past is indeed unique? Or, is this a sensible question?
Information Theory and Physics
207
To summarize, in this chapter we have considered some of the
problems of communicating electrically in our actual physical
world. We have seen that various physical phenomena, including
lightning and automobile ignition systems, produce electrical disturbances
or noise which are mixed with the electrical signals we
use for the transmission of messages. Such noise is a source of
error in the transmission of signals, and it limits the rate at which
we can transmit information when we use a particular signal power
and band width.
The noise emitted by hot bodies (and any body is hot to a degree
if its temperature is greater than absolute zero) is a particularly
simple, universal, unavoidable noise which is important in all communication.
But, at extremely high frequencies there are quantum
effects as well as Johnson or thermal noise. We have seen how
these affect communication in the limit of infinite bandwidth, but
have no quantum analog of expression 10.4.
The use of the term entropy in both physics and communication
theory has raised the question of the relation of the two entropies.
It can be shown in a simple case that the limitation imposed by
thermal noise on the transmission of information results in the
failure of a machine designed to convert the chaotic energy of heat
into the organized energy of a lifted weight. Such a machine, if it
succeeded, would violate the second law of thermodynamics. More
generally, suppose we regard a source of information as to what
state a system is in as a message source. The information-theory
entropy of this source is a measure of the energy needed to transmit
a message from the source in the presence of the thermal noise
which is necessarily present in the system. The energy used in
transmitting such a message is as great as the increase in free
energy due to the reduction in physical entropy which the message
brings about.
CHAPTER A 1 Cybernetics
Some words have a heady quality; they conjure up strong
feelings of awe, mystery, or romance. Exotic used to be Dorothy
Lamour in a sarong. Just what it connotes currently I don’t know,
but I am sure that its meaning, foreign, is pale by comparison.
Palimpsest makes me think of lost volumes of Solomon’s secrets
or of other invaluable arcane lore, though I know that the word
means nothing more than a manuscript erased to make room for
later writing.
Sometimes the spell of a word or expression is untainted by any
clear and stable meaning, and through all the period of its currency
its magic remains secure from commonplace interpretations. Tao,
elan vital, and id are, I think, examples of this. I don’t believe that
cybernetics is quite such a word, but it does have an elusive quality
as well as a romantic aura.
The subtitle of Wiener’s book, Cybernetics, is Control and Communication
in the Animal and the Machine. Wiener derived the
word from the Greek for steersman. Since the publication of
Wiener’s book in 1948, cybernetics has gained a wide currency.
Further, if there is cybernetics, then someone must practice it, and
cyberneticist has been anonymously coined to designate such
a person.
What is cybernetics? If we are to judge from Wiener’s book it
includes at least information theory, with which we are now
reasonably familiar; something that might be called smoothing,
filtering, detection and prediction theory, which deals with finding
208
Cybernetics 209
the presence of and predicting the future value of signals, usually
in the presence of noise; and negative feedback and servomechanism
theory, which Wiener traces back to an early treatise on the
governor (the device that keeps the speed of a steam engine constant)
published by James Clerk Maxwell in 1868. We must, I
think, also include another field which may be described as
automata and complicated machines. This includes the design and
programming of digital computers.
Finally, we must include any phenomena of fife which resemble
anything in this list or which embody similar processes. This brings
to mind at once certain behavioral and regulatory functions of the
body, but Wiener goes much further. In his second autobiographical
volume, I Am a Mathematician, he says that sociology and
anthropology are primarily sciences of communication and therefore
fall under the general head of cybernetics, and he includes,
as a special branch of sociology, economics as well.
One could doubt Wiener’s sincerity in all this only with difficulty.
He had a grand view of the importance of a statistical approach
to the whole world of life and thought. For him, a current which
stems directly from the work of Maxwell, Boltzmann, and Gibbs
swept through his own to form a broad philosophical sea in
which we find even the ethics of Kierkegaard.
The trouble is that each of the many fields that Wiener drew
into cybernetics has a considerable scope in itself. It would take
many thousands of words to explain the history, content, and
prospects of any one of them. Lumped together, they constitute
not so much an exciting country as a diverse universe of overwhelming
magnitude and importance.
Thus, few men of science regard themselves as cyberneticists.
Should you set out to ask, one after another, each person listed in
American Men of Science what his field is, I think that few would
reply cybernetics. If you persisted and asked, “Do you work in the
field of cybernetics?” a man concerned with communication, or
with complicated automatic machines such as computers, or with
some parts of experimental psychology or neurophysiology would
look at you and speculate on your background and intentions. If
he decided that you were a sincere and innocent outsider, who
would in any event never get more than a vague idea of his work,
he might well reply, “yes.”
210
Symbols, Signals and Noise
So far, in this country the word cybernetics has been used most
extensively in the press and in popular and semiliterary, if not
semiliterate, magazines. I cannot compete with these in discussing
the grander aspects of cybernetics. Perhaps Wiener has done that
best himself in I Am a Mathematician. Even the more narrowly
technical content of the fields ordinarily associated with the word
cybernetics is so extensive that I certainly would never try to
explain it all in one book, even a much larger book than this.
In this one chapter, however, I propose to try to give some small
idea of the nature of the different technical matters which come
to mind when cybernetics is mentioned. Such a brief resume may
perhaps help the reader in finding out whether or not he is interested
in cybernetics and indicate to him what sort of information
he should seek in order to learn more about it.
Let us start with the part of cybernetics that I have called
smoothing, filtering, and prediction theory, which is an extremely
important field in its own right. This is a highly mathematical
subject, but I think that some important aspects of it can be made
pretty clear by means of a practical example.
Suppose that we are faced with the problem of using radar data
to point a gun so as to shoot down an airplane. The radar gives
us a sequence of measurements of position each of which is a little
in error. From these measurements we must deduce the course and
the velocity of the airplane, so that we can predict its position at
some time in the future, and by shooting a shell to that position,
shoot the plane down.
Suppose that the plane has a constant velocity and altitude.
Then the radar data on its successive locations might be the crosses
of Figure XI-1. We can by eye draw a line A B, which we would
guess to represent the course of the plane pretty well. But how are
we to tell a machine to do this?
If we tell a computing machine, or “computer,” to use just the
last and next-to-last pieces of radar data, represented by the points
L and NL, it can only draw a line through these points, the dashed
line A 'B'. This is clearly in error. In some way, the computer must
use earlier data as well.
The simplest way for the computer to use the data would be to
give an equal weight to all points. If it did this and fitted a straight
Cybernetics 211
X X '«
A
Fig. XI-1
line to all the data taken together, it might get a result such as that
shown in Figure XI-2. Clearly, the airplane turned at point T, and
the straight line AB that the computer computed has little to do
with the path of the plane.
We can seek to remedy this by giving more importance to recent
data than to older data. The simplest way to do this is by means
of linear prediction. In making a linear prediction, the computer
takes each piece of radar data (a number representing the distance
north or the distance east from the radar, for instance) and multiplies
it by another number. This other number depends on how
recent the piece of data is; it will be a smaller number for an old
piece of data than for a recent one. The computer then adds up
all the products it has obtained and so produces a predicted piece
of data (for instance, the distance north or east of the radar at some
future time).
212 Symbols, Signals and Noise
x
A
Fig. X I-2
The result of such prediction might be as shown in Figure XI-3.
Here a linear method has been used to estimate a new position and
direction each time a new piece of radar data, represented by a
cross, becomes available. Until another piece of data becomes
available, the predicted path is taken as a straight line proceeding
from the estimated location in the estimated direction. We see that
it takes a long time for the computer to take into account the fact
that the plane has turned at the point T, despite the fact that we
are sure of this by the time we have looked at the point next after T.
A linear prediction can make good use of old data, but, if it does
this, it will be slow to respond to new data which is inconsistent
with the old data, as the data obtained after an airplane turns will
be. Or a linear prediction can be quick to take new data strongly
into account, but in this case it will not use old data effectively,
even when the old data is consistent with the new data.
X
X
x
x
X
X-----X------X------X
T
Fig. X I-3
Cybernetics 213
To predict well even when circumstances change (as when the
airplane turns) we must use nonlinear prediction. Nonlinear prediction
includes all methods of prediction in which we don’t merely
multiply each piece of data used by a number depending on how
old the data is and then add the products.
As a very simple example of nonlinear prediction, suppose that
we have two different linear predictors, one of which takes into
account the last 100 pieces of data received, and the other of which
takes into account only the last ten pieces of data received. Suppose
that we use each predictor to estimate the next piece of data which
will be received. Suppose that we compare this next piece of data
with the output of each predictor. Suppose that we make use of
predictions based on 100 past pieces of data only when, three times
in a row, such predictions agree with each new piece of data better
than predictions based on ten past pieces of data. Otherwise, we
assume that the aircraft is maneuvering in such a way as to make
long-past data useless, and we use predictions based on ten past
pieces of data. This way of arriving at a final prediction is nonlinear
because the prediction is not arrived at simply by multiplying each
past piece of data by a number which depends only on how old
the data is. Instead, the use we make of past data depends on the
nature of the data received.
More generally, there are endless varieties of nonlinear prediction.
In fact, nonlinear prediction, and other nonlinear processes
as well, are the overwhelming total of all very diverse means after
the simplest category, linear prediction and other linear processes,
have been excluded. A great deal is known about linear prediction,
but very little is known about nonlinear prediction.
This very special example of predicting the position of an airplane
has been used merely to give a concrete sense of something
which might well seem almost meaningless if it were stated in more
abstract terms. We might, however, restate the broader problem,
which has been introduced in a more general way.
Let us imagine a number of possible signals. These signals might
consist of things as diverse as the possible paths of airplanes or the
possible different words that a man may utter. Let us also imagine
some sort of noise or distortion. Perhaps the radar data is inexact,
or perhaps the man speaks in a noisy room. We are required to
214
Symbols, Signals and Noise
estimate some aspect of the correct signal: the present or future
position of the airplane, the word the man just spoke, or the word
that he will speak next. In making this judgment we have some
statistical knowledge of the signal. This might concern what airplane
paths are most likely, or how often turns are made, or how
sharp they are. It might include what words are most common and
how the likelihood of their occurrence depends on preceding words.
Let us suppose that we also have similar statistics concerning noise
and distortion.
We see that we are considering exactly the sort of data that are
used in communication theory. However, given a source of data
and a noisy channel, the communication theorist asks how he can
best encode messages from the source for transmission over the
channel. In prediction, given a set of signals distorted by noise, we
ask, how do we best detect the true signal or estimate or predict
some aspect of it, such as its value at some future time?
The armory of prediction consists of a general theory of linear
prediction, worked out by Kolmogoroff and Wiener, and mathematical
analyses of a number of special nonlinear predictors. I
don’t feel that I can proceed very profitably beyond this statement,
but I can’t resist giving an example of a theoretical result (due to
David Slepian, a mathematician) which I find rather startling.
Let us consider the case of a faint signal which may or may not
be present in a strong noise. We want to determine whether or not
the signal is present. The noise and the signal might be voltages
or sound pressures. We assume that the noise and the signal have
been combined simply by adding them together. Suppose further
that the signal and the noise are ergodic (see Chapter III) and that
they are band limited—that is, they contain no frequencies outside
of a specified frequency range. Suppose further that we know
exactly the frequency spectrum of the noise, that is, what fraction
of the noise power falls in every small range of frequencies.
Suppose that the frequency spectrum of the signal is different
from this. Slepian has shown that if we could measure the over-all
voltage (or sound pressure) of the signal plus noise exactly for
every instant in any interval of time, however short the interval is,
we could infallibly tell whether or not the signal was present along
with the noise, no matter how faint the signal might be. This is a
Cybernetics 215
sound theoretical, not a useful practical, conclusion. However, it
has been a terrible shock to a lot of people who had stated quite
positively that, if the signal was weak enough (and they stated just
how weak), it could not be detected by examining the signal plus
noise for any particular finite interval of time.
Before leaving this general subject, I should explain why I
described it in terms offiltering and smoothing as well as prediction
and detection■ If the noise mixed with a signal has a frequency
spectrum different from that of the signal, we will help to separate
the signal from the noise by using an electrical filter which cuts
down on the frequencies which are strongly present in the noise
with respect to the frequencies which are strongly present in the
signal. If we use a filter which removes most or all high frequency
components (which vary rapidly with time), the output will not
vary so abruptly with time as the input; we will have smoothed the
combination of signal and noise.
So far, we have been talking about operations which we perform
on a set of data in order to estimate a present or future signal or
to detect a signal. This is, of course, for the purpose of doing
something.
We might, for instance, be flying an airplane in pursuit of an
enemy plane. We might use a radar to see the enemy plane. Every
time we take an observation, we might move the controls of the
plane so as to head toward the enemy.
A device which acts continually on the basis of information to
attain a specified goal in the face of changes is called a servomechanism.
Here we have an important new element, for the radar
data measures the position of the enemy plane with respect to our
plane, and the radar data is used in determining how the position
of our plane is to be changed. The radar data is fed back in such
a way as to alter the nature of radar data which will be obtained
later (because the data are used to alter the position of the plane
from which new radar data are taken). The feedback is called
negative feedback, because it is so used as to decrease rather than
to increase any departure from a desired behavior.
We can easily think of other examples of negative feedback. The
governor of a steam engine measures the speed of the engine. This
measured value is used in opening or closing the throttle so as to
216
Symbols, Signals and Noise
keep the speed at a predetermined value. Thus, the result of the
measurement of speed is fed back so as to change the speed. The
thermostat on the wall measures the temperature of the room and
turns the furnace off or on so as to maintain the temperature at a
constant value. When we walk carrying a tray of water, we may
be tempted to watch the water in the tray and try to tilt the tray
so as to keep the water from spilling. This is often disastrous. The
more we tilt the tray to avoid spilling the water, the more wildly
the water may slosh about. When we apply feedback so as to
change a process on the basis of its observed state, the over-all
situation may be unstable. That is, instead of reducing small deviations
from the desired goal, the control we exert may make
them larger.
This is a particularly hazardous matter in feedback circuits. The
thing we do to make corrections most complete and perfect is to
make the feedback stronger. But this is the very thing that tends
to make the system unstable. Of course, an unstable system is no
good. An unstable system can result in such behavior as an airplane
or missile veering wildly instead of following the target, the temperature
of a room rising and falling rapidly, an engine racing or
coming to a stop, or an amplifier producing a singing output of
high amplitude when there is no input.
The stability of negative-feedback systems has been studied
extensively, and a great deal is known about linear negative-
feedback systems, in which the present amplitude is the sum of
past amplitudes multiplied by numbers depending only on remoteness
from the present.
Linear negative-feedback systems are either stable or unstable,
regardless of the input signal applied. Nonlinear feedback systems
can be stable for some inputs but unstable for others. A shimmying
car is an example of a nonlinear system. It can be perfectly stable
at a given speed on a smooth road, and yet a single bump can start
a shimmy which will persist indefinitely after the bump has been
passed.
Oddly enough, most of the early theoretical work on negative-
feedback systems was done in connection with a device which has
not yet been described. This is the negative feedback amplifier,
which was invented by Harold Black in 1927.
Cybernetics 217
The gain of an amplifier is the ratio of the output voltage to the
input voltage. In telephony and other electronic arts, it is important
to have amplifiers which have a very nearly constant gain. However,
vacuum tubes and transistors are imperfect devices. Their
gain changes with time, and the gain can depend on the strength
of the signal. The negative feedback amplifier reduces the effect
of such changes in the gain of vacuum tubes or transistors.
We can see very easily why this is so by examining Figure XI-4.
At the top we have an ordinary amplifier with a gain of ten times.
If we put in 1 volt, as shown by the number to the left, we get out
10 volts, as shown by the number to the right. Suppose the gain
of the amplifier is halved, so that the gain is only five times, as
shown next to the top. The output also falls to one half, or 5 volts,
in just the same ratio as the gain fell.
The third drawing from the top shows a negative feedback
amplifier designed to give a gain of ten times. The upper box has
Fig. X I-4
218
Symbols, Signals and Noise
a high gain of one hundred times. The output of this box is connected
to a very accurate voltage-dividing box, which contains no
tubes or transistors and does not change with time or signal level.
The input to the upper box consists of the input voltage of 1 volt
less the output of the lower box, which is 0.09 times the output
voltage of 10 volts; this is, of course, 0.9 volt.
Now, suppose the tubes or transistors in the upper box change
so that they give a gain of only fifty times instead of one hundred
times; this is shown at the bottom of Figure XI-4. The numbers
given in the figure are only approximate, but we see that when the
gain of the upper box is cut in half the output voltage falls only
about 10 per cent. If we had used a higher gain in the upper box
the effect would have been even less.
The importance of negative feedback can scarcely be overestimated.
Negative feedback amplifiers are essential in telephonic
communication. The thermostat in your home is an example of
negative feedback. Negative feedback is used to control chemical
processing plants and to guide missiles toward airplanes. The
automatic pilot of an aircraft uses negative feedback in keeping
the plane on course.
In a somewhat broader sense, I use negative feedback from eye
to hand in guiding my pen across the paper, and negative feedback
from ear to tongue and lips in learning to speak or in imitating
the voice of another. The animal organism makes use of negative
feedback in many other ways. This is how it maintains its temperature
despite changes in outside temperature, and how it maintains
constant chemical properties of the blood and tissues. The ability
of the body to maintain a narrow range of conditions despite
environmental changes has been called homeostasis.
G. Ross Ashby, one of the few self-acknowledged cyberneticists,
built a machine called a homeostat to demonstrate features of
adjustment to environment which he believes to be characteristic
of life. The homeostat is provided with a variety of feedback
circuits and with two means for changing them. One is under the
control of the homeostat; the other is under the control of a person
who acts as the machine’s “environment.” If the machine’s circuits
are so altered by changes of its “environment” as to make it
unstable, it readjusts its circuits by trial and error so as to attain
stability again.
Cybernetics
219
We may if we wish liken this behavior of the homeostat to that
of a child who first learns to walk upright without falling and then
learns to ride a bicycle without falling or to many other adjustments
we make in life. In his book Cybernetics, Wiener puts great
emphasis on negative feedback as an element of nervous control
and on its failure as an explanation of disabilities, such as tremors
of the hand, which are ascribed to failures of a negative feedback
system of the body.
We have so far discussed three constituents of cybernetics:
information theory, detection and prediction, including smoothing
and filtering, and negative feedback, including servomechanisms
and negative feedback amplifiers. We usually also associate electronic
computers and similar complex devices with cybernetics.
The word automata is sometimes used to refer to such complicated
machines.
One can find many precursors of today’s complicated machines
in the computers, automata, and other mechanisms of earlier
centuries, but one would add little to his understanding of today’s
complex devices by studying these precursors. Human beings learn
by doing and by thinking about what they have done. The opportunities
for doing in the field of complicated machines have been
enhanced immeasurably beyond those of previous centuries, and
the stimulus to thought has been wonderful to behold.
Recent advances in complicated machines might well be traced
to the invention of automatic telephone switching late in the last
century. Early telephone switching systems were of a primitive,
step-by-step form, in which a mechanism set up a new section of
a link in a talking path as each digit was dialed. From this, switching
systems have advanced to become common-control systems. In
a common-control switching system, the dialed number does not
operate switches directly. It is first stored, or represented electrically
or mechanically, in a particular portion of the switching
system. Electrical apparatus in another portion of the switching
system then examines different electrical circuits that could be used
to connect the calling party to the number called, until it finds one
that is not in use. This free circuit is then used to connect the calling
party to the called party.
Modern telephone switching systems are of bewildering complexity
and overwhelming size. Linked together to form a nation
220
Symbols, Signals and Noise
wide telephone network which allows dialing calls clear across the
country, they are by far the most complicated construction of man.
It would take many words to explain how they perform even a few
of their functions. Today, a few pulls of a telephone dial will cause
telephone equipment to seek out the most economical available
path to a distant telephone, detouring from city to city if direct
paths are not available. The equipment will establish a connection,
ring the party, time the call, and record the charge in suitable units,
and it will disconnect the circuits when a party hangs up. It will
also report malfunctioning of its parts to a central location, and it
continues to operate despite the failure of a number of devices.
One important component of telephone switching systems is the
electric relay. The principal elements of a relay are an electromagnet,
a magnetic bar to which various movable contacts are
attached, and fixed contacts which the movable contacts can
touch, thus closing circuits. When an electric current is passed
through the coil of the electromagnet of the relay, the magnetic
bar is attracted and moves. Some moving contacts move away from
the corresponding fixed contacts, opening circuits; other moving
contacts are brought into contact with the corresponding fixed
contacts, closing circuits.
In the thirties, G. R. Stibitz of the Bell Laboratories applied the
relays and other components of the telephone art to build a complex
calculator, which could add, subtract, multiply, and divide
complex numbers. During World War II, a number of more complicated
relay computers were built for military purposes by the
Bell Laboratories, while, in 1941, Howard Aiken and his coworkers
built their first relay computer at Harvard.
An essential step in increasing the speed of computers was
taken shortly after the war when J. P. Eckert and J. W. Mauchly
built the Eniac, a vacuum tube computer, and more recently
transistors have been used in place of vacuum tubes.
Thus, it was an essential part of progress in the field of complex
machines that it became possible to build them and that they were
built, first by using relays and then by using vacuum tubes and
transistors.
The building of such complex devices, of course, involved more
than the existence of the elements themselves; it involved their
Cybernetics 221
interconnection to do particular functions such as multiplication
and division. Stibitz’s and Shannon’s application of Boolean algebra,
a branch of mathematical logic, to the description and design
of logic circuits has been exceedingly important in this connection.
Thus, the existence of suitable components and the art of interconnecting
them to carry out particular functions provided, so to
speak, the body of the complicated machine. The organization, the
spirit, of the machine is equally essential, though it would scarcely
have evolved in the absence of the body.
Stibitz’s complex calculator was almost spiritless. The operator
sent it pairs of complex numbers by teletype, and it cogitated and
sent back the sum, difference, product, or quotient. By 1943,
however, he had made a relay computer which received its instructions
in sequence by means of a long paper tape, or program, which
prescribed the numbers to be used and the sequences of operations
to be performed.
A step forward was taken when it was made possible for the
machine to refer back to an earlier part of the program tape on
completing a part of its over-all task or to use subsidiary tapes to
help it in its computations. In this case the computer had to make
a decision that it had reached a certain point and then act on the
basis of the decision. Suppose, for instance, that the computer was
computing the value of the following series by adding up term
after term:
+ i/,i + ...
We might program the computer so that it would continue adding
terms until it encountered a term which was less than 1/1,000,000
and then print out the result and go on to some other calculation.
The computer could decide what to do next by subtracting the
latest term computed from 1/1,000,000. If the answer was negative,
it would compute another term and add it to the rest; if the
answer was positive, it could print out the sum arrived at and refer
to the program for further instructions.
The next big step in the history of computers is usually attributed
to John von Neumann, who made extensive use of early computers
in carrying out calculations concerning atomic bombs. Even early
computers had memories, or stores, in which the numbers used in
222
Symbols, Signals and Noise
intermediate steps of a computation were retained for further
processing and in which answers were stored prior to printing them
out. Von Neumann’s idea was to put the instructions, or program,
of the machine, not on a separate paper tape, but right into the
machine’s memory. This made the instructions easily and flexibly
available to the machine and made it possible for the machine to
modify parts of its instructions in accordance with the results of
its computations.
In old-fashioned desk calculating machines, decimal digits were
stored on wheels which could assume any of ten distinct positions
of rotation. In today’s desk and hand calculators, numbers are
stored in binary form in some of the tens of thousands of solid-
state circuit elements of a large-scale integrated circuit chip. To
retain numbers in such storage, electric power (a minute amount)
must be supplied continually to the integrated circuit. Such storage
of data is sometimes called volatile (the data disappears when
power is interrupted). But, volatile storage can be made very reliable
by guarding against interruption of power.
Magnetic core memory-arrays of tiny rings of magnetic material
with wires threaded through them provide nonvolatile metnory in
computers larger than hand calculators and desk calculators. An
interruption of power does not change the magnetization of the
cores, but an electric transient due to faulty operation can change
what is in memory.
Integrated circuit memory and core memory are random access
memory. Any group of binary digits (a byte is 8 consecutive digits),
or a word (of 8, 16, 32 or more digits) can be retrieved in a fraction
of a microsecond by sending into the memory a sequence of
binary digits constituting an address. Such memory is called
random access storage.
In the early days, sequences of binary digits were stored as holes
punched in paper tape. Now they are stored as sequences of
magnetized patterns on magnetic tape or magnetic disks. Cassettes
similar to or identical with audio cassettes provide cheap storage
in small “recreational” computers. Storage on tape, disk or cassette
is sequential', one must run past a lot of bits or words to get to the
one you want. Sequential storage is slow compared with random
access storage. It is used to store large amounts of data, or to
Cybernetics
223
store programs or data that will be loaded into fast, random
access storage repeatedly. Tape storage is also used as back up
storage, to preserve material stored in random access storage and
disk storage that might be lost through computer malfunction.
Besides memory or storage, computers have arithmetic units
which perform various arithmetical and logical operations, index
registers to count operations, input and output devices (keyboards,
displays and printers) control units which give instructions to other
units and receive data from them, and can have many specialized
pieces of hardware to do specialized chores such as doing floating
point arithmetic, performing Fourier analyses or inverting matrices.
The user of a computer who wishes it to perform some useful or
amusing task must write a program that tells the computer in explicit
detail every elementary operation it must go through in
attaining the desired result. Because a computer operates internally
in terms of binary numbers, early programmers had to write out
long sequences of binary instructions, laboriously and fallibly.
But, a computer can be used to translate sequences of letters and
decimal digits into strings of binary digits, according to explicit
rules. Moreover, subroutines can be written that will, in effect, cause
the computer to carry out desirable operations (such as dividing,
taking square root, and so on) that are not among the basic commands
or functions provided by the hardware. Thus, assemblers
or assembly language was developed. Assembly language is called
machine language, though it is one step removed from the binary
numbers that are stored in and direct the operations of the
hardware.
When one writes a program in assembly language, he still has to
write every step in sequence, though he can specify that various
paths be taken through the program depending on the outcome of
one step (whether a computed number is larger than, smaller than
or equal to another, for example). But, the numbers he writes are
in decimal form, and the instructions are easily remembered sequences
of letters such as CLA (clear add, a command to set the
accumulator to zero and add to it the number in a specified memory
location).
Instructing a computer in machine (assembly) language is a
laborious task. Computers are practical because they have operat
224
Symbols, Signals and Noise
ing systems, by means of which a few simple instructions will cause
data to be read in, to be read out, perform other useful functions,
and because programs (even operating systems themselves) are
written in a higher-order language.
There are many higher-order languages. Fortran (formula translation)
is one of the oldest and most persistent. It is adapted to
numerical problems. A line of Fortran might read
40 IF SIN(X) < M THEN 80
The initial number 40 is the number of the program step. It
instructs the computer to go to program step 80 if sin x is less
than M.
Basic is a widely used language similar to but somewhat simpler
than Fortran. The C language is useful in writing operating systems
as well as in numerical calculations. Languages seem to be easiest
to use and most efficient (in computer running time) when they
are adapted to particular uses, such as numerical calculations or
text manipulation or running simulations of electrical or mechanical
or economic systems.
One gets from a program in a higher-order language to one in
machine language instructions by means of a compiler, which translates
a whole program, or by an interpreter, which translates the
program line by line. Compilers are more efficient and more common.
One writes programs by means of a text editor (or editing
program) which allows one to correct mistakes—to revise the program
without keying every symbol in afresh.
Today many children learn to program computers in grade
school, more in high school and more still in college. You can’t
get through Dartmouth and some other colleges without learning
to use a computer, even if you are a humanities major.
More and more, computers are used for tasks other than scientific
calculations or keeping books. Such tasks include the control
of spacecraft and automobile engines, the operation of chemical
plants and factories, keeping track of inventory, making airline
and hotel reservations, making word counts and concordances,
analyzing X-ray data to produce a three-dimensional image (CAT
= computer-aided tomography), composing (bad) music, pro
Cybernetics 225
ducing attractive musical sounds, producing engaging animated
movies, playing games, new and old, including chess, reading
printed or keyed-in text aloud, and a host of other astonishing
tasks. Through large-scale integration, simple computers have
become so cheap that they are incorporated into attractive toys.
While children learn a useful amount of programming with little
difficulty, great talent and great skill are required in order to get
the most out of a computer or a computerlike device, or to accomplish
a given task with a smaller and cheaper computer.
Today, far more money and human effort are spent in producing
software (that is, programs of all sorts) than are spent on the
hardware the programs run on. One talented programmer can
accomplish what ten or a hundred merely competent programmers
cannot do. But, there aren’t enough talented programmers to go
around. Indeed, programmers of medium competence are in short
supply.
Even the most talented programmers haven’t been able to make
computers do some things. In principle, a computer can be programmed
to do anything which the programmer understands in
detail. Sometimes, of course, a computation may be too costly or
too time-consuming to undertake. But often we don’t really know
in detail how to do something. Thus, things that a computer has not
done so far include typing out connected speech, translating satisfactorily
from one language to another, identifying an important
and interesting mathematical theorem or composing interesting
music.
Efforts to use the computer to perform difficult tasks that are
not well understood has greatly stimulated human thought concerning
such problems as the recognition of human words, the structure
of language, the strategy of winning games, the structure of music
and the processes of mathematical proof.
Further, the programming of computers to solve complicated
and unusual problems has given us a new and objective criterion
of understanding. Today, if a man says that he understands how
a human being behaves in a given situation or how to solve a
certain mathematical or logical problem, it is fair to insist that he
demonstrate his understanding by programming a computer to
imitate the behavior or to accomplish the task in question. If he
226
Symbols, Signals and Noise
is unable to do this, his understanding is certainly incomplete, and
it may be completely illusory.
Will computers be able to think? This is a meaningless question
unless we say what we mean by to think. Marvin Minsky, a freewheeling
mathematician who is much interested in computers and
complex machines, proposed the following fable. A man beats
everyone else at chess. People say, “How clever, how intelligent,
what a marvelous mind he has, what a superb thinker he is.” The
man is asked, “How do you play so that you beat everyone?” He
says, “I have a set of rules which I use in arriving at my next move.”
People are indignant and say, “Why that isn’t thinking at all; it’s
just mechanical.”
Minsky’s conclusion is that people tend to regard as thinking
only such things as they don’t understand. I will go even further
and say that people frequently regard as thinking almost any
grammatical jumbling together of “important” words. At times I’d
settle for a useful, problem-solving type of “thinking,” even if it was
mechanical. In any event, it seems likely that philosophers and
humanists will manage to keep the definition of thinking perpetually
applicable to human beings and a step ahead of anything a
machine ever manages to do. If this makes them happy, it doesn’t
offend me at all. I do think, however, that it is probably impossible
to specify a meaningful and explicitly defined goal which a man
can attain and a computer cannot, even including the “imitation
game” proposed by A. M. Turing, a British logician, in 1936.
In this game a man is in communication, say by teletype, with
either a computer or a man, he doesn’t know which. The man tries
by means of questions to discover whether he is in touch with a
man or a machine; the computer is programmed to deceive the
man. Certainly, however, a computer programmed to play the
imitation game with any chance of success is far beyond today’s
computers and today’s art of programming, and it belongs to a very
distant future, if to any.
We have seen that cybernetics is a very broad field indeed. It
includes communication theory, to which we are devoting a whole
book. It includes the complicated field of smoothing and prediction,
which is so important in radar and in many other military
applications. When we try to estimate the true position or the
Cybernetics 227
future position of an airplane on the basis of imperfect radar data,
we are, according to Wiener, dealing with cybernetics. Even in
using an electrical filter to separate noise of one frequency from
signals of another frequency, we are invoking cybernetics.
It is in this general field that the contribution of Wiener himself
was greatest, and his great work was a general theory of prediction
by means of linear devices, which makes a prediction merely by
multiplying each piece of data by a number which is smaller the
older the data is and adding the products.
Another part of cybernetics is negative feedback. A thermostat
makes use of negative feedback when it measures the temperature
of a room and starts or stops the furnace in order to make the
temperature conform to a specified value. The autopilots of airplanes
use negative feedback in manipulating the controls in order
to keep the compass and altimeter readings at assigned values.
Human beings use negative feedback in controlling the motions
of their hands to achieve certain ends.
Negative feedback devices can be unstable; the effect of the
output can sometimes be to make the behavior diverge widely
from the desired goal. Wiener attributes tremors and some other
malfunctioning of the human being to improper functioning of
negative feedback mechanisms.
Negative feedback can also be used in order to make the large
output signal of an amplifier conform closely in shape to the small
input. Negative feedback amplifiers were extremely important in
communication systems long before the day of cybernetics.
Finally, cybernetics has laid claim to the whole field of automata
or complex machines, including telephone switching systems,
which have been in existence for many years, and electronic
computers, Much have been with us only since World War II.
If all this is so, cybernetics includes most of the essence of
modern technology, excluding the brute producton and use of
power. It includes our knowledge of the organization and function
of man as well. Cybernetics almost becomes another word for all of
the most intriguing problems of the world. As we have seen,
Wiener includes sociological, philosophical, and ethical problems
among these.
Thus, even if a man acknowledged being a cyberneticist, that
228 Symbols, Signals and Noise
wouldn’t give us much of a clue concerning his field of competence,
unless he was a universal genius. Certainly, it would not necessarily
indicate that he had much knowledge of information theory.
Happily, as I have noted, few scientists would acknowledge
themselves as cyberneticists, save perhaps in talking to those whom
they regard as hopelessly uninformed. Thus, if cybernetics is over-
extensive or vague, the overextension or vagueness will do no real
harm. Indeed, cybernetics is a very useful word, for it can help to
add a little glamor to a person, to a subject, or even to a book. I
certainly hope that its presence here will add a little glamor to
this one.
CHAPTER A .1 1 Information Theory
and Psychology
I have read a good deal more about information theory and
psychology than I can or care to remember. Much of it was a mere
association of new terms with old and vague ideas. Presumably
the hope was that a stirring in of new terms would clarify the old
ideas by a sort of sympathetic magic.
Some attempted applications of information theory in the field
of experimental psychology have, however, been at least reasonably
well informed. They have led to experiments which produced
valid data. It is hard to draw any conclusions from these data that
are both sweeping and certain, but the data do form a basis or at
least an excuse for interesting speculations. In this chapter, I
propose to discuss some experiments concerning information
theory and psychology which are at least down-to-earth enough
to grapple with. Naturally I have chosen these largely on the basis
of my personal interest and background, but one has to impose
some limitations in order to say anything coherent about a broad
and less than pellucid field.
It seems to me that an early reaction of psychologists to information
theory was that, as entropy is a wonderful and universal
measure of amount of information and as human beings make use
of information, in some way the difficulty of a task, perhaps the
time a man takes to accomplish a set task, must be proportional
to the amount of information involved.
229
230
Symbols, Signals and Noise
This idea is very clearly illustrated in some experiments reported
by Ray Hyman, an experimental psychologist, in the Journal of
Experimental Psychology in 1953. Here I shall describe only one
of several of the experiments Hyman made.
A number of lights were placed before a subject, as psychologists
call an experimentee or laboratory human animal. Each light was
labeled with a monosyllabic “name” with which the subject became
familiar. After a warning signal, one of the several lights flashed,
and the subject thereafter spoke the name of the light as soon as
he could. The time interval between the flashing of the light and
the speaking of the name was measured.
Sometimes one out of eight lights flashed, the light being chosen
at random with equal probabilities. In this case, the information
conveyed in enabling the subject to identify the light correctly was
log 8, or 3 bits. Sometimes one among 7 light flashed (2.81 bits),
sometimes among 6 (2.58 bits), sometimes one out of 5 (2.32 bits),
one out of 4 (2.00 bits), one out of 3 (1.58 bits), one out of 2 (1 bit),
or one out of 1 (0 bits). The average response time, or latency,
between the lighting of the light and the speaking of its name was
plotted against number of bits, as shown in Figure XII-1.
Clearly, there is a certain latency, or response time, even when
only one light is used, the choice among lights is certain, and the
information conveyed as to which light is lighted is zero. When
more lights are used, the increase in latency is proportional to the
information conveyed. Such an increase of latency with the logarithm
of the number of alternatives had in fact been noted by a
German psychologist, J. Merkel, in 1885. It is certainly a strikingly
accurate, reproducible, and a significant aspect of human response.
We note from Figure XII-1 that the increase in latency is about
0.15 second per bit. Some unwary psychologists have jumped to
the conclusion that it takes 0.15 second for a human being to
respond to 1 bit of information; therefore, the information capacity
of a human being is about 1 / . 15, or about 7 bits per second. Have
we discovered a universal constant of human perception or of
human thought?
Clearly, in Hyman’s experiment the increase in latency is proportional
to the uncertainty of the stimulus measured in bits.
However, various experiments by various experimenters give
Information Theory and Psychology 231
800
</) a
z
8 Ul If)
o
o
o
z
Ui
2
P
Z
o
b <
at
oc
0 .5 1.0 1.5 2.0 2.5 3-0 3.5
BITS PER STIMULUS PRESENTATION
Fig. XII-1
somewhat different rates of increase in seconds per bit. Moreover,
data published by G. H. Mowbray and M. V. Rhoades in 1959
show that, after much practice, a subject’s performance tends to
change so that there is little or no effect of information content
on latency. It appears that human beings may have different ways
of handling information, a way used in learning, in which number
of alternatives is very important, and a way used after much learning,
in which number of alternatives, up to a fairly large number,
makes little difference. Further, in one sort of experiment, in which
a subject depresses one or more keys on which his fingers rest in
response to a vibration of the key, it appears that there may be
little increase in latency with amount of information right from
the start.
232 Symbols, Signals and Noise
Moreover, even if the latency were a constant plus an increment
proportional to information content, one could not reasonably
assert that this showed that a significant information rate can be
obtained by dividing the increased time by the number of bits. We
will see that this can lead to fantastic information rates in the sort
of experiment which I shall describe next.
H. Quastler made early information-rate experiments in which
subjects played random sequences of notes or chords or read lists
of randomly chosen words as rapidly as possible, and J. C. R.
Licklider did early work on both reading and pointing speed.
Before we heard of this work, J. E. Karlin and I embarked on an
extensive series of experiments on reading lists of words, which of
all experiments gives the highest observed information rate, a rate
which is much higher than, for instance, sending Morse code
or typing.
Suppose the “sender” of the message chooses an alphabet of,
say, 16 words and makes up a list by choosing words among these
randomly and with equal probabilities. Then, the amount of choice
in designating each word is log 16, or 4 bits. The subject “transmits”
the information, translating it into a new form, speech rather
than print, by reading the list aloud. If he can read at a rate of 4
words a second, for instance, he transmits information at a rate
of 4 x 4, or 16 bits per second.
Figure XII-2 shows data from three subjects. The words were
chosen from the 500 most common words in English. We see that
while the reading rate drops somewhat in going from 2 to 4 word
vocabularies (or from 1 to 2 bits per word), it is almost constant
for vocabularies or alphabets containing from 4 to 256 words
(from 2 to 8 bits per word).
Let us now remember the alleged means for getting an information
rate from such data as Hyman’s, that is, noting the increase
in time with increase in bits per stimulus. Consider the dotted
average data curve of Figure XII-2. In going from 2 bits per
stimulus to 8 bits per stimulus the reading rate doesn’t decrease
at all; that is, the change in reading time per word is 0, despite an.
increase of 6 in the number of bits per word. If we divide 6 by 0,
we get an information rate of infinity! Of course, this is ridiculous,
but it is scarcely more ridiculous than deducing an information
BITS PER WORD
12.3
WORDS PER SECOND
$
Information Theory and Psycholo:
?
____
^OOO
WORD
DICTIONARY
Fig. XII-2
K> OJ OJ
234
Symbols, Signals and Noise
rate from such data as Hyman’s by dividing increase in number of
bits by increase in latency.
Directly from Figure XII-2, we can see that as reader A reads
8-bit words at a rate of 3.8 per second, he manages to transmit
information at a rate of 8 X 3.8, or about 30 bits per second.
Moreover, when the words put in the list are chosen randomly from
a 5,000 word dictionary (12.3 bits per word), he manages to read
them at a rate of 2.7 per second, giving a higher information rate
of 33 bits per second.
It is clear that no unique information rate can be used to describe
the performance of a human being. He can transmit (and, we shall
see, respond to or remember) information better under some
circumstances than under others. We can best consider him as an
information-handling channel or device having certain built-in
limitations and properties. He is a very flexible device; he can
handle information quite well in a variety of forms, but he handles
it best if it is properly encoded, properly adjusted to his capabilities.
What are his capabilities? We see from Figure XII-2 that he is
slowed down only a little by increasing complexity. He can read
a list of words chosen randomly from an alphabet of 256 about as
fast as words chosen from an alphabet of 4. He isn’t very speedy
compared with machines, and in order to make him perform well
we must give him a complex task. This is just what we might
have expected.
Complexity eventually does slow him down, however, as we see
from the points for an alphabet consisting of all the words in a
5,000 word dictionary. Perhaps there is an optimum alphabet or
vocabulary, which has quite a number of bits per word, but not
so many words as to slow a man down unduly. Partly to help in
finding such a vocabulary, Karlin and I measured reading rate as
a function of both number of syllables and “familiarity,” that is,
whether the word came from the first thousand in order of
commonness of occurrence or familiarity, from the tenth thousand,
or from the nineteenth thousand. The results are shown
in Figure XII-3.
We see that while an increase in number of syllables slows down
reading speed, a decrease in familiarity has just as pronounced an
effect. Thus, a vocabulary of familiar one syllable words would
Information Theory and Psychology 235
(a) READER A
FAMILIARITY
O 1 - 1,000
A 10,000-11,000
P 19,000-20,000
i
1 R n
N
-----------=
WORDS PER SECOND
5.0
4.5
4.0
3.5
3.0
2.5
2.0
1.5
(c) R EADER C
i
1
k
I
1 5
2 3 4
SYLLABLES PER WORD
F ig. X I I -3
236
Symbols, Signals and Noise
seem to be a good choice. Using the 2,500 most common monosyllables
(2,500 words means 11.3 bits per word) as a “preferred
vocabulary,” a reader attained a reading speed of 3.7 words per
second, giving an information rate of 42 bits per second.
“Scrambled prose,” that is, words chosen with the same probabilities
as in nontechnical prose but picked at random without
grammatical connection, also gave a high information rate. The
entropy is about 11.8 bits per word, the highest reading rate was
3.7 words per second, and the corresponding information rate is
44 bits per second.
Perhaps one could gain a little by improving the alphabet, but
I don’t think one would gain much. At any rate, these experiments
gave the highest information rate which has been demonstrated.
It is a rate slow by the standards of electrical communication, but
it does represent a tremendous number of binary choices—around
2,500 a minute!
What, we may ask, limits the rate? Is it reading through each
word letter by letter? In this case the Chinese, who have a single
sign for each word, might be better off. But Chinese who read both
English and Chinese with facility read randomized lists of common
Chinese characters and randomized lists of the equivalent English
words at almost exactly the same speed.
Is the limitation a mechanical one? Figure XII-4 shows rates for
several tasks. A man can repeat a memorized phrase over twice
as fast as he can read randomized words from the preferred list,
and he can read prose appreciably faster. It appears that the
limitation on reading rate is mental rather than mechanical.
So far, it appears that we cannot characterize a human being
by means of a particular information rate. While the difficulty of
a task ultimately increases with its information content, the difficulty
depends markedly on how well the task is tailored to human
abilities. The human being is very flexible in ability, but he has to
strain and slow down to do unusual things. And he is quite good
at complexity but only fair at speed.
One way of tailoring a task to human abilities is by deliberate,
thoughtful experiments. This is analogous to the process of so
encoding messages from a message source as to attain the highest
possible rate of information transmission over a noisy channel.
Information Theory and Psychology 237
WORDS PER SECOND
This was discussed in Chapter VIII, and the highest attainable
rate was called the channel capacity. The “preferred list” of the
2,500 most frequently used monosyllables was devised in a deliberate
effort to attain a high information rate in reading aloud
randomized lists of words.
We may note, however, that choosing words at random with the
probabilities of their occurrence in English text gives as high or a
little higher information rate. Have the words of the English language
and their frequencies of occurrence been in some way fitted
238 Symbols, Signals and Noise
to human abilities by a long process of unconscious experiment
and evolution?
We have seen in Chapter V that the probability of occurrence
of a word in English text is very nearly inversely proportional to
its rank. That is, the hundredth most common word occurs about
a hundredth as frequently as the most common word, and so on.
Figure V-2 illustrates this relation, which was first pointed out by
George Kingsley Zipf, who ascribed it to a principle of least effort.
Clearly, Zipf’s law cannot be entirely correct in this simple
form. We saw in Chapter V that word probabilities cannot be
inversely proportional to the rank of the word for all words; if they
were, the sum of the probabilities of all words would be greater
than unity. There have been various attempts to modify, derive,
and explain Zipf’s law, and we will discuss these somewhat later.
However, we will at first regard Zipf’s law in its original and
simplest form as an approximate description of an aspect of human
behavior in generating language, a description which Zipf arrived
at empirically by examining the statistics of actual text.
Zipf, as we have noted, associated his law with a principle of least
effort. Attempts have been made to identify the effort or “cost”
of producing text with the number of letters in text. However,
most linguists regard language primarily as the spoken language,
and it seems unlikely that speaking, reading, or writing habits are
dictated primarily by the numbers of letters used in words.
In fact, we noted in the information-rate experiments which we
just considered that reading rates are about the same for common
Chinese ideographs and for the equivalent words in English written
out alphabetically. Further, we have noted from Figure XII-3 that
commonness or familiarity has an influence on reading time as
great as does number of syllables.
Could we not, perhaps, take reading time as a measure of effort?
We might think, for instance, that common words are more easily
accessible to us, that they can be recognized or called forth with
less effort or cost than uncommon words. Perhaps the human brain
is so organized that a few words can be stored in it in such a
fashion that they can be recognized and called forth easily and that
many more can be stored in a fashion which makes their use less
easy. We might believe that reading time is a measure of accessibility,
ease of use, of cost.
Information Theory and Psychology
239
We might imagine, further, that in using language, human beings
choose words in such a way as to transmit as much information
as possible for a given cost. If we identify cost with time of utterance,
we would then say that human beings choose words in such
a way as to convey as much information as possible in a given
time of speaking or in a given time of reading aloud.
It is an easy mathematical task to show that if a speaking time
tr is associated with the rth word in order of commonness, then
for a message composed of randomly chosen words the information
rate will be greatest if the rth word is chosen with a probability
p{r) given by
p(r) = 2 ~ctr (12.1)
Here c is a constant chosen to make the sum of the probabilities
for all words add up to unity. This mathematical relation says that
words with a long reading time will be used less frequently than
words with a short reading time, and it gives the exact relation
which must hold if the information rate is to be maximized.
Now, if Zipf’s law holds, the probability of occurrence of the
rth word in order of commonness must be given by
p(r) = ± (12.2)
Here A is another constant. Thus, from 12.1 and 12.2 we must have
y = 2“ctr (12.3)
By using a relation given in the Appendix, this relation can be
re-expressed KjA = J^r — ctr
tr — a + b log r (12.4)
Here a and b are constants which must be determined by examining
the relation of the reading time tr and the order of commonness
or rank of a word, r. If Zipf’s law is true and if the information
rate is maximized for words chosen randomly and independently
with probabilities given by Zipf’s law, then relation 12.4 should
hold for experimental data.
Of course, words aren’t chosen randomly and independently in
constructing English text, and hence we cannot say that word
240
Symbols, Signals and Noise
probabilities in accord with relation 12.1 actually would maximize
information transmission per unit time. Nonetheless, it would be
interesting to know whether predictions based on a random and
independent choice of words do hold for the reading of actual
English text.
Benoit Mandelbrot, a mathematician much interested in linguistic
problems, has considered this matter in connection with
reading-time data taken by D. H. Howes, an experimental psychologist.
R.R. Riesz, an experienced experimenter in the field of
psychophysics, and I have also attempted to compare equation
12.4 with human behavior.
There is a difficulty in making such a comparison. It seems
fairly clear that reading speed is limited by word recognition, not
by word utterance. A man may be uttering a long familiar word
while he is recognizing a short, unfamiliar word. To get around
this difficulty it seemed best to do some averaging by measuring
the total time of utterance for three successive words and then
comparing this with the sum of the times for the words computed
by means of 12.4.
Riesz ingeniously and effectively did this and obtained the data
of Figure XII-5. In the test, a subject read a paragraph as fast as
possible. Certainly, a straight line according to 12.4 fits the data
as well as any curve would. But the points are too scattered to prove
that 12.4 really holds.
Moreover, we should expect such a scatter, for the rank r corresponds
to commonness of occurrence in prose from a variety of
sources, but we have used it as indicating the subject’s experience
with and familiarity with the word. Also, as we see from Figure
XII-3, word length may be expected to have some effect on reading
time. Finally, we have disregarded relations among successive
words.
This sort of experiment is extremely exasperating. One can see
other experiments which he might do, but they would be time
consuming, and there seems little chance that they would establish
anything of general significance in a clear-cut way. Perhaps a
genius will unravel the situation some day, but the wary psychologist
is more apt to seek a field in which his work promises a
definite, unequivocal outcome.
Information Theory and Psychology 241
LOG to r
O © CO h> (0 iO ^ M *; o
- 0 6 6 6 6 6 6 0 °
SQNOD3S Nl 3kMIJ_ 9NI0V3U
242
Symbols, Signals and Noise
The foregoing work does at least suggest that word usage may
be governed by economy of effort and that economy of effort may
be measured as economy of time. We still wonder, however,
whether this is the outcome of a trained ability to cope with the
English language or whether language somehow becomes adapted
to the mental abilities of people. What about the number of words
we use, for instance?
People sometimes measure the vocabulary of a writer by the total
number of different w'ords in his works and the vocabulary of an
individual by the number of different words he understands. However,
rare and unusual words make up a small fraction of spoken
or written English. What about the words that constitute most of
language? How numerous are these?
One might assert that the number of words used should reflect
the complexity of life and that we would need more in Manhattan
than in Thule (before the Air Force, of course). But, we always
have the choice of using either different words or combinations of
common words to designate particular things. Thus, I can say
either “the blonde girl,” “the redheaded girl,” “the brunette girl”
or I can say “the girl with light hair,” “the girl with red hair,” “the
girl with dark hair.” In the latter case, the words with, light, dark,
red, and hair serve many other purposes, while blonde, redheaded,
and brunette are specialized by contrast.
Thus, we could construct an artificial language with either fewer
or more common words than English has, and we could use it to
say the same things that we say in English. In fact, we can if we
wish regard the English alphabet of twenty six letters as a reduced
language into which we can translate any English utterance.
Perhaps, however, all languages tend to assume a basic size of
vocabulary which is dictated by the capabilities and organization
of the human brain rather than by the seeming complexity of the
environment. To this basic language, clever and adaptable people
can, of course, add as many special and infrequently used words
as they desire to or can remember.
Zipf has studied just this matter by means of the graphs illustrating
his law. Figure XII-6 1 shows frequency (number of times a
1 Reproduced from George Kingsley Zipf, Human Behavior and the Principle of
Least Effort, Addison-Wesley Publishing Company, Reading, Mass., 1949.
Information Theory and Psychology 243
word is used) plotted against rank (order of commonness) for
260,430 running words of James Joyce’s Ulysses (curve A) and for
43,989 running words from newspapers (curve B). The straight line
C illustrates Zipf’s idealized curve or “law.”
Clearly, the heights of A and B are determined merely by the
number of words in the sample; the slope of the curve and its
constancy with length of sample are the important things. The
steps at the lower right of the curves, of course, reflect the fact that
infrequent words can occur once, twice, thrice, and so on in the
sample but not 1.5 or 2.67 times.
When we idealize such curves to a 45 ° line, as in curve C, we
note that more is involved than the mere slope of the line. We start
our frequency measurement with words which occur only once;
that is, the lower right hand comer of the graph represents a fre
244
Symbols, Signals and Noise
quency of occurrence of 1. Similarly, the rank scale starts with 1,
the rank assigned to the most frequently used word. Thus, vertical
and horizontal scales start as 1, and equal distances along them
were chosen in the first place to represent equal increases in
number. We see that the 45° Zipf-law fine tells us that the number
of different words in the sample must equal the number of occurrences
of the most frequently used word.
We can go further and say that if Zipf’s law holds in this strict
and primitive form, a number of words equal to the square root
of the number of different words in the passage will make up half
of all the words in the sample. In Figure XII-7 the number N of
different words and the number V of words constituting half the
passage are plotted against the number L of words in the passage.
10 102 I03 10* 10s I06
L IN THOUSANDS OF WORDS
Fig. X II-7
Information Theory and Psychology
245
Here is vocabulary limitation with a vengeance. In the Joyce
passage, about 170 words constitute half the text. And Figure
XII-6 assures us that the same thing holds for newspaper writing!
Zipf gives curves which indicate that his law holds well for
Gothic, if one counts as a word anything spaced as a word. It holds
fairly well for Yiddish and for a number of Old German and
Middle High German authors, though some irregularities occur
at the upper left part of the curve. Curves for Norwegian tend to
be steeper at the lower right than at the upper left, and plains Cree
gives a line having only about three-fourth the slope of the Zipf’s
law line. This means a greater number of different words in a
given length of text—a larger vocabulary. Chinese characters give
a curve which zooms up at the left, indicating a smaller vocabulary.
Nonetheless, the remarkable thing is the similarity exhibited by
all languages. The implication is that the variety and probability
distribution of words is pretty much the same for many, if not all,
written languages. Perhaps languages do necessarily adapt themselves
to a pattern dictated by the human mental abilities, by the
structure and organization of the human brain. Perhaps everyone
notices and speaks about roughly the same number of features in
his environment. An Eskimo in the bleak land of the north manages
this by distinguishing by word and in thought among many
types of snow; an Arab in the desert has a host of words concerning
camels and their equipage. And perhaps all of these languages
adapt themselves in such a way as to minimize the effort involved
in human communication. Of course, we don’t really know whether
or not these things are so.
Zipf’s data have been criticized. I find it impossible to believe
that the number of different words is entirely dictated by length
of sample, regardless of author. Certainly the frequency with which
the occurs cannot change with the length of sample, as Zipf’s law
in its simple form implies. It is said that Zipf’s law holds best for
sample sizes of around 120,000 words, that for smaller samples one
finds too many words that occur only once, and that for larger
samples too few words occur only once. It seems most reasonable
to assume that only the multiple authorship gave the newspaper
the same vocabulary as Joyce.
So far, our approach to Zipf’s law has been that of taking it as
246
Symbols, Signals and Noise
an approximate description of experimental data and asking where
this leads us. There is another approach to Zipf’s law. One can
attempt to show that it must be so on the basis of simple assumptions
concerning the generation text. While various workers have
given proposed derivations, showing that Zipf’s law follows from
certain assumptions, Benoit Mandelbrot, a mathematician who was
mentioned earlier, did the first satisfactory work and appears to
have carried such work furthest.
Mandelbrot gives two derivations. In the first, he assumes that
text is produced as a sequence of letters and spaces chosen randomly
but with unequal probabilities, as in the first-order approximation
to English text of Chapter III. This allows an infinite
number of different “words” composed of sequences of letters
separated from other sequences by spaces.
On the basis of this assumption only, Mandelbrot shows that
the probability of occurrence of the rth of these “words” in order
of commonness must be given by
p(r) = P(r + V)-B (12.5)
The constants B and V can be computed if the probabilities of the
various letters and of the space are known. B must be greater than
1. P must be such as to make the sum ofp(r) over all “words” equal
to unity.
We see that if V were very small and B were very nearly equal
to 1, 12.5 would be practically the same as Zipf’s original law.
Instead of the straight, 45° line of Figure XII-6, equation 12.5
gives a curve which is less steep at the upper left and steeper at
the lower right. Such a curve fits data on much actual text better
than Zipf’s original law does.
It has been asserted, however, that the lengths of the “words”
produced by the random process described don’t correspond to
the length of words as found in typical English text.
Further, language certainly has nonrandom features. Words get
shortened as their usage becomes more common. Thus, taxi and
cab came from taxicab, and cab in turn came from cabriolet. Can
we say that the fact that the random production of letters leads to
the production of “words” which obey Zipf’s law explains Zipf’s
law? It seems to me that we can assert this only if we can show how
the forces which do shape language imitate this random process.
Information Theory and Psychology
247
In his second derivation of a modified form of Zipf’s law as a
consequence of certain initial assumptions, Mandelbrot assumes
that word frequencies are such as to maximize the information for
a given cost. As a simple case, he assumes that each letter has a
particular cost and that the cost of each word (that is, of each
sequence of letters ending in a space) is the sum of the costs of its
letters. This leads him to the same expression as the other derivation,
that is, to equation 12.5. The interpretation of the different
symbols is different, however. The constant B can be less than
unity if the total number of allowable words is finite.
Regardless of the meaning of the constants, P, V, and B in
expression 12.5, we can, if we wish, merely give them such values
as will make the curve defined by 12.5 best fit statistical data
derived from actual text. Certainly, we can fit actual data better
his way than we can if we assume that V = 0 and B = 1 (corresponding
to Zipf’s original law). In fact, by so choosing the values
of P, V, and B, equation 12.5 can be made to fit available data very
well in all but a few exception cases. In the cases of modem
Hebrew of around 1930 and Pennsylvania Dutch, which is a
mixture of languages, a value of B smaller than 1 gives the best fit.
According to Mandelbrot, the wealth of vocabulary is measured
chiefly by the value of B; if B is much greater than 1, a few words
are used over and over again; if B is nearer to 1, a greater variety
of words is used. Mandelbrot observes that as a child grows, B
decreases from values around 1.6 to values around 1.15 or to a
value around 1 if the child happens to be James Joyce.
Certainly, equation 12.5 fits data better than Zipf’s original law
does. It overcomes the objection that, according to Zipf’s original
law, the probability of the word the should depend on the length
of the sample of text. This does not mean, however, that Mandelbrot’s
explanation or derivation of equation 12.5 is necessarily
correct. Further, it is possible that some other mathematical
expression would fit data concerning actual text even better. A
much more thorough study would be necessary to settle such
questions.
Zipf’s law holds for many other data than those concerning
word usage. For instance, in most countries it holds for population
of cities plotted against rank in size. Thus, the tenth largest city
has about a tenth the population of the largest city, and so on.
248
Symbols, Signals and Noise
However, the fact that the law holds in different cases may be
fortuitous. The inverse-square law holds for gravitational attraction
and also for intensity of light at different distances from the sun,
yet these two instances of the law cannot be derived from any
common theory. "iVy con.
It is clear that our ability to receive and handle information is
influenced by inherent limitations of the human nervous system.
George A. Miller’s law of 7 plus-or-minus 2 is an example. This
states that after a short period of observation, a person can remember
and repeat the names of from 5 to 9 simple, familiar objects,
such as binary or decimal digits, letters, or familiar words.
By means of a tachistoscope, a brightly illuminated picture can
be shown to a human subject for a very short time. If he is shown
a number of black beans, he can give the number correctly up to
perhaps as many as 9 beans. Thus, one flash can convey a number
0 through 9, or 10 possibilities in all. The information conveyed
is log 10, or 3.3 bits.
If the subject is shown a sequence of binary digits, he can recall
correctly perhaps as many as 7, so that 7 bits of information are
conveyed.
If the subject is shown letters, he can remember perhaps 4 or 5,
so that the information is as much as 5 log 26 bits, or 23 bits.
The subject can remember perhaps 3 or 4 short, common words,
somewhat fewer than 7 — 2. If these are chosen from the 500 most
common words, the information is 3 log 500, or 27 bits.
As in the case of the reading rate experiments, the gain due to
greater complexity outweighs the loss due to fewer items, and the
information increases with increasing complexity.
Now, both Miller’s 7 plus-or-minus-2 rule and the reading rate
experiments have embarrassing implications. If a man gets only 27
bits of information from a picture, can we transmit by means of
27 bits of information a picture which, when flashed on a screen,
will satisfactorily imitate any picture? If a man can transmit only
about 40 bits of information per second, as the reading rate experiments
indicate, can we transmit TV or voice of satisfactory quality
using only 40 bits per second?
In each case I believe the answer to be no. What is wrong? What
is wrong is that we have measured what gets out of the human
Information Theory and Psychology
249
being, not what goes in. Perhaps a human being can in some sense
only notice 40 bits a second worth of information, but he has a
choice as to what he notices. He might, for instance, notice the girl
or he might notice the dress. Perhaps he notices more, but it gets
away from him before he can describe it.
Two psychologists, E. Averback and G. Sperling, studied this
problem in similar manners. Each projected a large number (16
or 18) of letters tachistoscopically. A fraction of a second later
they gave the subject a signal by means of a pointer or tone which
indicated which of the letters he should report. If he could unfailingly
report any indicated letter, all the letters must have “gotten
in,” since the letter which was indicated was chosen randomly.
The results of these experiments seem to show that far more than
7 plus-or-minus-2 items are seen and stored briefly in the organism,
for a few tenths of a second. It appears that 7 plus-or-minus-2 of
these items can be transferred to a more permanent memory at a
rate of about one item each hundredth of a second, or less than a
tenth of a second for all items. This other memory can retain the
transferred items for several seconds. It appears that it is the size
limitation of this longer-term memory which gives us the 7 plus-
or-minus-2 figure of Miller.
Human behavior and human thought are fascinating, and one
could go on and on in seeking relations between information theory
and psychology. I have discussed only a few selected aspects of a
broad field. One can still ask, however, is information theory really
highly important in psychology, or does it merely give us another
way of organizing data that might as well have been handled in
some other manner? I myself think that information theory has
provided psychologists with a new and important picture of the
process of communication and with a new and important measure
of the complexity of a task. It has also been important in stirring
psychologists up, in making them re-evaluate old data and seek
new data. It seems to me, however, that while information theory
provides a central, universal structure and organization for electrical
communication, it constitutes only an attractive area in
psychology. It also adds a few new and sparkling expressions to
the vocabulary of workers in other areas.
CHAPTER XIII Information
Theory and Art
Some years ago when a competent modern composer and
professor of music visited the Bell Laboratories, he was full of the
news that musical sounds and, in fact, whole musical compositions
can be reduced to a series of numbers. This was of course old stuff
to us. By using pulse code modulation, one can represent any
electric or acoustic wave form by means of a sequence of sample
amplitudes.
We had considered something that the composer didn’t appreciate.
In order to represent fairly high-quality music, with a band
width of 15,000 cycles per second, one must use 30,000 samples
per second, and each one of these must be specified to an accuracy
of perhaps one part in a thousand. We can do this by using three
decimal digits (or about ten binary digits) to designate the amplitude
of each sample.
A composer could exercise complete freedom of choice among
sounds simply by specifying a sequence of 30,000 three-digit decimal
numbers a second. This would allow him to choose from
among a number of twenty-minute compositions which can be
written as 1 followed by 108 million 0’s—an inconceivably large
number. Putting it another way, the choice he could exercise in
composing would be 300,000 bits per second.
Here we sense what is wrong. We have noted that by the fastest
demonstrated means, that is, by reading lists of words as rapidly
as possible, a human being demonstrates an information rate of
250
Information Theory and Art 251
no more than 40 bits per second. This is scarcely more than a ten-
thousandth of the rate we have allowed our composer.
Further, it may be that a human being can make use of, can
appreciate, information only at some rate even less than 40 bits
per second. When we listen to an actor, we hear highly redundant
English uttered at a rather moderate speed.
The flexibility and freedom that a composer has in expressing a
composition as a sequence of sample amplitudes is largely wasted.
They allow him to produce a host of “compositions” which to any
human auditor will sound indistinguishable and uninteresting.
Mathematically, white Gaussian noise, which contains all frequencies
equally, is the epitome of the various and unexpected. It is
the least predictable, the most original of sounds. To a human
being, however, all white Gaussian noise sounds alike. Its subtleties
are hidden from him, and he says that it is dull and monotonous.
If a human being finds monotonous that which is mathematically
most various and unpredictable, what does he find fresh and
interesting? To be able to call a thing new, he must be able to
distinguish it from that which is old. To be distinguishable, sounds
must be to a degree familiar.
We can tell our friends apart, we can appreciate their particular
individual qualities, but we find much less that is distinctive in
strangers. We can, of course, tell a Chinese from our Caucasian
friends, but this does not enable us to enjoy variety among Chinese.
To do that we have to learn to know and distinguish among many
Chinese. In the same way, we can distinguish Gaussian noise from
Romantic music, but this gives us little scope for variety, because
all Gaussian noise sounds alike to us.
Indeed, to many who love and distinguish among Romantic
composers, most eighteenth-century music sounds pretty much
alike. And to them Grieg’s Holberg Suite may sound like eighteenth-century
music, which it resembles only superficially. Even
to those familiar with eighteenth-century music, the choral music
of the sixteenth century may seem monotonous and undistinguish-
able. I know, too, that this works in reverse order, for some
partisans of Mozart find Verdi monotonous, and to those for whom
Verdi affords tremendous variety much modern music sounds like
undistinguishable noise.
Of course a composer wants to be free and original, but he also
252
Symbols,
Signals and Noise
wants to be known and appreciated. If his audience can’t tell one
of his compositions from another, they certainly won’t buy recordings
of many different compositions. If they can’t tell his compositions
from those of a whole school of composers, they may be
satisfied to let one recording stand for the lot.
How, then, can a composer make his compositions distinctive
to an audience? Only by keeping their entropy, their information
rate, their variety within the bounds of human ability to make
distinctions. Only when he doles his variety out at a rate of a very
few bits per second can he expect an audience to recognize and
appreciate it.
Does this mean that the calculating composer, the information-
theoretic composer so to speak, will produce a simple and slow
succession of randomly chosen notes? Of course not, not any more
than a writer produces a random sequence of letters. Rather, the
composer will make up his composition of larger units which are
already familiar in some degree to listeners through the training
they have received in listening to other compositions. These units
will be ordered so that, to a degree, a listener expects what comes
next and isn’t continually thrown off the track. Perhaps the composer
will surprise the listener a bit from time to time, but he
won’t try to do this continually. To a degree, too, the composer
will introduce entirely new material sparingly. He will familiarize
the listener with this new material and then repeat the material in
somewhat altered forms.
To use the analogy of language, the composer will write in a
language which the listener knows. He will produce a well-ordered
sequence of musical words in a musically grammatical order. The
words may be recognizable chords, scales, themes, or ornaments.
They will succeed one another in the equivalents of sentences or
stanzas, usually with a good deal of repetition. They will be uttered
by he familiar voices of the orchestra. If he is a good composer,
he will in some way convey a distinct and personal impression to
the skilled listener. If he is at least a skillful composer, his composition
will be intelligible and agreeable.
Of course, none of this is new. Those quite unfamiliar with
information theory could have said it, and they have said it in other
words. It does seem to me, however, that these facts are particu
Information Theory and Art
253
larly pertinent to a day in which composers, and other artists as
well, are faced with a multitude of technical resources which are
tempting, exasperating, and a little frightening.
Their first temptation is certainly to choose too freely and too
widely. M. V. Mathews of the Bell Laboratories was intrigued by
the fact that an electronic computer can create any desired wave
form in response to a sequence of commands punched into cards.
He devised a program such that he could specify one note by each
card as to wave form, duration, pitch, and loudness. Delighted
with the freedom this afforded him, he had the computer reproduce
rapid rhythmic passages of almost unplayable combinations, such
as three notes against four with unusual patterns of accent. These
ingenious exercises sounded, simply, chaotic.
Very skillful composers, such as Varese, can evoke an impression
of form and sense by patching together all sorts of recorded and
modified sounds after the fashion of musique concrete. Many
appealing compositions utilizing electronically generated sounds
have already been produced. Still, the composer is faced with
difficulties when he abandons traditional resources.
The composer can choose to make his compositions much
simpler than he would if he were writing more conventionally, in
order not to lose his audience. Or he and others can try to educate
an audience to remember and distinguish among the new resources
of which they avail themselves. Or the composer can choose to
remain unintelligible and await vindication from posterity. Perhaps
there are other alternatives; certainly there are if the composer has
real genius.
Does information theory have anything concrete to offer concerning
the arts? I think that it has very little of serious value to
offer except a point of view, but I believe that the point of view
may be worth exploring in the brief remainder of this chapter.
In Chapters III, VI, and XII we considered language. Language
consists of an alphabet or vocabulary of words and of grammatical
rules or constraints concerning the use of words in grammatical
text. We learned to distinguish between the features of text which
are dictated by the vocabulary and the rules of grammar and the
actual choice exercised by the writer or speaker. It is only this
element of choice which contributes to the average amount of
254
Symbols, Signals and Noise
information per word. We saw that Shannon has estimated this to
be between 3.3 and 7.2 bits per word. It must also be this choice
which enables a writer or speaker to convey meaning, whatever
that may be.
The vocabulary of a language is large, although we have seen
in Chapter XII that a comparatively few words make up the bulk
of any text. The rules of grammar are so complicated that they
have not been completely formulated. Nonetheless, most people
have a large vocabulary, and they know the rules of grammar in
the sense that they can recognize and write grammatical English.
It is reasonable to assume a similarly surprisingly large knowledge
of musical elements and of relations among them on the part
of the person who listens to music frequently, attentively, and
appreciatively. Of course, it is not necessary that the listener be
able to formulate his knowledge for him to have it, any more than
the writer of grammatical English need be able to formulate the
rules of English grammar. He need not even be able to write
music according to the rules, any more than a mute who understands
speech can speak. He can still in some sense know the rules
and make use of his knowledge in listening to music.
Such a knowledge of the elements and rules of the music of a
particular nation, era, or school is what I have referred to as
“knowing the language of music” or of a style of music. However
much the rules of music may or may not be based on physical laws,
a knowledge of a language of music must be acquired by years of
practice, just as the knowledge of a spoken language is. It is only
by means of such a knowledge that we can distinguish the style
and individuality of a composition, whether literary or musical.
To the untutored ear, the sounds of music will seem to be examples
chosen not from a restricted class of learned sounds but from all
the infinity of possible sounds. To the untutored ear, the mechanical
workings of the rules of music will seem to represent choice
and variety. Thus, the apparent complexity of music will overwhelm
the untutored auditor or the auditor familiar only with
some other language of music.
We should note that we can write sense while violating the mles
of grammar to a degree (me heap big injun). We might liken the
intelligibility of this sentence to an English-speaking person to our
Information Theory and Art
255
ability to appreciate music which is somewhat strange but not
entirely foreign to our experience. We should also note that we can
write nonsense while obeying the rules of grammar carefully (the
alabaster word spoke silently to the purple). It is to this second
possibility to which I wish to address myself in a moment. I will
remark first, however, that while one can of course both write
sense and obey the rules while doing so, he often exposes his
inadequacies to the public gaze by thus being intelligible.
It is no news that we can dispense with sense almost entirely
while retaining a conventional vocabulary and some or many rules.
Thus, Mozart provided posterity with a collection of assorted,
numbered bars in % time, together with a set of rules (Koechel
294D). By throwing dice to obtain a sequence of random numbers
and choosing successive bars by means of the rules, even the
nonmusical amateur can “compose” an almost endless number of
little waltzes which sound like somewhat disorganized Mozart. An
example is shown in Figure XIII-1. Joseph Haydn, Maximilian
Stadler, and Karl Philipp Emanuel Bach are said to have produced
similar random music. In more recent times, John Cage has used
random processes in the choice of sequences of notes.
In ignorance of these illustrious predecessors, in 1949 M. E.
Shannon (Claude Shannon’s wife) and I undertook the composition
of some very primitive statistical or stochastic music. First we
made a catalog of allowed chords on roots I-VI in the key of C.
Actually, the catalog covered only root I chords; the others were
Fig. XIII-1
256
Symbols, Signals and Noise
derived from these by rules. By throwing three specially made dice
and by using a table of random numbers, a number of compositions
were produced.
In these compositions, the only rule of chord connection was
that two succeeding chords have a common tone in the same voice.
This let the other voices jump around in a wild and rather unsatisfactory
manner. It would correspond to the use of a simple and
consistent but incorrect digram probability in the construction of
synthetic text, as illustrated in Chapter III.
While the short-range structure of these compositions was very
primitive, an effort was made to give them a plausible and reasonably
memorable, longer-range structure. Thus, each composition
consisted of eight measures of four quarter notes each. The long-
range structure was attained by making measures 5 and 6 repeat
measures 1 and 2, while measures 3 and 4 differed from measures
7 and 8. Thus, the compositions were primitive rondos. Further,
it was specified that chords 1, 16, and 32 have root I and chords
15 and 31 have either root IV or root V, in order to give the effect
of a cadence.
Although the compositions are formally rondos, they resemble
hymns. I have reproduced one as Figure XIII-2. As all hymns
should have titles and words, I have provided these by nonrandom
means. The other compositions sound much like the one given.
Clearly, they are all by the same composer. Still, after a few hearings
they can be recognized as different. I have even managed to
grow fond of them through hearing them too often. They must
grate on the ears of an uncorrupted musician.
In 1951, David Slepian, an information theorist of whom we
have heard before, took another tack. Following some early work
by Shannon, he evoked such statistical knowledge of music as lay
latent in the breasts of musically untrained mathematicians who
were near at hand. He showed such a subject a quarter bar, a half
bar, or three half bars of a “composition” and asked the subject
to add a sensible succeeding half bar. He then showed another
subject an equal portion including that added half bar and got
another half bar, and so on. He told the subjects the intended styles
of the compositions.
Information Theory and A rt 257
RANDOM
|frn~i i 1
When once nor ran - dom thoughts I turned
That Christmas day, did they fore- see Or did they learn of us who sing
l j r r 1 1 1 f i f
On Beth- le - hem and on that star
The Child grown Christ, and Christ de - nied
The day when an - gels sang be - fore
1* F .£ - I ....f ; -f : i i
Which drew three wise men from a - far
And Christ be -trayed and cru - ci - fled
And, lay - ing down the gifts they bore
Fig. XIII-2
In Figure XIII-3, I show two samples: a fragment of a chorale
in which each half bar was added on the basis of the preceding
half bar and a fragment of a “romantic composition,” in which
each half bar was added on the basis of the preceding three half
bars. It seems to me surprising that these “compositions” hang
together as well as they do, despite the inappropriate and inadmissible
chords and chord sequences which appear. The distinctness
258 Symbols, Signals and Noise
CHORALE •
ROMANTIC:
Fig. X III-3
of the styles is also arresting; apparently the mathematicians had
quite different ideas of what was appropriate in a chorale and
what was appropriate in a romantic composition.
Slepian’s experiment shows the remarkable flexibility of the
human being as well as some of his fallibility. True stochastic
processes are apt to be more consistent but duller. A number have
been used in the composition of music.
There is no doubt that a computer supplied with adequate
statistics describing the style of a composer could produce random
music with a recognizable similarity to a composer’s style. The
nursery-tune style demonstrated by Pinkerton and the diversity of
styles evoked by Hiller and Isaacson, which I will describe presently,
illustrate this possibility.
In 1956, Richard C. Pinkerton published in the Scientific American
some simple schemes for writing tunes. He showed how a note
could be chosen on the basis of its probability of following the
particular preceding note and how the probabilities changed with
respect to the position of the note in the bar. Using probabilities
derived from nursery tunes, he computed the entropy per note,
Information Theory and Art
259
which he found to be 2.8 bits. I feel sure that this is quite a bit too
high, because only digram probabilities were considered. He also
presented a simple finite-state machine which could be used to
generate banal tunes, much as the machine of Figure III-1 generates
“sentences.”
In 1957, F. B. Brooks, Jr., A. L. Hopkins, Jr., P. G. Neumann,
and W. V. Wright published an account of the statistical composition
of music on the basis of an extensive statistical study of
hymn tunes.
In 1956, the Burroughs Corporation announced that they had
used a computer to generate music, and, in 1957, it was announced
that Dr. Martin Klein and Dr. Douglas Bolitho had used the
Datatron computer to write “popular” melodies. Jack Owens set
words to one, and it was played over the ABC network as Push
Button Bertha. No doubt many others have done similar things.
It remained, however, for L. A. Hiller, Jr., and L. M. Isaacson of
the University of Illinois to make a really serious experiment with
computer music. Hiller and Isaacson succeeded in formulating the
rules of four-part, first-species counterpoint in such a way that a
computer could choose notes randomly and reject them if they
violated the rules.
Because the rules involve, except in connection with the concluding
cadence, only direct relations among three successive notes,
the music tends to wander, but over a short range it sounds
surprisingly good. A sample is shown in Figure XIII-4.1
Hiller and Isaacson went on to demonstrate that they could use
the computer to generate interesting rhythmic and dynamic patterns
and to generate “Markoff-chain” music, in which successive
note selection depended on probability functions computed from
tables derived from various considerations of overtones or harmonics.
In this case they generated a coda according to a simple
prescription.
As it stands, this music, which was brought together and published
as the Illiac Suite for String Quartet, has a good deal of local
structure but is weak and wandering as a whole. The imposition
1 Reproduced from L. A. Hiller, Jr., and L. M. Isaacson, Illiac Suite for String
Quartet, New Music, 1957, by permission of Theodore Presser Company, Bryn
Mawr, Pa.
260 Symbols, Signals and Noise
( G ) E B ( H )
ff
m j =^=
/
I t . , t In p
/ p
/ p
m
of some simple pattern or repetition might have helped considerably.
This could be of a strictly deterministic nature, as in the case
of the prescribed repetitions in a rondo, or it could be of the nature
of Chomsky’s grammar, which we have considered in Chapter VI.
It is clear, however, that it is foolish to try to attain long-range
structure simply by relating a note to the immediately preceding
notes by digram, trigram, and higher probabilities. The relation
must be among parts of the composition, not simply among notes.
The work of Hiller and Isaacson does demonstrate conclusively
that a computer can take over many musical chores which only
human beings had been able to do before. A composer, and
especially an unskilled composer, might very well rely on a computer
for much routine musical drudgery. The composer could
merely guide the main pattern of the composition and let the
computer fill in details of harmony and counterpoint, according
to a specification of style or period. Further, the computer could
Information Theory and Art
261
be used to try out proposed new rules of composition, such as new
rules of counterpoint or harmony, with whose use and consequences
the composer might have little experience and familiarity.
In these days we hear that cybernetics will soon give us machines
which learn. If they learn in a complicated enough sense of the
word, why couldn’t they learn what we like, even when we don’t
know ourselves? Thus, by rewarding or punishing a computer for
the success or failure of its efforts, we might so condition the computer
that when we pressed a button marked Spanish, classical,
rock-and-roll, sweet, etc., it would produce just what we wanted
in connection with the terms. Such thoughts are intriguing, but
they are of course nonsense in our day and will probably remain
so for a long time to come.
Music is not all of art. I began with music because it offers an
apt means for illustrating in an unusual context some ideas derived
from information theory. We could just as well draw our illustrations
from the use of language. Indeed, experiments with the
stochastic production of text have been perhaps more widely
cultivated than experiments with music.
A professor at the Grand Academy of Lagoda showed Captain
Lemuel Gulliver a word frame consisting of lettered blocks
mounted on shafts. The professor turned these at random and
sought new wisdom in the patterns of letters which appeared.
Here we see just the wrong application of a stochastic process
in the generation of text. Certainly, this will not give us new
knowledge. Who would take the uncorroborated word of a random
process? There are all too many unsubstantiated statements available;
what we need to know is what is so and what isn’t.
Nonetheless, a stochastic process can produce some interesting
effects. In Chapter III we noted Shannon’s approximations to
English text. These were made by using letter digram and trigram
frequencies and a table of random numbers. We have seen that they
contain some interesting “words.”
To me, deamy has a pleasant sound; I would take “it’s a deamy
idea” in a complimentary sense. On the other hand, I’d hate to be
denounced as ilonasive. I would not like to be called grocid; perhaps
it reminds me of gross, groceries, and gravid. Pondenome,
whatever it may be, is at least dignified.
262 Symbols, Signals and Noise
I repeat Shannon’s second-order word approximation here:
The head and in frontal attack on an English writer that the character
of this point is therefore another method for the letters that the time who
ever t.old the problem for an unexpected.
I find this disquieting. I feel that the English writer is in mortal
peril, yet I cannot come to his aid because the latter part of the
message is garbled.
In seeking less garbled material, as I noted in Chapter VI, I wrote
three grammatically connected words in a column from the top
down on a slip of paper. I showed them to a friend, asked him to
make up a sentence in which they occurred, and then to add the
next word in this sentence. I then folded over the top word of the
four I now had and showed the visible three to another friend and
got another word from him. After canvassing twenty friends, I had
the following:
When morning broke after an orgy of wild abandon he said her head
shook vertically aligned in a sequence of words signifying what. . .
Later examples are:
One day when I went to what was Dionysus thinking of women without
men go off half way decent impression . . .
I forget whether he went on and on. Finally he stipulated that this must
stop immediately after this. The last time I saw him when she lived.
It happened one frosty look of trees waving gracefully against the wall.
You never can . . .
We see that a seeming order of meaning persists over groups of
far more than four words. Eventually, however, the text wanders.
The long-range wandering is of course attributable to the fact that
there is no long-range, persistent purpose or meaning guiding the
choice of words. We sometimes see a similar quality in the utterances
of schizophrenics. W. H. Hudson illustrates a more gradual
wandering admirably by a character in his The Purple Land: Uncle
Anselmo never gets to the end of a story because he continually
wanders on to new ground.
One can add a certain amount of long-range order by writing,
in view, at the bottom of the slip of paper to which people add
Information Theory and Art 263
words a title which indicates what the passage is supposed to
be about.
Dr. Donald A. Dunn of the Stanford Electronics Laboratory has
kindly supplied me with some examples in which the person
adding a word saw only one preceding word, together with the title:
Men and Women
Eve loved intensely sentimental or not sufficient tonight wherever you
may die before yesterday again and whatever m’love misbehaves.
The seduced are compatible unusual family life seemed wonderful
experience for tenderness forever yours.
Orphans frequently visited his promiscuity and infidelity despite hate
and love for tomorrow sex ain’t nothing.
In the following examples, which were produced at the Bell
Laboratories, the person adding a word saw three preceding words
as well as the title:
About Life
Life has many good and wise men seldom condemn halfwits lightly! You
wonder why not. Human feelings but Savage tribes found . . .
Engineers
It is frequently said that they knew why forces might affect salaries.
However, all scientists can’t imagine . . .
Housecleaning
First empty the furniture of the master bedroom and bath. Toilets are
to be washed after polishing doorknobs the rest of the room. Washing
windows semiannually is to be taken by small aids such as husbands are
prone to omit soap powder.
Murder Story
When I killed her I stabbed Claude between his powerful jaws clamped
tightly together. Screaming loudly despite fatal consequences in the
struggle for life ebbing as he coughed hollowly spitting blood from his ears.
I think that it is hard to read such material without amusement.
I feel a little admiration as well. I would never write, “It happened
one frosty look of trees waving gracefully against the wall.” I
almost wish I could. Poor poets endlessly rhyme love with dove,
264
Symbols, Signals and Noise
and they are constrained by their highly trained mediocrity never
to produce a good line. In some sense, a stochastic process can do
better; it at least has a chance. I wish I had hit on deamy, but I
never would have.
Will a computer produce text of any literary merit by means of
grammatical rules and a sequence of random numbers? It might
produce fresh and amusing “words” and amusing short passages
of some shock value. One can of course imagine a machine
designed to write detective novels and equipped with settings for
hard-boiled, puzzle, character, suspense, and so on, but such a
device seems to me to be very far away.
The visual arts can be used to illustrate the same points which
have been made in connection with music and language. A completely
random visual pattern, like a completely random acoustic
wave or a completely random sequence of letters, is mathematically
the most surprising, the least predictable of all possible patterns.
Alas, a completely random pattern is also the dullest of all patterns,
and to a human being one random pattern looks just like another.
Figure XIII-5, which is an array of 10,000 randomly black or white
dots, illustrates this.
Bela Julesz, who works in the field of perception, caused an
Fig. XIII-5
Information Theory and Art
265
electronic computer to produce this random array of dots as a part
of his studies of stereoscopic vision and of the meaning of pattern.
He also programmed the computer to remove some of the randomness
from such a random pattern. He did this by making the
computer examine successively various sets of five points located
at the tips and at the center of an X, as shown by the points marked
X in Figure XIII-6 (other points are marked O). If the center point
was the same (black or white) as either points 1 and 4 or points
2 and 3, it was changed (from black to white or from white to
black). This tends to remove any black or white diagonals, except
when points 1 and 4 are black and points 2 and 3 are white or
vice versa.
As we can see from Figure XIII-7, making a pattern less random
in this way alters and improves its appearance profoundly. An
unpredictable (random) component is desirable for the sake of
variety or surprise, but some orderliness is necessary if a pattern
is to be pleasing.
This exploitation of both order and randomness is in fact old
to art. The kaleidoscope offers a charming effect by giving to a
random arrangement of bits of colored glass a sixfold symmetry.
O O O O O
o
1 2
x o x o
o o x o o
3
O X o
4
x o
o o o
Fig. X III-6
O O
266 Symbols, Signals and Noise
F ig. X I I I - 7
Many years ago Marcel Duchamp, who painted Nude Descending
a Staircase, allowed a number of threads to fall on pieces of black
cloth and then framed and preserved them. Jean Tinguley, the
Swiss artist, has produced, by means of a machine, partly ordered,
partly random colored designs of considerable merit; I derive
continuing pleasure from one which hangs in my office. I saved
for years a pile of solder droppings which I intended to mount on
a block of ebony and present to the Museum of Modem Art.
Finally, I lost both the solder and the desire to do so.
All of this has given me a sort of minimum philosophy of art,
which I will not, I hasten to assure the reader, blame on information
theory. It is a minimum philosophy because it says nothing
about the talent or genius which alone can make art worth while.
Successful art requires the appreciation of an audience as well
as the talent of the artist. Audiences are influenced by things other
than the object of art before them. If a person sets his mind against
it, anything will leave him cold. A desire to appreciate can, on the
other hand, lead to one’s liking even poor works. I like the hymnlike
compositions that Betty Shannon and I made. Authors sometimes
prefer inferior works of their own. Both small coteries and
large groups can be led to appreciate sincerely things which are for
Information Theory and Art 267
a time the fashion but which have little long-range appeal and
which probably have little merit.
Among other things, audiences want to have a sense of authorship,
a sense of an individual, in connection with works of art. To
bring appreciation to an artist, his work must have enough consistency
so that it is recognizable as his. How let down the sincere
appreciator must be if he always has to look at the label or wait
for the announcer in order to know that the painting or music is
the product of his favorite artist.
Suppose that one artist had actually produced in succession the
masterpieces we now accept as the works of a number of great
artists with diverse styles, long before the artists lived. This would
astonish us, but we could scarcely appreciate him as an artist,
however much we might admire the individual paintings. Picasso
is eminently recognizable, but he is disquieting. He has been skillful
in many styles, and yet he escapes our final judgment by going from
one style to another. How much easier it is to appreciate Matisse.
To be appreciated by an audience, art must be intelligible to the
audience. Even a good joke in Chinese will amuse few Americans,
and certainly ten jokes in Chinese will be no more amusing than
one. To a degree, to be appreciated art must be in a language
familiar to the audience; otherwise no matter how great the variety
may be, the audience will have an impression of monotony, of
sameness. We can be surprised repeatedly only by constrast with
that which is familiar, not by chaos.
Some artists adopt a language taught to their audience by earlier
masters. Brahms was one of these. Other artists teach something
of a new language to their audiences, as the impressionists did.
Certainly, the language of art changes with time, and we should
be grateful to the artists who teach us new words. However, we
should not doubt the originality of such artists as Bach and Handel,
who spoke ringingly in a language of the past.
While a language with intelligible words and relations between
words is necessary in art, it is not sufficient. Mechanical sameness
is dull and disappointing. I prefer the surprises of stochastic prose
to the vapid verses of Owen Meredith. Perhaps in some age of bad
art, man will be forced to stochastic art as an alternative to the
stale product of human artisans.
So much for information theory and art.
CHAPTER XIV Back to
Communication
Theory
Surely, it is wonderful if a new idea contributes to the
solution of a broad range of problems. But, first of all, to be
worthy to notice a new idea must have some solid and clearly
demonstrated value, however narrow that value may be.
An information theorist has criticized me for exploring in this
book possible applications of information theory in fields of language,
psychology, and art. To him, the relation between such
subjects and information theory seems marginal or even dubious.
Why distract the reader from the clearly demonstrated value and
importance of information theory by discussing matters concerning
which no clear value or importance can be demonstrated?
Partly, in writing this book I have felt an obligation to the reader
to discuss relations between information theory in its solid and
narrow sense and various fields with which it has been connected
in the writings of others. Partly, I believe that information theory
is useful in helping us in talking sense or at least in keeping from
talking nonsense in connection with some linguistic, artistic, and
psychological problems. However, there is a danger in overemphasizing
such matters in a book on information theory.
It would certainly be wrong to assert or to believe that information
theory is valuable chiefly because of wide-ranging connections
268
Back to Communication Theory
269
with a variety of fields such as language, cybernetics, psychology,
and art. To believe this would be to repeat mistakes which have
been made in connection with other important discoveries.
Thus, in Newton’s day his work was beclouded by controversy
and philosophy, and for many years thereafter it was associated
in people’s minds with a putative universality which confused its
real nature. Einstein, however, could see more clearly. He said:
“Reason, of course, is weak when measured against its never
ending task.” Einstein then described Newton’s contribution to
this task of understanding and observed, “and with that, the goal
was reached, the science of celestial mechanics was bom, confirmed
a thousand times over by Newton himself and those who came
after him.”
It is fair to add that since Newton’s day, Newtonian mechanics
has been useful in solving or contributing to the solution of problems
that never entered the minds of Newton and his contemporaries,
but it has not solved all problems of science, as some
optimistic philosophers expected it to.
To me the indubitably valuable content of information theory
seems clear and simple. It embraces the ideas of the information
rate or entropy of an ergodic message source, the information
capacity of noiseless and noisy channels, and the efficient encoding
of messages produced by the source, so as to approach errorless
transmission at a rate approaching the channel capacity. The
world of which information theory gives us an understanding of
clear and present value is that of electrical communication systems
and, especially, that of intelligently designing such systems.
It seems to me wise at the close of this book to turn away from
the broad, speculative possibilities (or impossibilities?) of information
theory and to ask the following question: Beyond the things
already described in this book, what have information theorists
done and what are they doing that is mathematically sound, well
founded, compelling? What, in other words, have they done that
qualifies as sound science which we must accept rather than as
intriguing speculation that we have the privilege of arguing about?
Here we find a broad range of work. To explain all of it fully to
the reader would take another book. Thus, this chapter will be a
brief summary of some of the work of information theorists since
270
Symbols, Signals and Noise
the publication of Shannon’s original paper. Its purpose is to
acquaint the reader with the scope of information theory in its
narrow sense and, perhaps, to entice him into following such
activities in greater detail.
One thing that information theorists have sought is some application
of the entropy of information rate of a message source to a
problem other than that of encoding and transmission of information.
Ambitious men want to bring meaning into the picture
somehow, but a more modest worker is willing to settle for any
application which is meaningful and rigorously correct.
The only application of information rate to a problem other
than efficient encoding which has been given so far and which
meets these criteria was advanced by J. L. Kelly, Jr., in 1956.1 It
concerns gambling on chance events in which the bettor has inside
information as to the outcome of the event bet upon. We might
imagine, for instance, that the dice are already thrown (or the race
run) and that the favored bettor knows this and has received some
knowledge of the outcome, but the person with whom he bets
doesn’t know this and gives the bettor fair odds on the basis of
the chance of the outcome.
The information which the bettor receives is doled out to him
in bits, that is, yes-or-no answers to questions. His informant
could, for instance, inform the bettor completely concerning
whether a coin tossed had turned up heads or tails by sending him
one bit of information. Or the informant could narrow for the
bettor the possible outcomes of the cast of a die from 6 to 3 by
using one bit of information to tell the bettor whether the outcome
was odd or even.
Following this introduction, I can best explain Kelly’s result by
quoting the abstract of his paper:
If the input symbols to a communication channel represent the outcomes
of a chance event on which bets are available at odds consistent with their
probabilities (i.e., “fair” odds), a gambler can use the knowledge given him
by the received symbols to cause his money to grow exponentially. The
maximum exponential rate of growth of the gambler’s capital is equal to
ll‘New Interpretation of Information Rate,” Bell System Technical Journal, Vol.
35 (July, 1956), pp. 917-926.
Back to Communication Theory
271
the rate of transmission of information over the channel. This result is
generalized to include the case of arbitrary odds.
Thus we find a situation in which the transmission rate is significant even
though no coding is contemplated. Previously this quantity was given
significance only by a theorem of Shannon’s which asserted that, with
suitable encoding, binary digits could be transmuted over the channel at
this rate with an arbitrarily small probability of error.
Numerically the factor by which the gambler’s initial capital is
increased after N bets is
2 AW
Here R is the average number of bits of information transmitted
to the bettor per bet.
If this seems a trivial application of the amount of information
in bits, the reader should meditate on the fact that it is the only
mathematically established interpretation, other than those concerned
with the rate of generation of probable messages and their
efficient encoding for transmission, that anyone has discovered.
In advancing information theory, one may seek a new use for
information theory rather than a new interpretation of information
rate. Thus, in 1949, C. E. Shannon published a long paper entitled
“Communication Theory of Secrecy Systems.” 2 It is doubtful
whether this paper has helped substantially in the deciphering of
messages, but it has provided, for the first time, a well organized
theory of cryptography and cryptanalysis, and it is highly regarded
by the expert cryptanalysts.
It would be hopeless to try to go into the details of Shannon’s
work here, but I will try to give an idea of some of its content.
The cryptanalyst who lays hands on a message enciphered by
an unknown means is ignorant of two things: the message itself
and a specification of the means used to encipher it, which we may
call the key.
Sometimes, the cryptanalyst may know the general scheme of
encipherment. To take a ridiculously simple example, he might
know that a simple substitution cipher had been used, that is, for
each letter of the alphabet some other letter had been substituted
according to a fixed scheme.
2 Ibid., Vol. 28 (October, 1949), pp. 656-715.
272
Symbols, Signals and Noise
The cryptanalyst may have a short or a long enciphered message
to work with. If the message had only three letters in it, say QXD,
these might stand for AND, or BET, or any other English word
made up of three different letters. As the message becomes longer,
however, the number of possible English texts which could have
been encrypted by means of a simple substitution cipher to give the
particular message at hand decreases; if the enciphered message
is long enough, there will be only one possible source message.
Shannon expressed this decrease of uncertainty as to what message
might have been enciphered so as to give the message in
question as a change in the equivocation. The equivocation Hv(x)
of Chapter VIII gives the uncertainty of what message was
enciphered by the general means in question in order to give the
received enciphered message. Shannon was able to compute in the
case of various ciphers how the equivocation decreases as the
number of characters in the message increases. When the equivocation
approaches zero, only one message could have been enciphered
to give the enciphered message, and, in principle, the
message can be deciphered uniquely.
What other sorts of problems have confronted or now confront
information theorists? Some of these problems concern the sampling
theorem. Information theorists use the sampling theorem in
order to represent a smoothly varying, band-limited signal by
means of a sequence of numbers; the sample numbers are the
amplitudes of the signal taken every \/2W seconds, where W is
the band width of the signal.
The samples which represent a given band-limited signal are not
unique; they can be taken at various times. Thus, in Figure XIV-1,
either the vertical solid lines or the vertical dashed lines are samples
which legitimately represent the function, and samples could have
been taken at many other locations. In fact, the samples don’t
even have to be equally spaced in time, provided that, on the
average, there are two 2 W samples per second!
Fig. XIV-1
Back to Communication Theory
273
A band-limited signal is represented uniquely by 2 W samples
per second only when all samples from the infinite past to the
infinite future are used. Sometimes we would like to talk about a
piece of band-limited signal or about a band-limited signal which
is almost zero except for some specified range of time, and we
would like to describe such a portion of a signal or a signal of
limited duration handily in terms of samples.
Our first thought might be, can we merely specify a short signal
or a portion of a signal by specifying the values of a finite sequence
of samples and saying nothing about samples before or after these?
Alas, specifying such a finite set of samples does not specify just
one band-limited signal; many different band-limited signals can
be passed through a finite sequence of samples, and, if the signals
are very large outside of the range of the specified samples, they
can be very different within the range of the specified samples.
This failing, we might say, let us specify certain successive
sample values and make all preceding and succeeding samples be
zero. Surely, we may think, the band-limited signal so specified
will conform closely to the sample values where these are not zero
and will be small wherever the samples are specified as zero.
Suppose, for instance, that we insist that all of a set of equally
spaced samples after a time to are zero, while the samples before
the time to are nonzero, as shown by the dots in Figure XIV-2.
Because the samples are specified for all times past and future, they
do specify a unique band-limited signal. Will this signal be nearly
zero for times after ?0?
Alas, H. O. Poliak, of the Bell Laboratories, has shown that this
need not be so. Suppose we ask, what part of the total energy of
........................................
Fig. XIV-2
274
Symbols,
Signals and Noise
the band-limited signal passing through such samples is carried
by the part of the wave which occurs ten seconds, or twenty
minutes, or fifty years after r0? Remember all the samples are zero
after to-
The surprising answer is that almost half of the energy of the
signal can be carried by the part that occurs later than any specified
time after the samples become zero. Thus, the signal can be zero
at all the samples after to and still be large in between them.
Efforts to use the sampling theorem rigorously to represent
signals of limited length are in mathematical trouble, and mathematicians
are trying to find some way out.
Work by Poliak and Slepian indicates that neither samples nor
sine waves are the most appropriate way to represent band-limited
functions of finite duration, and these mathematicians have used
a more appropriate group of functions called prolate spheroidal
functions for this purpose.
One puzzling matter about information theory may be illustrated
by the following example. Suppose that in telegraphy we let a
positive pulse represent a dot and a negative pulse represent a dash.
Suppose that some practical joker reverses connections so that
when a positive pulse is transmitted a negative pulse is received
and when a negative pulse is transmitted a positive pulse is
received. Because no uncertainty has been introduced, information
theory says that the rate of transmission of information is just the
same as before. Yet we feel that some damage has been done to
the communication system. The damage would be even more
appalling if, in a teletypewriter link, we consistently printed out
W for A, K for B, and so on, in a completely scrambled fashion.
This bothered Shannon, and he has worked out a theory to
cover the situation. In this theory, he establishes a fidelity criterion.
Thus, he might assign a given penalty for substituting a consonant
for a vowel and a lesser penalty for substituting one vowel for
another. He can then assess the damage done to a message by either
consistent or random errors. When the damage is done by the
random errors of a noisy channel, he shows in principle how tp
minimize it, and he shows how many bits per second are required
to transmit the signal with a given degree of fidelity.
Shannon has also done a considerable amount of work concern
Back to Communication Theory
275
ing the transmission of messages over networks in which one message
may interfere with another message. The simplest case is that
of transmission of messages in both directions over the same
channel between two points, A and B. As a very special case, we
will assume that the circuit acts the same from B to A as from
A to B.
Suppose that we plot the channel capacity for transmission from
A to B against the channel capacity for transmission from B to A,
as shown in Figure XIV-3. We can imagine two very simple cases.
In one case, transmission from B to A does not interfere with
transmission from A to B, and transmission from A to B does not
interfere with transmission from B to A. In this case, the curve
consists of the horizontal solid line giving the channel capactiy
from B to A and the vertical solid line giving the channel capacity
from A to B.
Or we can imagine that at one time we can transmit in one
m
O
IV "'-.NO INTERFERENCE
\
INTERMEDIATE
CASE
N\ \
\ \\
a
z
o
u Ui <0
<r Ui
o.
«o
t
m
\
COMPLETE^*
INTERFERENCE^
\ \ \
o
BITS PER SECOND, B TO A
Fig, XIV-3
276
Symbols, Signals and Noise
direction only, either from A to B or from B to A. Then if we are
transmitting from A to B one-third of the time, we can transmit
from B to A only two-thirds of the time, and so on. The sum of the
channel capacity from B to A and the channel capacity from A to B
must be a constant, and the result is the dashed 450 line of Figure
XIV-3.
In an intermediate case, in which there is some interference between
transmission in the two directions, we will get a curve
roughly of the form of the dotted line of Figure XIV-3.
The study of efficient encoding continues to command the attention
of information theorists. In the case of discrete channels, information
theorists continually better codes for correcting A errors
in a sequence of B digits.
Information theorists also seek best codes for transmitting information
over a noisy continuous channel. In 1959, Shannon
published a long paper in which he arrived at upper and lower
bounds on the attainable error rates for codes of various complexity
(that is, length) used in signaling over a continuous channel
with Gaussian noise. Currently, convolutional codes and Viterbi
decoding are favored, and ingenious men vie in making better and
cheaper decoders.
Further, engineers who wish to improve electrical communication
continually try to find new encoding and transmission schemes
which are simple enough to be useful. They try to encode television
and voice signals into as few binary digits per second as they can;
the approaches they use have been indicated in Chapter VII. Such
efficient encoding is growing in importance because digital transmission
of signals (as in pulse code modulation) is displacing
analog communication. It will grow in importance as the encrypting
of signals in order to obtain privacy or secrecy becomes more
common, for secrecy is best attained by digital means.
Engineers also look for simple and efficient error-correcting
means useful in correcting the multiple errors which occur in the
transmission of digital signals over existing telephone circuits. The
use of digital transmission in transmitting text and in transmitting
business and technical data is growing by leaps and bounds, both
in military and in civilian applications. Telephone circuits go
almost everywhere. To keep pace with data, we must use voice
Back to Communication Theory
277
circuits for data. Here error correction by error detection and retransmission
is the favored technique. But, error-correcting codes
have their place.
There are special circumstances that call for special means of
modulation. Mobile radio is one. In a city, signals reach the
vehicle from many directions after bouncing off many buildings,
and a short pulse would be received as a smear of pulses which
have traveled different distances over different paths. Here great
ingenuity is needed in finding the best way to use a large overall
bandwidth in improving transmission.
Military communication, especially in the face of jamming, poses
a host of problems.
Perhaps some would regard all this as engineering drudgery,
unexciting compared with the broad philosophical vistas which
information theory seems to open to us. Can an informed understanding,
a loving appreciation of the nature, virtues and distinctions
among the French Impressionists or the Dutch genre
painters ever be so meaningful as a sudden and bewildering confrontation
with a new and strange world of art, such as the
Japanese?
Yet, the connoisseur who pursues with devotion the details of a
field may well have as much insight and as sound values as the
rapturous dilettante. There is some intellectual obligation to
appreciate a field for what it is rather than for the reactions it
excites in the minds of the uninformed. I hope that this book has
its exciting aspects, but I also hope that it won’t lead the reader to
a view of information theory widely different from that held by
informed workers in the field. Hence, it is perhaps well to end
in a sober vein.
a p p e n d i x : On Mathematical
Notation
The reader will find a fairly liberal use of mathematical
notation in this book, including a number of equations. This may
incline him to say the book is full of mathematics.
Of course it is. Communication theory is a mathematical theory,
and, as this book is an exposition of communication theory, it is
bound to contain mathematics. The reader should not, however,
confuse the mathematics with the notation used. The book could
contain just as much mathematics and not include one symbol or
equality sign.
The Babylonians and the Indians managed quite a lot of mathematics,
including parts of algebra, without the aid of anything
more than words and sentences. Mathematical notation came
much later. Its purpose is to make mathematics easier, and it does
for anyone who becomes familiar with it. It replaces long strings
of words which would have to be used over and over again with
simple signs. It provides convenient names for quantities that we
talk about. It presents relations concisely and graphically to the
eye, so that one can see at a glance the relations among quantities
which would otherwise be strewn through sentences that the eye
would be perplexed to comprehend as a whole.
The use of mathematical notation merely expresses or represents
mathematics, just as letters represent words or notes represent
music. Mathematical notation can represent nonsense or nothing,
278
On Mathematical Notation 279
just as jumbled letters or jumbled notes can represent nothing.
Crackpots often write tracts full of mathematical notation which
stands for no mathematics at all.
In this book I have tried to put all the important ideas into words
in sentences. But, because it is simpler and easier to understand
things written concisely in mathematical notation, I have in most
cases put statements into mathematical notation also. I have to a
degree explained this throughout the book, but here I summarize
and enlarge on these explanations. I have also ventured to include
a few simple related matters which are not used elsewhere in this
book, in the hope that these may be of some general use or interest
to the reader.
The first thing to be noted is that letters can stand for numbers
and for other things as well. Thus, in Chapter V, Bj stands for a
group or sequence of symbols or characters, a group of letters
perhaps; j signifies which group. For the first group of letters, j
might be 1, and that first group might be AAA, for instance. For
another value of j, say, 121, the group of letters might be ZQE.
We often have occasion to add, subtract, multiply, or divide
numbers. Sometimes we represent the numbers by letters. Examples
of the notations for these operations are:
Addition
2 + 3
a + d
We read a + d as “a plus d ” We may interpret a + d as the sum of
the number represented by a and the number represented by d.
Subtraction
5 - 4
q - r
We read q — r as “q minus r ”
Multiplication
3 x 5 or 3*5 or (3)(5)
h X v or u • v or uv
If we did not use parentheses to separate 3 and 5 in (3) (5), we
would interpret the two digits as 35 (thirty-five). We can use
280
Symbols, Signals and Noise
parentheses to distinguish any quantities we want to multiply. We
could write uv as (u) (v), but we don’t need to. We read (3) (5) as
3 times 5, but we read uv as “nv” with no pause between the u and
v, rather than as “u times v.”
Division
6 3 or j or 6/3
- or 1/p
P
We ordinarily read 1 /p as “ 1 over p ” rather than as “ 1 divided
by />.”
thus
Quantities included in parentheses are treated as one number;
_(2 + 4) _ 6 _ 2
3 3
(4 + 8) (2) = (12) (2) = 24
(a + b)c = ac + be
We read (a + b) either as “a plus b" or as “the quantity a plush,”
if just saying “a plus b” might lead to confusion. Thus, if we said
“c times a plus b” we might mean ca + b, though we would read
ca + b as “ca plus b.” If we say “c times the quantity a plus h,” it
is clear that we mean c(a + b).
The idea of a probability is used frequently in this book. We
might say, for instance, that in a string of symbols the probability
of the j th symbol isp(j). We read this “p of/ ”
The symbols might be words, numbers, or letters. We can
imagine that the symbols are tabulated; various values of j can be
taken as various numbers which refer to the symbols. Table XVI,
shows one way in which the numbers j can be assigned to the letters
of the alphabet.
When we wish to refer to the probability of a particular letter, N
for instance, we could, I suppose, refer to this as/>(5), since 5 refers
to N in the above table. We’d ordinarily simply write />(N),
however.
What is this probability? It is the fraction of the number of
On Mathematical Notation 281
Table XVI
Value o f j Corresponding L etter
1 E
2 T
3 A
4 O
5 N
6 R
etc.
letters in a long passage which are the letter in question. Thus, out
of a million letters, close to 130,000 will be E’s, so
/>(£) = 130,000
1,000,000
Sometimes we speak of probabilities of two things occurring
together, either in sequence or simultaneously. For instance, x may
stand for the letter we send and / for the letter we receive. p(x, y)
is the probability of sending x and receiving /. We read this “p of
x, y (we represent the comma by a pause). For instance, we might
send the particular letter W and receive the particular letter B. The
probability of this particular event would be written p{W, B).
Other particular examples of p(x, y) are p(A, A), p{Q, S), p(E, E),
etc. p(x, y) stands for all such instances.
We also have conditional probabilities. For instance, if I transmit
x, what is the probability of receiving/? We write this conditional
probability px(y). We read this “p sub x of/.” Many authors write
such a conditional probability p(y | x), which can be read as, “the
probability of y given x.” I have used the same notation which
Shannon used in his original paper on communication theory.
Let us now write down a simple mathematical relation and
interpret it:
p(x,y) =p(x)px(y)
That is, the probability of encountering x and / together is the
probability of encountering x times the probability of encountering/,
when we do encounter x. Or it may seem clearer to say that
282 Symbols, Signals and Noise
the number of times we find x and y together must be the number
of times we find x times the fraction of times that y, rather than
some other letter, is associated with x.
We frequently want to add many things up; we represent this
by means of the summation sign 2, which is the Greek letter
sigma. Suppose that j stands for an integer, so that j may be 0, 1,
2, 3, 4, 5, etc. Suppose we want to represent
0 + 1 + 2 + 3 + 4 + 5 + 6 + 7 + 8
which of course is equal to 36. We write this
j = 8
We read this, “the sum of j from j equals 0 to j equals 8.” The 2
sign means sum. The j = 0 at the bottom means to start with 0,
and the J = 8 at the top means to stop with 8. The j to the right
of the sign means that what we are summing is just the integers
themselves.
We might have a number of quantities for which j merely acts
as a label. These might be the probabilities of various letters, for
instance, according to Table XVII.
If we wanted to sum these probabilities for all letters of the
alphabet we would write
26
2 PU)
J= 1We read this “the sum ofp(j ) from j equals 1 to 26.” This quantity
is of course equal to 1. The fraction of times A occurs per letter
plus the fraction of times B occurs per letter, and so on, is the
fraction of times per letter that any letter at all occurs, and one
letter occurs per letter.
If we just write
O n Mathematical Notation
Table XVII
Value o f j L etter R eferred to P robability o f
L etter, p ( j )
1 E .13105
2 T .10468
3 A .08151
4 0 .07995
5 N .07098
6 R .06882
7 I .06345
8 S .06101
9 H .05259
10 D .03788
11 L .03389
12 F .02924
13 C .02758
14 M .02536
15 U .02459
16 G .01994
17 Y .01982
18 P .01982
19 W .01539
20 B .01440
21 V .00919
22 K .00420
23 X .00166
24 J .00132
25 Q .00121
26 z .00077
283
it means to sum for all values of j, that is, for all that represent
something. We read this, “the sum of p(j ) over j ” If j is a letter
of the alphabet, then we will sum over, that is, add up, twenty-six
different probabilities.
Sometimes we have an expression involving two letters, such as
/ and j. We may want to sum with respect to one of these indices.
For instance, p(i, j) might be the probability of letter i occurring
followed by letter j, as, p{Q, V) would be the probability of
encountering the sequence QV. We could write, for instance
284
Symbols, Signals and Noise
2 p^’j^
j
We read this, “the sum ofp of i,j with respect to (or, over)j." This
says, let j assume every possible value and add the probabilities.
We note that
^ P O J ) = p(‘)
j
This reads, “the sum of p of i, j over j equals p of /.” If we add up
the probabilities of a letter followed by every possible letter we get
just the probability of the letter, since every time the letter occurs
it is followed by some letter.
Besides addition, subtraction, multiplication, and division we
also want to represent a number or quantity multiplied by itself
some number of times. We do this by writing the number of times
the quantity is to be multiplied by itself above and to the right of
the quantity; this number is called an exponent.
21 = 2
“ 2 to the first (or 2 to the first power) equals 2.” 1 is the exponent.
22 = 4
“2 squared, (or 2 to the second) equals 4.” 2 is the exponent.
23 = 8
“2 cubed (or 2 to the third) equals 8.” 3 is the exponent.
24 = 16
“2 to the fourth equals sixteen.” 4 is the exponent.
We can let the exponent be a letter, n; thus, 2n, which we read
“ 2 to the n,” means multiply 2 by itself n times. an, which we read
“a to the n means multiply a by itself n times.
To get consistent mathematical results we must say
flO = 1
“a to the zero equals 1,” regardless of what number a may be.
On Mathematical Notation 285
Mathematics also allows fractional and negative exponents. We
should particularly note that
a~n = — or 1/a”
an
We read a~n as “a to the minus n.” We read 1/a" as “one over a
to the n.”
It is also worth noting that
Q n Q i n — ^(n+ro)
“a to the n, a to the m equals a to the n plus m.” Thus
23 x 22 = 8 X 4 = 32 = 25
or
41/2 x 41/2 _ 41 _ 4
A quantity raised to the lA power is the square root
41/2 = the square root of 4 = 2.
It is convenient to represent large numbers by means of the
powers of 10 or some other number
3.5 x 106 = 3,500,000
This is read “three point five times ten to the sixth, (or ten to
the six).”
The only other mathematical function which is referred to extensively
in this book is the logarithm. Logarithms can have different
bases. Except in instances specifically noted in Chapter X, all the
logarithms in this book have the base 2. The logarithm to the base 2
of a number is the power to which 2 must be raised to equal the
number. The logarithm of any number x is written log x and read
“log x.” Thus, the definition of the logarithm to the base 2, as given
above, is expressed mathematically by:
2loe1 = x
That is, “2 to the log x equals x”
As an example
v log 8 = 3
23 = 8
286 Symbols, Signals and Noise
Other logarithms to the base 2 are
x logx
1 0
2 1
4 2
8 3
16 4
32 5
64 6
Some important properties of logarithms should be noted:
log ab = log a + log b
log a/b = log a — log b
log dc = c log d
As a special case of the last relation,
log 2m = m log 2 = m
Except in information theory, logarithms to the base 2 are not
used. More commonly, logarithms to the base 10 or the base e
(e = 2.718 approximately) are used.
Let us for the moment write the logarithm of x to the base 2 as
log2 x, the logarithm to the base 10 as logi0 x, and the logarithm
to the base e as loge x. It is useful to note that
log2 X = (log2 10) (logxo x) =
i o g i o ^
log2 x = 3.32 logxo -X
log2 x = (log2 e) (log, x) = ^g-y
l o g 2 x = 1.44 loge X
The logarithm to the base e is called the natural logarithm. It
has a number of simple and important mathematical properties.
For instance, if x is much smaller than 1, then approximately
log,(1 + X ) = X
Use is made of this approximation in Chapter X.
In the text of the book, by log x we always mean log2 x.
Glossary
address: In a computer, a number designating a part of the memory used
to store a number, also the part of the memory which is used to
store a number.
alphabet: The alphabet, the alphabet plus the space, any given set of
symbols or signals from which messages are constructed.
amplitude: Magnitude, intensity, height. The amplitude of a sine wave is
its greatest departure from zero, its greatest height above or below
zero.
attenuation: Decrease in the amplitude of a sine wave during transmission.
automaton: A complicated and ingenious machine. Elaborate clocks
which parade figures on the hour, automatic telephone switching
systems, and electronic computers are automata.
axis: One of a number of mutually perpendicular lines which constitute
a coordinate system.
band: A range or strip of frequencies.
band limited: Having no frequencies lying outside of a certain band of
frequencies.
band width: The width of a band of frequencies, measured in cps.
binary digit: A 0 or a 1.0 and I are the binary digits.
bit: The choice between two equally probable possibilities.
block: A sequence of symbols, such as letters or digits.
block encoding: Encoding a message for transmission, not letter by letter
or digit by digit, but, rather, encoding a sequence of symbols
together.
Boltzmann’
s constant: A constant important in radiation and other
thermal phenomena. Boltzmann’s constant is designated by the
letter k. k = 1.37 x 10 23 joules per degree Celsius (centigrade).
287
288 Glossary
Brownian motion: Erratic motion of very small particles caused by the
impacts of the molecules of a liquid or gas.
capacitor: An electrical device or circuit element which is made up of
two metal sheets, usually of thin metal foil, separated by a thin
dielectric (insulating) layer. A capacitor stores electric charge.
capacity: The capacity of a communication channel is equal to the
number of bits per second which can be transmitted by means of
the channel.
channel vocoder: A vocoder in which the speech is analyzed by measuring
its energy in a number of fixed frequency ranges or bands.
check digits: Symbols sent in addition to the number of symbols in the
original message, in order to make it possible to detect the presence
of or correct errors in transmission.
classical: Prequantum or prerelativistic.
command: One of a number of elementary operations a computer can
carry out, e.g., add, multiply, print out, and so on.
complicated machine: An automaton.
contact: A piece of metal which can be brought into contact with another
piece of metal (another contact) in order to close an electric circuit.
coordinate: A distance of a point in a space from the origin in a direction
parallel to an axis. In three-dimensional space, how far up or down,
east or west, north or south a point is from a specified origin.
CORE (magnetic): A closed loop of magnetic material linked by wires.
Cores are used in the memory of an electronic computer. Magnetization
one way around the core means 1; magnetization the other
way around the core means 0.
cps: Cycles per second, the terms in which frequency is measured.
cycle: A complete variation of a sine wave, from maximum, to minimum,
to maximum again.
delay: The difference between the time a signal is received and the time
it was sent.
detection theory: Theory concerning when the presence of a signal can
be determined even though the signal is mixed with a specified
amount of noise.
digram probability: The probability that a particular letter will follow
another particular letter.
dimension: The number of numbers or coordinates necessary to specify
the position in a space is the number of dimensions in the spade.
The space of experience has three dimensions: up-down, east-west,
north-south.
diode: A device which will conduct electricity in one direction but not in
the other direction.
Glossary 289
discrete source: A message source which produces a sequence of symbols
such as letters or digits, rather than an electric signal which may
have any value at a given time.
distortionless: Transmission is distortionless if the attenuation is the
same for sine waves of all frequencies and if the delay is the same
for sine waves of all frequencies.
double-current telegraphy: Telegraphy in which use is made of three
distinct conditions: no current, current flowing into the wire, and
current flowing out of the wire.
electromagnetic wave: A wave made up of changing electric and magnetic
fields. Light and radio waves are electromagnetic waves.
energy level: According to quantum mechanics, a particle (atom,
electron) cannot have any energy, but only one of many particular
energies. A particle is in a particular energy level when it has the
energy and motion characteristic of that energy level.
ensemble: All of an infinite number of things taken together, such as, all
the messages that a given message source can produce.
entropy: The entropy of communication theory, measured in bits per
symbol or bits per second, is equal to the average number of binary
digits per symbol or per second which are needed in order to
transmit messages produced by the source. In communication
theory, entropy is interpreted as average uncertainty or choice, e.g.,
the average uncertainty as to what symbol the source will produce
next or the average choice the source has as to what symbol it will
produce next. The entropy of statistical mechanics measures the
uncertainty as to which of many possible states a physical system is
actually in.
equivocation: The uncertainty as to what symbols were transmitted when
the received symbols are known.
ergodic: A source of text is ergodic if each ensemble average, taken over
all messages the source can produce, is the same as the corresponding
average taken over the length of a message. See Chapter III.
filter: An electrical network which attenuates sinusoidal signals of some
frequencies more than it attenuates sinusoidal signals of other
frequencies. A filter may transmit one band of frequencies and
reject all other frequencies.
finite-state machine: A machine which has only a finite number of
different states or conditions. A switch which can be set at any of
ten positions is a very simple finite-state machine. A pointer which
can be set at any of an infinite number of positions is not a finite-
state machine.
290 Glossary
fm: Frequency modulation, representing the amplitude of a signal to be
transmitted by the frequency of the wave which is transmitted.
formant: In speech sounds, there is much energy in a few ranges of
frequency. Strong energy in a particular range of frequencies in a
speech sound constitutes a formant. There are two or three principal
formants in speech.
frequency: The reciprocal of the period of a sine wave; the number of
peaks per second.
galvanometer: A device used to detect or measure weak electric currents.
gaussian noise: Noise in which the chance that the intensity measured
at any time has a certain value follows one very particular law,
hypercube: The multidimensional analog of a cube.
hypersphere: The multidimensional analog of a sphere.
inductor: An electric device or circuit element made up of a coil of highly
conducting wire, usually copper. The coil may be wound on a
magnetic core. An inductor resists changes in electric current.
input signal: The signal fed into a transmission system or other device.
Johnson noise: Electromagnetic noise emitted from hot bodies; thermal
noise.
joule: A measure or amount of energy or work.
latency: Interval of time between a stimulus and the response to it.
line speed: The rate at which distinct, different current values can be
transmitted over a telegraph circuit.
linear: An electric circuit or any system or device is linear if the response
to the sum of two signals is the sum of the responses which would
have been obtained had the signals been applied separately. If the
output of a device at a given time can be expressed as the sum of
products of inputs at previous times and constants which depend
only on remoteness in time, the device is necessarily linear.
linear prediction: Prediction of the future value of a signal by means
of a linear device.
map: To assign on one diagram a point corresponding to every point on
another diagram.
maxwell’
s demon: A hypothetical and impossible creature who, without
expenditure of energy, can see a molecule coming in a gas which
is all at one temperature and act on the basis of this information.
memory: The part of an electronic computer which stores or remembers
numbers.
message: A string of symbols; an electric signal.
message source: A device or person which generates messages.
Glossary 291
negative feedback: The use of the output of a device to change the
input in such a way as to reduce the difference between the input
and a prescribed input.
negative feedback amplifier: An amplifier in which negative feedback
is used in order to make the output very nearly a constant times
the input, despite imperfections in the tubes or transistors used in
the amplifier.
network: An interconnection of resistors, capacitors, and inductors.
noise: Any undesired disturbance in a signaling system, such as, random
electric currents in a telephone system. Noise is observed as static
or hissing in radio receivers and as “snow” in TV.
NOISE temperature: The temperature a body would have to have in order
to emit Johnson noise of any intensity equal to the intensity of an
observed or computed noise.
nonlinear prediction: Prediction of the future value of a signal by
means of a nonlinear device, that is, any device which is not linear.
ORIGIN: The point at which the axes of a coordinate system intersect.
Output signal: The signal which comes out of a transmission system or
device.
period: The time interval between two successive peaks of a sine wave.
periodic: Repeating exactly and regularly time after time.
perpetual motion: Obtaining limitless mechanical energy or work contrary
to physical laws. Perpetual-motion machines of the first kind
would generate energy without source. Perpetual-motion machines
of the second kind would turn the unavailable energy of the heat
of a body which is all at one temperature into ordered mechanical
work or energy.
phase: A measure of the time at which a sine wave reaches its greatest
height. The phase angle between two sine waves of the same
frequency is proportional to the fraction of the period separating
their peak values.
phase shift: Delay measured as a fraction of the period rather than as a
time difference.
phase space: A multidimensional space in which the velocity and the
position of each particle of a physical system is represented by
distance parallel to a separate axis.
phoneme: A class of allied speech sounds, the substitution of one of which
for another in a word will not cause a change in meaning. The
sounds of b and p are different phonemes, the substitution of one
of which for another can change the meaning of a word.
292 Glossary
potential theory: The mathematical study of certain equations and their
solutions. The results apply to gravitational fields, to certain aspects
of electric and magnetic fields, and to certain aspects of the flow
of air and liquids.
power: Rate of doing work or of expending energy. A watt is 1 joule per
second.
probability: In mathematics, a number between 0 and 1 associated with
an event. In applications this number is the fraction of times the
event occurs in many independent repetitions of an experiment. E.g.,
the probability that an ideal, unbiased coin will turn up heads is .5.
quantum: A small, discrete amount of energy and, especially, of electromagnetic
energy.
quantum theory: Physical theory that takes into account the fact that
energy and other physical quantities are observed in discrete
amounts.
radiate: To emit electromagnetic waves.
radiation: Electromagnetic waves emitted from a hot body (anything
above absolute zero temperature).
random: Unpredictable.
redundant: A redundant signal contains detail not necessary to determine
the intent of the sender. If each digit of a number is sent twice
(1 10 0 1 1 instead of 1 0 1) the signal or message is redundant.
register: In a computer, a special memory unit into which numbers to
be operated on (to be added, for instance) are transferred.
relay: An electrical device consisting of an electromagnet, a magnetic bar
which moves when the electromagnet is energized, and pairs of
contacts which open or close when the bar moves.
resistor: An electrical device or circuit element which may be a coil of
fine poorly conducting wire, a thin film of poorly conducting
material, such as carbon, or a rod of poorly conducting material.
A resistor resists the flow of electric current.
sample: The value or magnitude of a continuously varying signal at a
particular specified time.
sampling theorem: A signal of band width W cps is perfectly specified
or described by its exact values at 2 IT equally spaced times per
second.
servomechanism: A device which acts on the basis of information received
to change the information which will be received in the future in
accordance with a specific goal. A thermostat which measures the
temperature of a room and controls the furnace to keep the temperature
at a given value is a servomechanism.
sign: In medicine, something which a physician can observe, such as an
Glossary 293
elevated temperature. In linguistics, a pictograph or other imitative
drawing.
signal: Any varying electric current deliberately transmitted by an electrical
communication system.
sine wave: A smooth, never-ending rising and falling mathematical curve.
A plot vs. time of the height of a crank attached to a shaft which
rotates at a constant speed is a sine wave.
single-current telegraphy: Telegraphy in which use is made of two
distinct conditions: no current and current flowing into or out of
the wire.
space: A real or imaginary region in which the position of an object can
be specified by means of some number of coordinates.
stationary: A machine, or process, or source of text is stationary, roughly,
if its properties do not change with time. See Chapter III.
statistical mechanics'. Provides an explanation of the laws of thermodynamics
in terms of the average motions of many particles or the
average vibrations of a solid.
statistics: In mathematical theories, we can specify or assign probabilities
to various events. In judging the bias of an actual coin, we collect
data as to how many times heads and tails turn up, and on the basis
of these data we make a somewhat imperfect statistical estimate
of the probability that heads will turn up. Statistics are estimates
of probability on the basis of data. More loosely, “the statistics of
a message source” refers to all the probabilities which describe or
characterize the source.
stochastic: A machine or any process which has an output, such as
letters or numbers, is stochastic if the output is in part dependent
on truly random or unpredictable events.
store: Memory.
subject: A human animal on which psychological experiments are carried
out.
symbol: A letter, digit, or one of a group of agreed upon marks. Linguists
distinguish a symbol, whose association with meaning or objects
is arbitrary, from a sign, such as a pictograph of a waterfall.
symptom: In medicine, something that the physician can know only
through the patient’s testimony, such as, a headache, as opposed
to a sign.
system: In engineering, a collection of components or devices intended
to perform some over-all functions, such as, a telephone switching
system. In thermodynamics and statistical mechanics, a particular
collection of material bodies and radiation which is under consideration,
such as, the gas in a container.
294 Glossary
tessaract: The four-dimensional analog of a cube, a hypercube of four
dimensions.
theorem: A statement whose truth has been demonstrated by an argument
based on definitions and on assumptions which are taken to
be true.
thermal noise: Johnson noise.
thermodynamics: The branch of science dealing with the transformation
of heat into mechanical work and related matters.
total energy: The total energy of a signal is its average power times its
duration.
transistor: An electronic device making use of electron flow in a solid,
which can amplify signals and perform other functions.
vacuum tube: An electronic device making use of electron flow in a
vacuum, which can amplify signals and perform other functions.
vocoder: A speech transmission system in which a machine at the transmitting
end produces a description of the speech; the speech itself
is not transmitted, but the description is transmitted, and the
description is used to control an artificial speaking machine at the
receiving end which imitates the original speech.
watt: A power of 1 joule per second.
waveguide: A metal tube used to transmit and guide very short electromagnetic
waves.
white noise: Noise in which all frequencies in a given band have equal
powers.
Zipf’
s law: An empirical rule that the number of occurrences of a word
in a long stretch of text is the reciprocal of the order of frequency
of occurrence. For example, the hundredth most frequent word
occurs approximately 1/100 as many times as the most frequent
word.
Index
Abbott’s Flat land, 166
Absolute zero, 188, 292
Acoustics, 126; continuous sources in,
59; network theory in, 6
Addresses, defined, 222, 287; illustrated,
223
Aerodynamics, 20; potential theory in, 6
Aiken, Howard, 220
Algebra, 278; Boolean, 221
Automata, 209, 227; defined, 219, 287;
examples of, 287
Averages, ensemble, 58-59, 60; time,
58-59, 60
Averback, E., 249
Axes, defined, 287; in multidimensional
spaces, 167-169
Ayer, A. J„ on importance of communication,
1
Alphabet, defined, 287; see also Letters
of alphabet
Babbitt, Milton, xi
Ambiguity, in sentences, 113-114
Band limited, defined, 287; signals, 170-
Amplification, by radio receivers, 188—
182,272-274
191
Band width, 131, 173-175, 188-189,
Amplifiers, 294; broad-band and nar192; amount of information transrow-band,
188; gain in, 217; Maser,
missible
over, 40, 44; channel
capacity and, 178; defined, 38,287;
191; negative feedback, 216-218,
227
power and, 178; represented by
Amplitudes, 131; defined, 31, 287; in
amplitude, 171
pulse code modulation, 132-133; in
Bands, defined, 287; line speed and, 38
samples of band-limited signals,
Bell, Alexander Graham, 30
171-174; see also Attenuation
Berkeley, Bishop, 116, 117, 119
Antennas, 184-185; in interplanetary
Binary digits, 206; alternative number of
communication, 197
patterns determined by, 71, 73-74;
Approximations, see Word approximacontracted to “bit,” 98; computers
tions
and, 222, 224; defined, 287; encodArithmetic, as mathematical theory, 7,
ing
of text in, 74-75, 76-77, 78-80,
8; units in computers, 223
83-86,88-90, 94-98; errors in transmission
of, 148-150, 157-163; not
Ashby, G. Ross, 218
necessarily same as “bit,” 98-100;
Attenuation, defined, 33, 287; in distortionless
transmission, 289; by filters,
stored in computers, 222, 223; in transmission of speech, 148-150,
289; frequency and, 33-34; number
157-163; “tree of choice” of, 73-
of current values and, 38
74,99
295
296 Index
Binary system of notation, decimal system
and, 69-70, 72-73, 76-77; octal
system and, 71, 73
Bit rate, defined, 100
Bits, 8, 66; as contraction of “binary
digit,” 98; defined, 202, 287; as
measurement of entropy, 80-86,
88-94, 98-100; not necessarily
same as “binary digit,” 98-100; in
psychological experiments, 230-
231; per quantum, 195, 197
Black, Harold, 216
Blake, William, 117
Block encoding, 77, 90, 177, 182; check
digits in, 159-161; defined, 75, 287;
error in, 149, 156-157
Blocks, defined, 75, 287; “distance”
between, 161-162; entropy and,
90-93, 94, 97; Huffman code and,
97, 101; length of, 101-103, 127
Bodies, forces on, 2-3; hot, 186, 207,
290, 292; in motion, 2
Bolitho, Douglas, 259
Boltzmann, 209
Boltzmann’s constant, 188, 202; defined,
287
Boolean algebra, 221
“Breaking,” in modulation systems, 181
Breakthrough, 140
Bricker, P. D., xi
Brooks, F. B. Jr., 259
Brownian motion, 29; defined, 185,288
C language, 224
Cables, linearity of, 33; insulation of, 29;
transatlantic, 26, 29, 138, 143;
voltage in, 29
Cage, John, 255
Campbell, G. A., 30
Capacitors, 33; defined, 5, 288
Capacity, channel, 97, 98, 106, 155-156,
158-159, 164, 176, 275-276; defined,
288; information, 97
Carnot, N. L. S., 20
Channel capacity, 107; band width and,
178; for continuous channel plus
noise, 176-177; defined, 97, 106,
164; entropy less than, 98, 106;
errors in transmission and, 155-
Channel capacity (Continued)
156; measurement of, 164; with
messages in two directions, 275—
276; of symmetrical and unsym-
metrical binary channels, 158-159,
164-165
Channel vocoder, 138; defined, 288;
illustrated, 137
Channels, capacity of, 97, 98, 106, 155—
156, 158-159, 164, 176, 275-276;
error-free, 163; noisy, 107,145-165,
170-182, 276; symmetrical binary,
157-159, 164-165
Checkdigits, 159-161, 165; defined,288
Checker-playing computers, 224, 225
Cherry, Colin, 118
Chess-playing computers, 224,225
Choice, in finite-state machines, 54-56;
in language, 253-254; in message
sources, 62, 79-80, 81; see also
Bits, Freedom of choice
Chomsky, Noam, 112-115, 260; Syntactic
Structures, 113n.
Ciphers, 64,271-272
Circuits, accurate transmission by, 43;
contacts in, 288; linear, 33, 43-44;
relay, 220, 221; undersea, 25
Classical, defined, 288
Codes, in cryptography, 64, 118, 271—
272; error-correcting, 159-163, 165,
276; Huffman, 94-97, 99, 100, 101,
105; in telegraphy, 24-29; see also
Encoding, Morse code
Coding, see Encoding
Communication, aim of, 79; as encoding
of messages, 78; interplanetary,
196-197; quantum effects and, 192—
196; see also Language
Communication theory (Information
theory), 18, 126, 268-269; art and,
250-267; ergodic sources and, 60-
61, 63; as general theory, 8-9; as
mathematical theory, ix-x, 9, 18,
60-61, 63, 278; multidimensional
geometry in, 170, 181, 183; origins
of, 1, 20-44; physics and, 24, 198;
psychology and, 229-249; usefulness
of, 8-9, 269
Companding, defined, 132
Compilers, 224
Complicated machines, see Automata
Computers, 66, 209, 219, 287; cores in,
222, 288; Datatron, 259; decisions
of, 221, as finite-state machines,
62; grammar for, 115; literary
work by, 264; memories (stores) of,
221-222, 290; music by, 225, 250,
253, 259-261; prediction by, 210—
212; programming, 221-225; relay,
220-221; transistor, 220; “understanding”
in, 124; uses of, 224-226;
vacuum tube, 220; visual arts and,
264-266
Contacts, defined. 288; in relays, 292
Continuous signals, encoding of, 66-68,
78, 131-143, 276; entropy of, 131;
frequency of, 67, 131; noise and,
170-182, 276; theory of, 170-182,
203
Coordinates, 167-169; defined, 288
Cores, magnetic, addresses in, 222;
defined,288
Costello, F. M„ xi
Cps, defined, 31, 288
Cryptography, theory of, 271-272; see
also Codes
Currents, electric, detection of, 27, 290;
induced by magnetic field, 29n„ Maxwell
on, 5n.; in telephony, 30; values
of, 28, 36-38, 44; see also Doublecurrent
telegraphy, Noise. Signals,
Single-current telegraphy
Cybernetics, 41; described, 208-210,
226-227; etymology of, 208
Cycles, Carnot, 20; defined, 31, 288
Datatron computer, 259
Decimal system of notation, 69, 72-73
Delay, defined, 33, 288; in distortionless
transmission, 289; see also Phase
shift
Detection theory, 208, 215; defined, 288
Dielectrics, in capacitors, 5, 288
Digram probabilities, 51,56, 57, 92-93;
defined, 50, 288
Dimensions, defined, 288; four, 166-
167; infinity of, 167-170; phase
spaces as, 167; two, 166; up-down,
east-west, north-south, 288; see-also
One-to-one mapping
Diodes, defined, 288
Index 297
Discrete signals, theory of (Fundamental
theorem of the noiseless channel),
98, 106, 150-159, 203
Discrete sources, 66, 67; defined, 59,
289: noise and, 150-156
Distance, between blocks in code, 161 —
162
Distortionless, defined, 34, 289
DNA (Desoxyribonucleic acid), 65
Double-current telegraphy, defined, 27,
289
Dudley, Homer, 136
Dunn, Donald A.. 263
Eckert, J. P„ 220
Edison, Thomas, quadruplex telegraphy
of, 27-29, 38
Efficient encoding, 75-76, 77. 104, 106,
146-147, 276; by block encoding,
101-103, 156-157; for continuous
signals, 131-143, 276; by Huffman
code, 94-98, 101, 105; in Morse
code, 43, 131; number of binary
digits needed in, 78-80, 88; principles
of, 142; for TV transmission,
131, 139-143, 276; for voice transmission.
131; see also Encoding
Effort, economy of, in language, 238-
239, 242
Einstein, Albert, 145, 167; on Brownian
motion, 185; on Newton, 269
Electromagnetic waves, 5n.; defined,
289; generation of, 185-186; on
wires, 187-188
Electronic digital computers, see Computers
Encoding, 8, 76, 78; “best” way of, 76,
77, 78-79; into binary notation,
74-75, 76-77, 78-80, 83-86, 88-90,
94-98; channel capacity and, 97-
98; of continuous signals, 66-68,
78, 131-143, 276; cryptographic,
64-65; dangers in, 143-144; of
English text, 56, 74-75, 76-80, 88,
94-98, 101-106, 127-129; in FM,
65, 178-179; of genetic information,
64-65; Hagelbarger’s method of,
162; by Huffman code, 94-97, 100,
105, 128; into Morse code, 24-
27, 65; of nonstationary sources,
59n.; noise and, 42, 44, 144, 276;
298
Encoding (Continued)
by patterns of pulses and spaces,
68-69, 78; physical phenomena
and, 184; quantum uncertainty and,
194-197; of speech, 65, 131-139,
276; in telephony, 65,276-277; word
by word, 93, 143; see also Block
encoding, Blocks, Efficient encoding
Encrypting of signals, 276
Energy, 171; electromagnetic, 185; as
formants, 290; free, 202, 204-206,
207; of mechanical motion, 185;
organized, 202; ratio, in signal and
noise samples, 175-179, 182; thermal
and mechanical, 202; see also
Quanta, Total energy
Energy level, defined, 203, 289
English text, encoding of, 56, 74-75,
76-80, 88, 94-98, 101-106, 127-
129; see also Letters of alphabet.
Word approximations. Words
Eniac, 220
Ensembles, 41,42; defined, 57, 289
Entropy, 105; per block, 91, 106; channel
capacity and, 98, 106; in communication
theory, 23-24, 80, 202,
206, 207; conditional, 153; of continuous
signals, 131; defined, 66,
80, 202, 289; estimates of, 91, 106;
of finite-state machines, 93-94, 153;
formulas for, 81,84, 85; grammar
and, 110, 115; highest possible, 97,
106; Huffman code and, 94-98;
human behavior and, 229; of
idealized gas, 203-205; per letter of
English text, 101-103, 111, 130;
measured in “bits,” 80-81; message
sources and, 23, 81-85, 88-94, 105,
206; in model of noisy communication
system, 152-155; per note of
music, 258-259; in physics (statistical
mechanics and thermodynamics),
21-23, 80, 198, 202, 206, 207;
probability and, 81-86; reversibility
and, 21-22; of speech, 139; per
symbol, 91,94, 105
Equivocation, defined, 154, 289; in
enciphered message, 272; entropy
and, 155, 164; in symmetrical
binary channels, 155, 164
Ergodic, defined, 289
Index
Ergodic sources, defined, 59,63; English
writers as, 63; as mathematical
models, 59-61,63,79, 172; probability
in, 90, 172
Errors, correction of, 159-163, 165;
detection of, 149-150; in noisy
channels, _ 147-165; reduction
through redundancy of, 149-150,
163, 164-165; in transmission of
binary digits, 148-150, 157-163;
see also Check digits, Noise
Exponents, defined, 284
Fano, 94
Feedback, see Negative feedback
Fidelity criterion, 131, 274
Filtering, 208, 210, 215
Filters, 227; defined, 41,289; in smoothing,
215; in vocoders, 136-138
Finite-state machines, defined, 289; description
of, 54-56; electronic digital
computers as, 62; entropy of,93-
94, 153; grammar and, 103-104,
111-112, 114-115; illustrated, 55;
men as, 62, 103; randomness in, 62
Flatland, 166
Flicker, in motion pictures and TV, 141
FM, “breaking” in, 181; defined, 290;
encoding in, 65, 178-179
Foot-pound, 171
Formants, defined, 290
Fortran, 224
Fourier, Joseph, analyzes sine waves,
30-34, 43-44
Freedom of choice, 48; e ntropy and, 81,
see also Choice
Frequency, 131; defined, 31, 290; of input
and output sine waves, 33; of
letters in English, 47-54, 56, 63;
quantum effects and, 196; in speech,
135-138; of TV signals, 67; of voice
signals, 67; of white noise, 173
Frequency modulation, see FM
Friedman, William F., 48
Fundamental theorem of the noiseless
channel, 106; analogous to quantum
theory, 203; reasoning of, 150—
159, 163, 164, 165; stated, 98, 156
Gabor, Dennis, “Theory of Communication,”
43
Gain, in amplifiers, 217-218
Galvanometers, defined, 27, 290
Gambling, information rate and, 270-
271
Games, as illustrations of theorems and
proofs, 10-14; played by computers,
224, 225, 226
Gas, 62, 293; as example of physical
system, 203-206; ideal expansion
of, 20n.; motion of molecules in,
185
Gaussian noise, 177, 276; defined, 173-
174, 276; monotony of, 251; same
as Johnson noise, 192
Genetics, information theory and, 64-65
Geometry, 4; computers and, 224, 225;
Euclidean, 7, 14; multidimensional,
166-170; in problems of continuous
signals, 181, 182
Gilbert, E. N.,xi
Go lay, Marcel J. E., 159
Governors, as servomechanisms, 209,
215-216
Grammar, 253-254; in Chomsky, 112-
115; entropy and, 110, 115; finite-
state machines and, 103-104, 111-
112, 114-115; meaning and, 114—
116, 118; phrase-structure, 115;
rules of, 109-110
Guilbaud, G. T„ 52
Hagelbarger, D. W., 162
Hamming, R. W., 159
Harmon, L. D., 120
Hartley, R. V. L., 42; “Transmission of
Information,” 39-40, 176
Heat, motion and, 185-186; waves, 186,
187; see also Temperature, Thermodynamics
Heaviside, Oliver, 30
Heisenberg’s uncertainty principle, 194
Hertz, Heinrich Rudolf, 5
Hex (a game), 10-13
Hiller, L. A. Jr., 259
Homeostasis, defined, 218
Hopkins, A. L. Jr., 259
Horsepower, defined, 171
Hot bodies, 186, 207, 290, 292
Howes, D. H., 240
Huffman code, 94-97, 99, 101, 105, 128;
prefix property in, 100
Index 299
Hydrodynamics, 20
Hyman, Ray, experiment on information
rate by, 230, 232, 234
Hypercubes, defined, 290
Hyperquantization, defined, 132, 142;
effects of errors in, 149
Hyperspace, 170; amplitudes as coordinates
of point in, 172-175
Hyperspheres, defined, 290; volume of,
170, 173
Ideas, general, 119-120
Imaginary numbers, 170
Inductors, 33; defined, 5, 290
Infinite sets, 16
Information, definitions of, 24; as handled
by man, 234; learning and,
230-237, 248-249; measurement of
amount of, 80; uncertainty and,
24; per word, 254
Information capacity, 97
Information rate, 91; in man, 230-237,
238-239, 250-251; see also Entropy
Information theory, see Communication
theory
Input signals, defined, 32, 290
Institute of Radio Engineers, 1
Insulation, of cables, 29; in capacitors,
5, 288
Interference, current values and, 38; intersymbol,
29; see also Noise
Intersymbol interference, defined, 29
Isaacson, L. M., 259
Jenkins, H. M., xi
Johnson, J. B„ 188
Johnson noise (Thermal noise), 196-199;
defined, 290; formulas for, 188-189,
195; same as Gaussian noise, 192;
as standard for measurement, 190
Joule, defined, 171, 290
Julesz, Bela, 264
Karlin, J. E., 232
Kelly, J. L. Jr., 270
Kelvin, Lord, 30
Kelvin degrees, defined, 188
Klein, Martha, 259
Kolmogoroff, A. N., 41, 42, 44, 214
300
Language, basic, 242; choice in, 253-
254; classification in, 122-123; as
code of communication, 118; defined,
253; economy of effort in,
238-239, 242; emotions and, 116—
117; euphony in, 116-117; everyday
vs. scientific, 3-4, 121; experience
as basis for, 123, 242, 245;
human brain and, 242, 245; meaning
in, 114-124; random and non-
random factors in, 246; understanding
and, 117, 123-124; see also
Grammar, Words, Zipf’s law
Latency, defined, 290; as proportional
to information conveyed, 230-232
Learning, experiments in, 230-237,248-
249; by machines, 261
Least effort, Zipf’s law and, 238-239
Letters of alphabet, binary digit encoding
of, 74-75, 78, 128-129; entropy
per, 101-103, 111, 130; frequency
of, 47-54, 56, 63; predictability of,
47-52; sequences of, 49-54, 63, 79;
suppression of, 48
Licklider, J. C. R., 232
Line speed, defined, 38, 290
Linear prediction, 211-212, 227; defined,
290
Linearity, defined, 32-33, 290; of circuits,
33, 43-44
Liquids, motion of molecules in, 185
Logarithms, bases of, 285; explained,
36, 82, 285-286; Nyquist’s relation
and,36-38
Machines, desk calculating, 222; finite-
state, 54-56, 62, 93-94, 103-104,
111-112, 114-115, 153, 289; learning
by, 261; pattern-matching, 120;
perpetual-motion, 198-202, 206,
291; translation, 54, 123, 225; Turing,
114; see also Computers
Man, behavior of, 126, 225; emotions
in, 116-117; entropy and, 229; as
finite-state machine, 62, 103; information
rate in, 230-237, 238-239,
250-251; memory in, 248-249; as
message source, 61, 103-104; negative
feedback in, 218, 227; see also
Language
Index
Mandelbrot, Benoit, xi, 240, 246-247
Map, defined, 290
Mapping, continuous, 16, 179; one-to-
one, 14-15, 16-17, 179
Mars, transmission from vicinity of,
192-194
Maser amplifier, 191, 194
Mathematical models, function of, 45-
47; ergodic stationary sources as,
59-61, 63, 79, 172; in network
theory, 46; to produce text, 56
Mathematics, intuitionist, 16; new tools
in, 182; notation in, 278-279; potential
theory in, 6; purpose of, 17;
theorems and proofs in, 9-17; in
words and sentences, 278-279
Mathews, M. V„ xi, 253
Mauchly, J. W., 220
Maxwell, James Clerk, 198, 209; Electricity
and Magnetism, 5n., 20;
equations of, 5-6, 18
Maxwell’s demon, 198-200; defined,
290; illustrated, 199
Meaning, language and, 114-124
Memories, 221-222; addresses in, 222,
287; defined, 290
Memory, in man, 248-249
Merkel, J„ 230
Message sources, 8; basis of knowledge
of, 61; choice in, 62, 79-80, 81; defined,
206, 290; discrete, 59, 66, 67,
150-156, 289; entropy and, 23, 81-
85, 88-94, 105, 206; ergodic, 57-61,
63, 79, 90, 172; intermittent, 91;
natural structure of, 128; nonsta-
tionary, 59n.; simultaneous, 80,
91-92; stationary, 57-59, 172-173;
statistics of, 293; as “tossed” coins,
81-85; see also Signals
Messages, defined, 290; increase and
decrease of number of, 81; see also
Signals
Miller, George A., 248
Minsky, Marvin, 226
Modes, 195
Modulation, improved systems of, 179,
277; noise and, 180-181; see also
FM, Pulse code modulation
Molecules, Maxwell’s demon and, 198—
200; motion of, 185-186
Morse, Samuel F. B., 24-25,42, 43
Morse code, 24-25, 40, 65, 97, 129, 131;
limitations on speed in, 25-27
Motion, Aristotle on, 2; Brownian, 29,
185, 288; energy of, 185; of molecules,
185-186; Newton’s laws of,
2-3, 4-5, 8, 18, 20, 203
Mowbray, G. H., 231
Multiplex transmission, defined, 132
Music, 65, 66; appreciation of, 251-252;
composition of, 250-253; electronic
composition of, 224, 225, 250, 253,
259-261; Janet compiler in, 224;
rules of, 254-255; statistical composition
of, 255-259
Nash, see Hex
Negative feedback, 209, 227; in animal
organisms, 218; defined, 215, 291;
as element of nervous control, 219,
227; linear, 216; nonlinear, 216;
unstability in, 216,227; uses of, 218
Negative feedback amplifiers, 216-218,
227; defined, 291; illustrated, 217
Nervous system, as finite-state machine,
62; negative feedback and, 219,227
Network theory, 8, 18; in acoustics, 6;
defined, 5-6; generality of, 6-7;
linearity in, 33; mathematical
models in, 46
Networks, 5; defined, 5, 291; as filters,
289; simultaneous transmission of
messages over, 275-276
Neumann, P. G., 259
Newman, James R., ix, xi; World of
Mathematics, x
Newton, Isaac, 14, 269; laws of, 2-3,
4-5, 8, 18, 20, 145, 203
Noise, 43, 207; attempts to overcome,
29, 146-165, 170-182; “breaking”
to, 181; causes of, 184-185; continuous
signals and, 170-182; in
discrete communication systems,
150-156; electromagnetic, 188-190;
encoding and, 42, 44, 144, 276; extraction
of signals from, 42, 44;
number of current values and, 38;
power required with, 192; in radar,
41-42, 213-214; in radio, 145, 184-
185; 188-191, 207, 291; from random
error in samples, 131-132; as
“snow” in TV, 144, 146, 291; in
Index 301
Noise (Continued)
telegraphy, 38, 147; in telephony,
145, 162, 291; in teletypewriter
transmission, 146; see also Gaussian
noise, Johnson noise, White
noise
Noise figure, formula for, 191
Noise temperature, defined, 291; as
measure of noisiness of receiver,
190-192
Noiseless channel, theorem of, 98, 106,
150-159, 163, 164, 165, 203
Nonlinear prediction, 213-214; defined,
291
Numbers, imaginary, 170; see also
Binary digits, Binary system of notation
Nyquist, Harry, “Certain Factors Affecting
Telegraph Speed,” 35-39,
176, 182; “Certain Topics in Telegraph
Transmission Theory,” 39;
on Johnson noise, 188
Off-on signals, encoding by, 68-77; see
also Single-current telegraphy
One-molecule heat engine, 200-201, 204
One-to-one mapping, 14-15, 16-17, 179
Origin, defined, 291; see also Points
Output signals, defined, 32, 291
Parabolic reflectors, 189
Particles, indistinguishability of, 7;
energy level of, 289
Period, defined, 31, 291
Periodic, defined, 31, 291
Perpetual motion, defined, 291; machines,
198-202, 206
Phase, defined, 31, 291
Phase angle, defined, 291
Phase shift, defined, 33, 291; see also
Delay
Phase space, 167; defined, 291
Philosophy, Greek, 125; as reassurance,
117
Phoneme, defined, 291; vocoders, 138
Phrase-structure grammar, 115
Physics, 125-126; communication
theory and, 24, 198; theoretical, 4
Pinkerton, Richard C., 258
Pitch, in speech, 135
Plosives, 135-136
302 Index
Poe, Edgar Allen, The Gold Bug, 64;
rhythm of, 116-117
Poincare, Henri, 30
Points, in multidimensional space, 167-
169, 172, 179, 181, 182; see also
Origin
Pollack, H. O., 273-274
Potential theory, 6; defined, 292
Power, average, 172-173; band width
and, 178; concentrated in short
pulses, 177; defined, 292; as measure
of strength of signal and noise,
171; noise, 175-179, 182, 189, 192-
193; ratio (signal power to noise
power), 175-179, 182, 191; receiver,
193; signal, 173-179, 182, 189, 192;
transmitter, 193
Predictability, of chance events, 46-47;
of letters of alphabet, 47-52, 56,
283; randomness and, 61-62; see
also Probability
Prediction, linear, 211-212, 227, 290;
nonlinear, 213-214, 291; from radar
data, 210-214, 227
Prefix property, defined, 100
Probability, 293; conditional, 51, 281;
defined, 292; entropy and, 81-86;
of letters in English text, 47-52, 56,
283; in model of noisy communications
system, 151-153; ofnextword
or letter in message, 61-62; notation
for, 280-284; in word order,
86-87; see also Digram probabilities,
Predictability, Trigram probabilities
Programs (for computers), 62, 221-225;
decisions in, 221; compilers for,
224
Prolate spheroidal functions, 274
Proofs, by computers, 224, 225; constructive,
13, 15-16; illustrated by
“hex,” 10-13; illustrated by mapping,
14-17; nature of, 9-17
Psychology, communication theory and,
229-249
Pulse code modulation, 132, 138, 142,
147, 276; in musical composition,
250
Pupin, Michael, 30
Quadruplex telegraphy, 27-29, 38
Quanta, defined, 292; energy of, 196;
transmission of bits and, 197
Quantization, defined, 68, 132; in TV
signals, 142; see also Hyperquantization
Quantum theory, 125-126, 203; defined,
292; energy level in, 289; unpredictability
of signals in, 196
Quastler, H., 232
Radar, noise in, 41-42,213-214; prediction
by, 210-214,227
Radiate, defined, 186, 292
Radiation, defined, 186, 292; equilibrium
of, 187; rates of 186; see also
Quanta
Radiators, good and poor, 186
Radio, noise in, 145, 184-185, 188-191,
207,291
Radio telescopes, 189-190
Radio waves, as encoding of sounds of
speech, 76; see also Electromagnetic
waves
Random, defined, 292
Ratio, of signal power to noise power,
175-179, 182, 191
Reading speed, 232-237, 238-241; recognition
and, 240
Receivers, efficient, 184-185; measurement
of performance of, 190-192
Recognition, and reading speed, 240
Redundancy, 144; defined, 39, 143, 292;
as means of reducing error, 149-
150, 163, 164-165
Registers, 222; defined, 292
Relays, in circuits, 220, 221; defined,
292; malfunction of, 147
Registers, 223; defined, 292
189; as source of noise power, 189
Response time, see Latency
Reversibility, indicated by entropy, 21-
22
Rhoades, M. V., 231
Riesz, R. R„ 240
RNA (Ribonucleic acid), 65
Runyon, J. P., xi
Samples, 78; of band-limited signals,
171-175, 272-274; defined, 292;
fidelity criterion and, 131; interval
between, 66-68; random errors in,
131-132
Sampling theorem, defined, 66-67;
problems in, 272-274
Scanning, in TV, 140
Science, history of, 19-21; informed
ignorance in, 108; meaning of
words in, 3-4, 18, 121
Secrecy, through encrypting of signals,
276; see also Codes
Sentences, ambiguous, 113-114; generation
of, 112-115
Servomechanisms, 209; defined, 215,292
Shannon, Claude E., xi, 43, 87-89, 93,
94, 98, 129-130, 146, 147, 221, 254,
270, 274, 276; “Communication
Theory of Secrecy Systems,” 271—
272; estimate of entropy per letter
of English text by, 101-103, 111,
130; fundamental theorem of the
noiseless channel of, 98, 106, ISO-
159, 163, 164, 165, 203; “Mathematical
Theory of Communication,”
ix, 1, 9, 41-42; use of multidimensional
geometry by, 170, 173
Shannon, M. E. (Betty), 255, 266
Shepard, R. N„ xi
Signals, analog, 140; band-limited, 170-
182, 272-274; broad-band, 143;
choice of “best” sort of, 42; continuous,
66-68, 78, 131-143, 170-
182, 203, 276; defined, 196, 293;
demodulated, 181; discrete, 66, 67,
98,106, 150-159, 163, 164, 165, 203;
encrypting of, 276; energy of, 172—
175, 182, 294; faint, 214; future
values of, 209; input and output, 32,
290-291; as points in multidimensional
space, 172, 179, 181, 182; in
quantum theory, 196; representation
of, by samples, 66-68; unpredictability
of, 196; on wires, 187—
188; see also Messages, Modulation,
Redundancy, Samples
Signs, 119; defined, 292-293
Sine waves, amplitude of, 31,287; cycles
in, 31, 288; defined, 31, 293;
Fourier’s analysis of, 30-34, 43-44;
period of, 31, 291; see also Frequency
Single-current telegraphy (off-on telegraphy),
defined, 27, 293
Slepian, David, xi, 162, 214, 256, 274
Smoothing, 208, 210, filters in, 215
Index 303
Sources, see Message sources
Space, defined, 293; dimensions in, 166—
170; Euclidean, 169; function, 181;
phase, 167, 291; signal, 181
Space ships, transmission from, 196-197
Speech, encoding of, 65, 131-139, 276;
entropy of, 139; frequencies in,
135-138; pitch in, 135, 136; recognition
of, by computers, 224, 225;
samples in representation of, 67-
68; voiced and voiceless, 135-136;
see also Language, Words
Speed, line, 38; reading, 232-237, 238-
241; of transmission, 24-27, 36-38,
44, 130-131, 155-156, 164
Sperling, G., 249
Stadler, Maximilian, 255
Stars, radio noise from, 190
States, 203-206, 207; defined, 203; see
also Finite-state machines
Stationary, defined, 57-59, 293
Statistical mechanics, defined, 293;
entropy in, 22, 202, 206, 207
Statistics, defined, 293; of a message
source, 293
Stibitz, G. R„ 220, 221
Stochastic, 60, 63; defined, 56, 293;
music, 255-259
Stores, 221-222; defined, 293
Subject, defined, 230, 293
Summation sign, explained, 282-284
Switching systems, 219-220, 224, 227,
287
Syllables, reading speed and, 234-236
Symbols, as current values, 28, 36-38;
defined, 293; probability of occurrence
of, 88-94; primary and secondary,
40; selection of, 39-40
Symptoms, 119; defined, 293
Systems, 202-203; defined, 293; deterministic,
46-47; switching, 219-220,
224, 227
Systems of notation, binary, 69-77;
decimal, 69-70; octal, 71
Szilard, L„ 21, 198
Telegraphy, noise in, 38, 147; Nyquist
on, 35-39, 176, 182; as origin of
theory of communication, 20;
quadruplex, 27-29, 38; speed of,
25-27, 36—38; and telephony on
304
Telegraphy (Continued)
same wire, 38; see also Double-cut-
rent telegraphy, Single-current
telegraphy
Telephony, 126; continuous sources in,
59; currents in, 30; encoding in, 65,
276-277; intercontinental relay in,
277; military, 132; multiplex transmission
in, 132; negative feedback
in, 218; noise in, 145, 162, 291;
rates of transmission in, 130-131;
switching systems in, 219-220, 224,
227, 287; and telegraphy on same
wire, 38
Telescopes, radio, 189-190
Teletypes, errors in type-setting, 149;
noise in, 146
Television, approximation of picture
signal in, 141; color, 140, 142; efficient
encoding in, 139-143, 276;
errors in transmission over, 132;
intercontinental relay in, 277; interplanetary,
194; quantization in,
132; rates of transmission over,
130-131; scanning in, 140; “snow”
in, 144, 146, 291
Temperature, in degrees Kelvin, 188;
radiation and, 186-187; radio noise
power and, 189-190; of sun, 190
Tessaracts, defined, 167, 294
Theorems, defined, 294; illustrated by
“hex,” 10-13; illustrated by mapping,
14-17; nature of, 9-17; proven
by computers, 124, 126,224, 225
Theory, classification of, 7-8; function
of, 4, 18, 125; general vs. narrow,
4-7; physical vs. mathematical,
6-8; potential, 6, 292; of relativity,
145; unified field, 5; see also Communication
theory, Network theory,
Quantum theory
Thermal equilibrium, 199-200
Thermal noise, see Johnson noise
Thermodynamics, 19, 293; defined, 294;
entropy in, 21-23, 198, 202, 207;
laws of, 198-199, 206, 207
Thermostats, 216, 227, 292
“Thinking,” by computers, 226
Thomson, William (Lord Kelvin), 30
Tinguley, Jean, 266
Total energy, free energy in, 202; of
signal, 172-175, 294
Index
Transistors, 33; in computers, 220; defined,
294; malfunction of, 147
Translating machines, 54, 123, 225
Transmission, accurate, 127; of continuous
signals in the presence of noise,
170-182; distortionless, 34, 289;
error-free, 163-164; length of time
of, 40, 175; microwave, 277; of
Morse code, 24-27; multiplex, 132;
“predictors” in, 129-130; rates of,
24-27,36-38,44, 130-131, 155-156,
164; of samples, 131-132; of TV
signals, 139-143, 194; in two directions,
33, 275-276; from vicinity of
Mars, 192-194
“Tree of choice,” binary digits and, 73-
74,99
Trigram probabilities, 51-52, 93
Tuller, W. G„ 43
Turing, A. M., 226
Turing machines, 114
Uncertainty, Heisenberg’s principle of,
194; information as, 24; of message
received, 79-80, 105, 164; see also
Entropy
Understanding, language and, 117, 123—
124
Unstability of feedback systems, 216,
227
Vacuum tubes, 33; in computers, 220;
defined, 294; malfunction of, 147
Visual arts, appreciation of, 266-267;
computers and, 264-267
Viterbi, Andrew J., 192
Vocabulary, size and complexity of, 242-
245
Vocoders, defined, 294; described, 136—
142, illustrated, 137; types of, 138
Voice, see Speech
Voltage, in cables, 29; in detection of
faint signals, 214-215
Volume, in multidimensional geometry,
169-173
Von Neumann, John, 221-222
Watt, defined, 171, 294
Wave guides, 195, 294
Waves, heat, 186, 187; light, 186, 187,
196; radio, 76; of speech sounds,
133-135; see also Electromagnetic
waves, Sine waves
White noise, defined, 173, 294; monotony
of, 251
Wiener, Norbert, 41, 44, 227; Cybernetics,
41, 208, 219; I Am a Mathematician,
209, 210
Wolman, Eric, xi
Word approximations, to English text,
48-54, 86, 90, 110-111, 246, 261-
264; first-order, 49, 53,86,246,261;
fourth-order, 110-111; to Latin,
52; second-order, 51, 53, 90, 262;
third-order, 51-52, 90; zero-order,
49
Index 305
Word-by-word encoding, 93, 143
Words, associations to, 118-119; binary-
digit encoding of, 75, 77, 78, 86-88,
129; determined by qualities, 119-
121; sequences of, 53-56, 79, 110-
111, 122; as used in science, 3-4, 18,
121; see also Language, Zipf *s law
Wright, W. V., 259
Zipf’s law, 238-239; defined, 294; illustrated,
87, 243; other data and, 247
About the Author
Dr. John R. Pierce was born in Des Moines, Iowa, in 1910 and
spent his early life in the Midwest. He received his undergraduate
education at the California Institute of Technology in Pasadena,
and his M.A. and Ph.D. degrees in electrical engineering from the
same institution.
In 1936 he went to the Bell Telephone Laboratories, and was
Executive Director, Communication Sciences Division when he left
in 1971 to become Professor of Engineering at the California Institute
of Technology. He is now Chief Technologist at Caltech’s
Jet Propulsion Laboratory.
Dr. Pierce’s writings have appeared in Scientific American, The
Atlantic Monthly, Coronet, and several science fiction magazines.
His books include Man’s World of Sound (with E. E. David);
Electrons’ Waves and Messages; Waves and the Ear (with W. A.
van Bergeijk and E. E. David); and several technical books, including
Traveling Wave Tubes, Theory and Design of Electron Beams,
Almost All about Waves and An Introduction to Communication
Science and Systems (with E. C. Posner).
Dr. Pierce is a member of several scientific societies, including
the National Academy of Sciences, the National Academy of Engineering,
the American Academy of Arts and Sciences, the Royal
Academy of Sciences (Sweden), and is a Fellow of The Institute
of Electrical and Electronic Engineers, the American Physical
Society and the Acoustical Society of America. He has received ten
honorary degrees and a number of awards, including the National
Medal of Science, the Edison Medal and the Medal of Honor
(IEEE), the Founders Award (National Academy of Engineering),
the Cedegren Medal (Sweden) and the Valdemar Poulson Medal