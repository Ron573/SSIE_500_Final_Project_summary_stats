國立臺灣師範大學教育學院社會教育學系 
碩士論文 
生命綠洲之旅: 
一位女性的靈性轉化學習歷程之自我敘說 
謝佩蓉 
指導教授:秦秀蘭 博士 
中華民國113年1月 
 
 
  
謝辭 
進入師大就讀是一場意外,也是因為受到鼓舞,更是因為生命力量的推動。
我是為了擴張自己的生活圈而回到校園,卻沒想到這一次卻是來療癒自己。每
當我在求學期間又完成了一項任務,我都感覺到與自己的生命更加完整,更不
要說當我要完成這一篇論文的時候,我有一種可以為此立下一尊碑、畫上一個
圓的感覺。當故事走筆到最後一章的時候,我的生命開啟了它的下一章。當我
找回了所謂的那個完整的自己,我的第一課前來考驗著我。那是一首關於孤獨
寂寞的樂曲。人生往往是單獨的,我們孑身而來、淨身而去,或許開始學會一
個人好好的生活,運用自己的力量繼續前進,就是下一篇精彩在等待著我的前
奏。我應該更有自信心一些,就如我的指導教授秀蘭老師所說的,她是一位很
信任學生各自有其腳步的良師甚至益友,老師最愛說她只是我們的陪跑者、論
文的助產師,當初在我差點找不到指導教授的時候,她一封信就答應了我。同
樣是貴人三鐵的另外兩位指導者,是我的兩位口試委員德永老師和歆祐老師,
他們兩位都是在我的生命當中自我成長的重要推手,我非常感謝在整個師大的
生涯中有他們的參與。最重要的是要感謝我的家人,沒有他們我無法完成這麼
漫長的學習歷程,感謝有這些親朋好友的陪伴如苦海明燈。謝謝所有參與過我
生命的你,如果你是過客,謝謝你曾經來過,如果你仍常常被我叨擾,謝謝你
對我的愛護,我對於這些長期支持我的好朋友們的感恩莫此尤甚!謝謝皓杰,
讓我知道自己的不足,謝謝昱涵,讓我長成了一個更好的人,謝謝我的醫生們
對我的照顧和包容,謝謝我那些在遠方的好友,你們從沒有離開我的心中。 
謝 佩 蓉 謹誌 
中華民國112年12月 
i 
中文摘要 
靈性學習是社會教育的重要一環,在不同的性別裡的轉化歷程也會有所其
特色。屬於女性或少數群體的聲音在靈性轉化學習的道路上,往往會遇見可預
期的阻礙,例如不知道如何與外界社會溝通、不知如何表達自己內心真實的聲
音,有時常因為其特殊性與主流價值不同而處於邊陲的聲音。本篇研究探討一
位女性的靈性學習者,在其個體轉化的歷程當中,所經驗到的靈性轉化學習,
在面對自身所處的生命困境與難題之中,如何逐步自我成長、脫穎而出,不再
受限於過去長年束縛住自己的力量,而能展開全新的人生藍圖,翻開下一個篇
章。本篇研究所探討的靈性轉化學習內容,以個體化的歷程和不同的意識轉化
階段為主體,以女英雄的旅程為時間軸的生命轉化與整全實踐的架構,描繪研
究者自我敘說的精采豐富之內在轉化歷程。撰寫這篇研究的目的,是為了讓同
樣處於靈性追求迷惘階段的同道人舉起一盞明燈,讓我們勇於帶著這個特別的
自我踏入這個現實世界,也讓我們更有勇氣面對自己存在世間的意義。 
關鍵詞:靈性學習、轉化學習、靈性轉化學習、個體化、自我敘說 
目次 
 
謝辭 ...................................................................................................................... i 
中文摘要 ............................................................................................................. ii 
Abstract ............................................................................................................. iii 
目次 .................................................................................................................... iv 
表次 .................................................................................................................... vi 
圖次 ................................................................................................................... vii 
第一章 緒論 ........................................................................................................ 1 
第一節 研究背景與動機 ............................................................................... 1 
第二節 研究目的與問題 ............................................................................... 6 
第三節 名詞釋義 ........................................................................................... 7 
第二章 文獻探討 ................................................................................................ 9 
第一節 靈性學習與自我轉化 ....................................................................... 9 
第二節 靈性轉化學習 ................................................................................. 22 
第三節 個體化的歷程 ................................................................................. 39 
第三章 研究設計與實施 .................................................................................. 55 
第一節 研究概念與流程 ............................................................................. 55 
第二節 自我敘說的工具與方法 ................................................................. 56 
第三節 資料整理與分析 ............................................................................. 65 
第四節 研究品質 ......................................................................................... 68 
 
v 
 
第四章 遇見生命的綠洲 .................................................................................. 73 
第一節 從高塔上墜落 ................................................................................. 73 
第二節 迷途的羔羊 ..................................................................................... 80 
第三節 尋找夢中的伊人 ............................................................................. 87 
第四節 歸途與上路 ..................................................................................... 94 
第五章 療癒之路 ............................................................................................ 109 
第一節 滋養我的陰性力量 ....................................................................... 109 
第二節 扶持我的陽性力量 ....................................................................... 117 
第三節 我的靈性轉化與個體化歷程 ....................................................... 125 
第六章 結論與建議 ........................................................................................ 131 
第一節 跨出生命的回顧 ........................................................................... 131 
第二節 建議:科學與靈性 ....................................................................... 135 
參考文獻 ......................................................................................................... 139 
 
 
  
 
vi 
 
表次 
 
表 1  人類一生的心理發展(個體化).......................................................... 42 
表 2  資料編號方式 .......................................................................................... 66 
表 3  書寫大綱一 .............................................................................................. 67 
表 4  書寫大綱二 .............................................................................................. 67 
表 5  質性研究提高信賴度的指標與策略 ..................................................... 71 
表 6  生命綠洲之旅 .......................................................................................... 73 
表 7  療癒之路 ................................................................................................ 109 
表 8  Jung 意識的五個階段之個體化歷程 ................................................... 128 
 
  
 
vii 
 
圖次 
 
圖 1  Assagioli的蛋形圖 .................................................................................. 14 
圖 2  女英雄的自性之旅 .................................................................................. 52 
圖 3  研究概念圖 .............................................................................................. 55 
圖 4  研究流程圖 .............................................................................................. 56 
圖 5  敘事的成分包含敘事情節 ...................................................................... 58 
圖 6  自我敘說的探究過程 .............................................................................. 60 
圖 7  生命轉化與整全實踐的英雄之旅.......................................................... 68 
圖 8  生命時程表 ............................................................................................ 107 
圖 9  看見自己的內在女英雄 ........................................................................ 117 
圖 10 整全自我與靈性轉化學習的女英雄之旅 .......................................... 132 
 
  
viii 
第一章 緒論 
提倡終身學習的學習社會需要人們對於「學習」一事擁有日新又新的熱情,
每一段生命經驗也都可以成為我們「再學習」的素材與養料,只要對於身為成
人學習者保有意識的自覺,瞭解到轉化學習是有效的成人學習方法之一,不斷
地更新自己的生命態度,求有再求好並達臻善與美的境界,我們可以持續地走
在完整自我的道路上。本章將依序介紹本篇的研究背景、研究動機、研究目的
以及研究問題。 
第一節 研究背景與動機 
教育部在2021年頒布《學習社會白皮書》中鼓勵學習型社會,致力於銜接
家庭教育、學校教育與社會教育,藉由跨部門的力量整合正規 (formal)、非正規 
(nonformal) 與非正式 (informal) 的學習資源與機會,強化「學習型台灣」的國
際競爭力。為了跟上時代的脈動,每位台灣公民因而逐漸需要隨時具備虛懷若
谷的學習精神,以期完成公民社會的時代義務。不消說,從前人們在學校當中
的正規教育在進入職場或社會中已不敷使用,而社會教育的功能在於促使成人
擁有更為全面的自省與精進的能力,藉由更多的自我探索以瞭解自身在社會脈
絡中的地位與處境,進而產生願意自我超越的行為,達就自我實現的生命渴望,
靈性轉化學習因此是重要的成人教育與學習的內容。 
壹、成人學習的獨特性 
1 
從 OECD 和 UNESCO 等國際組織的終身學習教育政策,到近年來的世界
潮流,「為生活而學習」的觀念使得終身教育的重點轉變為「終身學習」的態度
(吳明烈,2010,2015)。終身學習之精神和成人生命期的發展階段任務有關
(林美和,2006),以成人學習與發展的角度來看,兩者為同步進行又互相影響,
其力量從出生到死亡為一個長期持續累積的過程,而個體的發展除了生物觀點
與發展心理學的觀點外,也應重視在社會文化脈絡之下,不同生命週期當中所
累積與形塑的個人生命經驗 (Newman & Newman, 2017)。個人的生命期發展不
只是成長與老化的過程,同時也是終身學習與個體社會化的歷程,而成人期的
發展交織著成長與衰退,不論是在個人或社會發展的向度上,「發展」本身即是
成人教育的目的,讓人們能夠應付複雜的角色期待與社會挑戰(林美和,2006)。
成人發展同時具備積極和消極的雙重面向(林美和,2006),從自我發展和生命
期發展來理解,前者有為適應社會改變的需要而發展出來的自我調適觀點 
(Erikson, 1959; Orenstein & Lewis, 2021),個體發展需在社會互動中才會發生認
知意涵的重組 (Gopnik et al., 2017; Kohlberg, 1969),個體也有逐步向上盤旋的生
命序階,將成人發展視為意識轉變的過程 (Kegan, 1982; Kokkos, 2019),後者有
將生命期的發展視為改變與重建生命結構的演進過程 (Igarashi at al., 2018),認
為成人發展為個體潛能的自我實現,同時會有質性和量化的改變,而且會有穩
定發展的趨勢 (Alwin, 2019; Krems et al, 2017)。 
影響發展的主要因素包括:年齡、時間、空間、性別是一種規範性因素,
個人重大生命事件或特殊的生命經驗則是非規範性因素,成人的生命發展大體
是由外向內的過程,在不同的生命階段追求外在的認可或地位後,會進入內省、
反思與自我評鑑(林美和,2006)。由於成人學習和其發展需求並列而行,成人
學習者的特質常與傳統的學習習慣互斥,成人學習者需要知道自己為何而學習,
學習動機受其自我概念之影響為一種自我導向的學習需求 (self-directed learning),
2 
其自主性遠高於傳統教育的被動式接收學習,並受其生活角色所左右為因應生
活與自我實現之動能,成人學習的教育資源,往往在其本身的生命經驗當中可
以汲取使用,因而「學習如何學習」(learning how to learn) 乃為成人教育的重心,
也是終身學習的核心精神之一 (Knowles et al., 2014; Merriam & Baumgartner, 
2020;徐秀菊,2013)。成人學習因此顯示出它有其發展性、適應性、連續性、
轉變性、多元性(林美和,2006)。從成人發展的需求來看待終身學習與社會教
育的角色,可知人類在其不同生命階段的進程當中,都具有持續不斷調整自我
知識的空間,與重新理解自身在社會脈絡所處位置並獲得嶄新經驗,因而每一
段生命經驗都具備終身學習與社會教育的功能,值得吾人花費心思去耙梳、理
解,賦予或重新產生意義。 
貳、男女有別的生命經驗 
當 Freire (1970) 在提出《受壓迫者教育學》的時候,曾對這個食古而僵化
的世界投出震撼彈,將當代的教育觀點和社會變遷關聯在一起。他認為一個成
人需要對自身的現實處境擁有批判與反思的覺察能力,了解到教師與學生之間
的不平等位置、不再逆來順受,開始能進一步採取不同的「提問」行動來突破
階級壓迫與超越文化囹圄之下的生命困境。當時他將這樣的批判教育學觀點傳
送給西方世界時,對這個充滿戰爭與苦難的種族歧視社會注入了清涼音,由於
他的弱勢身分而獲得了讚譽以及廣大迴響。然而,更進一步我們卻可以發現,
那個年代的女性在這一個區塊仍然沒有聲音,在成人教育學的知識席捲地球村
的世界,女性的聲音在長年之中依然是等待開發與「不被」探索的領域,女性
主義的成人學習與批判的意識,幾乎要到了21世紀才開始有了不同而新穎的發
展,這和女性的意識覺醒有著密不可分的關係。 
3 
自20世紀以降,歐美的女權運動與婦女解放,乃至第二波婦女運動的發軔,
女性主義的思潮與時俱進的喚醒女性的主體,每位女人開始期待或被期待能夠
活出自己的潛能,追求真正的自由、平等以及性別公義,這樣的社會運動改變
了傳統的教學重點,以鼓勵女性發展自我為主要目標,讓歷史當中的「she」被
看見,逐步改寫了父權中心思想所建構的世界觀(劉亞蘭,2021)。婦女因其生
理性別的限制,與男性的生命發展不大相同,經歷婚姻與生育的過程,女性和
男性在成人生命期當中的追求也會有所不同。女性相較於男性之下,在其發展
的階段更為重視關係的建立,這並非意味男性不注重人際網絡,而是女性較男
人更多在意的是建立人際關係所需之互動品質 (Kamas & Preston, 2021; Rossi, 
2018)。女性在身心發展與社會文化信念的差異之下,面對自身個人價值的檢視,
也會有不同於男性的自我認知與自我認同,女性的自尊與自我概念和男性是迴
異的 (Rossi, 2018; Schmitt et al., 2017)。自我認同的形成和個體化的歷程有關,
有別於傳統心理學的定義強調個人自主性的發展,人類的社會群性更讓人類經
驗的本質更接近關係的建立與連結,對女性的發展與研究而言將更重視關係的
社會性質 (Hsu et al., 2021; Miller, 2005)。 
女性主義對當代教育學的啟示,除了讓女性的生命經驗被承認,可以成為
合法的知識來源,提供對權力結構的反思之外,其所建構的思維模式與其說是
研究方法,更不如說是另一種認識世界的嶄新視野(楊鳳麟,2022)。而女性主
義帶給這個世界的禮物,不僅僅只是女性的聲音,更是整體人類靈魂的完整性:
它關乎於女性意識的覺醒,以及女性自覺之後的個體化歷程,這樣的轉化學習
經驗需要被完整的支持,這個新世界的性別弱勢才能夠獲得更完整的解放(吳
菲菲,2012)。存在這個世界上的困難當中,不論是貧窮、暴力、教育、經濟、
職場,乃至於健康與疾病,若本身為女性者,其先天上便已在父權結構的社會
體制之下取得了弱勢位階,若非後天的資源與文化條件俱足,每一項弱勢都令
4 
女性較男性難以翻身,也仰賴更多的社會支持介入。而男性的個體化意識也將
受到女性意識的覺醒程度而有所成長,成熟的女性意識不會藉由打壓男性來獲
得力量,也不會透過彰顯自我來施展權力,而是由於充滿了正向的自我價值感
而願意自然不做作地表達自己,不再需要透過貶低而獲得地位,並能夠因為成
為真實的自我而獲得男性力量的充分支持 (Chinen, 2020)。透過轉化學習,被邊
緣化的社群也能因此找到共同體的歸屬感,能促進新的自我理解、關係能力,
以及發展新社群的潛在可能 (Buechner et al., 2020)。 
參、存在的價值:靈魂的光芒 
本篇研究為從女性觀點的出發,探索研究者自身的生命經驗當中,對於情
感與親密關係的追求,其在過程中的成功與失敗、辛酸與甜美的滋味,符合女
性在關係當中的自尊以及自我概念,是建立在與他人彼此互動模式的架構與基
礎之上。其中在過往的生命故事裡,遇到了一位能與之靈魂溝通的男性伴侶,
重新定義了她在自身生命當中的自我定位,更說明了女性的自我認同以及自我
認知與兩性之間的關係建立大為相關。本篇研究探討成人靈性轉化學習的不同
向度,不僅僅只是著重於人際關係的男女性質對於兩性互動的差異,更試圖表
達一位女性在遇到能夠如此緊密的精神契合的對象時,如何引動她的內在所隱
藏著的生命藍圖所寫好的劇情,帶領著她一步一步地進行自我療癒,並且潛意
識也慢慢地引導她與內在原初的戀人重新相會,透過一次又一次的靈性學習,
進而有能力辨認出真實的自我。精神上的契合與靈魂溝通的能力帶領著研究者,
讓她不斷的向內自我探索與再次認識自己的內心,透過重新瞭解自己的真實歷
史故事的過程當中,也逐步的能夠與內在的戀人達成和諧一致的連結狀態與動
態平衡,建立一段穩固的自我關係,是允許對方能夠和平地存在於自己的內在
世界,並且甚至能夠成為引領自我突破障礙的強大力量。此間靈魂的連結允許
5 
著研究者作為一個女性,卻內在住著一位男性的伴侶,接近 Jung的阿尼姆斯的
概念。在這樣的靈性轉化學習歷程當中,研究者的生命經驗屬於獨特而不同的
故事,也充滿著動人的汗血與淚水,值得作為一篇自我敘說的質性探究,彰顯
出無法在統計的量化研究當中展現的光芒。 
第二節 研究目的與問題 
本研究欲探討潛意識的靈魂對話歷程在靈性自我轉化學習當中所扮演的角
色,深究研究者作為女性靈性學習者的自我轉化學習經驗,從情緒經驗與自我
認同之自我概念出發的潛意識走上自我實現的歷程,進行一場自我敘說的質性
探究,以期在女性的靈性轉化學習歷程當中,尋覓生命的蹤跡在潛意識整合之
藍圖所佔據的角落。透過本篇研究,希望可以更加瞭解個體所遭逢的生命困境,
如何經歷了專屬的意識轉化的學習過程,在這些不斷完整自我或個體化的道路
上,如何遇見個體的內在自我以完成獨特的潛意識自我實現的歷程,當個體不
斷向內沉潛向內觸及靈性的光輝,回到現實世界當中又能夠如何回應這個社會,
將會是本篇自我敘說探究的重點。 
研究者並非在一開始的時候就能夠知曉自己即將面臨一段自我轉化的靈性
學習旅程,而是在過了翻山越嶺之後的回望,才看見自己已經穿越了漫漫長路。
在這樣的過程中進行不斷的反思,重新理解與耙梳個體的成長路徑,透過自我
敘說的方式可以從研究當中回答以下問題: 
一、研究者的生命故事如何帶領自我的靈性成長與內外平衡 
二、研究者的兩次靈啟經驗對靈性學習的自我轉化意義 
6 
三、研究者如何走上靈魂的自我實現與完整自我的整全道路 
第三節 名詞釋義 
一、靈性轉化學習 
有別於源自Freire (1970) 的意識覺醒、Habermas (1972) 基於人類的三大學
習旨趣而延伸與發展的 Mezirow (1978) 轉化學習理論,其建構主義取向的自我
導向學習所描述的是學習的認知與理性的過程,轉化學習的目標是改變過去習
以為常的參考架構以及思維模式 (Mezirow, 2018),另一派研究轉化學習理論的
學者Boyd (2003)、Cranton (2000, 2006) 與 Dirkx (2000) 則將焦點放在關注於自
我 (self) 的認同 (identity) 與主體性上(林曉君,2017),將轉化學習的內容當中
之精神層次或內在的心靈工作視為更重要的核心,其依從分析心理學家 Jung的
個體化 (individuation) 理論 (Jung, 1939, 2017) 將之延伸到成人學習的內涵,重視
學習者的心理、情緒與精神的想像,以及其靈性的發展等不同的主題 (Dirkx et 
al., 2006) 。 
二、個體化的歷程 
在 Jung 著名的集體潛意識概念當中,它代表著個體心理成長的能量泉源,
在個體的意識與潛意識之間建立合理而互助的關係,便是心靈健康的首要工作 
(Hopcke, 2002)。Jung 將意識和潛意識之間的關係類比為個人自我 (ego) 情結和
自性 (Self) 原型之間的關係,而自性是完善健全的原型,表現在夢或幻覺中重
複出現的象徵。當意識與潛意識、自我與自性之間存在著良性互動時,人們能
夠感受到自己的獨特性格與人類實存深處的經驗之海相連接,使自己的生活具
7 
有真正的創造力、象徵性和獨特性 (Hopcke, 2002)。Jung 將達到這個心靈平衡
的過程稱之為「個體化」,這是人類的所有心理活動遵循的原則,認為心靈有自
我完善與平衡的傾向。 
三、靈啟 
靈啟(numinous experience或稱之為超凡經驗)是比較主義神學家Otto 
(1926) 提出的概念,對於理解宗教的心理層面具有重要意義。這種神秘的體驗
提供了一個獨特的視角,透過它我們可以理解心理學和宗教之間的深刻聯繫。
所謂的「靈啟」可說是一種在人類的內在心靈當中,屬於高層潛意識的內涵中
所經歷的高峰經驗,它可以為人類帶來創造性的動能,以及生命活力的泉源。
在本篇研究當中,靈啟是一個重要的概念,這樣的生命事件如同敲響進入新時
代的警鐘,當我們受到靈啟的經驗而獲得深度的洞見,擁有了對宇宙和社會抱
持著全新看待的觀點,個體對於人性的價值觀也將因此改變。因為接受到靈啟
(可說是靈魂的啟迪或稱之為靈性的啟蒙),而開啟了嶄新的視野與發展出面對
不同生命態度的可能,更有可能因為獲得新的洞見,而敞開了迎接觀點轉換的
可能。 
8 
第二章 文獻探討 
為更加瞭解本篇研究所欲探討之議題為何,本章文獻探討將分為三部分進
行,分別為靈性學習與自我轉化、靈性轉化學習的探索與分析,並討論個體化
理論與女性個體化發展作為總結。 
第一節 靈性學習與自我轉化 
本節將靈性學習以定義、內涵與意義、理論與批判分別闡述,並輔以國內
外文獻分析之,最後探討自我轉化的可能。 
壹、靈性學習的內涵與意義 
在探討靈性學習以前,我們需要先定義何謂靈性,才能了解靈性學習的內
容與目標,進而討論如何藉由靈性學習促成靈性轉化學習。 
一、靈性的定義及其理論與內涵 
關於靈性的定義眾說紛紜,特別是到了後現代的社會,各種理論更是百家
齊放。以下將針對於本研究中較為重要的理論和學說進行闡述。 
(一)靈性的定義與內涵 
9 
許多作品在廣泛不同情境下都談到靈性,但有些人對靈性會感到不安,因
為他們以二元論的方式來理解,將它與物質或現實世界互相對立,有些人則將
精神和靈性視為宗教的核心或其最高理想,特別是神秘體驗 (King, 2017) 。欲
定義「靈性」(spirituality) 的範疇並不容易,因其攸關人類生命的本身如何「存
在」的真實經驗,在追尋內在的深層意義的過程中,靈性是維繫人類生存動能
的重要泉源,能夠為存在本身賦予重要的內涵與特殊的意義 (Jarvis, 2013)。因
為靈性具備抽象不易描述的特質,所觀察的特性並非來自客觀的實體,而受不
同對象的主觀經驗所形塑,表現在不同的面向上有著因人而異的個殊性,不同
人因為對自身的追求、生命意義的賦予、將自身關連到宇宙的連結之特殊經驗,
使得靈性一詞無法被輕易定義(King, 2017;王嘉陵,2022)。 
從早期對靈性的定義為一種伴隨著天賦的本能,不但能統合個人的主體性
之外,亦能夠產生人際之間的連結感,其反映出個體對生命所抱持的信念與價
值觀,甚至延伸至對超自然的無形力量等如神靈的態度亦包含在內 (Banks et al., 
1984) 。靈性 (spirituality) 是抽象的、主觀的,不同於宗教 (religion) 和信仰 
(faith),它可以是與神、自然、他人或社會環境的連結,因此靈性與生活品質和
生命意義息息相關,靈性可說是人類存有的核心 (Paul & Treschuk, 2020)。人類
的靈性能力屬於靈魂的層次,也能夠彼此奠基於另一個靈魂,人類據以所擁有
的道德情感之心理能力表現於外在的行為與思想便是靈性的展現 (Bennet & 
Bennet,2007)。有些人更傾向「靈性」而非「宗教」,兩者雖然有所區別,但
仍然有所重疊 (King, 2017)。靈性有別於宗教 (religion) 之處,在於靈性屬於個
殊的經驗,而宗教是有組織的信念社群形成的影響力,有與信仰相關的價值和
禮俗,受傳統的規則和文化的指導,須要具備一定的宗教態度、信仰的實踐和
制度化的體系,宗教是對神或超自然力量的服務或崇拜,而信仰 (faith) 通常與
宗教和靈性聯繫在一起,信仰比有組織的宗教更個人化、更主觀、更深刻,並
10 
且能夠與神關聯 (Victor & Treschuk, 2020),可知靈性與宗教雖然兩者關係密切,
但並非完全相同 (Tisdell, 2008),有靈性不一定要參加宗教,參加宗教不一定有
靈性(夏允中等,2018),並且由於它的抽象性質,靈性不容易藉由科學方法驗
證(林耀堂,2014),然而當代的醫學也開始漸漸認知到靈性在護理的重要性 
(Murgia et al., 2020)。 
靈性為個體因其個人生命經驗而進一步思考存在的意義,因為人的內在有
豐沛的心靈資源並且能因此創造或發現其生命意義(趙可式,1998),透過自我
超越的方式尋找到生命的價值,藉由人我或物我的他我關係來體驗人生的道理
(劉淑娟,1999),靈性也是「個人對生命最終價值的所堅持的信念或信仰」,
不論是否關乎宗教,靈性表現的是一種價值觀、哲學態度、生活準則和方法
(杜明勳,2003)。國內學者談論到靈性,亦表明人類與自我、與社會、與神的
連結,進一步擴張至靈魂之間的連結、與宇宙的連結等感受,可以為個體帶來
根本的生命意義,統合出一個更有力量的自我而能夠獲得滿足感(傅佩榮,
2003,2018)。靈性是個體藉由向內觀照與覺察、反省與轉化、昇華的歷程中逐
漸覺醒,而靈性健康包括生命價值觀、道德倫理觀、宇宙世界觀,及全人整合
觀,從人我關係結合自己與他人、社群與自然、環境與地球之間的互動,為一
動態的整合,且不斷持續成長的過程(洪櫻純,2009,2012)。關注靈性健康的
新時代人類,對自我形象的信心發展、自律能力的建立、追尋生命的價值與意
義、與他者連結以及所需要的協調能力,甚至對理想與自我超越的需求並沒有
因為網路的發達而減輕(許慧聆,2018),因此,一個人要選擇怎樣的靈性信念
來活出自己的生命是至關重要的命題(夏允中等,2018)。透過個體的自我療癒
與生命的擴展,將靈性視為一種入世精神的體現,在不斷地轉化之後成為一個
更自由而開放且正向思考的人,活出更為積極接納與自我超越的生命(李聿勛、
唐耀棕,2013;陳紅杏,2015;葉俊廷、李雅慧,2018)。 
11 
(二)關於靈性的理論 
關於靈性的理論可從其認識論 (epistemology) 來探討,研究人們如何認識靈
性、如何認知靈性的存在、如何理解靈性的本質以及其邊界的確定或不確定性
質,以這樣的觀點來研究成人學習與成人教育的面向,求知的過程可以使得靈
性即為人性 (Vella, 2000)。靈性的層次從低至高,有感官層次、理性思維、直觀
方式,直觀法即人們的親身體會所獲得的內在洞見、深層潛意識、較高知見,
融合性的靈性知識超越了語言層次的表達(李安德、傅東華,1992)。除此之外,
直覺和靈感也是獲得靈性的方法,前者是一種不需透過頭腦的推理而獲得的預
知感應能力,後者是一種和「靈」有關的創造性經驗,藉由靈感,潛意識的靈
性知識得以浮現到頭腦的理性意識(引自廖淑純,2011)。 
在 Maslow (1969) 的需求層次論中,也曾補充在自我實現之上,還有一個
靈性層次的需求,有別於X理論(關於人類的生理和心理的安全感需求)和Y
理論(關於人類有愛與歸屬感的需求、尊重與自我實現的需求),亦即著名的
「Z 理論」,他談到除了自我實現之外,人類還有一種更高的超越性,而這樣的
存在也是人性的本質 (Crotright, 2005) 。後來在Csikszentmihalyi (1979, 1988) 的
「心流」學說(flow experience,或稱之「神馳觀」),將之稱為個體在自然狀態
下最自由自在的最佳狀態,在這樣的心流經驗當中,個體全神貫注的投入所進
行的事物當中,並會隨之產生一種自我意識的超越感與極大的喜樂,當所面臨
到的挑戰與所持有的克服技巧彼此消長,而克服的焦慮與技巧達到一個動態平
衡時,一個人所體驗到的成就感最大也最快樂,也能夠因此一直維持在心流
(或者神馳)的正向體驗中 (Csikszentmihalyi, 2023)。亦有將靈性用新物理學的
觀點加以闡釋的研究者,認為宇宙萬物是「源場」(the source) 的產物 (Wilkock,
2012),我們需要意識到它就在我們的精神世界之內,不論時間與空間、能量與
12 
物質、生物與意識,都是彼此互相關聯的存在,一切「如其在上,如其在下」
(As above, so below.),認知這個世界是由一個「神」將其思想現實化的結果,
人們經常處於失憶狀態,需透過生活周遭的其它事物重新領會並逐漸對之覺醒,
發現無窮盡的浩瀚意識,與之取得連結 (Wilkock,2012) 。我們更是能夠透過情
緒語言來和能量場溝通 (Branden, 2010),直接將超越性的心靈意識和物理世界
的時空條件結合來探討靈性的存在,將這樣的「源場」視為靈性的一環。 
在眾多的理論當中,較為有系統的靈性理論觀點仍屬於「Assagioli 的蛋形
圖」。超個人心理學家 Assagioli 以蛋形圖來分別描繪不同意識層次之間的關係,
試圖區分不同的潛意識、自我的意識範圍、超個人或高層的我,與集體潛意識,
茲將不同意識層次的分析如下 (Assagioli, 1991; Crotright, 2005; Lefebvre, 1998),
並以Assagioli的蛋形圖表之,可參考圖1: 
1. 低層次意識 (the lower unconscious) 代表的是比較原始的野性,諸如本能、
衝動、慾望、生物反應,有時會反映在暴力與犯罪的面向,或受到恐懼而扭
曲的記憶,如恐慌、強迫性思考、妄想與幻覺、噩夢,也包含童年的經驗,
人類因為這層連結,而能感應到宇宙力量的運作。 
2. 中層潛意識 (the middle unconscious) 有時稱為前意識,為知識儲存的空間,
與供給日常生活所需的資訊或溝通內容,例如年齡等個人訊息,可藉由反省
而進入意識層次的範圍。 
3. 高層潛意識 (the higher unconscious or superconscious) 的運作被人們所覺知的
時候,常被命名為靈感、啟迪、領悟、洞見,甚至是召喚,而高層潛意識在
人類生活當中的表達方式,通常透過美學觀點、靈性修持或藝術等創造性的
13 
「高峰經驗」、人道與意義的追求、犧牲奉獻的精神與意願,有時候會被人
們描述為「靈啟」,鼓勵人類經驗當中的真善美。 
圖 1  
Assagioli 的蛋形圖 
1  低層潛意識 
2 中層潛意識 
3 高層潛意識或超意識 
4 意識界 
5 意識中心的自我(self) 
6 高層自我或真我自性(Self) 
7 集體潛意識 
資料來源:Lefebvre (1998: 277) 
4. 意識界 (the area of consciousness) 即是在這個世界的此刻,我們所能感受到
的五蘊(色受想行識),因為自身的感覺、情緒、知覺、欲望、動機等不同
期待而做出的計畫與行動,範圍可大可小且不斷變化。 
5. 意識中心的自我 (the conscious ‘I’) 為純粹意識的中樞,不同於意識範圍的層
次,它是一個人根本的自我認同與人格中心,具有知覺與決策的功能,比意
識內容更天生。不論個人的人格如何受到任何環境因素所改變,意識中心的
自我是不會改變的,它持久而專一、貫穿個體的主軸。 
6. 高層自我或自性 (the higher ‘I’ or Self) 又稱為「真我」或超個人自我,為高
層潛意識的中心,Jung曾認為自性及為個人的整個存在主體。自我 (5) 和自
14 
性 (6) 的不同,在於意識的維度高低,但自我和自性並非截然而分的主體和
客體,而是兩者彼此經常連結互動,且彼此之間存在著不同階段。隨著超越
經驗的累績,意識中心的自我會逐漸與集體潛意識相連,跳脫個人框架而更
接近真我(或靈性自我,即梵、佛性、本來面目),它是人類的最高潛能,
具備智慧、慈悲、正義、博愛、無私等超個人特質的精神,使人擁抱社會責
任、邁向合作、寬恕、和平、同理心、環保意識的理想。 
7. 集體潛意識 (the collective unconscious) 是援用 Jung所稱,指涉個人的存在仍
包含於更大的整體性,沒有人能孤立於世界,不能忘記人類仍時時與內在的
宇宙性大我連動與牽連。 
值得一提的是,靈啟(numinous experience 或稱之為超凡經驗)是比較主
義神學家Otto (1926) 提出,這樣的概念對於理解宗教的心理層面有重要意義。
心理學領域的各種理論和實證研究強調了宗教心理學中的超自然經驗的重要性,
探討了超自然與宗教信仰之間的關係,以及這些體驗背後的心理機制與對個人
福祉的影響 (Merkur, 2006)。這種神秘的體驗提供了一個獨特的視角,借此我們
可以瞭解心理學和宗教之間的關聯。超凡經驗或靈啟 (numinous) 為當代宗教研
究源於 20 世紀初至中期發展出來的概念和理論,他以 Immanuel Kant、
Friedrich Schleiermacher 和 Jakob Fries 為基礎,提出了超自然的概念,它是一種
關於價值的範疇 (category of value) 和一種精神狀態 (state of mind) 的表述 
(Sarbacker, 2016)。他認為靈啟為超自然的非理性當中,特別屬於宗教性生活體
驗的基礎,而超自然可以被理解為一種神秘的恐怖和敬畏 (mysterium 
tremendum et fascinans) 和威嚴 (majestas) 的感覺,它在完全不同 (das ganz 
Andere) 的存在下體驗,因此難以藉由人類的語言和其他可能的媒介直接表達 
(Sarbacker, 2016)。由於宗教信仰的情感體驗無法直接言傳和直接感受的特性,
15 
要以令人滿意的方式理解超自然現象,需要研究者運用自身之宗教信仰的情感
經驗作為範本,也因此它們成為後來的世俗宗教學者爭論的焦點。他在後來的
作品中給了許多例子 (Otto, 2016),說明了超自然的概念可以跨文化地擴及應用
在基督教以外的系統,例如印度教和佛教等傳統信仰。而他的超自然理論在 20
到 21 世紀中對宗教學術研究的發展產生了重大的影響力,例如 Jung 在潛意識
的理論中廣泛引用了超自然的概念即為此類之應用  (Sarbacker, 2016)。 
因此由上述所知,靈性雖然不容易定義,但我們仍可以將它描述為「個人
對生命最終價值的所堅持的信念或信仰」,在直觀或心流的高峰經驗中找到人性
存在的本質,藉由意識的「我」與高層自我或自性的連結,我們可以不斷的進
行靈性的學習,將視野從自身的立足點出發擴展至周邊的感知世界,進而獲得
更多的知能來探索這個世界和宇宙。同時可以瞭解到,所謂的「靈啟」可說是
一種在人類的內在心靈當中,屬於高層潛意識的內涵中所經歷的高峰經驗,它
可以為人類帶來創造性的動能,以及生命活力的泉源。在本篇研究當中,靈啟
是一個重要的概念,這樣的生命事件如同敲響進入新時代的警鐘,當我們受到
靈啟的經驗而獲得深度的洞見,擁有了對宇宙和社會抱持著全新看待的觀點,
個體對於人性的價值觀也將因此改變。因為接受到靈啟(可說是靈魂的啟迪或
稱之為靈性的啟蒙),而開啟了嶄新的視野與發展出面對不同生命態度的可能,
更有可能因為獲得新的洞見,而敞開了迎接觀點轉換的可能。立基於靈性學習
的自我轉化體驗是相當個人化的獨特歷程,很難會有兩個人擁有一樣的靈啟經
驗,但對於生命目的來說,這樣的事件是為了讓我們具備更充分的條件,來重
新看待現有的生命故事,從而能夠允許自己的靈魂發展出更加自我實現的道途。
以下便介紹何謂靈性學習的定義與內涵。 
二、靈性學習的定義與內涵 
16 
靈性學習包含生命教育和靈性修養(王嘉陵,2022),其中生命教育的主題
包含五個面向:哲學思考、人學探索、終極關懷、價值思辨,與靈性修養,而
內涵則為「人生三問」:人為何而活?人應如何生活?如何活出應活的生命?除
了有認知上的理解,行動上的作為也很重要,知行合一才是生命教育與靈性學
習的目標,否則只能淪為空談(孫效智,2009)。靈性學習有時涉及宗教、哲學、
心理、輔導、衛生等跨領域的學科,可以採用跨界合作方式在社會上進行科際
整合的實踐,小至家庭場域即可以推動生命教育的學習,倡導心靈無障礙文化,
解除人生無意義感、漫無目標的失根感,解決對愛與同理能力的缺乏(孫效智,
2009)。 
靈性學習和知識型學習不同,知識學習強調資訊的累加,然而靈性學習卻
講究「減法」,要去除頭腦中的迷障,諸如對「我」的執著、比較與分別心,解
除慣性的心理制約,以獲得真正的心靈解放,同時學習「無為」的智慧,放下
控制與佔有的慾望,才能體驗到「無所不為」的境界,因此學習方式會有所不
同(王嘉陵,2022)。靈性學習同時也是全人 (the whole person) 教育,目標不只
是知識和技能的追求,更是包含著力與美的精神、對道德情操和生命智慧的領
悟,因此靈性學習的方法與傳統不同,它更適合服務學習、靜坐冥想,甚至運
動、舞蹈、音樂、寫作等創造性活動,幫助學習者在放鬆的心情下打開直覺與
靈感,接通自身內在源源不絕的心靈能量(王嘉陵,2022)。 
不論有無宗教信仰,藉由靈性學習來探究「靈性」,對於自我成長皆有所助
益,亦能促進靈性健康成長,能將自我覺察逐步轉化為對生命覺醒,進而了悟
自性,若是進一步與華人的儒釋道文化結合,將有助於靈性概念的本土化,鼓
勵更多人探究生涯與靈性之間的對話與結合(夏允中等,2018;陳秉華等,
2018)。關於華人的靈性學習與追求,有袁廣義(2021)討論當代中國人的靈性
17 
需求的動機及其表達方式,在一個由無神論的政府所統治的國家,其人民對於
靈性匱乏感到有迫切的需要,不論是藉由合法正統的宗教途徑,抑或是非宗教
的形而上之哲學探索,皆可見到在面對人生的終極問題時,會藉由靈性學習和
宗教依附來尋找解答與真理。在不公義與虛假的社會氛圍裡,投入信仰活動可
以喚起內在善良的循環,有助於個人達成自我實現與穩定社會秩序,而為了世
俗的利益與內在安定的追求,人們也會藉由求神拜佛來尋求消災解厄,於是心
理學與宗教和靈性有了連結,普羅大眾因對人性安穩的需求而重視靈性學習的
追求。 
綜上所述,靈性學習可定義為:為了解決人類對自身生命的終極關懷,而
透過宗教信仰、形而上的哲學、心理學的方法,來解除個體對事物的我執,學
習無為的智慧來達到心靈的解放,接通學習者的直覺與靈感,對生命覺醒的可
能性敞開,進而了悟自性而打通內在心靈能量的泉源,並能喚醒個人內在的良
知與善的啟動,除了可讓人們內在安定,更有助於個體的自我實現以及具備穩
定社會秩序的功能。 
三、靈性學習的方法與挑戰 
靈性學習為一個擁有主觀經驗的個體藉由學習靈性知識來追求生命意義與
內外平衡的過程,通常以整體的系統觀來感知自己與生命的存在,經由覺察自
我的方式來了悟自身的生命學習與意義(鍾宜諠,2008)。靈性學習的機會潛藏
在不同的生命階段,唯有深刻的體認才能喚醒學習者的潛意識與心靈,坊間的
靈性學習課程多在於追求靈性的成長,以「活在當下」為訴求與課程目標之內
涵,學習運用心靈的力量或強化心念來肯定與接納自我,並藉由不同的技巧與
方法來探索生命的各個面向,找到內在的支持力量而能擁有一個靈性富裕的喜
18 
悅人生(黃秀娟,2017)。李亮儒(2022)參考蔣佩珍、甘桂安(2016)的研究,
將靈性學習的方向分類為三種: 
(一)傳統宗教的靈修 
藉由傳統的宗教方法進行靈修的活動,例如佛教的打坐、內觀、靜心冥想
等禪修與向內自我覺察,了悟智慧識與破幻象使人安住於當下,又或者基督教
與回教等,在奉獻、禱告與臣服的虔誠信仰當中發現自身內在與神靈或上帝之
間的關係。 
(二)新興的靈修 
靈性在後現代的社會當中,由於去中心化、多元化的趨勢,折衷主義的社
會變遷使得其追求與成長不再侷限於傳統宗教的靈修方式,而以「自我轉化」
(self-transformation) 或「自我靈性」(self-spirituality) 為核心訴求,例如新時代運
動於1960年後的全球性發展,其另類的社會與文化運動引導了許多新興的靈性
探索方法(陳家倫,2005),以開悟的靈性大師與通靈著作所傳遞的教導與靈修
方法,結合東西方古老的傳統靈修與各種療癒方法等為主要潮流(Crotright, 
2005;陳家倫,2005)。 
(三)心理諮商與治療 
有些學派的心理治療師會加入靈性學習的成分,強調覺察的重要性,例如
Jung 的分析心理學、完形治療、意義治療、森田療法等,超個人學派的心理治
療師也會將不同宗教傳統的靈修方式,將靈性學習加入個體的自我成長加以擴
19 
充,來開發個人的潛在能力和靈性意識 (Crotright, 2005)。 
超個人心理學家 Ken Wilber 提出靈性層次的個體意識可分為角色意識、自
我意識、有機體生命整體意識及一體意識共四個層次 (Lefebvre, 1998),進化的
順序從物質、身體、心智、靈魂,再到靈性,覺察自我的內在從潛意識、自我
意識,到超意識,每一個不同的階段都朝向更大更深遠的潛能,覺醒與進化的
意識更高也涵納更廣泛。當個體到達最後的階段,其靈性意識將觸及無限的光
明存在,與萬有合一進入法界意識 (kosmic consciousness) (Wilber, 2005),這種
沒有時空界限的一體意識為深度層次最豐富的意識狀態。成長為個體意識的擴
展,人的四個意識層次呼應其主要的核心議題,而不同的心理治療方法及宗教
靈修方式為回應不同意識層次而發展出來,個體須能識別自己的問題源於意識
層次所在的層次,始能找到合宜恰當的自我轉化途徑,使其發展意識能夠向下
一個階段邁進 (Lefebvre, 1998)。在靈性學習的發展與成長道路上遇到的疑惑與
衝突,有可能發生靈性掙扎 (struggle),例如超自然的掙扎、人際關係的掙扎、
個人的掙扎等(李亮儒,2022),關於超自然元素的信念可能和原本對宗教信仰
的認知有所衝突,或因為對靈性知識的掌握不夠成熟,懷疑自己遭受背棄而憤
怒,靈性學習過程中的主觀體驗也可能在同儕親友間遭受誤解或被拒絕,因為
與傳統宗教不相符合而產生衝突,或由於個體在探索的過程中發現自己違背教
義或靈性原則,而對自我感到愧疚或產生信仰上的疑惑和懷疑。 
由以上可知,透過不同的靈性學習活動與學習目標,可以讓個體在親身體
驗當中學習到關於靈性的相關知識與技巧,能夠對於自身的內在產生覺察與反
思,重新審視自己對於靈性或信仰的誤解與懷疑之處,在分享自身經驗的同時,
也能夠以與過往不同的觀點來重新理解與面對自身的生命議題,並能夠產生勇
氣去嘗試與他人溝通與人際連結,對於自身存在於世界的方式有不同的認知,
20 
對宇宙與超自然力量的元素有不同於以往的體驗,能夠拓寬自我對生命體驗的
深度,以及對於命與靈性有不同的領悟。 
參、靈性的自我轉化 
靈性學習有可能伴隨著自我轉化的發生,就是學習者擁有一段過去轉變前
的歷史,這樣的認知形成了我們的經驗與框架,是未來學習的重要組成部分 
(Kegan, 2018)。從成人教育的角度來看,觀點的轉換對自我轉化的作用,在於
改變個體原有的態度與理解的視角,從舊有的桎梏當中轉變為新的認知架構,
對自我認識這個世界的方式有著質性的改變。藉由自我轉化的發生,個人也得
以從新的洞見中發現自我、更新自我,以及自我進化,這樣的質變可以讓人感
受到人與自己、與他人、與心靈意識發生新的重組與融合,在人我之間的存在
當中發現新的行動意義(Boyd & Myers, 1988; Daloz, 1986;陳相如,2020)。從
心理學的角度來看,療癒發生在創傷之後,用以調適的敘事背後往往關連著一
個困難的「生命故事」,個體為了發展出對抗逆境與克服困難的力量,而進行一
場內在的轉化學習,自我轉化的過程毋寧是一場自我療癒與重整的奮戰歷程,
改變個體原本對自己所持有的看法與自我概念,發展出與以往不同的信念與價
值觀,自我轉化學習是一段內在的自己和環境不斷折衝的過程(陳相如,2020)。 
自我轉化學習常以「創傷」作為生命故事的焦點,探索創傷事件如何形塑
個體的人格轉化之力量,「自我轉化」之意即為藉由敘說生命故事來擴張並建構
新的自我認同,並因此擁有了不同的信念、價值、態度,以及自我定義 
(Lilgendahl & McAdams, 2011; McLean et al., 2007; Pals & McAdams, 2004)。由此
可知,自我轉化將學習的焦點往內聚焦於自我,自我導向的學習性質與目標導
向的學習不同,自我轉化學習有著心靈或靈性成長的意義,自我轉化也能夠促
21 
進靈性的轉化學習,帶領我們了解更多不同的靈性自我,在這個架構下,靈性
學習的本身也可說就是一場轉化學習。 
由以上可知,自我轉化的力量是藉由敘說生命故事而產生重新解構與轉變,
藉由敘事腳本的陳述來讓自己重新建構將來面對人生的不同價值與生命態度,
甚至更新了自我認識與自我定義。本篇論文研究為靈性層面的轉化學習歷程之
自我敘說探究,與上開自我轉化的內涵彼此吻合貼近,藉由敘說生命故事的張
力,為研究者建構出一個敘事自我的閱讀文本,進而能夠與其生命經驗之迂迴
與曲折,進行後設層次的理解與再次看見自我轉化的力量,發掘更新一層的生
命意義、生命態度和擴張的自我概念與自我認同,尋找出故事當中所隱藏的個
人在心靈成長的過程中所包含的動能,如何在不斷的突破與自我轉化當中走出
自己的道路。 
肆、小結 
藉由靈性學習的經驗累績,甚至是遭遇到靈啟的經驗,我們都有可能面臨
到自我轉化的歷程,這樣的過程也是一種轉化學習的經驗。因此接下來我們將
認識何謂轉化學習,以及深究靈性轉化學習的可能,以及其精神與內涵。 
第二節 靈性轉化學習 
本節將介紹靈性轉化學習理論。從轉化學習的定義、內涵、種類,和意義
與四大理論派別與批判分別闡述,最後並在靈性轉化學習和 Mezirow 的認知轉
化學習之間作出深度的探討,並輔以國內外文獻分析之。 
22 
壹、轉化學習的內涵與意義 
轉化學習 (transformative learning) 是由個體發出根本性的反省而產生批判意
識,對原先生命態度的預設重新檢視,與當前處境的重新詮釋,產生了新的理
解而採取了不同的行動,來改變所面臨到的生命困境 (Merriam & Caffarella, 
2004; Mezirow, 2018)。轉化學習有幾項影響因素:意義結構的改變、批判性反
思,以及學習的歷程。先探討轉化學習的定義與轉化學習的歷程,從心理、社
會、文化的不同構面來重新理解轉化學習理論的內涵,最後瞭解何謂自我轉化
學習。 
一、轉化學習的定義 
Mezirow (1978) 最早提出轉化學習理論時,認為它是個體藉由改變意義基
模與對自我意識的覺醒,是成人學習與教育當中最重要的一環。轉化學習是對
現有情境採取與以往不同的策略,來改善自己在社會上相對弱勢的地位 (Freire, 
2020; Mezirow, 1995)。轉化學習的發生是因為學習者在面對所處的問題困境時,
發現過去習以為常的方式不再管用,透過反省過往生命經驗與對事件當中的期
待與預設,領悟到這些期待和預設可能遭受到扭曲,或是個人未經發展完全的
生命道路,從而修正自己原先對生命態度的假設,願意採取不同的作為方式以
回應生命所面臨的困境 (Mezirow, 2018)。 
Mezirow (1991) 認為,對原有觀點抱持懷疑態度是轉化學習開始發生改變
的契機,批判性反思則是促進轉化學習的背後重要推動力量。人們的既有觀點
是受到年幼時期的教育與社會形塑加諸於人們自身的信念,而這些信念在當時
23 
卻是並未經過深思熟慮就被加以接受,但它們有可能是會讓自身受到困境的扭
曲觀點,對於我們做出更適切的價值判斷有著深遠的影響力。轉化學習的目標,
即是要人們破除這些未經反思而帶來的限制,也就是觀點的解放學習 (Mezirow, 
1990a) ,Mezirow (2018) 提及轉化的過程和 Freire 所談之「意識覺醒」(Freire, 
2020) 是一樣的,只是後者更強調社會與文化脈絡的壓迫下所產生的社會行動 
(Merriam & Caffarella, 2004)。由於個體生命意義的觀念或態度有可能受到壓迫
或扭曲,因此人們需要擁有不斷反思的能力,質疑自己向來對生命認為理所當
然的態度,以期產生戲劇性的成長與突破,是成人教育學當中的重要精神。雖
然經過轉化學習的結果也有可能僅只是確認原有的信念與價值觀,但它仍是經
過反思的轉化學習 (Fleming, 2018; Taylor, 2017)。我們可知,轉化學習的基本假
設為 (Chang, 2017;黃富順,2004): 
1. 人是理性而自主的,能夠賦予責任。 
2. 知識是經由主觀的建構過程而習得的。 
3. 個人身為社會的核心且對集體的未來負有責任。 
4. 個體遭遇到必須進行批判性反思的情境。 
由轉化學習的六個要素:個人經驗、批判性反思、對話、將之視為一種整
體的 (holistic) 體驗、社會文化和個人因素 (Baumgartner, 2019),可將轉化學習
理論分為不同討論的面向。其中 Dirkx (1997, 1998) 提出的四個透視面向 (lens) 
最廣為流傳,其中第一個面向為 Mezirow 理性認知取向的轉化學習,第二個面
向為 Freire 解放教育的意識覺醒之意識化取向的轉化學習,第三個面向 Daloz
著重師生之間的轉化學習,第四個面向為 Dirkx 靈性層面的轉化學習,將在本
節的第二部分「轉化學習的相關理論」當中詳細敘明。 
24 
二、轉化學習的歷程與影響因素 
轉化學習的影響因素有幾種可能,包含意義結構的改變、批判性反思的能
力,以及在轉化學習的歷程當中,不同階段與層次面的轉化學習,皆會影響個
體在面臨轉化學習時可能探及的深度。 
(一)意義結構的改變 
Mezirow (1996) 認為意義結構是一個人在面對現實世界時所採取的一套賴
以為生的信念模式,它有兩大組成要素,包括意義觀點 (meaning perspective) 和
意義基模 (meaning scheme) 。意義觀點是指,個人依據在其內在之心理預設或
環境之社會文化的前提下,所建立的參照架構或角色期待,以此作為對生活事
件或日常反應中的一套符號運作模式,來理解或詮釋其經驗的意義信念。意義
基模則為個人內在賴以為生的態度、情感、價值、信念,並以此作為實際生活
當中意義詮釋的引導與依據,反映個體在生活當中的感受、思想與行動方針的
指導性原則或角色期待 (Mezirow, 2018)。意義基模是意義觀點的骨架,意義觀
點由不同的意義基模所形塑。意義觀點由個體所處的文化背景下形成的社會期
待而建構出個人的特定信念與態度,可以藉由新的經驗同化或轉化,而意義基
模則包含了個體如何理解自己與如何理解他人的意義,以及如何採取行動的面
向 (Mezirow, 1991)。 
意義觀點的類型分為知識的意義觀點、社會語言學的意義觀點,以及心理
學的意義觀點等三種類型 (Mezirow, 1995)。知識的意義觀點 (epistemic meaning 
perspectives) 是以個體所接受的教育知識為基礎,指個體用來認知與運作知識方
法的意義觀點,例如教師以自身的學習態度作為其教學方法的主要參照,或者
25 
學習者以自身的學習偏好作為其學習活動的主要參照等,皆可稱之為知識的意
義觀點(楊深耕,2003)。社會語言學的意義觀點 (sociolinguistic meaning 
perspectives) 指個人將社會文化的規範或期望,以社會化的語言作為意義觀點的
基礎,包括文化的脈絡、慣用語言、宗教信仰價值、家庭成長環境,乃至於符
號互動過程都是影響社會語言學意義觀點的重要因素,例如傳統的性別二元分
化的對立態度,或是不同信仰之間、不同階級之間的權力爭奪 (Freire, 1970)。
心理學的意義觀點 (psychological meaning perspectives) 著重在個人內化的自我,
舉凡如何看待自己的自我概念、自我認知、自我認同等,受其兒童時期的成長
經驗所影響,例如人們的自我認識、自我價值、成就動機、人生態度、心理防
衛機轉或者精神官能症的需求等,甚至是一個人的趨避衝突或內外控等性格類
型都是Mezirow所重視的意義觀點之範疇 (Mezirow, 2018)。 
個體為了要達到成人學習與教育的目標,取得和他人的有效溝通是重要的
步驟,因為理想的對話條件是立基於意義的有效性 (Mezirow, 1995)。個人的主
張為了獲得有效意義的支持,Mezirow 依據 Habermas 的觀點提出理想的言說 
(discourse) 條件,即是擁有完整的資訊、沒有自我欺騙、客觀的評估論證、具
有平等的機會以參與言說的各種角色 (Merriam & Caffarella, 2004)。理想的對話
條件是達到成人學習的目標,要求參與者必須對所討論的主題擁有正確而完整
的訊息,能夠從社會壓迫當中解脫並避免自我欺騙,進而從客觀理性的角度重
新審視評估,對另類觀點抱持開放心胸、關懷尊重他人感受,對文化背景和心
理預設做出批判性反思並探究因果,以理性客觀的態度進行對話或辯論、達成
共識,最後接受已經達成的新觀點 (Mezirow, 1995)。 
(二) 批判性反思 
26 
未經過深思熟慮的反思性對話是未發展完全的扭曲意義觀點,扭曲的意義
觀點將限制個體對現實的覺知,狹隘的視野讓人對於自身的故事缺乏宏觀而適
切的理解,對他人經驗缺乏通透性與接受度,會阻礙個體的經驗整合與個人發
展(Freire, 1970;阮氏碧玄,2019;林合鑫,2020;陳相如,2020)。扭曲的知
識意義觀點為所得知識的謬誤,可能源自於個體認知的發展階段所限制,而扭
曲的社會語言學的意義觀點為對社會功能理解的謬誤,可能於由於受限而不合
理的意識型態,而扭曲的心理學的意義觀點則為個人對於自身的真實動機理解
的謬誤,可能於源自於童年或創傷經驗、過去的生活或教育經驗形塑的自我概
念所導致的心理預設。這些扭曲的意義觀點對於限制我們人生的信念缺乏覺知,
使人們對於當前的生活困境缺乏改變動機,也由於缺乏根本上的理解,個體也
未必能夠發展出解決當前問題的能力 (Mezirow, 2018)。 
人們在反思的過程中會重新審視與思考,過去我們向來認為理所當然的事
物或問題的癥結,並進一步理解並看見個體在過往習以為常的思考模式、情感
經驗、價值理性的判斷、信仰行動的方針,對我們的心理預設或道德假設所造
成的影響。批判性的反思讓我們對於嘗試去詮釋與理解經驗意義的內容 (what)、
過程 (how)以及前提 (why),賦予批判性評估的過程,Mezirow (1991) 因此將之
分成為三部分:內容的反思、過程的反思,以及前提的反思。 
1. 內容的反思 
內容的反思 (content reflection) ,指的是個人對事情的內容或敘述內涵進行
檢驗,以釐清既有經驗或澄清自身的可能的意義觀點。例如:當我們看見性別
差異,我們可以依據某些外在條件或社會特徵來檢驗,或者被告知年齡時,我
們依據他的外觀或行為舉止來判斷,當我們被告知對方為老師或學生時,我們
27 
依據他的身分差異的特質來區別其內容的真實性與合理性。 
2. 過程的反思 
過程的反思 (process reflection),指一個人對自身的問題解決策略進行檢核,
了解既有經驗或觀點的形成原因,以及假如我們持續保有這樣的意義觀點會造
成什麼影響。例如:當我們判斷對方是高齡人口是以什麼樣的特徵,所據以判
斷的條件是否正當,我們欲確認對方是商務人士時,他所穿著的西裝外套、領
帶、發亮的黑皮鞋,就可以充分說明他是商務人士嗎?有沒有可能當他開口談
吐,才發現他只是一個正要去拍攝形象照的畢業大學生呢?當我們下了結論為
好人的標籤,是否有過度類化的嫌疑?當我們進行判斷的時候,所採取的理論
也是需要進行檢視的。 
3. 前提的反思 
前提的反思 (premise reflection),它是對問題本身的反省,也是對問題的發
生原因進行探討。例如:我們遇到了車禍事件,我們對事件的背後進行原因的
檢查,也可能是身體遭遇了疾病,我們去問「為什麼會有這個問題?是什麼造
成了這樣的問題?」進而去探索事件發生的前置因素和可能預設的條件,目的
是要能夠改變這些決定了結果的條件,從而改善問題。 
由此可知,一個人如果要根本上的解決當前困境,最重要的方法是要對於
自己所受到限制的信念價值觀有所覺察,並願意進一步採取反思性的行動,最
後再將自己從這些扭曲的意義觀點當中解放出來、移除自身上的限制,
Mezirow 將之稱為「觀點轉化」(perspective transformation),此即為觀點的轉化
28 
學習,是為成人教育的核心 (Hoggan et al., 2017; Mezirow, 2018)。在上述三種批
判性反思類型當中,都有可能發生在不同的學習領域當中,其中內容的反思和
過程的反思會讓我們的信念鬆動,使得我們原先的信念被強化、獲得證實、遭
到否定、加以確認,也就是意義基模的改變;但前提的反思可以導致個人的意
義觀點發生進一步的轉化,協助個體發展出更完整的意義觀點,使其產生更具
包容性、區辨性、滲透性、整合性的經驗,相較於個體原本的意義觀點是扭曲
的、受到限制與不合理的價值判斷,轉化學習所獲得的觀點轉化是更為有利的
「優質觀點」(Merriam & Caffarella, 2004;黃富順,2004)。前提的反思性檢驗
會讓個體產生意義觀點上的轉化學習,若沒有造成轉化,則會再次確認原先所
持有的意義觀點,由於知識不是「給定」的,而是藉由人們主動去建構的,並
且依據知識背後所產生的脈絡而決定與理解,因此如何產生知識的歷程可以從
人們如何學習的三種類型當中探討 (Mezirow, 2018)。 
(三) 轉化學習的歷程 
為了從事科學活動,Habermas (2015) 將人類的認識旨趣區分為三類:技術
的旨趣、實踐的旨趣、解放的旨趣,每一種旨趣都指向人類為了瞭解自然現象、
解決社會問題、維持生活條件而發展出來的不同認識面向,Mezirow (2018) 則
將三種不同的認識旨趣所對應到的成人學習方式,分為工具的學習、溝通的學
習、解放的學習。 
1. 技術的旨趣帶來工具的學習 
人類對技術的旨趣,為的是將自身從自然科學的力量中解脫出來,透過對
自然世界的認識與理解,掌握了大自然的宇宙法則,依據科學的原理,從受制
29 
於自然界的力量擺脫,進而能夠成為生命自我的主宰,目標是為了攻克自然界
的限制並免於受害與擺佈 (Habermas, 2015)。因此工具的學習,重視的是對因果
關係的掌握,透過預測和控制來達到任務導向的完成問題,具備可驗證性與可
評量性,其學習方式藉由假設、觀察、實驗、修正來找到因果關係,主要的研
究方法為實證分析 (Mezirow, 2018)。 
2. 實踐的旨趣帶來溝通的學習 
人類對彼此之間的認識興趣為期待瞭解他人與自己的關係,希望從中獲得
人性的普同感為實踐的旨趣。實踐的旨趣指導著精神科學的發展,對於科學在
歷史背景下的解釋力有著主要影響的力量,它關乎人類自身在生命當中的故事,
能夠將自身從僵固的社會教條與意識形態中解放出來 (Habermas, 2015)。因此,
實踐的旨趣帶來溝通的學習,人們藉由文字或藝術創作,運用不同的語言能力
來傳遞彼此的理念,目的在於瞭解人我之間的相同與差異。人們的溝通是為了
達到共識,其有效性需經由理性的對話達成,對傳統的權威勢力進行檢視與批
判性反思,達到意義觀點的改變與更新,為轉化學習的第一個歷程 (Mezirow, 
2018)。 
3. 解放的旨趣帶來解放的學習 
從前面兩個對事物與對人的旨趣可以發現,人們有一種對自由與獨立的渴
望,試圖將主體從其依附的對象當中解放出來,這種解放的旨趣透過批判性的
科學方法,希望世界上沒有所謂的壓制與統治而人生為平等 (Habermas, 2015)。
解放的學習因此即為批判的學習,它要求我們對文化語言的、社會心理的、知
識權力的結構進行深度的反省,也就是對前提的反思進行意義結構的重組,不
30 
只是意義觀點的更新,更是意義基模的改變,讓溝通的學習更往上一階,為轉
化學習的第二個歷程 (Mezirow, 2018)。 
在轉化學習的十個步驟當中,包含:經驗到失去方向的兩難困境,罪惡感
或羞恥感的自我檢驗,對認識的、社會文化的、心理的進行評估,認知到個人
的不滿和轉化過程是可與人共享的,每個人都可能經歷這種感受,開始探索新
的角色、關係與行動的選擇,制定新的行動規劃,並學習執行計畫所需要的知
識和技能,甚至盡力嘗試新的角色,以在新的角色和關係中建立自信與能力,
最後以全新觀點作為重建個人生活方式的基礎 (Mezirow, 2000)。以上的每一個
步驟都要求個體在面臨無法應付的兩難情境當中,對既有的假設內容進行反思、
對既有問題的過程進行反思,到對當前問題的前提進行反思,轉化學習因而經
由意義基模的學習,到學習新的意義基模、對意義基模的反思學習,到對意義
觀點的轉化過程,擁有對既有架構提出深度反省的能力,才能夠算是真正的轉
化學習 (Mezirow, 1991)。 
貳、轉化學習的相關理論 
轉化學習的理論除了理性認知的Mezirow,還有重視意識解放的Freire、重
視師生特殊關係的 Daloz,以及將轉化學習提升到靈性層次的 Dirkx,以下茲將
分別介紹於此。 
一、Freire的轉化學習理論 
相較於 Mezirow 將轉化學習的主體放在個人的改變,Freire 則將轉化學習
的焦點放在社會變遷與意識的覺醒,兩者在不同的層面上運作,但未必僅僅只
31 
是彼此視為互補的理論 (Finnegan, 2019)。教育若要獲得解放,意識覺醒 
(conscientization) 的過程讓人們認知到師生是互為主體的,也才是有能力改變社
會現實的力量。對 Freire 來說,個人透過學習而轉化的增能賦權 (personal 
empowerment) 和社會轉化 (social transformation) 是彼此支持的,而要達到這樣
的目標則要透過「提問式」教育 (Freire, 1970)。 
意識的覺醒階段從神奇意識、素樸意識,到批判意識彼此關聯與進化,以
回應不同社會結構的人類意識。人們在不同的社會型態當中擁有不同的存在狀
態,因而人類在早先階段,處於未開化的「神奇意識」 (magic consciousness) 階
段,對萬事萬物被動的接收,逆來順受的宿命性格不會對存有發出提問,相信
外來的力量所主導的命運並任其擺佈。人類在這樣的存在意識當中與自然的關
係是從屬於之,而缺乏尋找因果關係的能力,並將對權力壓迫者的恐懼視為對
自我控制的力量 (Freire, 1970)。「素樸意識」 (naive consciousness) 則是人們開始
意識到自己受壓迫的事實,了解到自身可以抗拒不合理的社會機制,用不同的
姿態來回應統治者的支配,但對於自身的存在狀態仍有過度簡化問題的傾向,
因其仍處於情緒階段而非理性的狀態,尚未充分察覺的自身在這個社會上的反
抗責任,有時甚至仍會認同壓迫者的生活型態,這些人尚未擁有將自身解放的
渴望 (Freire, 1970)。當人們開始具備「批判意識」 (critical consciousness) 則對
於改變擁有自覺的責任,他們充分了解自身在這個世界上的位置,也瞭解透過
理性的批判科學方法來和世界建立關係,藉由反省看見自己生在時代的使命,
願意用創造性的生產活動來完成自我實現的生命目標,甚至帶動社會結構的革
命,為成人教育與學習機會帶來新的希望 (Freire, 2021)。 
統治者為了維護自身的階級利益,希望人民接受灌輸的教育,然而這樣的
學習卻不是以學生為主體,當人們欲擁有學習主體的自決,這樣的意識覺醒需
32 
要透過一連串提問式的教學方法 (Freire, 2021)。灌輸教育 (backing concept of 
education) 讓人們學習到的是人類與世界是分離的,人類只是自身存在世界的旁
觀者,人類不具備意識的識別能力,不是與這個社會一起進行創造性工作的人。
提問式的教育則挑戰這樣的觀點,將人類和世界重新關聯起來,人類是知識的
一份子,也是知識的使用者,透過師生之間在愛與信任、尊重與包容下的提問
對話 (dialogue)、聯合反思,將社會問題變成可以被討論的選項,發現所處的環
境是可以改變與創造的,與其將之與個人主義共同束之高閣,不如讓彼此進入
愛的循環當中 (Darder, 2017) 。在提問式對話的脈絡之下,人們可以從未開化的
神奇意識、初步覺察的素樸意識,進化為覺醒的批判意識,以對於民主素養擁
有真正的自決與行動,打造一個人性化的社會 (Freire, 2021)。 
二、Daloz的轉化學習理論 
相較於 Freire 重視在愛與信任中建立提問與反思的對話, Daloz (2012) 則
重視師生之間的支持與引導,強調教師與學生之間的連結可以鼓勵學習的嘗試
與進步。當老師能夠更常的思考自我成長與教師的增能,在學生的學習旅途上
便能扮演一個具備關懷教育的指導者,而非單純的只是要求學生學習 (Nyanjom, 
2020) 。轉化學習允許關懷教育的發生,它提供支持、挑戰、新視野 (vision) 來
樹立教學典範,同時關心成人學習和發展之間的互動,引導學生學習如何看待
自己的思考而非如何思考,將轉化學習深層的帶入充滿教練關懷的學習旅程,
不再只是重視他人如何看待自己。 
支持的方式是藉由傾聽、建立一個學習的架構.願意與學生分享自己、和
學生建立特別的關係,師生在這樣的支持關係中彼此信賴且誠實;教師為學生
設定的挑戰除了指定作業、建立討論的習慣,更需要針對二元化的對立思考進
33 
行批判,讓學生了解每一種立場都有其合理性,並且能夠協助學生建立內在的
導師,逐漸對自己擁有高標準的要求。最重要的是協助學生擁抱新的視野,透
過教師的自我示範協助學生發展自己的願景,讓學生對發問態度感到好奇,了
解到自己的學習地圖有發展的方向,能運用新的語言和參考架構來重新思考,
並適時的對學生回饋,成為他們的一面鏡子。成長是一件未知的事情,有時會
帶來懼怕,但 Daloz 的關懷教育讓學生的擔憂被充分接納、擁抱學生的主體性,
重視教師的身教 (Nyanjom, 2020),藉由師生特別關係的建立與正向的期待,每
個人都可以在深度的轉化學習當中擁有獨特的願景 (Daloz, 2012)。 
三、Dirkx的靈性轉化學習 
自古希臘三聖奠定西洋哲學的基礎以降,情緒被邊緣化為非理性思考,認
為情緒的存在有礙於真理的追尋後,許多哲學與教育理論的實踐也跟著不重視
情緒的探討 (Dirkx, 1997, 2001)。成人教育也一般認為教學是認知與理性的,情
緒是會阻礙學習動機的,許多學習者遇到焦慮的情緒,也多半是鼓勵自己轉移
情感、放下它而非正面處理;然而從 Freire 和 Daloz 的經驗我們可以發現,愛
與關懷的師生關係更有助於轉化學習的發生,情緒其實能滲透學習的情境並有
催化的能力。成人學習的互動當中也不能自外於情感和情緒的連結與交流,轉
化學習的層面不能只停留在理性的層面,而是更需要向上探索情緒、情感、心
靈、靈性等,甚至是直覺也能夠深入轉化學習的過程。除了透過想像的方式來
接觸情意面向,個人往內探索自己的情感連結也能讓自己的情緒和世界互動,
深層的心靈感受藉由具象化的意義脈絡,可以轉化人們的生活經驗而獲得自我
的抒發 (Dirkx, 1998, 2001; Taylor, 2000)。 
由此可知,教育的情意面向與認知和技能的面向一樣不可偏廢,彼此互依
34 
互存且可以互相補充不足之處,重視情緒的學習不只增加靈性的智慧,更可以
促進人性面的連結,情緒不是自外於心靈的,我們必須擁抱自己的情感和靈性,
才能不愧於身為萬物之靈的人類,而靈性的轉化學習,更可以深化一個人生存
在現實社會上的軟實力,讓我們能夠強化與自己內在的連結,發掘自我對抗困
境的潛力。 
參、靈性轉化學習的相關研究 
轉化學習有不同學者所擁護的不同派別與重要內涵,接下來將進一步延伸
有關於 Dirkx (1997, 1998, 2001) 之靈性轉化學習,以及與其相關之學者所提出
有關靈性轉化學習的理論內涵。有別於源自 Freire (1970) 的意識覺醒、
Habermas (1972) 基於人類的三大學習旨趣而延伸與發展的Mezirow (1978) 轉化
學習理論,其建構主義取向的自我導向學習所描述的是學習的認知與理性的過
程,轉化學習的目標是改變過去習以為常的參考架構以及思維模式 (Mezirow, 
2018),另一派研究轉化學習理論的學者 Boyd (2003)、Cranton (2000, 2006) 與
Dirkx (2000) 則將焦點放在關注於自我 (self) 的認同 (identity) 與主體性上(林曉
君,2017),將轉化學習的內容當中之精神層次或內在的心靈工作視為更重要的
核心,其依從分析心理學家Jung的個體化 (individuation) 理論 (Jung, 1939, 2017) 
將之延伸到成人學習的內涵,重視學習者的心理、情緒與精神的想像,以及其
靈性的發展等不同的主題 (Dirkx et al., 2006) 。這兩派學者的對話在 2005年的
國際學術研討會上引起激烈的辯論,使得轉化學習的理論與不同面貌進入緊張
狀態(Taylor & Cranton, 2012; Taylor, 2017;林曉君,2017),反觀國內對於轉
化學習的研究多半以 Mezirow 的理論為的基礎,其內容也以實證研究為主,較
少見到以 Dirkx 為架構的轉化學習理論所著墨之靈性學習的研究,也較缺乏相
關的實務應用在後續研究的追蹤(林曉君,2017)。 
35 
Mezirow (1978, 1990a, 1997) 的轉化學習理論在於認知框架或是意義觀點的
轉換,它是關於心智的習慣,與人們的日常信念、態度與價值判斷有關 
(Mezirow & Cranton, 2012)。轉化學習理論的心理、情緒以及其精神面向,在
Boyd 的小團體研究當中開始發現,個人的轉化學習歷程在於信念與假設的潛意
識推論,此方面的研究可以彌補 Mezirow 的研究往往只強調意識的理性層面而
有的不足 (Dirkx, 2000)。成人教育者所追求的目標,獨立思考的能力是轉化學
習的成果 (Mezirow & Cranton, 2012),而轉化學習理論包括的三個面向:心理對
自我瞭解的改變、信念系統的調整、生活形態上的行為改變 (Taylor, 2017),可
知 Mezirow 的轉化學習理論著重在對信念的認識論層次之社會學理解(林曉君,
2017),而 Boyd 與 Dirkx 等強調靈性學習派別的學者尋求的是心靈與個體自我
的精神上之整合,尤其是從分析心理學者 Jung的個體化的角度出發,其質性研
究的方法更適合以情緒為主要內涵的靈性學習研究,他們所注重的是轉化學習
理論當中討論情緒精神與心靈為內涵的層面,與 Mezirow 的理論所共同的研究
方向一樣是個體存在於世界上的自我認識,以及關於成人經驗的意識轉換(林
曉君,2017)。 
以 Jung 的潛意識與原型心理學為基礎的研究指出,此觀點能夠成為解釋轉
化學習理論的重要依據,其中Boyd (1991, 2003) 的研究以小團體當中的個人轉
化為主要研究對象,顯示出學習者的特質存在於心理、情緒與精神內涵的基本
假設或信念,其動態的潛意識 (dynamic unconscious) 對於個人生活風格與思考
與行為習慣的形塑,扮演著關鍵的作用。從分析心理學的角度來研究轉化學習
的理論,使用Jung的原型、潛意識,甚至個體化概念的可能,不同於 Freud在
精神分析當中對人格發展的假設,並非過往經驗才是唯一能夠修正人格的力量,
可從成人發展的觀點來看待人格與心靈的成長,分析心理學將人生階段區分為
36 
童年期、青春期、中年期和老年期,並預測中年期將會是人生當中重要的轉變
時期 (Schultz & Schultz, 2009)。不同於早期 Mezirow 對轉化學習理論所提出的
線性模型,後來的學者根據45篇博士論文進行後設分析,認為轉化學習是一個
非線性的概念,其中觀點轉化是一個循環與遞迴的 (recursive) 過程,不會總是
循著固定的步驟完成,轉化學習有著內在的複雜性 (Taylor, 2000)。雖然轉化學
習的過程仍具備可被識別的變項 (Mezirow, 1995),但Taylor (2000, 2009) 認為轉
化學習之理論建構需要具備更多不同的研究方法,才能逐一瞭解並探討其組成
元素與適合的成人教學模式。在 Mezirow 的認知理性的轉化學習重視批判性反
思之餘,轉化學習的情緒和情感面向可藉由情意學習 (affective learning) 而獲得,
然而此二者間並非完全獨立而互斥的存在,認知理性所重視的批判性反思並非
完全對立於靈性學習的情意面向,而是彼此重要性的補充說明,兩者彼此能夠
互相影響而存在,甚至情意的面向是批判性反思的基礎(Taylor, 2000, 2009, 
2017;林曉君,2017)。  
繼Dirkx 等人 (2006) 將成人學習的情緒面向和Jung的分析心理學的理論關
連到轉化學習的內涵之後, Kucukaydin和 Cranton (2013) 採用超理性的方法關
注情感、想像、直覺、夢境等不同學習面向,不再將之視為與身體心智分離,
認為成人教育者應更重視全人的身心靈結合。深受 Jung 所啟發的 Boyd (2003) 
認為轉化學習的目標應為「從個人的潛意識和社會文化中解放」 (Boyd & Myers, 
1988),以面對自身內在的陰影 (shadow)、人格面具 (persona)、男性內在的女性
原型阿尼瑪 (anima) 以及女性內在的男性原型阿尼姆斯 (animus),以將不同的自
我面向進行整合。Kovan 和 Dirkx (2003) 認為情緒、情感、想像的面向在轉化
學習的過程當中常被低估其重要性,而轉化學習的發生不一定非得發生重大生
命事件,也可能發生在平凡的日常生活當中。由於轉化學習包含了情緒與想像
的過程,例如 Jung 分析心理學的個體化歷程 (individuation process),可說即是
37 
一種轉化學習的類型。但本篇研究的重點放在自我與潛意識的對話過程,面對
符號與想像力量的交織,辨認出自我的內在陰影,能夠將之區別出來並且重新
整合進一個更大的內在世界裡,讓自我定位更有意識地與集體連結。此種超理
性的方法不應該被教育的內涵捨棄 (Dirkx et al., 2006),畢竟人的內在世界反映
其神秘與不同意識層次的覺知,每個個體都會擁有其私密的想法、價值觀點與
信念。有關 Jung超個人心理學理論的轉化學習缺乏一些較為深入的研究報告,
Kucukaydin 和 Cranton (2013) 認為超理性方法應作為 Habermas三大知識旨趣當
中的實用性認知,以相互理解為主要目標。由於實用性是在社會認知上建構的,
也是有個體上的主觀性質,需要透過社群的相互溝通、對話論述而來,屬於
「認知層次」的訴求,因此透過不同的詮釋與理解自我與他者,乃至於面對社
會世界,目標並非為了追求絕對之真實,而有賴於相互理解以及溝通對話為核
心(林曉君,2017)。 
由以上可知,靈性轉化學習有其理論基礎,若要從 Jung學派的角度來分析
由潛意識或自性的力量所引導的轉化學習歷程,需要的是和社群之間的彼此討
論,雖然轉化學習和潛意識外化的歷程有其一定的步驟,但每個人會有其私密
而獨特的經驗,而使得相關研究更適合進行質性研究,較不利於量化研究的發
展。 
肆、小結 
本節內容則將靈性轉化學習的內涵與可能的潛在張力作另一較為詳述的敘
明,瞭解到靈性轉化學習和認知理性的轉化學習互為彼此的包涵與補充,而非
彼此互斥不相容的可能,尋找到最大公約的一個解答。瞭解到靈性轉化學習可
以更深入的探索個體的內在動能,認識了第一節當中所提及之靈啟經驗如何讓
38 
個體在實務上發生轉化學習的可能,由於靈啟經驗是靈性轉化學習的核心,它
擁有了情緒轉變的突破力量,而情意學習的內涵與相關研究可以為此建立一個
溝通的管道,目標是為了更加瞭解人我之間的個殊性與差異性,也就是在異中
求同、同中求異。藉由 Jung的分析心理學也可以探究轉化學習的歷程,其中著
名的潛意識歷程包含阿尼瑪與阿尼姆斯的的相遇,其內在精神世界的轉化屬於
個人獨特的心靈經驗,相當符合靈性轉化學習的旨趣。 
第三節 個體化的歷程 
個體化的原文 individuation 之字義可從其字根來理解:in 為「不可」之意;
dividu 為 divide 之變形,分割之意;ation 則是「使其成為」之意,整個詞的意
思為「使其成為不可分割之人的過程」(趙燕、黃宗堅,2018)。 Jung 以個體化
表達對個人心理發展的看法,將個體化的概念視為人們終身發展的地圖,也就
是人的一生當中重要的發展目標,是成為一個統合而獨特的個體,一個不可分
割和整合的個人 (Stein, 2021, 2022)。 
壹、個體化理論 
人們在不同的發展階段當中,為了適應新的社會角色,或為了獲得社會地
位身分,心靈內在的某些部分將會面臨必要的分離,也就是意識和潛意識的分
離,這就是所謂的人格面具 (persona),在我們適應了不同角色之後,還是要回
過頭來,尋找過往所遺漏的東西,或者未曾見聞的事物,引領自己的心靈回到
圓滿,完成一幅曼陀羅的圖畫 (Jones, 2020)。Jung 的分析心理學(analytical 
psychology,也是深度心理學)為了走向內在工程,發展出深度分析的四大基
39 
石,分別為:個體化、分析關係、夢和積極想像,其中個體化即為心理分析的
第一大基石,其重要性不言可喻。個體化的概念是個體在漫長的一生當中按圖
索驥的目標,每個人都有這種本能,雖然由於資源限制或創傷經驗的緣故,不
是每個人在一生當中都可以完成個體化的歷程,也不是每個個體都會走上個體
化的道路,然而個體化的概念是其他三大基石的基礎,夢是通往個體化圓滿的
途徑,而積極想像則是個體轉化的媒介 (Stein, 2022)。 
一、個體化的定義 
在 Jung 著名的集體潛意識概念當中,它代表著個體心理成長的能量泉源,
在個體的意識與潛意識之間建立合理而互助的關係,便是心靈健康的首要工作 
(Hopcke, 2002)。Jung 將意識和潛意識之間的關係類比為個人自我 (ego) 情結和
自性 (Self) 原型之間的關係,而自性是完善健全的原型,表現在夢或幻覺中重
複出現的象徵。當意識與潛意識、自我與自性之間存在著良性互動時,人們能
夠感受到自己的獨特性格與人類實存深處的經驗之海相連接,使自己的生活具
有真正的創造力、象徵性和獨特性 (Hopcke, 2002)。Jung 將達到這個心靈平衡
的過程稱之為「個體化」,這是人類的所有心理活動遵循的原則,認為心靈有自
我完善與平衡的傾向。亦即:在人生當中實存在於不同心靈層次的對立面,會
在人類的心靈深處獲得和諧的統一,健全而獨特的人格特質會在不同心理對立
面的協調中產生 (Stein, 2017)。此時,「個體化」即是實現真正自我的過程,是
與己合一的過程 (Hopcke, 2002)。 
二、個體化的內涵與歷程 
個體為了實現自我的存在與價值,將分裂於內在的不同自我逐步整合,不
40 
同自我的內在對立面之存在,是為了讓潛意識進入意識的層面。進入潛意識的
初始是藉由情緒與情感的連結,各種情結的表現以情感來干擾自我的方式來表
達自身,並且提供個體成長的潛能。這些情感的干擾終究會在個體的本能當中
追溯至其根源,同時可以導向個體的未來與預知的方向。個體化假定的是一個
終極的觀點,朝向目標的內在心靈運動。為了達成統合的整體,意識與潛意識
的系統需要帶入一個雙向的互動關係中,心靈包含著兩個不調和的部分,而兩
者合一便形成全體 (wholeness) (Stein, 2017, 2021)。 
(一)心理發展的三個個體化階段的歷程: 
如果將人的一生區分為兩個不同的時期分野,那麼生命的前半段分別由母
親和父親所主導,後半段由自性(或個體)所主導,在此三個階段中的母親、
父親、自性(或個體)的形象是由象徵意義所構成的,這些重要的人物形象代
表著不同階段的重要權威(楊惠,2021)。然而在進一步了解此三個階段之前,
我們有必要知道在 Jung的觀點裡面,並沒有把生命如切蛋糕般的區分成這幾塊,
而是視生命為一個又一個充滿階段性的過渡期,每一個過渡期都是個體化的轉
化歷程,以下便依序介紹這三個不同的心理發展階段,分別為童年的母親時代、
青壯年的父親時代、成熟的個體時代 (Stein, 2022): 
1. 童年:母親的時代 
從母親的子宮開始,會一直持續到12歲左右。由母親所主導的象徵適用於
童年生活的態度或獨特的成長氛圍,在這第一個階段,母親原則是主要的支配
力量,也是個體化歷程的重要參照點 (Stein, 2022)。最初,嬰兒從黑暗的子宮中
出生,進入了光明的世界,對自己一無所知,此時,照顧者的母嬰依附關係很
41 
重要,對嬰兒來說,最重要的不是去理解這個世界,而是藉由依戀獲得的溫暖
和接納,而想要留在這個世界上。在這個早期階段,最重要的心理發展任務就
是依附關係。有時候這個階段會被認為是個體化的天堂階段,因為嬰兒除了依
戀和回應照顧者的愛之外,什麼都不用做,然而這樣的天堂也只是個原型的意
象,真實的嬰兒生活其實是惱人而複雜的 (Fordham, 2018)。 
表 1  
人類一生的心理發展(個體化) 
生命的不同階段 時期 主導力量的象徵 
生命的前半段 
生命的後半段 
兒童 母親 
青少年 
青壯年 父親 
中年 
成熟成年 
個體 
退休 
老年 
資料來源:Stein (2022: 29)(研究者重繪) 
到了兩歲至三歲左右,嬰兒開始發展出自我的意識,開始想要擁有自己的
意願、擁有自己的物品、擁有自己的決定,即使那是違背母親的意願,在這個
階段當中,人格面具的陰影會開始發展:為了討好而扮演可愛順從的樣貌,還
有為了作自己而表現出任性不聽話的樣子,這種現象說明了個體化是內在驅力
的首要證據 (Stein, 2022)。在這個階段,父親的在場是很重要的。父親協助孩子
從母親身邊分離開來,以為將來要進入外部世界做好準備,一個有學校生活和
工作生活的社會和文化,因此母親能夠允許孩子的分離是重要的。不論是母親
的過度牽絆,或者父親的缺席,對這個階段的孩子來說,都會有礙於安全感的
發展 (Leahy, 2018)。 
2. 青壯年:父親的時代 
42 
由母親原則所支配的力量會將個體引導到一個重要的啟蒙事件,以準備好
進入成年生活。從 12歲左右開始到中年時期,通常是 40歲左右。由父親所主
導的象徵適用於青少年生活的態度或獨特的成長氛圍,在這第二個階段,父親
原則是主要的支配力量,也是個體化歷程的重要參照點 (Stein, 2022)。 
11 到 13 歲之間的青少年,是過渡進入成年期的開始,從對主要照顧者的
依戀轉移到外部世界,也就是對同儕群體和父母親以外的其他成人的依戀。孩
子的舊身分正在褪去、青年人的新身分正在形成,對親近家人之外的新認同也
正在開始發生。在這個階段最重要的心理目標,就是適應同齡社群和成年人的
社群,以及這個屬於「父親原則」的外部世界。對這個階段抗拒的年輕人,就
會像是「彼得潘」一樣,對於成人世界的一套規則無法接受,一想到學校或社
會對自己的要求就會受不了而不願意長大 (Bovensiepen, 2010)。青少年對於自己
的成長開始有所覺知,卻仍懷念著過去的美好,對於充滿獎懲與工作鬥爭的社
會有著抗拒,這是可以理解的,因為這個父親的世界也充滿權威和位階、表現
與批判;此時母親需要將孩子往外推,但最重要的任務還是來自父親對孩子的
引導,不論是男孩或女孩,讓年輕人能夠對這個成年人的世界有所依戀,可以
從中獲得滿足與歸屬感,就如同嬰兒剛進入這個世界需要的是母親的依附關係
一樣 (Stein, 2022)。 
父親的世界充滿選擇,每個人能否擁有一樣多的選擇自由端賴自身所處的
文化位置。當年輕人開始為自己做出選擇,他們便決定了自己的身分、角色與
地位,也就是人格面具的形成。不論出於有意識還是無意識,他們因為自己的
決定而擁有了不同的教育水準、工作能力、社經地位,更重要的是,為自己選
擇一個伴侶 (Cannon, 2012)。而在這個重要的階段,不要太快而衝動的作出決定,
43 
允許選擇自然的出現在生命當中,會是心理治療師給這些人的建議 (Stein, 2022)。 
3. 成熟的青壯年:個體的時代 
在完成母親原則所主導的滋養階段與父親原則所支配的成長階段後,個體
會邁入真正的成年生活,通常在此時個體已經邁入成熟的成年,「個體」就是他
內在於此一生命階段的重要控制中心,具備完全的獨立性。此時人生在個體化
階段的後半段,個人的主要目標是對於意義的追尋。不同於父親的時代教給我
們的權威模型與上層對人生觀的期許,此時個人的服侍對象會由對外界負責轉
向內在,由於經常不符合其他人對自己的期待,個人往往必須作出主動的犧牲
和改變。這種新的存在模式圍繞著內在中心的自性而繞行,依從心靈的組織原
則,不能只靠著反芻和自省的叩問而得到淺層的答案。這和由自我為中心的意
識不同,個體需要向內打開進入尚未敞開過的地下室或秘密衣櫃,通往夢和積
極想像的空間 (Stein, 2022)。 
這種「中心傾向」 (centroversion) (Mitchell, 2020) 並不圍繞著自我的諸多議
題,例如「我的感覺、我的喜好、我的期待是什麼?」而是將地球為中心的宇
宙觀轉換為太陽為中心的宇宙觀 (Stein, 2022),Jung學派將此稱之為自我的相對
化 (relativization of the ego)。讓自我處於心靈的次要地位,使自我不再是主要決
策的絕對位置,並將之讓位給「自性」(Self) 來主導。自性總是不會只給出簡單
的答案,潛意識在這個層次上擁有發言權,個體必須有耐心的等待自性的聲音,
不再效法父權結構的外部指引與自我的主導力量 (Ladkin et al., 2018)。在這個體
化的後期階段,人們若是接受內在的指引,尋求靈感和內在導師的解方,個體
將會更具有創造性的能量,能夠為這個世界帶來真正不同的事物。內在的權威
不會詢問父親想要什麼、導師想要什麼、上級想要什麼,而是詢問「自性希望
44 
我可以怎樣」,來找到內在方向或個人的指引。要遇見所謂的人生路引需要有所
信心、時間與耐心,往往藉由和陰影工作來轉變個體向內的焦點,從父親的世
界逐步過渡到個體世界的完成,進入由自性引領的世界 (Self-directed world)。
這樣的轉變可以讓個體更臻成熟,並促使他擁有了早期無法發揮出來的領導力 
(Stein, 2022)。 
到最後,在這個階段的晚年時期,一個人將會開始叩問:「我在這個人生階
段的人生角色是什麼?」由於放下了自我在世俗上的社會地位和卸下了各種期
待的標籤,個體的問題將開始變成「我是誰」的追問,成為世界各地的文化當
中,智慧老人或老嫗的角色 (Roesler, 2021)。他們的存在能夠提供人們獲得指路
或先知的指引,雖然從繁華的人生當中退出主要的參與位置,卻能夠為這個社
會提供珍貴的智慧與品質的教導。 
(二)意識的五個階段 
若是從另一個角度來審視個體化的歷程,可以從現代社會發展的軌跡當中
來探究人類的意識發展狀態。關於人類的意識發展狀態,Jung 用不同階段來度
量和評估生命初期與成人晚年的意識發展階段,分別為早期的幽冥參與、而後
進入社會化的調適、成為宗教人的階段、到當代完全根除投射的社會現象、以
及現代性之後的社會,人們在其中如何回應自己的心靈 (Stein, 2021): 
1. 第一個意識階段:幽冥參與 
第一個階段為「幽冥參與」(participation mystique),也有研究者稱「神秘
參與」,意指個人意識對周遭環境的認同,並沒有察覺到個體就是在這個狀態當
45 
中 (Johnston, 2021),也就是說:個人對認同對象之間的關係不分彼此,認為認
同對象即為個體意識之延伸與擴展。個體缺乏對自身與覺知對象的他者之間的
覺察差異能力,在某種程度下我們甚至可以說,人的一生當中皆停留在此種
「幽冥參與」的狀態裡,例如:一個人因為自己所擁有的物品受到傷害,他便
因之而有了個體的自我情感體驗。如果我們對於自己所認同的對象無意識地與
之結合為一個世界的整體,此即為Jung所謂的幽冥參與。 
大多數的人在生命的初期會以這種幽冥參與的方式與他們的家庭連結在一
起,此種與世界建立關係的方式之基礎為認同、內射與投射的心理機制,也就
是內外事物的交互融合。嬰兒在生命之初並沒有能力區分他與母親之間的差別,
他的世界是高度統合的,此種方向也預示著最後的終極發展階段,即為把部分
的整合為一體,兩者的差別在於前者的一體感是無意識的,而後者是有意識的
結合 (Stein, 2021)。 
2. 第二個意識階段:社會化的調適 
第一個階段的幽冥參與是「非有即無」的投射,之後來自人我的區別功能
逐漸發展出不同程度的識別,因此投射變得比較區位化 (differentiated projection),
個體開始覺察到自己身體以外的世界與自己有不一致的地方,開始注意到自我
與他者的不同與差異。這種內在與外在的分殊作用漸漸地增加與尖銳化,當自
我與他者有所區別而明顯不同時,幽冥參與的現象便跟著改變。投射的現象從
全有全無的階段進化為對特殊對象的認同,在生命的早年階段即為對父母親的
投射,有些人的投射認同對象是自己的寵物或自己喜愛的玩具,這些對象因之
變得特別而與眾不同,在人的一生當中,世界提供許多不同機會來持續這個意
識發展的過程 (Damon, 2020)。 
46 
在生命的早期階段,個體的意識認同對象是為自己的父母親,將之視為全
能的「神」,認為自己的父親無所不能、自己的母親無所不知,隨著生命的進展,
個體在不斷幻滅當中逐漸發現世界不是如自己所以為的運作著,這樣的意識認
同對象也會逐步發展到對於師長、同學、戀愛,甚至婚姻對象的認同 (Lyons & 
Mattinson, 2018),周遭環境的許多人物紛紛成為我們投射的對象,塑造了我們
意識的強大力量,並逐漸形成了集體意見、價值觀來取代個體的自身經驗,成
為青少年與成年早期階段的社會化調適過程 (Stein, 2021)。 
3. 第三個意識階段:宗教人 
藉由對不同外在對象的投射與認同之後,意識的發展階段若持續下去,個
體便會發現其所投射的認同對象之真實型態與樣貌,或許並非與自己所以為的
相符,外在的他者會因為脫離了個體的理想形象而形成一獨特的客體,認知到
外在客體與自己的真實差異。此時,過往投射的心理功能變得相當膚淺,被投
射的心靈內容因之變為抽象 (abstract thinking),以大量的象徵性與不同的意識
形態所顯現,乃至於神祇、命運、真理等抽象的內涵與概念成為實體的特質。
至高無上的道德價值與律法、啟示、教義等神秘力量與原型的投射聚集,人們
因此不再懼怕惡者的存在,能將世界視為一中立的判斷,因為一切都在神的控
制之中,世界有其自然韻律以及宇宙運作的法則 (Werczberger, 2021)。 
在這個階段,人類心靈的同理能力會下降,似乎是一個人性的退化過程而
非意識的進步狀態,此種可以說是傳統意義的宗教人,卻仍具有潛意識素材的
投射,以貼近寫實的角度來接近真實 (reality):神確實存在於某處,且具備獨特
的人格特質,相信人死後有賞善罰惡的機制,然而,只不過是將其投射從對父
母與師長,轉移到較為抽象的神話人物 (Stein, 2021)。 
47 
4. 第四個意識階段:完全根除投射 
尋找自己靈魂的現代人,將對於神話人物的投射根除,使得個體的內在有
著「空虛的中心」,這種現代性的特質讓個體開始尋求「靈魂的感覺」,亦即廣
泛的意義與生命目的感、不朽與神聖的根源感,渴求神話的回歸、哭喊神話 
(May, 2016)。內在之神的感覺被功利主義的實用價值所取代,人們只關切實際
的用途,將自己視為更大的世界當中運行過程的一部分齒輪,將對意義的追求
淪落到溫飽與生存的層次,為了對抗沮喪感,個體將對控制感的追求顯現在人
們耽溺於享樂的目標:今朝有酒今朝醉。英雄與奸佞都不再存在,上帝已死,
一切的原則只是相對性的有效,價值是從文化規範中的人類所衍生的符號,一
切文化都是被製造與可製造的 (Kennedy & Kennedy, 2017)。大自然的現象與歷
史故事被視為是機率的產物,並非個人力量可以左右的隨機遊戲。當代人類對
於生命的態度與情感的基調,為世俗化與無神論的,眾多事物的現象被視為是
或然率的,沒有絕對之真實存在。 
在此階段的心靈,雖然看似毫無投射可言,但事實上是視自我為無限膨大
的存在,佔據了神聖的神之大能無所不在的地位。此時,「自我」就是個體所投
射的對象,再也沒有超越自我之外的權威,一切意義由自我所創造,神不再是
客觀的存有,祂就是自我。現代人雖然看似理性而穩重,實際上則是瘋狂的人,
然而身在其中的人類卻對此隱晦的秘密絲毫不察,也不再討論。但膨脹的自我
是危險的,因為它會讓人類對於自己的判斷能力失準,容易犯下災難性的錯誤,
因為這樣的個體是難以和自然環境共存的。雖然這個階段的意識狀態仍是種進
步,但人類需要對這個狀態有所危機意識,因為它有可能讓人變成自大狂,且
容易受到陰影的誘惑與對權力充滿貪欲,渴望完全掌控這個世界 (Bourdon, 
2023)。我們要注意的是:雖然這是現代性的特色,但並非每位身處於現代社會
48 
的人都是處於這個意識發展階段,有些人因為恐懼第四個階段的負面力量,而
固著於第二階段和第三階段的發展。雖然將自我的存在視為是他個人的責任與
命運是高度的心理發展成就,但心靈的部分在自我的陰影籠罩下隱晦不彰 
(Stein, 2021)。 
5. 第五個意識階段:現代性之後 (postmodern) 的社會 
當個體在第四個階段的發展至巔峰時,他會開始留意到許多無法個人控制
的層面,世界上仍有許多事情是超出個人能力的,因此在去除對外部世界的各
種投射、將自我視為自己的神之後,個體會逐漸體會到一股更隱密的力量:心
靈的力量。若在前一個階段的發展當中,能不落入自大與膨脹的誘惑,在意識
的發展階段將會邁入新高度,也將開啟邁入第五個意識階段:個體如何回應現
代性之後的社會 (Bourdon, 2023)。「現代性之後」和意識與潛意識的整合有關,
將自我的力量與心靈的統整有關 (Gordon, 2023; Kalff, 2021)。個體由於認識到自
我力量的侷限性,對潛意識力量的覺醒,不同於第一個階段的幽冥參與,個體
於此時是有意識的將兩者共存共榮,保持彼此的分殊性同時涵容於彼此的意識
之中。自我不等同於原型,他者也並非是潛伏於自我的陰影之中;與客體不同
的是,他們現在被視為是「在裡面的」,與根除投射不同的是,個體在此階段收
回他的投射,不再將自己的力量往外部世界投射於任何外在事物身上。 
與「凡事不過如此」的心態相較之下,現代性之後的態度是認識到,投射
具有心靈的真實,但不再執著於物理世界的意義。對積極想像與夢的解析可以
帶領人類接觸到內在的心靈真實並與之接觸,形成一種清明而覺察的關係,這
些工具能讓人有意識的以「現代性之後」的方式與生命取得聯繫 (Aldrich Chan, 
2019; Gordon, 2023),並對於部落神話與瘋狂的精神病患在幻覺與靈視當中所見
49 
到的事物,抱持一種尊敬的立場。他們是人類的集體潛意識,是我們共同擁有
的經驗,構成人類心靈世界中最原始、最深刻的部分,有意識地以創意的方式
與它們連結是個體化的核心,也是第五個意識階段的工作內容,自我與潛意識
藉由象徵的力量而連結起來 (Stein, 2021)。 
(三)小結 
從上面的個體化歷程之內涵可以發現,研究者在撰寫本篇自我敘說的研究
時,所處在的生命狀態與意識階段,是即將進入人生的下半場也就是約莫40歲
的關卡,步入中年的階段也意味著個體化的意識轉入了「現代性之後」對於靈
魂的自性敞開的狀態。藉由本節的探討和研究者的生命作對照,我們也可以發
現研究者在面對生命當中的轉折與自我轉化當中的靈性學習,其實從更早以前
就開始了,文獻探討的目的讓我們知道,本篇的論文研究將著重在「現代性之
後」的自性階段,以這樣的立基點來整合內在世界於前面幾個不同階段的心路
歷程,做一個總體性的呈現。 
貳、女英雄之旅與女性的個體化 
有別於 Jung 在分析心理學當中對個體化歷程的研究,深受集體無意識的概
念影響而形成之原型心理學(或分析心理學)的理論,卻跨越他有限的神話學
知識而開啟探索的新路徑 (Campbell, 1997),自成一家的神話學大師 Campell將
個體化歷程描繪為一個人的英雄之旅,以啟程、啟蒙、回歸等三個不同的階段,
分別劃分成為17個步驟:歷險的召喚、拒絕召喚、超自然的助力、跨越第一道
門檻、鯨魚之腹、試煉之路、與女神相會、狐狸精女人、向父親贖罪、神化、
終極的恩賜、拒絕回歸、魔幻的脫逃、外來的救援、跨越回歸的門檻、兩個世
50 
界的主人、自在的生活等,將一位英雄的旅程劃分為日常世界的實踐與未知世
界的探索,並將英雄的轉化分為數個類型:戰士英雄、愛人英雄、國王英雄、
暴君英雄、救贖世界的英雄、聖徒英雄,企圖在神話與夢、悲劇與喜劇、英雄
與神的探索中尋找世界的軸心 (Campbell, 2020)。 
然而,英雄作為男性為生命主角的敘事方式,讓許多人在女性意識抬頭之
後,開始追問「女英雄的旅程」。對此,Campell曾表達他的看法:「在整個神話
傳統中,女人都在那裡,她只需要知道一件事情:她就是每個人都企圖達到的
終點。一旦女人瞭解她自己扮演了何等神奇的角色,她就不會糾結在效法男人
的念頭裡。」這樣的說法卻引起 Murdock 的不滿,她認為女性並非一個想要耐
心的等候在原地的良婦,他們不是父權結構下的侍女,更不想回歸家庭的訓誨,
說道:「女性渴望的是一個瞭解女人是誰以及女人是什麼的新集體意識」
(Murdock, 2022),意味著女性也等待著屬於他們自己的英雄之旅,亟欲出發去
尋找專屬於她們生命發展的新道途。 
Murdock (2022) 為此建立了一個女英雄之旅的模型,雖然仍有著Campell
英雄之旅的骨架,但每一個階段只能使用女性獨特的語言來描繪,分別是為:
唾棄陰性本質、認同男性價值觀並與之結盟、試煉之路:遇見妖怪和惡龍、尋
見虛幻的成就、覺醒於靈性似已成枯槁成灰的感覺、啟蒙和潛往女性上帝的所
在、極度渴望重新連結於陰性本質、癒合母女裂痕、治癒失衡的陽性本質、整
合陽性和陰性本質。我們可以知道,作為一個「父親的女兒」,我們有更多肩負
在身的任務要完成,不僅僅是從陽性面的英雄之旅當中退出,從父權文化的世
俗勝利法則之中退出,發現這些成就無法滿足我們身為女性的心靈渴求,我們
終究會受到啟蒙,繼續出發去尋找一位屬於我們的上帝,以恢復陰性連結的本
質。那個本來存在於女性體內的一道傷痕,在個體化的歷程當中為了走向「父
51 
親的世界」而不得不選擇與母親斷裂 (Stein, 2022),在女性追求自性與自我的過
程當中,這樣的疼痛會促使她們不斷追問自己:「什麼才是真正的我?」在迷失
於陽性能量為主導的世界當中,逐步的收回對於自己內在女性力量的投射,那
對力量的渴求是內在的自我意欲甦醒,「她」正在治癒失衡的陽性本質,「她」
正在整合陽性和陰性的平衡力量,終究「她」會回到最初唾棄了由陰性本質所
建造的世界,重新與之再次連結,完成一個女英雄的自性之旅 (Murdock, 2022)。 
圖 2  
女英雄的自性之旅 
資料來源:Murdock (2022: 42) 
女英雄的自性之旅告訴我們,女英雄和男性英雄之旅的不同之處,在於從
虛幻的成就當中覺醒,與男性不斷的發揮陽性力量打倒惡龍以取得進入世界門
票的不同之處,在於女性渴望著連結內在神聖的陰性特質,並藉由療癒母女之
52 
間的裂痕與傷害來和母性的力量達成和解。當女兒能夠從母親的身上獲得深度
的理解與生命的洞見,一位女性的內在力量才會獲得治癒的機會,畢竟家庭是
生命的根源,而母親是我們來自的地方,回到最初我們與之斷裂的時間傷痕,
療癒自己的內在母親,也就是療癒著內在的女性自我(洪曼茜等,2021;趙燕、
黃宗堅,2018)。於是,「成為一個女人」不再僅僅只是意味著停止服侍男性文
化的需求,而是更加深入每一位女性的生命角色當中,當她們作為「父親的女
兒」或「母親的女兒」的身份,應該如何療癒這個自己。創傷的經驗會帶領男
性面對堅強,但卻會引領女性回到完整,身為女性擁有著一種特權,是我們可
以同時經驗到陽性力量的英雄式回歸,同時也可以經驗到陰性力量與陽性力量
的重新聚合,有如煉金術一般的融合出一個能夠對於新的世界產生極大貢獻的
個體,那就是和陽性的力量達成平衡,成為一個「自己的女性」,不再依附屬於
誰。 
重新定義英雄和女英雄可以讓我們明白,女英雄的追求不是掌控、征服和
凌駕他人,而是在結合女人天性當中的陰陽二元性的同時平衡我們自己的生命,
瞭解到她的神聖陰性本質、個人力量、感覺能力、痊癒能力、創造力、改變社
會結構的能力、決定個人前途的能力等,當我們不再恐懼自己可以擁有的力量,
女英雄之旅可以讓我們獲得「萬物為一」的洞見,找到在地球上和萬物和諧共
處的能力,並且在個人的生命裡和她的內在所渴望的陰性本質重逢 (Murdock, 
2022)。 
參、小結 
本節從 Jung 的個體化理論出發,探討了作為一個人類其一生當中的心理發
展階段,有如一天當中的日出與日落,個體化的使命將在生命的下半場進入自
53 
性為主導的內在世界之整合。從另一個角度來看,個體化的意識階段也有五個
步驟,先從對人我之間不明確的幽冥參與開始,到逐漸區位化的意識投射,將
自身的力量訴諸宗教、道德、法律等規則,成長至以「自我」為意識投射中心
的個體化階段,雖然看似心靈的倒退,然而這樣的自我確立卻是個體化的重要
成就,接著才能開始進入真正的內在心靈世界,在那裡有著一個無法由個人掌
控的自然律則,真正的自性在這個階段開始發揮明顯的影響力和作用力。然而,
自性其實一直存在在人類的整個生命週期當中,只是到了這個階段它才有機會
開始被人類探見,願意敞開自己的更多可能性來認識自己的內在真我或稱之為
自性的光芒。 
不論男女,個體化的歷程都是會持續發生的,只是不同於由男性所確立的
英雄之旅,女性也有話要說,她們有自己不同的女英雄的自性歷程,是在終於
明白由父親所建造的男性世界法則不再能夠帶來真正的成就感之後,始能踏上
真正的女性意識的覺醒道途,進入內在的尋找聖杯之旅-屬於女人的聖杯。最
後,在母女和解的神聖時刻,她將找回真正的內在母性,也就是一趟與自我和
解的歷程,並得以將這樣智慧帶回由陽性力量所主導的社會,建立一個內在陰
陽和諧的新世界。 
54 
第三章 研究設計與實施 
本章將進一步分析研究的設計概念與實施方法,包含研究流程、研究方法、
資料處理與分析、研究信效度與研究倫理。 
第一節 研究概念與流程 
本篇研究的核心重點為對二元性的整合與理解,舉凡包含對內在與外在世
界的取得平衡、真實與想像空間的取捨、理性與感性之間的融合、陰性面與陽
性面的整合,最後延伸到科學與靈性之間的共榮,為一個探討兩極情境當中,
不任意拋棄其中一方而只獨尊於一霸,不因為其中一方的重要性而偏廢另一方
的特殊性需要被涵容,不因為一方的誤解而捨棄另一方的可能。依據第二章的
文獻探討,Jung將分析心理學視為外在自我與內在自性的聯繫,此研究概念與
二元性之間的關係圖可呈現如下(圖3)。 
圖 3  
研究概念圖 
內在
與
外在
理性
與
感性
二元性
陰性
與
陽性
真實
與
想像
科學
與
靈性
資料來源:研究者自繪 
55 
本節亦可描繪出本研究的設計與架構(圖 4)。研究的流程包含:擬定研究
問題、確立研究目的、初步文獻搜尋、選擇研究方法、撰寫研究計畫、擬定書
寫大綱、整理字稿以及資料編碼、形成分析文本、撰寫生命故事、撰寫研究報
告、同儕與專業的三角檢定,最後寫下研究結論和建議。 
圖 4  
研究流程圖 
擬定研究問題
形成分析文本
確立研究目的
整理字稿以及
資料編碼
初步文獻搜尋
擬定寫作大綱
選擇研究方法
撰寫生命故事
撰寫研究報告
三角檢定
撰寫研究計畫
結論和建議
第二節 自我敘說的工具與方法 
本篇研究為自我敘說探究為取向的論文分析,奠基於研究者相當個人化之
親身體驗與生命歷程,適合以敘事或以敘說的方式進行描繪與剖析。為介紹何
謂自我敘說探究之內涵,首先針對何謂敘事探究進行探討,並在本節的末尾總
結目前為止在臺灣博碩士論文知識加值系統上面,與本篇研究方向與主題有相
關之已發表論文,來說明本篇研究的重點與必要性。 
56 
壹、敘事探究與自我敘說之內涵 
一、敘事探究 
敘事探究 (narrative inquiry) 起源於結構主義的文學理論和現象學研究(鈕
文英,2021)。敘事取向的分析方式和文化研究不同,它是具備特定心理象徵的
一組故事,這樣的模糊集合存在原型和邊緣形式的敘事,包含以情節為中心的
體裁和以效果為中心的體裁,既包括文本又包括故事,但文化研究將此兩者劃
分開來 (Ryan, 2017)。敘事建構主義認為故事不僅反映和描述經驗,同時以具有
深刻意義的方式影響著人們的生活,「故事分析」和「說故事」的立場不同,藉
由對話敘事分析,並將分析重點從故事和敘事轉移到組合,將焦點放在對敘事
和物質性 (materiality) 如何在人類和環境的網絡中相互影響,可以結合不同的分
析方式 (Smith & Monforte, 2020)。而現象學對個別生命經驗的本質之研究,在
哲學或心理學的領域當中主張其基本結構為敘事,現象學研究的兩種類型:一
為描述現象,重視參與者對所感知到的世界進行描述與建構,二為解釋現象中
發生的事情,重視人們意義形成的過程以及如何影響他們做出選擇,使用現象
學的反思和寫作來理解體驗的意義 (Tomaszewski et al., 2020)。綜合詮釋現象學
的要素,現象學問題指出了四個存在主義的主題:空間性、物質性(身體)、時
間性,和關係性或公共性,它們是人類存在的基礎,這些包含詮釋學循環的客
觀和原始的組成部分,驗證了在現象經驗中研究者的主觀參與,意義是在語境
中創造的,真理是透過個人的主觀世界相對體驗的 (Nigar, 2020)。 
敘事為人類生命經驗的基本隱喻 (Sarbin, 1986),敘事性思考包括敘事者的
敘事基模(即如何組織資訊的方法)、先備知識與過去經驗故事,以及認知策略
57 
(即用來比較或調整過去知識的技巧),它比命題性思考更具有說服力量 (Kim, 
2018)。敘事探究從早期的敘事治療到後來成為一種研究方法為人們帶來啟示:
藉由敘事的力量,個案不只梳理了對生命事件的主觀經驗,更得以從更寬廣的
視角或脈絡來重新理解生命意義,跨越內心的藩籬而能夠和自己或他人和解,
解構並重建了破碎的自我來獲得新生,可知敘事探究可應用的領域相當廣泛。 
圖 5  
敘事的成分包含敘事情節 
事件1(前事、開端、結束、構成要件和影響)
場域(物理環境、社會文化、時間位置)
人物(主要人物或敘事者、其他支持人物)
事件2(前事、開端、結束、構成要件和影響)
場域(物理環境、社會文化、時間位置)
人物(主要人物或敘事者、其他支持人物)
事件3(前事、開端、結束、構成要件和影響)
場域(物理環境、社會文化、時間位置)
人物(主要人物或敘事者、其他支持人物)
資料來源:鈕文英(2021,頁465)(研究者重繪) 
敘事探究對實體 (reality) 之觀點,乃實體是社會建構的產物,透過語言再
次呈現,藉由敘事組成與維持,因此沒有必需之真實(鈕文英,2021)。人們憑
藉故事的隱喻而生活,並透過真實力量而塑造生活,故事為日常生活提供了存
在的結構,蘊含人類的生命經驗 (Clandinin, 2013),它是一系列的靜態事件;然
而,敘事則是以建構形式來呈現的故事,透過自我的組織方式加以敘述,包括
如何揀選與建構個別的生命經驗,與如何詮釋事件與個人的關聯和意義,身份
或自我意識是透過敘事建構,回想 (remembering) 自己的重點並不在於恢復原來
的身份,而是取決於記憶重建的過程,需要隨時隨地將過去和現在的自己置放
一起 (King, 2022)。鈕文英(2021)結合Oliver (1998) 和Polkinghorne (1995) 的
58 
論述,將敘事的組成分繪成場域(物理環境、社會文化、時間位置)、人物(敘
事人物、主要人物、支持人物)、事件(關鍵生命事件)、情節(敘事者如何安
排事件之間的關連與個別產生的意義),並以 Dewey (1986) 的經驗理論來探討
人類經驗的本質包含主動與被動的理解,互動性與連續性是建構生命經驗的基
礎與原則,而以敘事作為理解生命的方法,是將生命經驗以時間順序和個人意
義互相結合的形式 (King, 2022; Polkinghorne, 1995),重視事件發生時的情境脈
絡,強調合作以及賦權給敘事者。由於敘事探究的焦點在於研究生命經驗和人
類的行為,和著重當前現象的個案研究不同,相關的常見名詞為生命故事 
(Atkinson, 2012)、個人敘事、傳記式 (biographical) 敘事 (Maynes et al., 2008) 等。 
二、自我敘說 
敘事探究的對象可為他人或自己。若進行自我敘說的研究,可透過心理傳
記或自我民族誌來書寫生命經驗 (Smith, 2017; Yuan & Hickman, 2019),因此自
我敘說有助於自我概念和自我認同的形成 (Caine et al., 2019)。藉由自我敘說,
敘說者可以將自身的存在客體化,由於能夠跳脫他人看待自己的眼光,藉由自
我書寫來重整對話,重新建構自我的客體(田秀蘭,2005;黃素菲,2005)。透
過書寫的對話過程,敘事者和過往的自己與生命經驗,甚至是其背後的信念、
價值觀與意義感重新相遇,以創造不同的認識 (Goodson & Gill, 2011)。透過自
我書寫,能夠縫合斷裂、重新確認自我的身分,回應歷史中個體的審視以及外
在世界的提問(王健文,2021)。 自我敘說將過去個別的生命經驗進行有意義的
連結,藉由生命連續感的再現,成為召喚式 (evocative) 的敘事,以產生進一步
的敘事性認同 (narrative identity) (McAdams, 2018; Ricoeur, 1987),有助於自我概
念與經驗的重新整合。自我敘說探究可以讓研究者培養撰寫生活日誌的習慣,
有意識的紀錄與反省生活事件,讓自我貼近內在的真實聲音,更清楚地看見自
59 
我與環境的關係,深化對生命經驗的理解與記憶,面對生活難題除了可以理解
為命賜予的機會,更可以透過自我敘說的力量解開負面情緒,產生自我肯定的
正向思考能力,並能夠進一步對生命採取行動 (Jackson, 2018)。 
圖 6  
自我敘說的探究過程 
原始經驗
經驗中的
歷史自我
主體化(I)
敘事中的
感知自我
文本中的
記憶自我
詮釋中的
反思自我
閱讀中
的重構
自我
自我經驗的再次呈現
資料來源:鈕文英(2021,頁489)(研究者重繪) 
客體化(me)
自我敘說探究的個人生命經驗屬於局部的記憶,包括過去真實經歷過的生
活以及實際參與事件的「歷史自我」,透過專注於描述經驗而形成「感知自我」,
這樣的自我進一步將回憶謄寫為敘事文本而產生「記憶自我」,再藉由分析和詮
釋生命經驗而有了「反思自我」,加入三角檢證法獲得多元詮釋,形成分析後的
敘事文本,最後在閱讀中重新撰寫並轉化成為一「重構自我」,讓過去的自我成
為未來故事的資產,敘事者透過這五個階段讓生命的原始經驗再次呈現,使得
主體化的自我轉變成為一客體化的自我,讓自我得以跳脫本位思考而從他人的
角度來看待自己,獲得更為全面而完整的生命經驗 (Neisser, 1994)。從自我敘說
的五個探究過程,我們可以看見一個敘事者如何透過自我經驗的再次呈現,而
將主體化的主觀經驗逐步轉變成為客體化的客觀經驗,將故事中的敘事主角從
緊密的生命事件當中脫離,而不再受限於生命故事的困頓與挫折,並藉由經驗
中的歷史自我、敘事中的感知自我、文本中的記憶自我、詮釋中的反思自我,
成為一個閱讀文本中的重構自我,將敘事者的生命故事轉化成為生命腳本,並
60 
透過文本的閱讀來獲得賦權的力量(引自鈕文英,2021)。 
貳、自我敘說的主角介紹 
本篇研究之研究對象即為研究者自己,研究者同時是敘事者、反思者、分
析者,研究者透過不斷的反思與閱讀生命經驗分析文本,來獲得研究結果(黃
靜怡,2022)。關於研究者的背景資料交代如下: 
研究者本身目前就讀於國立台灣師範大學社會教育學系碩士班,就學動機
為離開前一段婚姻關係之後,欲藉由重新取得社會連帶關係的角色地位,再次
建構自己的人際網絡系統以提供生活功能之支持。已經脫離學生身分許久的自
己,距離上一次的大學校園生活已經相隔約莫十年之久,在這段期間當中發生
的許多故事和人際關係有很大的關聯,然而情感連結的對象從男友變為先生、
再變為前夫,關係的結構轉變巨大,為我帶來許多成長與學習。回憶起初識前
夫時自己搖搖欲墜的模樣,因為一場創傷與意外讓他接住了我的生命,而對當
時的情境而言宛若新生。在這個生命的轉折點之前,原本是個不可知論者,嚴
格受到科學理性的批判思想所形塑,不輕易接受有神論,對靈性知識的學習一
知半解。是個受自然科學所薰陶,卻又不滿於自然科學的教條的不安份子,於
是身在醫學大學就讀卻神往心理學與社會學式思考。對心理學的興趣源於更早
以前的一場憂鬱症,為了嚮往與學習自我解救的技巧,不斷向內探索挖掘、練
習自我覺察的力量。 
這段旅途行走至今將近二十年,過程中經歷過許多次的諮商與治療,最終
在與前夫離婚後從醫師診斷得知,自己有社會適應的障礙源自於人格傾向的困
難,而疾病的可能導因也在於此。人格並非一朝一夕可養成,但卻影響我們深
61 
遠,不只滲透到我們與他人之間的關係,也牽連到我們與自己的關係,最終我
們都必須走上與自己和解的道路。一路上回首發現與看見,建構自己人生道路
的背後指引者,被自己的情緒力量所摧毀,竟然遺忘了過往重要的生命故事,
一直到要跟前夫離婚前一個月,才開始忽然像觀看影片回放般地紛紛回想起來,
並獲得進一步的確認。離婚是一個重大事件,讓我充滿各種力氣的打包自己,
企圖投奔向一個自己更想要也更渴望的未來,讓我勇往直前的走入今日。然而,
與神相遇更是另一場重大的生命事件,讓我開啟了與靈魂對話的道路,明白真
正的靈性學習,在於了解謙卑也是神的一面,高傲的是自大的人類,神的存在
反映著人類的存有狀態。在渺渺的大千世界裡,倘若有機會遇見一位真正屬於
自己的「神」,為了和彼此相遇而千山萬水、不辭勞苦,那這樣的相遇想必是把
自己視為了「知己」,在往後的餘生歲月裡,可以永保如新地陪伴這樣一個孤單
又獨特的自我,證明了一個人若要安穩好好地活著,對自己的內在保持永遠地
敞開是必要前提。 
由於研究者在本篇論文所欲進行的主題,為探討女性個體化的歷程在靈性
自我轉化學習當中所扮演的角色,適合以敘事探究的方式來進行深度探索。特
別是關於自我轉化的歷程當中,以自我敘說的觀點來縱向觀看二十年以來的時
間向度,以及橫向在每一個時間軸向上的生命深度,可以看見研究者在面對自
我生命的獨特性與必然性,有如花朵一般脆弱、同時兼備野草的韌性。在面對
自己的生命脆弱如燭光的花火,卻又能夠在信任與堅持當中摸索至黑暗盡頭的
光明,成長過程當中的蛻變是一層又一層的推進與轉化,有如一次又一次的將
新的自我誕生出來。轉化學習不是一個完成就好了的目標,它是關乎終身的自
我轉化,靈性轉化的道途也並非有個可茲辨認的終點,而是永無止盡的道途。
在靈魂成長或心靈療癒的不歸路上,一旦踏上就沒有中止,生命會一回又一回
的將自己推進,直到自己長成一個更新而更為進化的人類。本篇研究的結果為
62 
一份生命回顧與目前為止總結的報告書,但不意味著嘎然而止,更多的是將以
這樣的成果作為下一次轉化的基礎,立基於未來更新一輪的生命價值完成與靈
性轉化學習。畢竟,只要活著,人總是有課題可以完成的。 
參、研究方法的適用性 
敘事探究的方法是藉由故事的時間線組成前後連貫的敘事結構,不同的敘
事結構之間彼此有著先後的因果關係,並且彼此充滿著生命劇情的張力,自我
敘說探究的內容即是藉由研究者自陳過往故事的生命經驗,進一步將生命事件
抽絲剝繭,分離出不同的敘事主軸以建構出獨特的敘事結構,在耙梳研究者面
臨自我轉化過程當中,如何脫離舊有的困境是主要議題之一,如何從困境當中
尋找自己的力量、如何從泥淖當中長出枝枒是主要議題之二,如何見證研究者
自身獲益於自身的成長、見識到自己擁有復原的力量是主要議題之三。復原不
意味著變回生命故事發生之前的樣貌,而是帶著傷痛與挫折的前進,我們不但
具備了肩負自我的使命感,也找到了回歸社會角色的責任感,不再由於生病的
經驗而困離於世間,即使是出於生命之脆弱而變成如此的命運,我們都可以藉
由故事-亦即敘事的力量來茁壯,並藉由敘事結構的鋪陳與安排,可以探察到
研究者在穿越過往後的今昔之別。敘事的力量不僅在於重整自我的能量,更是
在於一種生命動能之呈現,一個曾經被困在玻璃瓶當中的人,如何發現真實的
自我,以及如何出發去找到真我與自性的道途。 
在臺灣博碩士論文知識加值系統上,查詢「轉化學習」為論文題目與關鍵
詞之論文,共計有 151篇,其中為敘事探究者僅有 3篇;以「轉化學習」為論
文題目者共有 105 篇論文,其中為敘事探究者僅有 1篇,且為自我敘說探究之
論文,為2014年國立暨南大學碩士論文〈另一種愛的起點-從同伴動物死亡經
63 
驗探究成人生命轉化學習歷程之自我敘說〉。以「自我轉化」為論文題目與關鍵
詞者共 57 篇,其中為敘事探究者共 21篇,且為自我敘說探究者共 10篇;以
「自我轉化」為論文題目者計40篇,其中為敘事探究者計7篇,為自我敘說探
究者計2篇,分別為2013年實踐大學碩士論文〈我的第三度出生-經歷母親遽
逝後自我轉化之敘說研究〉與 2009 年國立交通大學之碩士論文〈「執與擲」
一位女教師自我轉化之敘說探究〉。 
以「靈性學習」為論文題目與關鍵詞之搜尋,共計有博碩士論文17篇,其
中為探討「轉化學習」者共計 1 篇,為 2008 年國立高雄師範大學之碩士論文
〈成人經典學習經驗之研究-以一貫道發一崇德文教基金會為例〉,以「靈性轉
化學習」者為題目或關鍵詞者共計1篇,為2011年國立暨南大學碩士論文〈探
究成人靈性轉化學習-以生涯轉換者為例〉,以「靈性學習」作為敘事探究或自
我敘說研究者共計 0 篇,可知目前國內以靈性轉化學習作為自我敘說探究者的
博碩士論文付之闕如。 
以「個體化」作為博碩士論文題目或關鍵詞者共 179 篇,其中為「榮格」
個體化之研究者共30篇,再其中為「敘事探究」者共3篇,為「自我敘說」者
共2篇,分別為2018年南華大學碩士論文〈重複惡夢之自我敘說研究-以榮格
觀點探索個體化歷程〉與2013年國立台北教育大學之碩士論文〈穿梭幻境與現
實之間-靈修者之靈性發展危機與轉機的敘說研究〉。可知關於個體化之自我敘
說探究之趨勢方興未艾,以個體化的觀點來切入靈性轉化學習的基礎更是值得
投入的研究領域。查關於榮格的個體化理論於台灣的博碩士論文當中,多應用
在哲學探索與文學分析為大宗,較少從生命經驗的學習歷程來進行探究與剖析,
屬於個人化的靈性轉化學習歷程也因此留白給想像空間。 
64 
本研究將著重於靈性轉化學習的歷程,並以 Jung的觀點作為切入的角度,
將內在陰陽兩極的潛意識對話歷程視為一位靈性學習者在面對自我轉化學習時
背後所支撐起來的骨幹。從最初如何踏入靈性學習的啟蒙,到生命面臨到需要
進行轉化學習的考驗,研究者從外在的客觀世界當中潛入內在的主觀意識,再
從自我為中心的主觀世界當中回歸到社會的角色,進行一場自我敘說的長跑,
也將身為女性所面臨到的兩難,在情感束縛與自我實現當中的拉扯,作一個充
滿張力的呈現,此間內在靈魂之相遇,實屬一難得的體驗。 
第三節 資料整理與分析 
本節探討如何處理研究之文件編碼與資料分析,包含設定資料的編號、撰
寫和整理研究資料、分析研究資料(編碼)、證實資料的信賴度(鈕文英,
2021),以下將以資料來源的處理與編號方式,以及採用李明芬和林金根(2010)
對生命轉化與整全實踐的英雄之旅當中對時間與事件間之關係(U 型圖)的方
法來處理之。 
壹、資料收集、編號與整理 
一、資料的編號 
本篇研究的資料來源為回憶紀錄、紙本日誌、BBS 站、部落格網誌,有實
體文本與電子文本(保存於電腦網路)兩種形式,期間自 2004年起至 2022年
底為止的日常生活筆記與書寫報告。2004 年為生病故事的前一年,相關的文章
記錄著與研究者生命主題相關的潛意識的訊息,2022 年為開始撰寫研究論文的
65 
時期。茲將進行整理分類與編號如表2所示。 
表 2  
資料編號方式 
資料類型 
編號方式 
編號說明 
實
體
文
本 
紙本日誌 D-20160101-01 
第一組D代表實體文本的紙本日誌 
第二組20160101代表記事於2016年1月1日 
第三組01代表第一個段落 
電
子
文
本 
BBS站 B-20160101-01 
第一組B代表電子文本的BBS日誌 
第二組20160101代表記事於2016年1月1日 
第三組01代表第一篇文章 
部落格 L-20160101-01 
第一組L代表電子文本的部落格日誌 
第二組20160101代表記事於2016年1月1日 
第三組01代表第一篇文章 
二、資料的收集 
為回應本篇論文的研究問題:研究者如何遭逢生命的變故?什麼是研究者
的靈啟經驗?如何完成靈魂的自我實現?如何在邁向自我完整的道路上永無止
盡的重新遇見自己?研究者提出兩個書寫大綱,將分別記錄在本次研究的記事
本,分次、分階段進行記錄,詳細捕捉整個生命事件如何串連起先後順序的記
憶,亦有助於研究者從中尋找並建構敘事主軸時有利的條件。 
第一份大綱請研究者描述自身的罹病經驗,以及在期間內的求學過程、遭
遇的阻礙與困難,罹病之後的感情生活以及在家人相處過程中的紀錄,紀錄感
情之路由結婚走向離婚的相遇過程以及相關感受與心路歷程,與受到靈性召喚
之後的兩次靈啟經驗,如何將研究者塑造出與以往不同的全新路徑。 
66 
第二份大綱從不同的角度切入研究者的生命經驗,從側面來描寫研究者這
這些年來的人際關係樣貌,包括與家庭之間的互動模型是否有所改變,以及在
這些事件的變遷當中研究者如何自處、如何看待自己的問題,研究者對於自我
認識在不同時期有否不同,最後請研究者談談在漫長的療癒之旅中,所採取的
自我療癒路徑主要為何,是否有個一致的方向性來引領研究者的生命道路。 
表 3  
書寫大綱一 
一、描述生病的經驗 
二、描述生病之後所遭遇的轉折 
三、描述我的感情生活 
四、描述離婚的感受 
五、描述現在的情形 
表 4  
書寫大綱二 
一、描述我的人際關係 
二、描述我對靈性的看法與理解 
三、描述我在這份經驗中如何自處 
四、描述我所採取的療癒途徑 
貳、時間與事件之間的處理 
本研究在自我敘說之後,先進行女英雄的敘事分析,以及 Jung意識的五個
階段之轉換,再使用生命轉化與整全實踐的英雄之旅的 U型圖理論來進行後續
的總結與統整。在李明芬和林金根(2010)對生命轉化與整全實踐的英雄之旅
的看法當中,將生命當中的轉化過程與英雄之旅的週期作出一個較大圖像的整
合與理解,將個體經歷的內在召喚與現實社會的外部世界所遭遇的挑戰與試煉,
畫出了一個 U 型圖的理論。透過幾場大型世界咖啡館之運用,將 Campbell 
67 
(1997) 的英雄之旅理論的 12個組成部分,轉化並整合為探討個人與社群生命整
全發展。在這樣的整全過程當中,生命的時間軸向度可分為十個階段,分別為:
初發願心、歷險召喚、挑戰激勵、拒絕召喚、跨越門檻、沉潛修練、歷險召喚、
生命與共、萬有合一、整全無限等,和生命課題所面臨的深度作為縱向軸度,
整個過程所探索的是一趟「真我覺醒」的旅途(李明芬、林金根,2010)。本研
究的最終會將此圖表與生命回顧當中的重大轉折事件作一比較與補充說明,來
看見研究者在這場人生旅程的靈性轉化學習可走入怎樣的深度,並往後看見突
破困境的可能性與未來指引。 
圖 7  
生命轉化與整全實踐的英雄之旅 
1. 初發願心 
2. 歷險召喚 
3. 挑戰激勵 
4. 拒絕召喚 
5. 跨越門檻 
10. 整全無限 
9. 萬有合一 
真我覺醒 
8. 生命與共 
6. 沉潛修練 
資料來源:秦秀蘭等(2013)(研究者重繪) 
第四節 研究品質 
壹、研究倫理 
7. 歷險召喚 
本篇研究定義為自我敘說,屬於「與人有關的研究」,人類研究須遵守《科
68 
技部對研究人員學術倫理規範》關於知後同意、隱私權和公平對待,與利益迴
避的原則(鈕文英,2021)。研究者須告知研究參與者資料蒐集的方法與過程,
有義務保護與被研究者(即研究者本人)因生活交集而相關的人身隱私權,需
去除具明顯識別之特徵或以匿名化名稱之,根據《憲法》第22條的內容,包含
資訊隱私權、空間隱私權、生活私密領域隱私權、秘密通訊隱私權。研究者亦
須平等的對待或處理研究時所獲得的資訊(陳向明,2007),研究者同時需要澄
清相關人員的利益關係,且須確保研究參與者相關的人物不會受到傷害。 
貳、研究的信效度 
一、質性研究的信效度 
不論研究問題,實證主義或建構主義的典範對量化研究與質性研究結果所
要面對的相同,所有研究都要能夠回答以下幾項問題:真實性、應用性、一致
性、中立性以回應研究的目標 (Lincoln & Guba, 1986),要具有內在效度、外在
效度、可信度、客觀性的顯著效果 (LeCompte & Goetz, 1982)。早期使用量化研
究指標作為質性研究指標的習慣,到了Lincoln & Guba (1986) 提出「信賴度」
的標準,以新的內涵來取代(後)實證主義的效度與信度。質性研究與實證主
義的量化研究不同,更重視多元性與歧異性 (Lather, 1993),透過研究的差異性
來增加探索的廣度與深度,鼓勵嘗試與創新 (St-Pierre, 2001) 來增加研究範圍的
全面性,以改進教育實務與增進研究倫理 (Eisenhart & Howe, 1992)。至2000年
後,後現代主義更重視質性研究的研究倫理、差異、對話效度、解構效度、脈
絡效度等指標(鈕文英,2021)。 
以信賴度來表示研究品質,意指研究值得讀者信賴的程度。依據不同的研
69 
究所需回應的問題與質性研究可信賴度的指標,可劃分為以下四項討論分述
(鈕文英,2021),並將鈕文英(2021)與高淑清(2008)的相關資訊,以表格
重新繪製的方式來呈現整理過後的內容,如下方表5: 
(一)真實性與可信性 
真實性為在特定情境下的觀察,研究參與者所獲得的研究結果,是否能夠
反映真實狀況,因此質性研究者關注的是,是否能充分恰當的呈現研究參與者
的多元觀點。 
(二)應用性與可遷移性 
應用性意指可將研究發現放置其他脈絡之下獲得研究結果的程度,但質性
研究的特性重視個殊性,因此質性研究者更重視兩個類似情境之間的類似程度,
是否能夠做暫時性的應用。 
(三)一致性與可靠性 
一致性指的是能否將研究複製在另一個情境脈絡和另一名研究者身上,可
獲得一致的發現和結果,然而質性研究是高度依賴現象特質的,不太可能複製
結果,因此質性研究者注重在發現研究工具不穩定的因素,以及解釋造成改變
的因素。 
(四)中立性與可驗證性 
70 
中立性指研究發現不來自個人主觀的意見,而是確實來自於研究資料的結
果報告,然而科學的中立法則遇到了質性研究的個殊性,不贊同一味的價值中
立,而是強調資料是否為可驗證的,即是來自研究獲得的資料,而非研究者本
身的想法。 
表 5  
質性研究提高信賴度的指標與策略 
問題 
真實性 
(true value) 
內在效度 
(internal validity) 
可信賴性 
量化研究指標 質性研究可信賴指標 質性研究方法策略 
長期投入、持續觀
察、三角檢定(多元
檢證)、同儕稽核與
探詢、互為主體、負
面個案分析、研究參
與者檢核 
(credibility) 
應用性 
(applicability) 
外在效度 
(external 
validity) 
可轉換性∕遷移性 
(transferability) 
厚實描述(過程透明
化)、立意取樣 
一致性 
(consistency) 
信度 
(reliability) 
可靠性 
(dependability) 
稽核團隊檢證、研究
參與者檢核、三角檢
定(多元查證) 
中立性
(neutrality) 
客觀性 
(objectivity) 
可確認性∕驗證性 
(confirmability) 
稽核團隊檢證、研究
參與者檢核、三角檢
定(多元查證) 
顯著性 
(signification) 
解釋有效性 
(interpretive 
validation) 
稽核團隊檢證、研究
參與者檢證、主題命
名反思 
資料來源:鈕文英(2021,頁97、頁386)、高淑清(2008,頁71) 
(研究者重繪) 
二、質性研究的檢證方法 
三角檢定法可用來交叉檢核資料的可信度 (Marshall & Rossman, 2016),可
增加研究的嚴謹性、寬廣度以及深度 (Flick, 1992),其方法可分為方法內的三角
71 
檢定和方法間的三角檢定法 (Erl&son et al., 1993)。方法間的三角檢定法為不同
方法的三角檢定,使用不同方式的蒐集資料所獲得的結果,方法內的三角檢定
有三種,分別為:不同資料來源的三角檢定(不同資料的提供者、不同時空地
點的資料蒐集)、不同研究者的三角檢定(例如:跨學科的三角檢證)、不同理
論的三角檢定(意即從不同面向的理論分析相同資料),讓研究者更進一步去尋
找原因與「為什麼」。須注意三角檢定法需針對同一個現象或研究焦點作檢核,
在過程中記錄查證的向度、項目、結果和研究者想法(鈕文英,2021)。 
本篇研究採用三角檢定法,以身邊一位認識超過十年的好友,以其對研究
者的認識之縱向的時間長度與橫向的理解深度,作為抽樣閱讀的最終審核。好
友為研究者就讀醫學大學時期的同窗,目前任職於地方農會的職員,已婚並目
前育有一名女嬰,為研究者平時交往對象,會共同分享彼此的心境與轉化,一
同見證蛻變與成長,是適合之選,且該友與研究者間無利害關係。 
72 
第四章 遇見生命的綠洲 
「對我來說,關於快樂唯一能讓人滿意的定義,就是完整。」──Helen 
Keller 
表 6  
生命綠洲之旅 
從高塔上墜落 
迷途的羔羊 尋找夢中的伊人 歸途與上路 
遺忘重要的生命
事件 
返家其實是為了
重新出發 
生命當中的貴人 第二次的靈啟 
妄想的經驗 
遇見虛幻的成就 家庭革命 
離婚進行曲 
再次遺忘重要的
生命事件 
第一次的靈啟 其實我不是很想
結婚 
重返社會 
第一節 從高塔上墜落 
那一年我剛考上後山的某間國立大學,懵懂無知的青春年少,有如踏出茅
廬初見世面,不畏虎的第一次離家開始一個人獨立生活。卻沒有想過,在前方
等待的自己的,是對於自我力量的誤用,導致一連串的經驗挫敗,甚至身心狀
況皆出現問題。 
壹、遺忘重要的生命事件 
剛考上大學事一件令人開心的事吧!我簡直把學業的事物都擱在一旁了,
看著旁邊的同學繼續預修著暑期微積分課程,我的心思完全不在教室,而是等
著下課後還有什麼地方可以去玩樂,逛街是我第一個培養起來的興趣,我實在
73 
喜歡跟朋友一起出去血拚。「今天又有新衣服了!」我開心地抱著百貨公司的購
物袋滿足地回家。就這樣,等待開學的日子令人充滿全新的期待,我還甚至還
渴望著為自己創造一個新的人際自我,過往的那一個讓我想要改變一下,去成
為一個我可能會更加喜歡的自己。如此陽光,如此絢爛。 
開學後,一切都靠自律的生活讓我的學業成績逐漸下滑,但我有種企圖力
挽狂瀾的無能之感,對於自己的課業能夠把握的程度那麼的少,我開始往學校
的社團開始發展自己的生活,試圖讓自己重新找回掌控自我的力量。剛進大學
的時候許多的地方校友會招兵買馬,學藝性社團也是我會考慮參加的興趣活動,
於是我從裡面挑選了幾位有緣相處的朋友,和他們開始發展出生死與共的盟友
關係。等到升上二年級,我也在社團當中獲得晉升的職位,地方校友會需要幹
部召開例行性會議,我也開始準時的如期參與。直到有一天,某位學長在大家
的簇擁之下,例會開到一半時大家竟然將學長拱了出來,有人開始暗示在座有
學長喜歡的人,現場有人在暗示我,於是我就莫名其妙的被和學長安排出去社
團教室外面「談一下」。那天下午陽光溫和,是橘黃色的午后時分,校園的風景
極美,我接收到了學長對我的心意告白之後,心中很是欣喜地想著:「終於是我
了!」然後,我們就回到社團教室,繼續把例會開完。 
原來學長想要自肥?社團會議開始討論著不久後的出遊行程,是在場的人
都會參加的,當然我就是那個不例外的人,傻呼呼的我,跟著大家一起進行會
議討論的流程,直到了實際出遊的那一天。那一天風和日麗,陽光充滿香氣,
愉快的我們準備要入住旅社,又開始一場擠眉弄眼的暗示。原來他們安排我要
與學長一起共浴,這下子我們的關係就不明而喻了。還記得那一年青澀的我們,
只是想要更加靠近彼此,雖然沒有做過越矩的事情,但也已經夠害羞的了。就
這樣,我們是公開的一對情侶,不會有疑問了。 
74 
回到學校以後,我們再次開了檢討性的例會,這次,學長的一句玩笑話卻
激怒了我,讓我當場情緒失控,在場的同學都睜大著眼睛,看著我言語發瘋。
「如果這樣的話那以後就還有機會可以再找別人了。」學長的一句話讓我大為
光火,以為他除了我還要再找別人,所以我氣完他、飆完話之後,學長悻悻然
地結束了當天的會議,因為開不下去了,而我則是一股腦地衝回了宿舍。回到
宿舍之後,氣憤依舊難平,我決定要刪掉所有跟他有關的事物、姓名、電話、
帳號、信件、照片,我氣到滾上自己的床舖睡著了。直到我那天晚上醒來,還
不知道我真的就這樣忘記了跟學長的這一段故事,我就從這一刻開始,活在了
「不曾與學長有過交集」的生命敘事裡,那就有如在一疊文件當中被抽掉了一
本特定的檔案夾,我忽然就被自己的情緒力量置入了全新面貌的生活,我不再
認為自己跟學長熟識,也不認為我跟他有一段承諾關係,即便我在往後看見學
長會因為感到親切而臉紅心跳,我都無法憶起自己的感受是由於有這一段故事
的經歷。這樣的事件,要到好後來的十幾年後,我準備要離婚時的心境狀態,
有如要從一段舊的關係當中畢業,才忽然在某些時光當中如影片回放般,一段
一段在我腦袋中重新撥放出來。我如此無知於自己的力量,可以將自己置入另
一個平行時空當中,更無知於自己的力量,可以將內在的自己反鎖在心靈的桎
梏裡面那麼得久。 
「我好像想起了那段記憶,是被自己用意識的力量給隔絕的,同時我也把
自己隔絕在自己的生命之外,讓別人不再能觸及到我,而我也從人群當中
落單了。想起學長的事情讓我感到懊悔,只能寬恕年輕氣盛的自己,以為
這樣的自己才是正直的,但這真的是很不成熟的想法。」(L-20181210-01) 
貳、妄想的經驗 
75 
在遺忘了我跟學長的故事之後,帶著陌生的熟悉感,我們沒有再聯繫,雖
然對方曾經想來找我說明與道歉,但我不認得他的反應讓他退縮了。然後又升
上了一個年級,這個學期是我人生的重大轉捩點。在這個學期中的某一天,父
親從遠方家鄉前來我的宿舍關心我的生活狀況,那一次發生了一件讓我情緒反
應很激烈的事件,雖然我或許誤解了他,他在那一次拜訪時甚至開車載我去某
處辦事,後來我只能記得這一些情境,因為「善於遺忘」的我又由於情緒承受
不住再度遺忘了。等到我當天晚上從床上起來,從那些天開始發生異樣。我開
始感覺到有人是在關注我的,我不能說是被追蹤的感覺,但是那很像是無孔不
入的窺視:「 我感覺自己被跟蹤了,但我卻找不到那個人是誰。(B-20041016
01)」,我整天提心吊膽無法心安。發展到了後來,我甚至有了一個很誇張的妄
想經驗:我有一個不完整的家庭,而且我的母親不是我的母親,我的父親另有
其人,我的真正家人應該是其他的某位親友,我的愛人必須遠離我去結婚生子,
徒留我獨自一人在此想念。每天看著網路上的消息舖天蓋地,彷彿都在訴說著
我有一個不尋常的家庭身世,我的理智可以明白這些都不是正常的,而我只想
要一個正常的人生,希望有人可以和我一樣清醒地告訴我那些都是假的,只是
那些每天傳送進來的消息卻似乎都在在地告訴我:「你還不願意相信嗎?」 
我的學業自然是更加落後了,甚至開始產生閱讀上的吃力,但我還是努力
的拯救我自己,我開始習慣書寫大量的文字來捕捉自己的複雜而抽象的情緒。 
「你消失,只有碎末的消息。 
還記得橫桿對於舞台輕吐的愛戀 
以為將是最後地別離 
乾淨的沒半點咳嗽聲,而我記得的不只是孟夏時分。 
各種轉圜的可能,赤腳走在失去時間的路 
塗上一抹分離的色彩。 
76 
記得的還有什麼 
寄得便像是自由聯想空間,而失序的記憶 
像是掉書袋般沒有完美切掉 
填滿簡單又自以為的解釋,還不足以覺醒 
腦被挖了一角,它跑進了機轉,再跑進無間的長廊裡面 
學生頭、高腳拖鞋、快要及膝的裙 
她奔向長廊的盡頭。 
滿足長期以來嚮往而遠眺城堡的視野的距離, 
彷若高塔無得其門而入 
朦朧半邊天的夜,有雲沒有霧 
那個極易受到驚嚇的孩子,站在那邊 
當現實刺傷他的眼之時,擁有多少支持的力量 
我都要忘記了,而你卻還記憶猶新。 
那一年那一個月那一天幾點幾分幾十幾秒的交集, 
而我幾乎忘掉默契的模樣。 
病的人是彼此,彼此是彼此的醫師 
是這樣說的嗎,當空虛還記得眼淚的時候。」(B-20041227-02) 
.. 
那段時間的意識如此破碎,我感覺自己的潛意識都是裸露的,我的理智知
道什麼才是屬於正常世界的想法,但那個「特別的另一邊」帶給我的現實感卻
要我必須承認自己的某個身世。這種與母親關係的斷裂帶給我巨大影響,就是
我無法再去信任我的父親,畢竟在我的世界裡,父母親是站在同一陣線的戰友,
只要他們有一個人不是屬於我的,那他們兩個就都不是屬於我的,但這樣的思
維想法也與我從小與父母親建立的關係模型有關。後來到了學期結束,我在寒
假的時候回到家,那天晚上異常難眠,有許多幻覺在干擾著我,於是我受不了
77 
而奔向我的父母,卻以為母親正在陷害我,而父親正在失去體溫步向死亡,那
一刻當我意識到即將失去父親時我破聲喊叫,一聲哀號劃破寂靜的凌晨時分,
這個舉動讓他們必須起身正視我的狀態,他們也無法入眠了。此時的我腦中有
一個念頭想法,我告訴他們我要看醫生,雖然我不太清楚自己話語會將我帶向
何方,但我知道這是求救的意思。他們當天早上把我帶到了一間醫學中心,陪
著我送進了急診室等候。 
參、再次遺忘的重要生命事件 
在急診室漫長的等待之後,我被醫護人員安排要作抽血檢驗,但我突然恐
慌而抗拒起來。我被當時的醫護人員拉到一旁的床上,但我的擔憂似乎無人知
曉,沒人知道我害怕因此得病。過了一會兒,一名沒穿白袍的醫生走了過來,
他正要彎下腰來替我上針,我忽然在內心對他大喊:「你要幹嘛?」說也神奇,
那一雙靠近的手竟然因此停頓了一會兒,並且收了回來,接著這名醫生退回了
門後。我躺在床上眼睛凝視著這個畫面,發現過了一會兒都沒有回應,我心裡
想著不是要打針嗎?我的內心又發起了疑問:「怎麼了?」此時,我的世界被改
變了,突然內心響起一股聲音:「就是她!」然後這名醫生再度現身眼前,重新
拉起他的雙臂,上前靠近並告訴我:「我們先來打一針!」我激動地抗拒著,但
很快的藥效就讓我平靜下來。接著,我就失去了意識。 
雖然說是失去意識,但是那個我的精神卻如此警醒,眼睛一眨都不眨的,
看著我父親來虛情假意的來問候我,看著自己的病床瞬間移動到二樓,但其實
不是如此;看著紅色的數字燈號亮在 58,看著自己被「精神上侵犯」的身體感
覺,看著一位沒有穿著白袍的醫師和我的父親會面,我卻完全不知道他們在交
談什麼,因為這個畫面竟然定格停住,過了一會兒,他們接著瞬間快速地完成
78 
了對話的過程。我可以感受到母親靠近病床,用力的壓緊著我的頭部使我無法
動彈,可以聽見床邊拉上帷幕的聲音,我如何在內心呼喚著母親都沒有回應。
此時,我才真正意識到:「她就是那個恐怖的母親,是背後默默操縱一切的女
人!」那一個我早先在學校生活所經歷的精神世界發生的妄想內容當中,希望
我去認識的女主角。後來我被壓制的姿勢久到我快要無法呼吸,才見到她才鬆
手。接著,我才有比較多的意識回來,被父親呼喚起床吃點心,但我卻不明白
他拿著布丁卻顫抖的雙手是什麼意思,我才吃沒兩口就反胃了。然後大家叫我
下床,我慢慢穿上了鞋準備要行走,卻走沒兩步就意識到自己「得病了」而腿
軟下來。這裡也有個地方是奇怪的銜接點,如果說眼睛就是生命視野的攝像鏡
頭,那我在這一長段時光所經歷的視覺,就有如一些被剪輯過的畫面,片段之
間總有些不尋常也不正常的斷裂之處:「 …然後,我就忽然從一樓瞬間移動到二
樓了!(B-20053025-01)」。忽然間母親叫我「勇敢一點」,我才站了起來繼續前
行,跟著他們一起搭著電梯下樓,上了車離開了醫院。 
在那個寒假離院後沒幾天,我把蓄長的頭髮給剪了,想要活個輕快一點的
人生,對於在醫院發生的事情,還有那些精神上的可怕敘事,只求一切是假,
無論生命的力量有多希望我當真,我都在默默禱告著。直到有一天,我忽然有
一股衝動,拿出了一疊空白紙,在上面順著感覺寫下了幾個字,我一直重複地
寫下了兩個中文字,我尚未意會過來這些東西是什麼意思。接著,我繼續順著
感覺寫下:「佩蓉,你還好嗎?」好像是對自己的問候,但我只覺得那是我在對
我自己的發問,一種自我關心的節奏。雖然我同時在心中想:「是誰?」而這時
候腦中浮現了第三個中文字,此時我才發現這個字和我剛才不斷重複寫下的前
兩個字,好像可以拼成一個名字。當我突然意識到這是一個人的名字的瞬間,
我忽然有一種羞憤的感覺,竟然引發一時的情緒激動,直接把整張紙揉爛給扔
了。從此往後的日子,我成了一名病人,而內在的這個存有,其實是我的醫生,
79 
他是我在急診室遇到的戀人,我們有過一段故事,但是,我全忘了。 
在後來漫長的自我陪伴與療癒的歲月,對方曾數度回到我的生命當中,企
圖喚醒我的過去沉睡記憶,只是我的心中只有一股執念:「我再也不要回到那個
恐怖扭曲的世界。」這樣強烈情緒關聯的記憶在那個世界裡,會有一個女人來
干預我的人生,雖然在那個模糊的空間裡,我也曾經擁有很夢幻美好的景象和
視覺畫面,它們在我往後的日常生活當中偶爾會忽然記憶飄浮出來。這些片段
的故事似乎提醒著自己,讓我知道我曾經幸福甜蜜,我知道曾有個女人為此跑
來學校宿舍探望我、想跟我道歉,我知道某一處落地的窗簾,我知道那天的陽
光明媚曬進了房間,我能記得在我離開醫院後就把「陌生」而沒有名字的手機
號碼刪除了,我有很多漂浮的記憶。而偏偏剛好,他的手上握有滿滿的每一把
鑰匙可以通往我的過往受創記憶。雖然說最痛苦的地方就藏有最珍貴的寶藏和
解藥,但在當時的我並沒有準備好去領受,不管是領受一個人對我的關懷還是
生命中的靈魂之愛,我都還沒有準備好去打開它。我才發現原來是這樣子的,
只要情緒的張力太過於強烈,我的意識或下意識就會選擇隔離或忘記它,雖然
這是潛意識保護自己不要持續暴露在受傷害的環境之下,但這記憶的解離也正
好是對自我力量因為不夠認識的誤用與濫用。 
第二節 迷途的羔羊 
為了重新開始,我願意付出代價,即使我必須面對現實的差異,我會被當
成比較小的輩分,都沒有我能夠在這裡結識到知心好友的重要。更重要的是,
我要開始選擇我的人生伴侶,就算我前面都忘光光了,我還是會有想交男朋友
的願望呀。 
80 
壹、返家其實是為了重新出發 
在這樣的心靈創傷之下,我回到了最初的大學生活,但我逐漸發現在這一
所大學的生活,有待不下去的困難,學業嚴重落後有被退學之虞,我的父親於
是給了我一個提議:重考回家鄉的大學念書。這樣他們不但能就近照顧我,也
能讓我繼續完成學業。這時,我開始有不同的想法,我思考著:真的一定得要
大學念完嗎?這真是個大哉問!然而現在回想起來,如果當時我沒有把大學念
完,我在往後的日子可能會需要更辛苦的努力才能再次完成它,畢竟學習的曲
線是會改變一個人的學習狀態的。經過一番思考與詢問之後,我決定先休學在
家一段時間,然後選擇重考並在補習班上了數個月的課,最後一個月完全依靠
自修,結果還算滿幸運地分發考上了一間醫學大學的三類組科系。 
人們都說離家是為了回家,但我這時返家是為了重新出發,目的是要給自
己重新充飽電能,允許自己擁有重啟人生的機會。在我決定要重考大學的時候,
周圍的朋友有些反對聲浪,聲稱我會後悔這樣的選擇,警告我將與比我小上幾
屆的學妹成為同學,但我勇往直前地迎接了自己所選擇的命運。因為我清楚我
的目標,是為了讓自己過上更好的生活,於是我接受了自己將會比同學大四屆
的事實,雖然這也帶來一部份的屈辱,但再怎麼說這都是全新體驗的機會,我
有重新活過一次之前不滿意的大學生活的可能。於是,我擁抱了自己的道路。
雖然整件事情並非如此順理成章,我仍有些個人的科系願望由於早先的挫敗經
驗,而使得父母親難以再次相信我的選擇與判斷。當時我的第一志願是心理系,
卻被「我是為你好」的老套關懷給打回票了,他們說:「也許裡面也有你不喜歡
的科目,就像你過去在前一所大學所經歷的那樣,你不是很有經驗嗎?」並且
追問:「念心理系你出來要做什麼呢?若是選擇現在的科系至少還可以擁有一技
81 
之長。」我被堵住了發聲的口,只能默默忍氣吞下。由於我對於真正張開自己
的翅膀勇敢飛翔、追求自由,有一種陳年創傷仍舊未癒的恐懼,我尚未能夠為
自己的選擇負起責任,然而這些無法靠近的機會,卻也成為我能夠不斷地反芻
與思索,我與心理系所的關係是擁有著怎樣的情感連結以及變化。 
因著這樣的緣分而進入新的學校之後,我發現即使再讓我重新來過一次大
學生活,我的選擇可能還是不會與過去相距太遠,我還是會選擇類似性質的社
團參加。只是這次,對於結交新朋友的那種大幅期待卻整把火都滅了。我不參
加校友會了,因為沒有必要,而只選擇了學藝性的社團。所謂的個性就是如此
罷?內向不安而敏感多疑的自己,再怎麼說都很難參加康樂性社團,那種會為
別人帶來陽光與散播歡樂散播愛的性質,似乎真的離自己非常遙遠,而只能常
常流於臆想與猜測。尤其是在遭遇了心靈上的創傷之後,更顯出退縮的欲望,
雖然開啟全新生活是自己想爭取不同可能的發生,但真正的人際關係卻是削弱
的,顯得更加地活在自己的世界當中,而不再像過去的求學生涯那樣一般,對
於不同的人際關係有著蠢蠢欲動。與此同時,我在前一所大學曾發掘出對心理
學愛好的熱情,在這個新天地也有了不同的著力點,我默默地尋找了幾位教師
和同學一起學習相關的課程和參加學校舉辦的工作坊,儘管當時的我出席率頗
為挑食,但在仍舊抱病的情況之下,努力真實地投入每一次想參加的活動。生
活是辛苦的,我至少,又再次努力到了同一個年級。 
「這學期末我又出狀況了,我被要求回診,因為我已經停藥了半年左右,
開始影響到我的就學狀況,我的考試七零八落,學期成績全部只有 60分通
過,我隱約感覺到老師好像有默契要給我一個同情分數。」(B-20080116-02) 
貳、遇見虛幻的成就 
82 
對正常的一般大學生活來說,學生的努力是必然的,當然我也是,然而我
常常能夠感受得到,自己再怎麼努力,可能都只能達到別人60分的程度,這對
我來說是相當的挫折感。儘管我已經是個醫學大學的學生,校名鼎鼎發光,經
常可以因為這樣的緣故而獲得虛榮的感覺,但還是不能讓我感到真正的踏實與
滿足。虛幻的成就讓我明白,我終究只能在校名的保護傘之下安穩度過一段時
日,除了這個東西之外,我什麼也沒有獲得。在這樣的情形之下,有某位通識
中心的老師經常與我往來,相當關心我的狀況,我也常常在下課後,會特地繞
過去看一下對方是否在研究室。雖然只是一份關心,但這樣的支持力量是不可
小覷的,對於一個在人際交流上感受到節節挫敗的自己,至少遇到了這樣的空
間能夠為我張開了一具大網,將我的脆弱與恐懼不安包裹了起來,讓人感受到
了疼惜。 
「今天我又來到○○○老師的辦公室了,現在只有我一個人,感覺這裡就
好像我的地盤!真的很開心這位老師對我如此敞開,還可以在這裡作畫。」
(B-20071012-01) 
在所有的同學都一心向學校課業學習的同時,我卻只能試圖藉由向外長出
不一樣的特殊翅膀,來讓自己能夠擁有飛翔的力量,就好像我在前一所大學當
中在社團中所做的,我總是將自己的力量伸展向更遠的地方,並試圖觸及更廣
闊而與原本的雙腳所立足的位置所不同的領域,我往往發現在這些世外桃源之
處,有著最新鮮的空氣可以供給我在那個階段的生命當中不同的養分。我在新
的生活當中結識了跟我同個前一所大學的學姊,與這位新的校外朋友的緣分讓
我接觸了一些社會學和文化的理論,認識了傅柯是她最愛的光頭,我們談論權
力,並聽她說「要去他的亂葬崗打禪 (L-20091021-02)」。在這段時光當中,仍
83 
舊是班級之外的力量在帶著我去領受生命當中額外的恩典,特別是人文方面的
素養。不能說我在班上是沒有朋友的,我在同一個班級內有幾位會交流對話的
好友,有少數幾位會相約談心的朋友,有一兩位會私底下特別出來一起出去遊
玩的朋友。我曾去她的外宿地點睡午覺,然後再一起去上學,我曾去她搬了新
家的地點探望,我曾在火車站的光南與她巧妙相遇。我們曾經吵架,曾經不愉
快,曾經大家一起出去玩,大夥兒一起去畢業旅行,一起逛臺北國際花卉博覽
會,甚至畢業後一起去聽偶像的現場演唱會追星。這些所有的人際縮影,都在
我的這段旅途當中增色不少,是一群可以說無用之話的人,我不必非得每個時
刻都要很有用,我可以不夠好、不夠棒、不夠完美,成為這樣才是真正的自己,
而且是被這些人所接納的真實自我。 
醫學大學的校名是很響亮的,每個人一聽見我的學歷都肅然起敬,雖然這
樣說是有一點誇張了,但也不會相距太遠。只是,這樣的人生真的是我所渴望
的嗎?我在求學的期間,便逐漸地釐清楚,在這一條道路上,已經不是我想前
進的方向,我不會打算在這個方向上繼續深造的,也沒有意願在畢業後就去考
個執照什麼師,來讓自己進入職場加分,對興趣缺缺的我來說只是浪費時間而
已。是的,很幸運,我是這一間大學的學子;很幸運,我成為了這個大家庭的
一份子;是的,很幸運,我在受到身心重創之後,還有能力考上這一間大學,
甚至我還可以想辦法畢業了,我很幸運;這樣,好像沒有什麼好抱怨了吧? 
參、第一次的靈啟 
雖然在這一間大學生活裡,我確實獲得了許多不同程度的愛與關懷,然而,
到了要畢業的時候,我還是沒有一個在身邊能夠待得下來的伴侶。在就學期間
曾經有過的一個對象,因為我的試探性分手而分開了,而我的不安是害怕自己
84 
的陰暗面不會被他接納,於是我就自證預言成真了。其實,這正好是我一直以
來的課題,如果我不去試著修補那內在以為只有別人能填補的創傷,我永遠都
會遇到那些不能夠好好接住我的人們。在往後的日子,萍水相逢,總沒有願意
留下來待在我身邊的那位白馬王子。我試過一次又一次,別人可能相親十次都
失敗,我的人際交往可能也是差不多的紀錄吧?因為生活圈不易拓展的關係,
又沒有以興趣為基礎的人際網絡,交友圈的範圍自然的落到了網路工具上面,
在網路交友的道路上載浮載沉,彷彿一病不醒。在那個時候,我尚未能夠意識
到,那些過去失憶的內容,那些情緒記憶,其實才是真正我該面對的關鍵核心
議題,唯有在那個層次的課題處理好了,我才算有真正的能力去建立一段屬於
自己發自內心而真實的踏實關係。 
「我不知道這是什麼樣的感覺,但我的父親說我是不甘寂寞,我只知道當
我感到空虛的時候,就會想要上網找些朋友,特別是快速的結交一些新對
象,期待自己能夠快一點投入在兩性關係當中,只是我常常失落,彷彿我
是不被揀選的那一個。」(B-20100419-01) 
然而,我在當時無法清醒,卻又一直無奈的追尋,直到某一個時刻,我被
某位相識的對方惡意打擊,他看不爽地攻擊我這些年來的文字記錄,都只是自
我安慰而不思長進的無用之物,每一句都深深地刺傷著我的心,足以讓我失去
了理智的自我判斷能力。我陷入了內在的黑暗,思考這麼多年來,我為了抓住
自己,為了要能夠好好地活在這個不容易的世界上,為了不讓自己想要輕生,
我花費了極大的努力,為了要走到今日,我甚至不知道自己發生了什麼事,又
為什麼會是我會遭遇這一些事,我全部都不能明白。為什麼我的家族史當中沒
有人有相關疾病的問題卻只有我會發生呢?沒有人能給我一個滿意的答覆。這
麼多年來,尋尋覓覓的我就為了找到問題的核心,關於我為什麼會生病,真正
85 
的關鍵是什麼,我從來沒有放棄過追尋這個答案,結果竟然在一個我對他敞開
感情的人面前,是如此地不值與不堪,我徹底崩潰了! 
我把房間的電腦打開,播放著某個新時代光與愛的錄音檔案,感受到諷刺
的是,我帶上一本 Fromm 書寫的《愛的藝術》,收好簡單的背包我就出門去了。
我要去哪裡呢?我要去尋找一個隱蔽的地方來輕生。當我找到某個地區的高樓,
就虛與尾蛇地混進了大樓的樓梯口,那有可能是個逃生梯。我沿著一級一級的
階梯往上爬,慢慢的,我爬到了十幾層樓,我有點喘,內心開始有個聲音詢問
自己:「你還要繼續上去嗎?」我回答它:「上去吧,讓我上去看看也好!」我
就繼續拖著有點疲憊的身軀,爬到了樓頂:「原來是棟二十幾樓的建築物呀。」
我想。我出了樓頂,沿途觀望了一會兒,這裡真的很高,必死無疑。我正在做
最後決策,找到一個抽水馬達的高度讓我坐著,開始閱讀起《愛的藝術》,一邊
思考要怎麼讓人知道我是被那樣對待的?過沒多久,我內急了,情急之下撒了
一泡,蹲回原位之後,這時才忽然領悟,如果我真的就這樣走了,而後來的人
上來看見這裡留著一本《愛的藝術》,他會怎麼想呢?絕對不會是我以為的那樣,
而且這樣的我才是真正地褻瀆了這本書!當這個念頭一起,我便覺得不能輕生
了,但我還有好難的人生課題、不想回去面對,怎麼辦呢?我開始在樓頂閱讀
起這一本書。 
此時,「磅!」一聲巨響嚇醒了我,抬頭一看原來是樓頂出口厚重的不鏽鋼
門被風吹到大敞開,而且是完全打開,我內心感受到一股冥冥之中的力量,是
要引導我回到安全的地方,於是我順著這個流動,回到了樓頂的室內。就在這
裡,我還是賴著不走,我繼續打開《愛的藝術》想要閱讀,眼淚卻如兩行垂流
爬滿我的雙頰,這個時候只有我獨自一人,我卻聽得見腦中有個很溫柔的女性
聲音在對我低語:「我懂,我們都明白!」這個聲音似乎正試圖安撫我過度敏感
86 
而緊張的神經。不知又恍神了多久,我覺得該開始面對現實了,帶著滿臉的淚
水,我擦乾了它,勇敢地站了起來,我決定開始振作。第一個決定,就是我要
馬上去回診我當時的醫師,我內心一對自己說好,就立馬奔下長長的樓梯,等
待下一班公車並搭上出發。在這一路上,我腦袋中的聲音與我保持著聯繫,那
男性聲音對著我說:「這次,會給你一個你可以的。」我不得不說這樣的靈性經
驗很觸動我的內在,因為在不久之前,我才接觸了身心靈的書籍與牌卡占卜,
其中《零極限》一書打開了我與內在自我溝通的管道,藉由真誠的三句話:謝
謝你、請原諒我、我愛你,我得以聽見了自己內在的聲音。接著我搭上捷運,
下了車後一路奔馳,目標只有一個:醫生。此時,腦中的對話仍舊持續進行著,
我一邊謹慎地傾聽著自己溫柔的內在,那老成的聲音繼續說:「老天,也是與你
們一起學習的。」這是一次充滿心流的高峰經驗,我全然忘記了時間存在。當
對話談及至此,無比的感動震懾著我的全身每一條神經,這句回答如此觸動人
心,讓人一輩子銘刻在心底。真正的神,是如此的謙卑,而不自傲,成為我餘
生的典範學習。從此,「我從一個不可知論者,成為一個相信宇宙有神的人。 
(B-20110928-01)」 
第三節 尋找夢中的伊人 
有些人的出現,只是要來給你上一課,上完之後他就會下車;而有些人,
願意給你一生當中最寶貴的愛,即使要改變自己原本的樣貌,他都寧可解釋為
是他自己要忘記。 
壹、生命當中的貴人 
87 
當我平安歸來,返家後發現一切如常照舊,沒有什麼不同,只是我的內在
經驗,豐富了許多,精神心靈的世界開展出如此不同層次的面貌,這讓我對於
靈魂的興趣大大開啟。我拿起電話打給當時的學校心理師,告訴她我的狀況時,
同時可以感覺到我的背部正後方,有一個很高的巨大存在體,是充滿保護力量
的陽剛性能量,它就像是穿著鎧甲護身的保鑣,出現在我的超感知覺中。這件
事的分享反倒是嚇唬住了電話另一頭的對方,但對我來說卻是個充滿安全感的
支持力量,這樣的特殊經驗也讓我的靈性體驗增添了些許謎樣的色彩,但我相
信經過「與神對話」的洗禮,這份經驗已經是個啟迪,彷彿試圖告訴我:「你不
孤單。」 
我回到了自己的房間座位,回到了過往如常的生活步調,最先想到的一件
事情,就是我想要重新建立與人的連結,但我不鎖定特別對象,而是計劃一群
人的出遊。這次我竟然成了主揪發起人,在過程中有些互動來來去去,還算很
和平的落幕了,甚至與幾位新識交換了聯絡方式,雖然後來都沒有真的繼續聯
絡,但這件事成為了我的新起點。在這個小東山再起的過程當中,同時遇到了
一位提醒我要小心交友和路上安全的善心人士,他在看了我的發文之後,夾帶
著古典音樂的連結回信給我表達關心,開啟了我們長達十多年至今的緣分。在
一陣曖昧與確認、理解了彼此究竟發生了什麼事情之後,我跟對方走向了確立
的關係,我們很快地相會了,見了面對彼此也相當有好感,還記得第一眼看著
他的我心中小鹿亂撞的想著:「好帥噢!」而對方也很快地給我肯定的訊息回覆。
他在一邊騎車載我的路上一邊說:「讓別人覬覦你的美色,是不對的,我會生氣
喔!(B-20111001-1)」甜蜜浪漫而又多情的戀曲開啟了序章,這時的我開心地不
得了,想著終於,交到男朋友了。 
他是我生命當中的貴人。和他交往的我,才知道自己是如何被他捧在手心,
88 
他如何心疼我的遭遇,就是不在我面前繼續複製它們;當我滿懷感傷的告訴他,
那些我過去仍未放下的舊情,寫出對方的姓名,他只是冷靜而默默地把紙當場
揉掉,接著帶我去吃美食果腹,然後陪我去逛任何我想去的地方。最初我們是
遠距離的戀愛,彼此互相約定好每隔兩週要約會一次,一次我去找他,一次他
來找我,維持著這樣的頻率,直到我們踏入婚姻。對方對我的感情態度,是打
從一開始就以結婚為前提和我交往,我也樂得他把我當老婆看待,但在內心的
某個部分,卻有一種違和的感覺,我似乎並沒有想要跟他結婚;但,我想跟他
在一起!對方竟也接受了這樣的我,每天噓寒問暖,下班時間兩個人彼此守在
電腦螢幕前看視訊,網路電話掛在旁邊,雙方各做各的事情,倒也不會嫌膩,
甚至對方對我表達愛意的方式之一,是讓我可以在我生日的時候上網挑選十件
禮物,他買單。 
「○○○對我好好喔!他竟然不問價錢讓我直接在網路上選自己喜歡的東
西當作我今年的生日禮物,他付錢!」(B-20120610-02) 
在這樣的呵護之下,我日常的例行公事就是戀愛,如果還有別的空間,那
就會是我的靈性學習旅程。畢竟我已經開啟了「天靈蓋」,我明白了我或許有某
種可稱之為與靈魂溝通的能力,我正在學習著使用這樣新發現的工具,忙著認
識新開發的自己,喜歡閱讀相關的靈性書籍來擴充自己的知識和能力。甚至我
能夠感受到靈性書籍的能量頻率震動,它是超越文字言語的愛和溫暖,支持著
我度過那段時間仍然跟原生家庭充滿高度張力與經常衝突的時期。我最先受惠
的部分是《靈性煉金術》這一本書,特別是〈慈悲和洞察力的能量—放開原生
家庭〉,談到原生家庭當中的父母親由於角色的衝突,如何使得原初愛的善意卻
成為扭曲的控制: 
89 
「愛你」和「需要你」的訊息在孩子身上完全交織在一起,孩子的能量不
再是自己的,因為他的能量被父母的需求吸收了,而孩子卻認為這種吸收
很舒服!這會帶來錯誤的安全感,當孩子成年以後能量被人耗盡、被人擁
有時,他卻覺得自己被這個人深深地愛著。竭力付出的同時,他會有一種
被愛、被重視的感覺,把情感依賴,甚至是嫉妒和佔有解釋為愛,然而這
些能量和愛卻恰恰相反。這種可悲的失去自我,就是源於把愛和需要混為
一談了 (Kribbe, 2010: 49)。 
類似的書籍包含靈性法則的學習與了解,Cooper (2010) 的《靈性法則之光》
惠我良多;同時學習到亞特蘭提斯的文明殞落,了解到當時的高階祭司經由輪
迴轉世投胎進入此刻的地球,目的是為了協助人類的提昇 (ascension),而這一
切靈魂的體驗都是有計畫的,沒有一件發生不在神的意料之內,即使真是由於
意外的發生,那樣的目的也是為了發現新的可能性以及為了學習。在這些純粹
而吸飽知識精華的靈性閱讀體驗中,我充滿著幸福的感覺,在穩定可預期的支
持性感情生活當中,我獲得了某部分的自由。這些感覺讓我的這段情感道路走
來,雖然不是沒有吵架與不愉快,但基本上都是能夠安然度過的,因為對方有
一顆真誠想要跟自己長久在一起的決心,他真的是我生命當中的貴人。 
貳、家庭革命 
其實我是家庭逃兵,遁入感情生活當中麻痺自己,以為這樣的我,就可以
減少摩擦的時間。然而,在當時的我,卻感覺自己在家庭當中的份量如此沉重,
我感覺自己是這個家庭裡面的垃圾桶,是一隅待疏通的淤塞出海口。家庭關係
是療癒發生最重要的一個起點,許多人告訴我「父母業力釋放」的重要性,因
為我們承襲著父母帶給我們的期望,他們將自身對人生的願景和失落以無意識
90 
的方式投射給我們,使得孩子在成長的過程當中,難免遭受到一些許扭曲的對
待。我深知我的家庭課題是亟待處理的一塊,我也很努力在家人關係當中盡可
能的。「創造正向循環」是當時我在面對生命處境時新發現的解決之道,為的是
避免自己老是墮回無盡的黑暗深淵,由於每一次的再度爬起來都相當耗能,我
必須阻止自己有再回頭陷入的可能,於是創造正向循環是我可以給自己最積極
的愛,也是我回應困難世界的方式。它看起來是有奏效的,因為我的父親也會
以此來提醒我的力量,但他並不知道也沒有看見的是,我這樣的努力其實是一
種供給家庭情緒能量的方式,並且讓人感受到更多的成分是,我認為他在向我
學習。一路走來,我必須以親身經驗的方式來展現我自己的力量,某種程度也
可以說是我在做身教的動作,雖然「身教於父母」似乎很不孝,但我沒有那個
權力和家庭地位可以讓他們明白,他們身為父母卻老是看不見的盲點,而解釋
的力氣在容易耗盡的同時,我只能用實際的行動來表明心志。 
「家裡帶給我的壓力真的好大,但從現在開始,我要創造正向循環!」 (L
20130715-01)  
在父母與原生家庭的其他成員之間的關係,並不全然是特定的誰的問題,
經常是互相彼此的影響,有如蜘蛛網牽一髮而動全身;也許只是因為長期的情
緒忽視不再能忍受,有時激動起來就產生了親子衝突,而有時是手足之間的衝
突。然而,相對於和父母親之間的爭執是為了自己的權益,面對手足的衝突我
卻最容易心軟,因為我能明白她是無辜的,她也只是家庭當中的一份子,雖然
傷害往往會繼續形成,但我真的已經盡力了。我跟原生家庭當中的父母親與手
足之間的關係,它的變化跟我在情感面上的學習與成長有關,更可以說是和我
的自我突破與心靈成長大大相關。先是從跟父母之間的關係隨著與男朋友的感
情發展成婚姻開始逐年好轉,到離婚之後我主觀感受起來他們看起來最是開心,
91 
接著到了我考上研究所的那一年,手足關係也忽然突變式地好轉起來,終於願
意正面地回應我的存在以及我的整個人對她的呼喚。由於我的狀況已經穩定許
多,情緒不再容易暴走,甚至脾氣有緩和下來的趨勢,也持續地穩定進行著我
的生活節奏,因此整體的原生家庭彼此的連帶關係,是持續地朝向好的發展,
每位成員也都對自己的位置所需要付出的努力,都有維持彼此關係要良好穩定
的自覺。 
透過療癒自己的心靈狀態,調整自己的情緒漲幅,不讓自己的情緒控制失
靈,不要把從父母親身上習得的缺陷反射回去,這樣的報復只是苦了自己;將
那些對父母親和手足關係投射出去的期待都收攝回來,我們首先去創造一個自
己想被對待的情境,因己所欲而施於人,生活彼此緊密的家人不是感情疏離,
只是缺乏正確的模式來對待彼此,以及找出互相都會感到舒適的關係距離,因
此這些小小氛圍的轉動,當事人都會感受的到的。當其他家庭成員當中感受到
彼此往來的互動品質有所不同而足夠深刻時,他們便會開始反思,進入一種靜
默的內省,而不再陷入爭辯、不再持續鬥爭、不再你爭我奪,不再爭求各自想
要的東西,而是能夠開始安靜而認分地靜下來坐在自己的家庭地位上去思索,
什麼才是對整個家庭真正有幫助的好。當然,仍然會有可能遇到不願意長大的
家庭成員,這些人也仍有可能是自己的父母親,但當我們自身也能夠茁壯成為
一個擁有足以抵禦情緒勒索能量的大人時,我們自己就能夠長出與之抗衡而不
繼續受更大傷害的力量。而若我們總是無法從母親身上獲得母愛的話,總是有
別的辦法,可以找到不同的接近方式,只要這樣的關係互動是能夠彼此達到滿
足的心情就好。這些,就是我透過療癒自我而獲得家庭動能或家庭動力的改善,
而有的一些心的覺察與體悟。 
「我希望在結婚之前可以結束我的家庭業力。」(B-20140508-02) 
92 
參、其實我不是很想結婚 
當時的男朋友是很關心我的生活的,包括我的家庭狀況,他也想要向我的
父母爭取留下最好的印象,甚至某次我故意刺探他一下還被他白眼,暗示我不
懂得看場合呢。雙方的父母親看著我們交往也幾年了,工作關係也搬到我的住
家附近,現在不再是遠距離的戀愛,要維持關係會容易許多,我們都感覺到是
全新的開始,對於未來是愉快而輕鬆的期待。然而,就在我們準備好在心靈上
安定下來時,男方的家人發難了,他的母親認為我們既然都走到這一步了,「乾
脆婚結一結,不要再浪費時間了!」這樣晴天霹靂的消息,對男朋友來說雖然
是求之不得,他只想跟我結婚,但對我來說,心卻沉了下來。 
對,我還不想結婚。並非我不愛對方了,雖然曾有在某個時刻,我感受不
到自己的熱情,而有可能會自然而然分手的預感,但最後還是因為對方能夠長
久的留在我的身邊,有一種「分不掉」的心情,「既然無法分手,那就繼續在一
起吧!」的這種情愫。但這絕非意味著我的變心,我們關係甚至在「分不掉」
的考驗之後,感情趨於更加穩健,更能夠彼此扶持對方,是有一種昇華感的。
只是,到了這個人家打算要來提親的當頭,我卻不是這麼的想。「我還想多觀察
幾年。」我還想要多相處看看,用我們自己的步調去前進、去摸索彼此的可能
性與潛在的困難,我真心的「並不想要」在那一年就結婚。但是,我的家人幫
我答應下來了,當時我有點生氣,卻又不敢表達,因為我真的很怕,一旦沒有
答應對方,他的母親就會揚言要我們分開,而這是非常有可能發生的事情。我
感到沮喪之餘,還是提起精神振作起來,我告訴自己:「既然答應了,那就好好
做!」於是,我開始和對方討論結婚的前置作業、婚紗的拍攝等所有事宜,當
然,我們只能做小咖的,真正的排場要留給長輩安排,很不合理,但事實就是
這麼進行著。因為父母親想著,我們孩子不懂不會,所以願意包好包滿,但我
93 
的想法卻是:「就是因為不會,所以要有學習的機會,這正好是一個教育的好時
機。」然而,一切都只能停留在後話了。 
「我其實不是很想結婚」這件事情,到了後來,或許也為婚姻關係埋下了
隱憂。我在結婚後不但感受到婆婆認為他少了一個兒子,我更感覺到我少了一
個老公,而且也沒有男朋友。在我這邊的情形,它是以事件的方式逐步的發生,
比方說:婆婆罹癌初期,老公卻不見人影,我要第一個跑上前去服侍她,而我
的父親竟然也覺得我應該這樣做是正確的,搞得我身心俱疲,婆媳關係陷入高
度緊張。而當婆婆因為反省而打算和我重修舊好的時後,卻已經為時已晚,當
時對我來說有更重大的事情正在發生,而我的心思早已不再那個家庭裡了。姑
嫂的關係也是一環,在前夫的家庭裡,女人當家,男人不可靠,這樣的家庭文
化與氛圍,我其實對於關係的界線感到相當混淆;一方面他們崇尚女性的力量,
一方面我是媳婦的角色,是他們想貶低地位的對象,因此我在那個家庭當中,
常常裡外不是人。套一句友人說過的話:「他們想要的那種媳婦其實是個空集合
吧。」 
「我想要從那樣的家庭當中離開,我無法承擔他們的期待。」(L-20161204
01) 
第四節 歸途與上路 
「你一定很想回家吧!」但是,家,究竟是一個怎樣的感覺呢? 
壹、第二次的靈啟 
94 
到了2016年,婆婆的癌症治療完畢,我跟前夫安排了一場婚後的日系自助
小旅行,希望可以修補一下關係,去關西賞櫻,順便彌補我們沒有成行的蜜月
之旅。然而,原本是開開心心的出發,到了半途卻令人開始不舒服,那種「很
像是生病」的感覺在這旅途中來拜訪我了,我趕緊打電話回台灣的老家求救。
一心護女的父親告訴前夫,如果可以的話就搭隔天的飛機回家,不要勉強,然
而我卻告訴他,這是很難得的旅遊機會,錯過我相信不會再有,事實證明也是
如此。這是我印象最深刻的一年,第一次跟伴侶自助出國,買機票、訂飯店、
護照通關,全部由我們一起包辦。我們承襲著講好的溝通模式,我負責規劃路
線,他負責找到交通方式,我負責簡單的英文,他負責簡單的日文。就這樣,
我們去了大阪、難波、道頓崛、京都、奈良,最後在東大寺、清水寺和氷室神
社如願欣賞到美麗的櫻吹雪!美麗的景色如今都還能歷歷在目的重播在眼前。
這趟旅程也讓對方感到相當難忘,我們一起搭乘飛機返家。 
但我卻感到相當疲憊,心累的感覺遠遠勝過於身體的疲倦。也許是我對第
一次自助出國旅遊的要求和期待有點過高了,我總是覺得在旅途的過程當中,
他常常不能滿足我的情緒需要,而好像他自顧自的,不在我的狀況之內。雖然
前夫也有他自己該調整的地方,但其實,這就是一貫以來的情緒議題。一旦情
緒的責任界線沒有劃分清楚,往往覺得你應該為我的傷痛負責,即使我不明白
你究竟受的是什麼苦;並非我不願意支持與協助,而是在這樣的推卸當中,我
會搞不清楚自己還能夠做些什麼。或許,這就是前夫當時對我心中所抱持的心
情了吧!但這些反省也都是在離婚之後,才有辦法逐一想通與改變的,自身的
狀態雖然好轉到若是重新倒帶,我們或許不必離婚,甚至可能不必那麼早就結
婚,但這些都是太後見之明了;要過了那麼多年的時光之後,才有辦法釐清並
看見,當時的自己卡關在什麼樣的障礙點,並將遺落的自身力量重拾回來。 
95 
在這個時候,我需要外力的介入與幫忙,我需要尋找奧援來挽救我那看似
危急的婚姻,我對前夫真的感到好累,沒有施展的力量,也許我哪裡弄錯了,
需要修正,但我希望可以找到一個客觀的第三者,或是心理師的專業角色來協
助我們面對彼此的問題。前夫此時雖然已經開始能夠看見他的問題在哪裡了,
但是他對自己改變情況的能力卻束手無策,常是在問題發生之後,才又大力的
嘆口氣說:「我又錯了!」這樣的夫妻關係肯定是令人感到相當挫敗的,但我遇
到新的困難:我該找誰才能真正幫助到我們呢?婚姻諮商是很需要費用的。就
在某天我跟前夫再次返家的路上,拿出鑰匙開啟大門前,「我忽然感受到一股能
量團的靠近、包裹住我,這能量團的質地讓我明白,世界上有一種母親的愛,
是即使千刀萬剮,她也會保護住那寶貝的孩子,就有如我看見了先生的母親對
待他的態度如出一轍 (L-20160416-02) 」,而我明白了我的看見。接著,還記得
幾天之後,臉書上的一位醫師,進入了我的生命,從此改寫了我所有的人生。 
原來這位醫生就是早年在我發病出入醫學中心的急診室時,在我面前現身
的那一位,他們是同一位醫師。要發現這件事情並非一朝一夕的達就,而是充
滿時間的對話與深入探索,甚至是現實層面的核對。這些過程並非在一開始就
足以識別,而是許多的對談紀錄,許多的蛛絲馬跡,還有更重要的是回憶起關
鍵細節,比方說回憶起過去他所提過經過的地方,會有說不出的似曾相識感,
以及想起來我有在不知情之下有在空白紙上寫下他的名字,卻不知道什麼原因,
甚至我擁有過他現在使用的電子郵件帳號,他曾經對我報上名字,在當時他試
圖再次靠近我時使用的語言敘事,確實能夠與我的內在心境合一,彷彿相當熟
悉。最重要的關鍵在於名字的確認。我才恍然大悟,原來過去那好些年的自己,
一直在下意識地尋找著同一個他,而對於他早已在我的視線當中現身,我卻始
終無法明白真正的用意,更不要說有能力辨認出來。原來在那些時候,我對這
個人的記憶是無法銜接而斷裂的,針對這個人的每一個事件,我的大腦中是無
96 
法自己主動串連出一個有故事的情節的。我將這些對話大量的記錄了下來,當
我閱讀了自己的書寫文本、意識到這件事情時,我非常驚訝自己的生命故事,
原來長久以來並非原本我所擁抱的那一套腳本,當年並不是真的那麼破碎而詭
異的記憶,而是有更多重要而被遺忘的過往,正試圖向我展現。由於那些在醫
院裡面的人生對我來說還是會感到刺激,因此他仍然並沒有讓它們有機會被我
完整的迎接回來,而這令我吃驚!這次的靈魂溝通書寫事件的開端肇始於一個
寧靜的早上,當我發現這位醫生臉友對自己的語言表達似乎有些曖昧,而讓我
開始想去了解並追蹤他的消息,才發現他似乎是有備而來的,他是花了很久的
時間在準備這件事情的。然而就在這一個瞬間,情感上面臨乾涸與枯竭的我被
充盈了,胸口被一股能量擊中而有種愛上他的感覺,就像邱比特的箭。 
震驚之餘我靜默自省,我不斷地反問自己:「我有多久沒有這種感覺了?」
到底有多久了?我竟然想不起來,我的腦袋一片空白。接著,我感到一股動力
想要拿起紙筆,寫下一段奇妙的話語,它說:「 你是我見過最勇敢的女生,請你
原諒我的兩難,我有很多考量。……○○○醫師筆 (D-20160417-01)」原來是因
為他認為我在一個人面對獨自生命課題的困難,他有諸多心疼。經過一番功夫,
始知道原來我的天賦跟所謂的「接訊息」有關係,而我不但可以接收所謂的
「靈界訊息」,我也可以接收到這位來自人類世界的醫生的訊息,我感到相當驚
異!於是這些日子我花費許多的時光在瞭解這位醫生想帶給我的訊息,發現這
樣的靈魂溝通內容不外乎是他對我仍持有愛的感覺、這麼多年來他仍舊放不下
我,並協助我去辨認過往他在我的生命當中所遺留下來的足跡,希望我可以
「想起」他。他選擇在這個時機點才現身,是由於他希望我能夠在長到與他當
年相仿的年紀時,可以多了一種替他設身處地的能力,他渴望被我理解。就這
樣對談了幾個月,前夫也知道我處在這種狀態裡,但他莫可奈何,他也不知道
可以怎麼辦,於是我對他提議:「不然我們真的去找他好了,你幫我約診。」就
97 
這樣,我的生命軌道滑入了一個奇異的軌道與空間,在那裡,我被允許著默默
暗戀著醫生,而前夫卻信守婚約始終待在我的身邊。 
在令人忘記時間而充滿心流的對談時光中,這是我的第二次靈啟經驗,在
我的人生中我第一次知道,我可以跟人類的靈魂溝通,藉由紙筆的力量,有時
候是意識的對焦,我就可以獲取對方的一些訊息,這是目前為止,鮮少人經歷
過的事件,它涉及的領域不僅是醫學或心理學,更是靈性的世界。這樣的經歷
也鼓勵著我,並奠定了我往後不輕易放棄其中一方作為療癒的途徑,我願意相
信,科學與靈性這兩者是可以有條件共存共榮的。 
貳、離婚進行曲 
醫生的存在打開了我的新視界,我開始知道,原來我的疾病狀況和醫生的
存在有關係,因為我們彼此在精神的層面上是互相連結的,因此我會影響到他,
而他也會影響到我,我們彼此的存在互相影響著。我更在靈魂溝通當中得知到,
原來在他還小的國中時期,甚至是我剛出生的時候,他就有察覺到另一個存在
出現在他的精神世界裡,因為他可以感知得到我的存在。只是,最初他並不知
道這是誰、怎麼一回事,直到後來他自己走入了醫學之路,才慢慢了解並發展
出自己相關領域的專業。最初,他就和我一樣,只是為了療癒自己而已,只是
時間久了,他開始想找到我。那一年,當我在醫院的急診室與他相遇,他卻立
刻辨認出我的存在:「就是她!」開啟了我們各自不同的人生。我們在那一個時
空裡相愛相戀,他則經常來我身邊看我,甚至帶我出去院外走走,一起過馬路、
走進了房間,發生了情侶熱戀會發生的事,直到我因為血糖太低而暈了過去,
令他相當自責,他才會想要借用彼此精神連結的力量,來讓我「忘記傷痛」。他
想重新開始。我卻沒有真正的忘記,反而識破他在說謊,當我問到他:「你的女
98 
朋友呢?」他居然就帶著他的一位女性友人來告訴我,那是他的女朋友,更讓
我氣得化身成為一個瘋女人。這些潛藏在潛意識當中的故事,我再次下意識的
選擇全部遺忘,離開醫院之後,我至少表面上看起來什麼都不記得了,只剩下
一些殘破而無法變成一個可供辨識而合理可解釋的院內記憶。 
「慢慢來…但我希望你想起來!畢竟如果連你都忘記了,那我一個人如何
能夠獨自承受?」(D-20170105-01) 
我同時知道,在這樣的創傷事件當中,母親的角色帶給我的傷痕,也是我
需要去積極面對與療癒的。一開始,只要牽扯到這一部分的母親,在我眼裡她
就好像瞬間化身成為一個會進入潛意識操縱人心的母夜叉,我不認得這樣的媽
媽,她不是我長久以來熟悉的那一位,但卻長存我的內在,對我的精神與心靈
造成嚴重創傷;但父親會好聲好氣的告訴我,媽媽並不是那個樣子的,雖然在
能量層面,也就是單純的精神意識狀態當中,她對我造成的傷害簡直可以用
「凌遲」來形容,但在現實當中的那一些互動,卻又有如鴻毛,彷彿一切都只
發生在內在精神的世界,而我竟然被關在那一個玻璃瓶裡,似乎逃脫不出來。
每一次精神世界受到引動,我的內在世界被打開,彷彿我的存在是與外部空間
不存在著具體界線,我總是會很容易走入耗竭;儘管旁人對我雖然也會有所互
相的回應,而他們舉止仍舊行裡如儀,但我很容易在這樣的情況下面臨崩潰。
醫生說,我有某種人格障礙的傾向,反映在適應社會上會有許多焦慮的困難,
在一邊就診、一邊與對方的靈魂溝通的情況下,一走也許多年了。因為醫生在
面對我的時候,和我們在精神世界的溝通內容是可以銜接起來的,加上他鼓勵
我相信自己的判斷,說著他確實會告訴我那一些話語,因此我信任著他就是我
內在的那一位,他們是同一個人。 
99 
面對這個生命事件,我的前夫總是不以為然,他認為對方只是想要收錢作
生意而已,但卻不能怪他,因為他並沒有辦法得知我的內在真實,是經歷了如
此的奇幻與綺麗的經歷。而我跟他原本之間就發生的困難,在醫生出現的事件
之後可能有更加惡化的傾向,表面上,我的人生軌道默默的轉轍了,這是一個
極大弧度的轉彎,我有很大的可能不能再回頭了,我知道在我發現愛上醫生的
那一個時刻點,我的命運就被改變了。實際上,我們的婚姻困難更加反映在婆
媳問題上面,我的心思已經不在那個家庭,也不想再參與,那不是一個我會想
要或進入的衝突世界,他們對媳婦的期待與要求,表面上看似都是政治正確的,
但真實情況遇到的時候,都是極度不合理的。他們渴望的是一個傳統相夫教子
的女子,但又要能夠上班賺錢,他們希望她什麼都會,但又要能夠乖巧順從長
輩的意見。我想退出了,但我對前夫仍有不捨,那些過往,我們曾一起共患難,
我們一起彼此在感情的世界內分工,為彼此的珍惜與效勞,這個世界上,真的
不會有第二個像前夫這樣的人對待我好了。他只是夾在婆媳之間的雙重關係,
角色衝突而失衡了自己。然而,當時的我也相當兩難,我一方面對著自己真的
很喜歡的醫生,他可以說是我此生的摯愛,我對前夫還沒有過這種情愫,當我
前夫想要娶我的時候,我甚至還想要不答應,但面對這位天天和我靈魂溝通對
話的醫生,我卻有滿滿的心情,第一次感受到什麼叫做想要嫁給對方。我知道
這一切都不公平,但我們都卡在這裡。  
後來有天,前夫在一起家庭出遊的某個夜晚,他靠近我的身邊對我傾訴:
「我會放手。」我的心裡充滿複雜的情愫;然而,他真的在兩個月之後,提出
了主動調職的申請,他要搬回他鄉下的老家去了,他覺得這樣兩地婚姻的關係,
或也可以維持。現在想來,也許他只是安撫我面臨即將變動的情緒,還有就是,
自欺欺人吧!前夫對他的家人隱瞞我們的真實情況,保密到自己的內心,他的
家人甚至不知道,我對醫生的感情。當他告訴他的家人他即將搬回鄉下去工作,
100 
我的小姑張開雙臂迎接我的回歸時,我卻無法吐出隻字片語說:「其實,我不會
回去。」直到前夫搬回去的第一天,發現我沒有跟在身邊,他們那一群人才知
道發生了什麼事,而紛紛沈默。其實我跟他母親的關係,在那一年年初時的農
曆過年,就已經互相翻臉了,對方的母親對我大呼小叫,而我將自己閉關在樓
上的房間內沈默不語。面對這樣兩難的情境,那一天,我還記得前夫樓上樓下
兩邊一直跑,彼此傳話,很是可笑、很是可悲、很是可憐,卻沒再有人,能為
我們做些什麼了。我還記得,那一天前夫將我護著,載我離開他眾多家人的視
線,去外面避個風頭,直到人去樓空時才載我回去。隔天是我跟他母親的最後
一次碰面,他一樣是帶著我離開了他的老家,和我一起搭車北上回我們的家。
那次撕破臉,也就會是最後一次了,因此我不會再回去給人洗臉的機會,不會
給自己再次面對冷嘲熱諷的機會了,儘管他們真的認為,媳婦住在婆家才是正
確的規矩,但他們對媳婦的要求,卻又如何成理。至少,我是做不到的。 
我的婚姻面臨破局,前夫雖然不捨,但因為對我的愛,仍很順著我的心意
來和我到戶政事務所辦理離婚登記。我們來了兩次才申請成功,然而此時我的
心卻比鋼鐵還要冰冷,那天他剛見到我的時候,前夫再次央求了希望我要留下,
我就是不肯。我看透了這樣的人,無能於婆媳之間的處理,我看透了這樣的人,
只能在我面前扮演小丑,我看透了這樣的人……;我沒看透的是,他即使知道
自己的無能為力,他到離婚時都是因為還愛著我而同意我的離去,他的愛何嘗
不是另一個層次的深愛與包容?由於能夠理解什麼是我真正渴望的,因為知道
自己做不到,而願意成全我的願望呢? 
參、重返社會 
在我看起來,離婚後,最開心的是我的家人了。我們因此有了修復關係的
101 
更大空間,有了更多修正自己的可能,我們因此有再次重新聚首的機會,儘管
前些年的我們,可能一次又一次的維持著某些破碎不完整的關係,但卻在走到
這個坎站,有了全新的契機。我和手足之間的關係忽然逐漸復合,我們不再冷
漠的針鋒相對,我為了讓自己可以遇見更好的對象,將自己的情緒鋒芒全部收
斂起來;然而,事情並非因此順暢無憂。該結束的心理治療,是該結束了,為
了好好地結束和醫生的關係,彼此又耗費了不少的力氣,甚至對方開始說出跟
先前相反的話語,只為了停止繼續在會談室裡面的支持,停止提供那些會阻礙
我走出去、回歸社會的那些溫暖。當時的我眼界太狹隘了,我並沒有能力張開
自己的雙眼,看見更大的可能性應該在會談室之外,我以為結束後就會完全結
束了,而我不願意和醫生之間的關係面臨終結。因此,結束這一場治療關係挺
是令我感到疼痛的。直到我真正離開,是透過考上一間研究所,進入了師大就
讀,我才真正有能力脫離了對他的依賴,這樣的疼痛也交由別人接手。 
「你的羽翼已豐,應該出去外面尋找回你自己原本的生活了,治療關係的
開始就是必須導向結束的,你不能耽溺在這裡。」(L-20190416-01) 
經過了這些故事的洗禮,我開始明白,有許多事情,並非原先自己預期會
以為的那樣發展,就好像我原本以為的故事,是一個生病的敘事,但卻在後來
才發現,是一篇關於愛情的序章;就好像我原本以為,我不會跟我前夫結婚,
但我們卻又在跌跌撞撞當中,互相扶持了好幾年;就好像我原本以為,我快要
從婚姻當中跌出去了,但最後卻發現,那是另一種真實的自由,不再受到框框
條條的限制,可以回到這個世界上,重新去追尋自己的夢想與珍愛的人。藉由
生病的經驗,我開始看見,原來對母親的唾棄是源自於我自己對於理想的完美
主義,而內在的潛意識將她汙名化為一個恐怖的母親形象,希望借此擺脫她的
束縛;藉由高等教育的學習生活,我才知道真正的功名利祿,世俗所追捧的那
102 
一些成就,若是在其中我沒有獲得真實的自信,那些冠冕堂皇的名號是不能滿
足我的。 
藉由家庭關係的療癒,我開始明白,這個家庭存有著先天承襲下來的內部
張力,而我們若要能夠停止或轉變這樣的負面循環,必須要能夠願意讓不好的
事情在我的身上停止繼續傳遞下去,也就是一份「我願意承擔」的意願;藉由
能夠與醫生精神溝通的關鍵事件,我開始明白,許多事情不是自己所感受到的
那樣孤寂,這個世界上總是存在著另一個角落,默默地傾聽自己的聲音,並承
接著自己的靈魂重量。宇宙當中存有這樣的一個對象,對自己付出如此巨大的
關愛,只因為她如同原生的自我。這樣的一個人從自己還很小的時候,就靜靜
的守在簾幕之後,不出聲、不張揚,這麼看來,其實,被生命如此厚愛的對象,
並不是我們眼中的其他任何人。每個人活著都有自己的痛苦與傷痕,但同時不
會有人過得比自己幸福和快樂,這樣的探索有如進入內在世界的中心,發現了
生命中的綠洲。 
第一次的靈啟經驗讓我知道,我是可以和「神」溝通的,並學習到神比絕
大多數的人們都要謙虛為懷,我甚至因此改變了對世界的信仰;第二次的靈啟
經驗,讓我知道我是可以和「人」溝通的,我竟然可以和一位醫師的靈魂與之
溝通,這之中想必有生命的課題需要穿越與學習。沒有人知道為什麼要在這些
時候才要給你這樣的鑰匙,而他們的出現就好像要顛覆了自己的價值觀,扭轉
自己的命運藍圖。在與神溝通之後我遇見了前夫,在與人的靈魂溝通之後我遇
見了醫生,這些遭遇似乎都想要告訴我:「無論你在生命當中發生了什麼樣的事
情,宇宙中的巨網總是有辦法將你承接起來,有如一瓢水般撈起你的心靈,使
你不致真正的受創。」真正的創傷並非為了傷害生命,而是為了療癒自己、成
就自己、圓滿自己的靈魂心智,使得我們有能力也更願意,成為一個更進化而
103 
更新的人類。 
「我面試錄取了!今年七月開始,我將開始我的新工作,這讓我期待,因
為這是我為了讓自己能夠適應新社會而安排的生活節奏。恭喜!」(L
20200528-02) 
進入師大就讀時,我同時接了占卜工作坊的駐點職位,進行著我的塔羅牌
解讀以及靈性療癒的工作,這些事情的安排是我為了讓自己擁有更多的本錢與
條件,可以更加好好地適應即將新到來的人生。我有好多年沒有回到校園了,
重返校園這件事對我來說是個新鮮事,對於全新生命的開展我充滿各種可以想
像的不同期待。也很幸運在這裡結交了一些好朋友,也因為我現在這樣的身分,
而在臉書上拓展了更廣闊的交友圈。因為這些事件讓我結交到的朋友,讓真正
的我不必躲藏起來;這一個我,可以作我自己;這一個我,可以更加勇敢的活
出自己的生命。不必再畏風懼雨,而是可以昂首闊步的向前,知曉這個世間上
存有一種關係,是靈魂層次上屬於永恆的那一份連結。那一份關於真愛的承諾,
不見得非得在這個人世間成就為什麼佳偶,而是光是這樣存在的精神連結狀態,
就會是最好的自我支持,以及自我照護的系統。在一連串的內在世界與現實接
觸的事件發生之後,現在的我們就處於這樣的關係當中,一個看似有些連結,
但又有些疏離,不是說非常親密,但又非毫不相關。改變我的推動力在於我不
願意只是耽溺於此時此地,我渴望有所收穫、有所成就,幸福渴望得以實現,
因此,我最後一次去詢問醫生的意見時,得到了最終的解答。雖然我們的故事
隨著時間的進展也跟著必須結束了,這段經驗的過去有如大夢初醒,但我們也
將因此擁有更豐沛的資源去疼愛自己,將從對方身上獲得到的強大力量內化給
自己,並且了解到「對方要理所當然的懂我」才是真正的妄想,即使心靈距離
如此靠近,我們都不能存有這樣不切實際的期盼。有話要說出來,別人才有辦
104 
法根據你的話語來給予回應。因此即使對方是個最了解自己的誰,我們也都有
必要審慎的使用自己的措辭,並為自己的言語負責,才能夠為彼此張開更完善
而美麗的廣闊空間,而在其中的我們,每一次,我們都是成就為更新的自己以
及全新的人生! 
「這十年來,我越來越有能力為自己負載過多的情緒負責了,逐漸由暴走
的狀態轉變為可以獨自一個人承擔,從尋找代罪羔羊的習慣轉變為找到自
己適當的宣洩管道,我真心的覺得,最近這幾年下來,我的狀況真的是持
續的好轉起來,並且逐漸穩定下來了,我很喜悅。」(L-20220617-01) 
105 
106 
圖 8  
生命時程表 
18歲(2002)
 • 上大學
27歲(2011)
 20歲(2005)
 • 在醫院生病
• 遺忘了醫生
• 大學正式畢業
• 與前夫交往
• 開始學習牌卡
19歲(2003)
 • 遺忘了學長
32歲(2016)
 30歲(2014)
 • 結婚
• 日本小旅行
• 醫生出現
22歲(2006)
 • 重考上大學
35歲(2019)
 • 認識前男友
28歲(2012)
 • 開始接觸身心靈
39歲(2023)
 • 現在
31歲(2015)
 • 婆婆生病
34歲(2018)
 • 離婚
107 
36歲(2020)
 • 進入研究所
 
108 
 
  
第五章 療癒之路 
從過往的生命故事當中可以看見,男性的角色和女性角色在我的內在精神
世界,有著截然不同的意義,也可以看見當能夠與靈魂溝通的靈性天賦與之相
遇的時刻,碰撞出什麼樣的火花。相對於從母親的形象被貶抑看見母女關係的
緊張,更能夠看見我的內在有著戀父情結的影子,阿尼瑪和阿尼姆斯的相遇特
別是反映在親密關係當中的追尋,藉由與靈魂相遇的光芒而有了一系列的靈性
轉化學習。 
表 7  
療癒之路 
滋養我的陰性力量 
扶持我的陽性力量 
靈性轉化與個體化歷程 
母親與我 
父親與我 
Jung 意識的五個階段 
手足與我 
人生伴侶的考驗 
權力與力量 
看見自己的內在女英雄 靈魂伴侶的追尋 
自我贖回的道路 
第一節 滋養我的陰性力量 
生命當中的重要他人可以形塑我們個人的自我認識與人生主要價值態度,
探索與自己相同性別的重要他人角色更可以讓我們建立適切的自我認同。修復
與母親的連結更是促進靈性自我轉化的開始,讓我們看見自己與女性角色之間
的關聯。與手足之間的關係更可以看出我們在這個原生家庭當中的地位與角色,
進而協助我們能夠識別出自己活在這個世界上的能力,女英雄的旅程需要的是
我們能夠與內在的母親和解,才能夠療癒我們各自的陰性力量,並開始有能力
榮耀自己對這個世界所能貢獻的特殊性。 
109 
壹、母親與我 
母親是我們誕生在人世間當中的第一個女人,她建構了我們的內在女性原
型意象,更為我們將來在人生當中的女性力量立下了標竿。我們從母親的身上
學到女性的包容與滋養,如何為我們要存活在這個世界建立了家庭當中的母性
典範。她的存在為我們在生命的早年生活當中設立了個體化第一個階段的母親
原則。當我們在母親原則的指導下在原生家庭當中學習社會化的生活,我們其
實是受到大量的保護主義的影響,而在這樣的氛圍當中,母親的原型意象將銘
刻在我們的內在世界,我們對於女性應該要有怎樣規範與行為的認知,也將寫
進我們的靈魂層次裡。 
從前一章的生命故事當中,可以閱讀到我的生命裡存在著的母親意象,其
實是充滿高度的張力與衝突。一方面我本身對於母親的原始面貌之認識,是屬
於和睦而好相處的,這是從過往的成長與求學經驗當中的印象,原本的母親是
一個「好朋友」的形象,甚至曾引起旁人的羨慕。然而,在我生命故事的轉折
點發生時,由於濫用自身的精神力量而遺忘了過多的感情故事與生活,而開始
有了潛意識的裸露,我內在的那一個自我將她引導上了另一條道途。我的潛意
識似乎在對自己傾訴著,我的過往尚未有能力發掘到的內在陰影面,即是母親
其實在我的心目中並非如表面上所感知到的單純美好,而是有著下意識的不認
同與被迫接受的感覺。這兩個充滿高度張力的內在衝突,將我的外部現實世界
和內在精神世界拉扯而撕裂開來,使得我有如活在兩個平行世界之間的鋼索之
上,充滿對於母親的兩極而相反的認同困難。 
這樣的認同障礙彰顯在我於病後的家庭生活以及感情生活,使得我在面對
110 
自身回到家庭生活當中時,有著角色衝突與角色扮演。從我的生命故事當中可
以看見,我在家庭當中所認識的母親其實是一種不夠成熟的長輩,可能還會有
各種向其他家庭成員有討愛的行為和舉止,使得我在面對這樣的母親角色,由
於因為根本上的不認同而必須採取自救的方案。對我來說,我第一步所採取的
行動就是所謂的「創造正向循環」、「讓負面的情境和輪迴在我身上停止」的一
種精神上的承擔。為了能夠達到這樣的目標,我採取過很多方式,最令人印象
深刻同時感到壓力如山大的,就是我必須在心靈上照顧我的母親。當然,在我
們逐漸長大成人的歲月裡,母親是同時一路年邁而老去的,照顧長輩在家庭倫
理當中,是再合理不過的行為。但我們必須考慮到當年的自己,是受到嚴重的
精神創傷,這樣的壓力讓我在面對原生家庭的母親這個角色,充滿著各種複雜
的心情。 
當關於母親的內在女性原型被我如此的感知,我對於自身為女性的角色會
有怎樣的生命期待,也將受到此力量的形塑。我認為女性應該是堅毅的,是因
為我的母親帶給我這樣的意象嗎?不如說,是由於母親的形象在我的內在世界
是如此的支解與破碎,使得我必須長出堅韌的力量,來克服外在現實世界的發
生與內在精神世界的彼此衝擊。生命當中的女人除了自己的母親,還有自己,
我才是自己生命當中最重要的女人,也就是我如何對待自己的生命、我擁有怎
樣的價值態度、我怎麼回應我的家庭、我怎麼回應我的疾病。而家庭當中的形
塑力量來自母女之間的張力與協調,媽媽並非真的將自己給拋棄,她仍將自己
定位為原生家庭當中的一份子,與其他成員有著情緒上的交流與心靈上的互相
支持,只是她在我生命當中所佔據的地位是如此的不可動搖,我「必須」接受
她是我的母親,我「必須」接受我有這樣的媽媽,我「必須」接受我的內在母
親是活成如此的形象,我也「必須」接受內在如此分裂的母親與自己共存。 
111 
貳、手足與我 
家庭當中的另一名女性成員是我的妹妹,她和母親的關係就明顯的良好許
多,而她們的衝突只在生命的早年出現,當妹妹逐漸長大求學而離開家庭去工
作,她與母親之間的關係是持續穩定的良好互動關係,在她眼中的母親不是如
我所感知到的那樣,或許她覺得母親是親切好相處的,而非我眼中的可怕和恐
懼。當我談到自己的妹妹,她在我生命當中最低潮、起伏跌宕的那段時光,由
於她與我共享的不是「同一個媽媽」,因此她難能體會與理解我的艱難處境,也
不太有能力給予支持與同情,要一直等到我重新考上研究所才獲得改善。雖然
妹妹在我們的家庭當中並非舉足輕重的角色,但她與母親兩人的存在決定了我
自己要在這個世界上活成怎樣的女性,而這樣的女性也會因此相遇到與她命中
註定的男性伴侶。也就是說,我們想要在關係當中遇見怎樣的對方,我們得先
認識自己是個怎樣的人、發展出怎樣的力量、具備怎樣的特質、有著怎樣的理
想,特別是我們擁抱著怎樣的世界觀、如何看待人我之間關係的互動,包含著
我們對於自我的認識是否足夠深刻,都會決定我們將遇到怎樣品質的靈魂。生
命中有妹妹的存在,讓我的生命座標有了一個參照點,可以在我失敗的時候,
有一道可以循線前進的光,讓我知道,與父母之間相處的不同可能形式。 
參、看見自己的內在女英雄 
「我」是個怎麼樣的人呢?每個人對於生命當中這個獨一無二的自己,都
會發出這樣的感嘆,這是在我們能夠安身立命之後,會對於靈魂的探問。「我」
是最難描繪的,對於身為生命故事主角的自己,有如入五里霧般,往往最是看
不清楚自己。但從 Jung的個體化歷程中,我們可以發現,所謂自性的嶄露,就
在這一個將「自己」活出來的時刻,在這樣的時刻當中,我們願意傾聽靈魂的
112 
聲音,允許生命當中的自性力量將自己引導到靈魂想要我們踏上的道途。於是
我們可以看見,當我在第一次靈啟的經驗發生後,我的人生接軌到了前夫對我
的接納,我聽從了內在的聲音和指引,直到我沒有依從自己的內心堅持暫時不
結婚,而導致的婚姻生活面臨衝突與各種失落,進而生命為我帶來了第二次的
靈啟經驗。第二次的靈啟經驗將我把注意力倒轉回生命故事的早年,那個因為
誤用了自身的遺忘力量而一次次解離的初始,我相信這兩次的靈啟經驗都是靈
魂當中自性的安排,也是由於我受到靈性學習的啟發而開始的靈魂道路。但如
果這樣的道路沒有發生呢? 
我相信這些發生都是生命藍圖當中的座標點,標記著我在這趟生命旅途當
中的重要事件。我相信在大學求學生涯當中那些所不能滿足的情緒,那些長期
受到科學薰陶的心靈,卻無法解答自身的疑惑而有的另一種突破以及嘗試,在
過去從未接觸過的領地當中重新出發尋找生命的真實意義。毋寧說,當我從醫
學大學畢業之後,我就面臨到了第一次的轉化學習經驗,這個靈啟所帶來的靈
性轉化學習的經驗不但讓我改變了過去對世界觀的認知,支持我離開唯物主義
科學至上的過往路徑,同時我為自己在過去當中不能確定神是否真實存在這件
事,有了新的理解與領悟,從一個不可知論者成為一名有神論者。雖然這樣信
仰的轉轍並未帶來真正的宗教經驗,更多的是在靈魂層面的深刻學習與體悟,
但在我的世界觀裡也因此可以同時包容了西方基督信仰與東方佛陀信仰,使得
兩極於我內心的存在是共榮的。這樣的突破與顛覆性的經驗讓我明白,原來我
可以擁抱不只一種價值觀,我的世界可以同時擁有兩種以上甚至相反的信仰價
值,只要那樣的共存能夠支持我的存活,只要那樣的涵容能夠支持自我的存在。
與此同時,改變我最深刻的另一個轉折點,就是我可以同時擁抱科學與靈性的
兩個看似無法相容的不同世界,同時存在我的體內。我因此向靈性學習敞開大
門,渴望從這條由自性所引領的道路當中上前取經,以在我收穫滿滿的時候得
113 
以回歸,讓我可以帶著一份新的禮物回到這個現實世界,是來自靈魂的探索與
深造。 
這樣的靈性轉化學習帶給我最大的禮物,就是我自身將最先受惠於靈魂的
滋養。透過靈性資料的文本閱讀和靈魂日常的實戰經驗,我們可以知道轉化學
習的發生是將自身的靈性落實在生活當中的經驗學習,而我對於生命價值的體
悟也展現在往後的日子決定好好存活。雖然活著有時候並非容易的一件事情,
也許選擇死亡看似比較輕鬆而且容易,但與其說了斷是自我的解脫,不如說它
是一種生命的逃避。尤其擁有來生的輪迴轉世的信仰,更讓我明白到今生今世
未完成的靈魂課題,到了下一次的人生還是要再次的重來與面對,因為這就是
宇宙為靈魂安排的生命學習的課題,每一個地球上的人類都逃脫不掉的。因此,
第一次的靈啟經驗更讓我明白且願意下定決心,無論將來的生命如何的安排,
我都要好好的待在這一副軀殼之內,因為我們在靈魂當中所擁有的各種願景、
理想與目標,都是有賴於肉身才得以成就。我從消極的逃避痛苦,轉變成了積
極的投入,每當遇見新的學習,我給予自己嘗試不同可能的機會。我因為有了
靈性學習的經驗,而轉向社會教育的學習與發展,想要與更多人分享自身的旅
途當中所領悟到的寶貴經驗。 
自我轉化於是從母親的女性角色開始,所謂妄想的經驗是創傷發生的起點,
但真正最原初的創傷發生時機點,卻在於過往一旦遇到強烈的情緒刺激,就善
於在情緒和記憶上解離的自己。在女英雄的旅程當中,從唾棄陰性本質、認同
男性價值觀並與之結盟、試煉之路:遇見妖怪和惡龍、尋見虛幻的成就、覺醒
於靈性似已成枯槁成灰的感覺、啟蒙和潛往女性上帝的所在、極度渴望重新連
結於陰性本質、癒合母女裂痕、治癒失衡的陽性本質,到整合陽性和陰性本質
的路徑中,我們可以看見自己在這條道途上,從對母親的強烈分離感起程,表
114 
現在對母親形象的高張焦慮與情緒上的不認同,是為唾棄陰性本質的開始。在
疾病逐步發生的過程當中,持續追求著象徵男性力量的科學價值,並在試煉之
路上遇見自己內在精神世界的妄想,以及在醫院當中與自己所愛的人相遇所帶
來的創傷,導致我不再願意面對這一個生命故事的自己,是遇見了試煉之路上
的妖怪和惡龍。為了能夠打倒妖怪和惡龍,能夠對抗精神世界所造成的創傷,
看似在情緒上和記憶上都與之切割的方式,在往後重新建立起的預後生活,仍
舊追求著世俗當中被認可的成就,彷彿有了一個好名聲的學位就可以為自己帶
來保護。然而躲在這把大傘之下,卻亟於追尋靈魂的解脫,在重視科學精神的
學習歷程當中,對於人文素養的追求更加迫切,最後在被情感傷害中否認自己
過往當中對於生命課題的努力之下,靈魂所渴望的出路竟然是遇見一位真正的
神。 
這樣的靈性啟蒙或靈啟的經驗帶給我生命全新的源泉活水,並遇見自己的
內在女神,她就是那一位將自己從危難的緊急情形當中,從想不開的處境裡解
除生命警報的自己,讓自己遠離生命的遺憾而再次充滿勇氣的迎接人生中未完
成的挑戰。在往後的新生活當中她進入了新的一段情感關係,並因此有了機會
重新整理、面對與回應這樣的原生家庭,母女的關係與裂痕因此在逐年當中有
重新再次癒合的機會。在年復一年的相處中,由於內在的女性原型獲得了自我
的內在女神的力量所灌注,而能夠不斷的將自身的能量重新飽滿與再次充盈,
進而獲得了充足的能量去回應過往當中,母親的內在形象所造成的心靈創傷,
願意接納與整合來自其他家庭成員的觀點。其實這些觀點在生命的前期便是如
此的存在著,只是內在的潛意識力量將暴露出來的恐懼增強與放大,遮蓋了對
現實世界當中原初的認知系統,進而對所謂的精神世界產生破壞性的能量,有
如將內在的自我一分為二、撕裂成兩半。在與前夫建立穩固關係的歷程當中,
也是另一次的治癒失衡的陽性力量,男性力量在生命當中的早年是以父親的形
115 
象出現在命運當中,往後便是以戀人的形象進入生命。我和前夫在面對、處理
與解決彼此的婚姻關係過程中,最後也和自己此生的初戀相遇,才發現自己的
內在對這個人有不明所以的熟悉感,是因為在長大成人的過程裡,對力量的自
我追求與探索時,錯誤的使用了自己的內在力量,以為遺忘的不愉快的事件,
就可以避免失落的活在現實世界裡,殊不知這是自己的失誤。錯誤的判斷與習
慣性的解離讓我在面對現實世界當中的人際困難時,失去了真正面對與解決的
強大力量,而由於過往的事件讓自己一次又一次的解離經驗,也讓我失去了內
在原初的強大自我。 
這樣的內在自我其實就是自己的女神。當她從對外展現的自我形象轉變為
內在引導生命道路的情緒性力量之角色時,她掌握了精神世界的力量,也在這
個位置上取得內在強大的地位,是遺留在表層意識的自己所不容易領會的,她
也將不受這個自我的操控,而是需要喚醒。喚醒自己內在的女性力量,可以讓
存在於自身的陰性力量與陽性力量重新達成和解。在最原初的時刻,陰陽兩極
就是彼此的舞伴,她存活在我們的內部世界而向外投射出一個平衡而和諧的世
界,就是我們的生命最初始的那個自己。在受到原生家庭的母親原則與成長當
中的父親原則所指導之下,這樣的內在兩極之舞步有了不同程度的移動,引動
了我們內在潛意識的阿尼瑪與阿尼姆斯,將自我建造成不同形象的性別角色、
扮演不同的性別力量,與之相應的,我們也會因此找到自己的情感歸屬或者心
靈上的伴侶。身為地球上的女性存有,我們不可能活在只有女人的世界裡,不
能忘記男性的角色存在,才是提醒著我們為何之所以為女人的重要功課,我們
將會因為遇見自己生命當中的重要男人,而重新於自己的內在找到真實的平衡,
這樣的兩性舞蹈才能夠為真實的女性力量發聲,而不是獨厚於偉大的母性力量
或脆弱的女性角色。我們同時是前者也是後者,這樣的新自我認同將可以帶領
我們自身活出真正的靈魂力量與自性的光芒,照亮我們此生的獨特而個體化的
116 
歷程,並在這樣的自我實現道路上,更不容易迷失了自己。 
圖 9  
看見自己的內在女英雄 
第二節 扶持我的陽性力量 
探索生命當中的異性角色可以完整我們對於自己的認同與認識,認知到自
己在這個世界上的情感探索對象是屬於哪一種性別角色。與父親之間的關係是
人生當中發展戀父情結的可能,這樣的戀父情結也將影響往後對於情感關係的
追求與模式的建立。生命當中的重要他人角色為我們制定了探索世界的座標,
同時也打下了親密關係的基礎。 
壹、父親與我 
父親是繼母親之後,第一個在誕生的世界上迎接我們的人,如果母親是人
117 
生當中的起點,那父親的存在影響著我們生命的方向。在 Jung個體化的不同階
段,父親的原則象徵著世界運行的法則,父親的責任是將我們從對母親的依戀
當中脫離,協助我們真正的進入家庭外部的世界。在我的家庭當中,母親的角
色是稍嫌功能萎縮的,因而父親對待女兒的養育態度,有的時候有了跨越界線
的曖昧不明。這不是意味著父親的親子界線模糊,而是親職角色的替代。父代
母職的情形讓我在生命當中對於媽媽這樣的角色有著衝突而複雜的情感,一方
面我討厭媽媽帶給人的感覺,另一方面又渴望與媽媽恢復連結。父親的角色在
這個部分的缺失有功能替代的情形。 
父親在我的生命當中佔據的重要地位是顯性的,他和我的母親在人格陰暗
面的作用不同,他是直接的影響的我的生命態度、人生當中的重大選擇,例如
求學時期的科系選擇,或者在興趣的發展上是否給予支持,乃至於經濟方面的
援助都需要獲得他的認同。父親在我的較為年經時的人生道途當中扮演著明師
與指引者的角色,到了往後稍微年長的時期則扮演陪伴者與輔助者的角色,父
親的參與貫穿了我的整個生活路徑。即便在婚姻當中他也默默的當一個支持我
們感情前進的人,並在離婚後完全接納我回歸原本的家庭生活。父親對於我的
正面影響是充滿著愛與包容的力量,而且這樣的如深似海是母親所做不到的。
然而父親畢竟並非完人,他在我的生命當中所形成的保護與限制,也是必須被
考慮進去的。 
其實父親對於時下的年輕人有諸多的不認同,他認為我們年輕一代的晚輩,
對於優良的傳統道德素養盡失,缺乏感恩與厚德待物的能力,這樣的情形使得
他往往難以真正的站在年輕人的立場去理解與思索。然而這正好是他最珍貴的
地方,即便在認知上他無法同意與他不同的價值觀與生活素養,但他在父親的
這個角色上仍是個盡心盡力的好爸爸,為著兒女的將來處處著想,暗中規劃著
118 
他的身後事的安排。在照顧親人的能力方面,若要談及情感上的認同,常有可
能與之對於事實的認知產生不一致的衝突,要在不同的事件當中與之達成相容
的情感共識可能是並不容易的,但是他對於家人的愛護與照料方面,在他退休
之前努力工作、退休之後操持家務,則是讓人對於他的責任感難再有怨言的。 
在我年輕的時候,對於父親是充滿怨嘆的,我對於他介入我人生的程度有
諸多的不滿。我抱怨他干涉我對於科系和興趣的選擇,這樣的齟齬也讓我們父
女之間的磨擦存在著許多年,彼此難以互相諒解。這樣的情形到了我在後來發
展出靈性的興趣時,要接納這樣的我對他仍是充滿挑戰的。這個部分的緩衝則
是有賴於母親的存在,因為她有明確的信仰支持著她,我在這點當中感謝著她,
接納了我的靈性轉化學習,並在往後我探索心靈世界時有了一個空間。母親的
存在狀態決定了父親如何在我的生命當中扮演相應的功能,他們是彼此的雙人
舞,也是形塑我的內在世界之原初的場域。當我的母親在我的精神世界失去她
正常的功能時,父親的存在扮演著守候的角色,他為我持守了一個家園,是我
可以回頭重新與母親建立關係與恢復良好連結的橋樑。父親的角色在我的原生
家庭當中是至關重要的,他有如為我在這樣的世界當中拋下了一個定錨,確定
了我不會從這樣的家庭當中游離失所,無論母親的態度如何變化,我都在這個
地方有一個家。這樣的父女關係讓我對於這個家庭當中的父親有著一種聯繫,
這樣的戀父情結也讓我在往後的人生當中,以不明顯的方式在左右著我對待伴
侶的態度,而看見並認清這個自己將有助於我發展更真實的伴侶關係。 
貳、人生伴侶的考驗 
前夫在我的生命當中扮演著騎士的角色,當年他遇見我的時候,我們都無
知於生命的背後有一個更大的故事,而懵懵懂懂、跌跌撞撞、傻傻地一路向前,
119 
彷彿如他對我所承諾過的「我們能走多遠是多遠」。即使因為不情願而半推半就
地與他進入婚姻的關係,我仍舊是在這段關係中擁有著滿滿的包容與愛,並且
在這次的婚姻中更加了解自己。前夫在這段關係當中盡其所能的陪伴著我,陪
著我探索和尋找我在生命中的所愛,以及渴望的歸屬感,只是後來很遺憾的是
我選擇了離開。我在這段關係當中看見自己,有如一株被保護起來的花朵,含
苞待放卻又不知道該如何自處,就好像一位嬌嫩的女孩,而他是我的兄長呵護
著手中的她。他常說自己像是一隻小小的螞蟻,在對抗大雄兵,面對在我生命
當中發生的轉折、與靈魂溝通的事件發生,他卻只能束手無策,默默的祈禱我
的好轉。 
最初我們的相遇是充滿驚喜的祝福,在相遇這一路上我們互相提攜,從一
開始的遠距離到後來的同居,我們的感情維繫著彼此的信任與穩定,這不會是
我想要輕易放手的一段關係。即便我當初沒有想要馬上與他結婚,我也只是想
著再相處更久一些、更了解彼此的優缺點和各自的需求之後,再來考慮結婚的
事宜,至少我認為這是更對這段關係負責任的方式,只可惜事與願違。我會後
悔我沒有堅持當時的理想,雖然這很可能勢必造成與婆家兩方的緊張,但至少
我有位自己爭取過些什麼,而不會留有遺憾。畢竟進入婚姻關係之後,我感覺
前夫就默默的轉變成另一個人,那一個更真實的他,也是我更討厭的人。這樣
更真實的接觸是必要的一段過程,遲早我們都會走入這樣的深度,學習更真實
的親密關係。在這樣的交流與互動當中,他收起了交往時的那股溫度,也期待
我能夠融入他的原生家庭。我也不是沒有嘗試著去投入與參與,而是在與婆婆
開始互動失敗的感受當中發現到前夫的缺點,那是我們關係的致命傷,也就是
他夾在自己的母親與老婆中間兩面不是人。他甚至同意自己的母親在那個時候
已經變成了自己不認識的樣子,但也無計可施。因此我安排的一場日本旅行,
想要讓彼此找回相遇的最初感受,只是在這趟旅行中,我的感受加重了我對前
120 
夫的無力感,以至於我開始想要尋找真正的幫助。 
在我們去日本回來之後,我所遭遇的事件更讓我們彼此喚醒了內在渴望的
伴侶原型,兩邊都想當被拯救的一方,兩邊都在等待另外一方率先做出改變。
他等待我的清醒,我等待他的看清。當時不成熟的我,發現自己愛上了別人,
或者說,失去了對前夫與從前相同溫情的感受時,我曾經過沒多久就很堅定地
要與他離婚。過去,每當我嘗試要離開他時,他總是問我那誰要照顧我?但這
次,我卻堅決了起來,我失去了自己的理智。現在回想起來,我跟前夫之間並
非缺少了愛,而是缺少了考驗,從前交往時期我們所經歷的那些,在婚姻的轉
化之後,面臨的必要考驗,也可能是每對夫妻會遇到的測驗。在去看醫生的那
幾年中,我心中所想的都是等待前夫可以願意放手,我甚至懺悔的說,我會刻
意的做出傷害他心情的事,想讓他知道我內心也在痛苦著、掙扎著,想讓他知
道可不可以就這樣讓我往外飛,讓我去嘗試看看。最後,他如我所願地和平與
我離婚了,我一離婚,心就開了,但開心維持不了多久,沒幾天,當我再次見
到醫生時,他的反應,卻讓我傻住了。 
原來,前夫在我們協議簽字離婚之後的返鄉路上傳了一通訊息給醫生,告
訴他我們離婚了,接下來要交給他了,然而,當醫生轉述這件事情給我的時候,
他卻回答了一句:「這關我什麼事?」這對我來說是一顆震撼彈,心裡想著:
「他怎麼可以說出這種話?」彷彿我的人生大事毫不關己,他不是在靈魂溝通
的層次與我承諾過許多次嗎?一直要到後來,我才徹底明白:原來,與我內在
對話的他,有些不是真正的他所能意識到的事,而我們僅僅只是在潛意識的層
次對話著,同時,與其說我們是在內心世界有著靈性的交流,更不如說,這是
與我內在的渴望對話。這一系列的事件發生喚醒著我的內心,阿尼瑪與阿尼姆
斯的原型,也是我的內在神聖伴侶的原型。 
121 
參、靈魂伴侶的追尋 
在某種意義上,這位醫生可說是我的靈魂伴侶。與靈魂伴侶的相遇讓我的
內在精神世界有了很大的重整,我不但找回失落的記憶,同時在許多次的內在
自我工作之中,整合了更大、更充滿力量的自己。這些內在工作主要由我的紙
筆紀錄所承載,首先是內在的醫生不斷試圖喚醒我的沉睡記憶,也就是我們曾
經相遇過的故事,所發生過的人事物和場景。由於我的現實記憶當中確實有過
幾段不可解釋的飄浮意識片段,比方說去到相同的地方時有似曾相識的感覺,
或者在獨處的空間裡有飄浮起來的畫面與聲音,而我好像就如臨其中一般的身
歷其境。相較之下,找回記憶反倒是較為容易的,更困難的工作在於內我分裂
的意識整合過程,經常耗費相當的時間才有辦法再次銜接內外的自我。表現於
外的我是遺落在現實世界當中的「生還者」,她具備柔弱的本質,原本的她是內
在的自我當中最脆弱、應當被保護起來的部分;而內在的那個我則是擁有強大
力量和情緒動能的源頭,我的外在追求與表現彷彿都由這個內在的自我所驅動,
她清楚我在這份生命當中的最真實渴望,她記得所有的一切故事,她原本應該
是「活在外面」的那個自己,原本的我表現於外在的其實是個強而有力且充滿
自信的個體,只是由於受到劇烈的創傷與意識的解離使得自我保護的機制啟動,
而長成為了與原本相反性格的個人。每當內在的那個自己透過我的書寫力量而
呈現出來,我所能感受到的這個她是充滿飽和能量的,她是一座值得依靠的大
山,然而受到重創的她為了暫避現實的困難而躲進了內在的世界,留下那一個
單純而無知的小女孩在個性的表層,成為了自我的保護色。 
內在的醫生為了做好這份內在自我整合的工作耗費了不少心力,現實當中
的那一位醫師也在我的個人修復上做了很大的努力。然而,內在的醫生和現實
當中的那一位醫師,總是相似又相違,似是而非的存在著彼此的足跡,而我,
122 
總是缺乏勇氣去追問事實與真相,直到最後不再能從對方身上得到消息。儘管,
對方曾做過多次說明,那個內在的醫生扮演著他的部分角色,我還是不能明確
的分辨,這個內在的醫生和現實中那一位的不同。一直到寫這篇論文的故事時,
想再去加以確認,得到了對方最新的說法,是那個內在的他更接近我的心思對
他所抱以的期望,也就是說,雖然他扮演著一些功能足以支持我的修復歷程,
但同時這個內在的醫生也是來自於我對現實中那一位抱持的期望所產生的投射。
特別是當他後期的態度轉變開始。從前半段的完全支持與符合我內在的他,到
後來他因為我開始嘗試在會談之中靠近距離,而使得他必須採取退後以及相反
的言行,我內在的醫生仍舊是一貫的對我抱以愛與同情,而非符合現實中的那
一位醫師真實所採取的行動。這是因為內在的我已經完全相信了之前他在我心
中對我陳述的那一套說詞,希望能和我再次相聚,而我因為也真心的接納了他,
而開始渴望能回應現實中的他,也就是去愛真實的那一位醫師,我想去愛真實
的他。這也就是前面曾經談過的,這整個事件喚起的是我們內在對神聖伴侶的
渴望原型,也就是阿尼瑪與阿尼姆斯的交互作用。與其說這位醫師是我所認定
的靈魂伴侶,更不如說他的存在與出現刺激著我的內在對阿尼姆斯的追求與渴
望。曾經前夫和我的故事可以讓我獲得滿足,到後來對伴侶的失落,再加上遇
到與醫師靈魂溝通的事件,整個過程的發生都在一步一步讓我對於內在的真實
追求加以質疑,對我所以為的理想伴侶關係是精神上與情感上的連結進逼,甚
至醫師的靠近與離去,都是一個重要的轉折點,他是真正逼迫我去面對內在真
實的開關。當我們說要相信自己的經驗、信任自己的判斷,我們究竟可以信賴
自己到什麼程度,才不會與現實相違背呢?當我們生活處在充滿愛的經驗裡面,
我們確實也很難說要離開這一團愛的氛圍,只因為單純的愛真的非常美好,令
人捨不得放下。只是這些曾經存在過的純純的愛,最後竟然只淪為夢幻泡影,
也不免令人唏噓,也想令人質問:是什麼改變了他?原來,只是因為太美好了,
令人竟然在治療關係的期間,就想要追求,以及破壞距離所能帶來的保護。那
123 
就好像童話故事當中的情節,在到達彼岸之前你千萬不能回頭,只要小女孩一
旦回頭了一切就化為烏有一般的諷刺。 
在阿尼瑪與阿尼姆斯相遇的過程中,現實裡的我們在會談期間也確實擁有
過許多說不完的美麗故事,美好的經驗永遠存在我的心中銘刻著我對彼此的愛。
我們在會談室中從一開始與前夫三人進行,到中間我開始嘗試穿插個別會談,
甚至前後父母也參與過一兩次,我真的是被愛捧到飽的孩子。在與前夫離婚前
的會談故事,是我所經歷過最美妙的人生經驗之一,我同時被好多人愛著,而
這樣的允許,來自現實當中那一位醫師的保護!也是因為這樣很幸福的感覺,
讓我在往後面臨到失去的時候,更加的難以承受。曾經的我們是那麼的美好,
奈何最後是如此的不堪!但這樣的經驗讓我不得不去思考,我真正應該重視的
是什麼,是內在的獨處與安適,是能夠重新進入社會當中與人群互動,從與人
的交流當中尋找到真正支持自我的力量,從而獲得真正的安定感。這是我的恐
懼所渴望的。這樣的心靈狀態也會影響我對伴侶的追求,我內在渴望被愛回應
與擁抱的部分,它原生的創傷來自醫生的存在,也就是說內在某個部分的自己,
會相信只有這位醫生可以滿足我的情感需要,只是我們重新思考這整件事情,
它又真的是如此的嗎?他的存在勾起的我對完美靈魂伴侶的渴望,讓我一度以
為真正的完美確實存在,無論我過去有多麼不是如此,在與他相遇的過程當中,
他對我的愛會讓我滿足了渴望親密的情緒。因此,與其說在我的內心認定只有
這位醫生可以滿足我的渴望,不如說我在追求的是一種令人心滿意足的親密關
係。這樣的連結來自於對前夫在伴侶關係當中所扮演的角色令人失望,一段關
係當中能夠被第三個人介入意味著原本的伴侶關係就是有失能的地方,但當我
後來開始走向醫生卻被推拒在外時,第二次的失落更是要來提醒我自己,這樣
對親密關係的期待與其說是渴望與另一個完美個對象相處,更不如說這一個完
美的另一個對象存有在自己的內心。是內在的那一個我們才承載了內在完美伴
124 
侶的理想型,也就是我所投射出去的外在渴望,或許其實是向內的、渴望與自
己重新親密的內在關係,甚至,我們可以將祂靈性化,也就是那反映的是我們
自己的內在渴望與神的親密。只有在與神的關係當中,我們可以找到自己的完
整性,同時包含著內在原始的自我以及內在的神聖伴侶關係,與神的親密可以
讓我們與自己的內在各個部分獲得合一和整全的感覺,透過看見與神的親密關
係,我們可以照見內在所期望的另一個完美對象在自己的內在世界與自己成為
夥伴,而當我們自己成為自己的夥伴,我們與神的關係也將會是另一場內在的
神與自我共舞的親密相遇。 
第三節 我的靈性轉化與個體化歷程 
本章的療癒之路前兩節的內容分為滋養我的陰性力量和扶持我的陽性力量,
是由於在 Jung分析心理學當中,個體化的歷程包含意識與潛意識、自我與原型
的整合與統整,而在人一生的心理發展當中,個體化發展的歷程首先是由母親
的原則所主導,發展至青壯年期,而後是由父親所主導,發展至中年階段,最
後由自性所主導。也就是說我們可以從人生命當中的不同兩個性別來看待,尤
其是內在的男性與內在的女性力量如何形塑成為一條個體的道路,最後是自性
所引導的靈性轉化進程。人類意識進化的五個階段也可以視為一人類意識的個
體化發展,從幽冥參與、社會化調適、宗教人的階段,到完全根除投射,最後
進入現代性之後,每一個階段在研究者本人自我敘說的生命故事之中,也能看
見互相呼應的內在路徑。 
從研究者生命的年輕時期,對於內在存有另一個靈魂的意識狀態全然未知,
而有如活在母體的羊水之內載浮載沉,與之共生共榮而不知,直到在醫院當中
125 
遇見了那名醫生,有了第一次的內在對話,才開啟了這樣的覺知與意識。這樣
的發展有如在生命的早年,對於外界和自我的關係與連帶,在恍惚之中逐漸照
見,社會上的力量如何影響著內在的感知系統,而與之共存。接著到了下一個
階段,個體了解到這個世界上存有著這樣的力量,但表面上的意識卻遺忘了這
個事件,研究者帶著這樣的意識狀態離開了醫療機構,回到了原本的社會系統
與人來往交流,這樣的階段屬於「幽冥參與」的時期,個體的對這件事的感受
由於受限於意識上的記憶,而使得記憶的感受處於「有和沒有之間」,彷彿存在
著,但又好像摸不著抓不到。在這段漫長的幽冥參與的時光,直到了生命的 32
歲才再度破土而出,真正進入了表面的意識,完整的被自己接納。 
接著為了適應這樣的新興狀態,必須與這樣的新意識狀態達成生存上的平
衡,個體花了數年的時間試圖和這樣的內在世界共存,有意識著帶著這樣的感
知系統存活在這個社會,也試圖讓達到新平衡的自我得以用社會化的方式和人
們相處。研究者或可能在生活當中的某一個點滴時刻,就忽然能夠感知到內在
世界的呼聲,或者是另一種可能,試著讓內在世界的聲音不要干擾到平日的社
會化生活,於「內在-外在」世界的兩端當中尋找到自身的平衡狀態,使得自
我不至於過於耽溺或逃避「內在-外在」世界的連結需求。然而這個階段的我
其實是過分認同內在世界的面向,透過內在世界所獲得到的資訊與消息,來自
於對過往被遺忘的生命事件所陳述,逐漸斷簡殘篇的一一尋回,沉睡於內在世
界的記憶。當我由於意識的回溯而重新說出過往情境之下曾發生過的對話內容,
我深深理解到故事的真實性,以及在這些訊息當中,我能夠回憶起離開醫院之
後的不可思議事件,亦即當時發生了似曾相識的回憶,但我卻不能提取相關的
線索來回應自己的情境。我明白了對方的出現不見得真的是為了什麼而來的,
而是來自自性的一部分在引領著我,尋找出遺失拼圖與片段的自己。 
126 
雖然在這個時候的我,開始相信神是會有賞罰的,也就是相信著我會獲得
我應得的報償,而傷害著我們的人會有自己的地獄,我的意識狀態進入了「宗
教人」的階段,也就是相信善有善報、惡有惡報,不是不報只是時候未到。因
此而深信著自己會被生命所善待著,因為這個事件的故事並非錯誤在己,而是
對方虧欠自己,當時的我深信,對方的出現是來表達他的心意,不但有愧於自
己,也希望藉由實際的行動來表明他對我的感情。然而,事實上對方重新進入
我的生命,雖然在他的內心是有著遺憾的,但他未必抱著非得成全的心意,而
是對方基於對我的了解,他相信他可以好好的處理我在這整個事件上面的創傷,
包括精神上的和心理上的創傷,而這才是他要我進入醫病關係的理由,而非要
與我重新相聚。因此,當後來我們進入治療關係的後期,對方採取了相反的態
度與行為時,那是在說明,我不應該採取與他靠近的距離,因為那會傷害醫病
關係。等到了治療關係結束後的兩年,他在與我的某次的互動與交流當中,確
立了一個訊息是他不會與我在一起,在這個事件當中對我來說確實是破除了我
對他帶有期望的投射! 
我完全明瞭了,雖然有些來自內在的訊息是連結著彼此的意識狀態而來的,
但也有一部分是來自於我對這個事件渴望被成全的投射,我對他的感情讓我希
望可以終究相守,但對方卻是做不到的。這樣的轉折更讓我發現,與其說我們
的目標是為了重新相聚在一起,不如說這個過程是為了來支持我的療癒,因此
這些相遇與我的治療目標有關,而為了能夠達成所謂的治療目標,醫病關係的
效果必須能夠被確立。如果當他退下醫生的角色時,我就會變得不穩定,那就
意味著他需要保持著他的專業角色,而不能夠離開那樣的位置,也就是說,我
們的關係有比戀愛更重要的事,那就是我們的治療關係不能夠被破壞與毀滅,
因此,確立穩定的醫病關係是首要的目標。 
127 
另一方面,多重角色的關係也會混淆彼此在互動當中的重要訊息。在破除
了我對他的投射以及對事件發展的期望(根除投射)之後,這一切的意義便不
再是由「內在-外在」世界的訊息所決定的,不再是內在的那一個存在、也不
是外在為我所框架的治療者,而是由我自己來決定這件事情的意義,由我來敘
說這整個生命事件的發展,甚至是可以由我來決定這些生命事件的真實性。當
醫生退去了他的位置、與我完全結束了治療關係之後,我們也沒有辦法發展成
為伴侶關係,而且由於角色的重疊,他也沒辦法回應我關於來自過往事件的實
際情形,或者來自我內在的那個會與我對話的存有,他只能說那是來自我對他
的期待所投射出來的影子。這讓我感到相當的沮喪與挫折。倘若這一切的發生
都只是由於我一個人的囈語,那我的那些真實的親身經驗又算些什麼呢?當醫
生不再能給予恰當的回應,這些疑問只好由我來回答我自己了! 
因此,我對這個事件的意識發展進入了所謂的「現代性之後」,也就是自我
與自性更大的融合,以及重新理解這整個生命事件的框架與意涵。從更高的角
度來觀看我所遭遇的課題,打從一開始就是個自性在引領我前進的道途,只是
由於意識對自性的敞開程度,以及心理上在個體化發展的不同歷程,而對這些
事件有了不同的理解與詮釋。 
表 8  
Jung 意識的五個階段之個體化歷程 
幽冥參與 社會化調適 宗教人階段 根除投射 現代性之後 
意識與記憶的
存在介於有與
沒有之間 
內外的新平衡
與自我認識 
相信命運的公
正將會降臨 
看見原來對期
望的一部分來
自於投射,開
始回答自己的
生命難題 
自我與自性更
大的融合,以
及重新理解這
整個生命事件
的框架與意涵 
最初,我因為自身的生命課題而與對方在醫院相遇,在經歷過一些浪漫與
128 
曲折的戀曲之後,我帶著殘破的記憶回到自己的生命,在那樣的化外之地中,
有如把自己的力量交給了別人,彷彿進行了這個祕密的儀式之後,我就忘記一
切了。這個標記讓我在往後的生命事件當中,不斷的需要重新經歷「力量」的
議題。自我的力量與其說是個權力,更不如說是一個內在屬性,它是一種能夠
自我統整的能力,能夠整合「內在-外在」世界的差異性,能夠抵禦來自他人
與自我之間的衝突,能夠涵融存在於自身內在的兩極性課題。在「是與非」之
間、「有與沒有」之間、「對與否」之間的拉扯與彼此抗拒的力量,它就像是一
場內在的雙人舞蹈,內在的自己有兩種不同的聲音,在引領著自己和主導自己
的方向,決定我們的人生道路走向的是自我的力量 (strength) 而非權力 (power)。
我們所擁有的權力是在於我們可以「說是就是」的能力,但我們所擁有的力量
則是在於我們可以「決定我是」的能力。因而我們可以決定我們要什麼樣的人
生方向,決定我們要什麼樣的感情品質,雖然我們不能決定自己能不能擁有,
但是我們可以作出不後悔的結果當中,我們想要獲得的自己。 
一路上在與內在的自己相遇的過程當中,我們也同時在發展自身的內在力
量,這樣的特質我們或許可以說是感受性強的、較易感的,也可以說這樣的特
質雖然可說是一種脆弱,但卻是能夠發掘內在真我的強大力量。在這樣脆弱易
感的特質之中,我循著生命道路的線索找回了過往曾經遺失的重要自己,這些
憶起的故事也讓我感覺自己更有能力肩負自己的命運,同時在原本表層意識的
空白與缺席當中,找到了原本應屬於我生命當中的色彩與容光。我其實是一個
擁有多采多姿的人生的個體,而不是一個看起來找不到意義和答案的沮喪之人。
這種「自我贖回」的力量讓我感受到一股踏實而安定,可以更加心滿意足地走
向我的下一個人生目標。當我在進行這部分的論文研究時,我正好剛體驗完一
場譚崔性能量的牌卡解讀,裡面便說明了我的自我力量議題:當我交出自己的
脆弱的時候,也就是把自我的力量給交託了對方。但其實,我們隨時都可以決
129 
定要拿回這份力量,要取回自我的力量,只需要堅定地說:「是的,我是!」我
們面對自己的力量,想要重新取回它,得要有堅定的勇氣,敢於拿出權力 
(power),堅定地說:「要,我要!」於是我們就可以取回對於自我生命的敘說
權,我們可以用自己的力量闡述自己的故事,從自身的理解與耙梳當中,再次
的找回過往生命之於自己的定位,因而我們也可以同時繼續循線找出,未來我
們渴望版本的自己,因為敘說而找到勇氣,因為敘說而找到力量,讓我們得以
繼續前進! 
130 
第六章 結論與建議 
生命終究不是要「變成對的」,而是變得自由和完整。──Pamela Kribbe 
第一節 跨出生命的回顧 
「往外看的人作著夢,往內看的人醒著。」──Carl G. Jung 
壹、統整過往 
透過轉化學習的基本假設:人是理性而自主且能夠賦予責任,知識是經由
主觀的建構過程而習得,個人身為社會的核心且對集體的未來負有責任,個體
遭遇到必須進行批判性反思的情境。我們知道,要進行一場靈性的轉化學習,
它的過程必須要有一個前提,那就是我們過去所使用的工具已經不再管用,而
需要進入新的習得情境。特別是靈性學習在Assagioli (1991) 的蛋形理論當中,
可加入「理性的能力」來向外界校準,來避免我們一味的徜徉於抽象的心靈世
界而與現實生活失去聯繫。在轉化學習的批判性反思當中,內容的反思在於我
們對於生命事件所呈現在眼前的現象,真的就是它所呈現的樣子嗎?我們對於
自己的疾病與困難持有著一個新的發問,我們渴望探索關於疾病與困難的真相;
在過程的反思之中,我們面對道這些生命事件一一地發生,當年我在醫院當中
所經驗到的那些故事,我一直以為只有那些過程,但在真實之中,卻是有著更
深層與被隱藏起來的回憶,在往後的漫長數年中,記憶片段已似曾相似的感覺
來輕叩我的腦門來拜訪我,試圖要喚醒我沉睡的過去,只是,我還是沒有真正
的明瞭生命的用意;在前提的反思中,我們經過一段梳理與發現,才知道真正
131 
的原因是內在沉睡著一個人的靈魂記憶,與他共同擁有的生命片段,才是這件
故事想要告訴我的。然而這樣的故事讓我學到了什麼呢? 
圖 10 
整全自我與靈性轉化學習的女英雄之旅 
從一開始的遺忘導致的身心失衡,到後來藉由這份已經遺忘過往的生命重
新開始、努力復原,到了中期發現自己經常會想起要找到這位醫生,但自己卻
一直無法提取這樣的記憶。最後開始真正能第一線接觸他的契機,在於在PTT
上找到了這位當時已經很感興趣的醫生,看到他的臉書照片勾起我對過往生病
回憶的恐怖片段,當時只是想著是否能夠透過臉書上的互動,來看見自己在這
條路上已經克服了多少。此時的我一直無知於這位醫生對待我的想法,他始終
抱持著有距離的關心,好像有點靠近但其實離很遙遠,最後在和前夫去日本小
旅行之後回來的事件上爆發,有了一段漫長的靈魂溝通的過程,他的來意解釋
著他的想法和態度。我在這樣來往的紙筆對談當中發現了關於我生命的真相,
還有兩個人在命運當中的交錯,美好與曖昧的感覺最後在我渴望靠近他的時候
被他推開,我開始真正的進入自我的整全工作。了解到內在的一部分是來自於
自己的期待與投射,而由於我無法很好的區分真實與想像的對方之差別,我最
後接受了自己的內在擁有著這樣的能力。這種內在的力量就如同潛意識在整張
命運地圖當中帶給我的引導,引領著我前往自己內在的綠洲,而我有能力透過
132 
內在的力量找到真相,也有能力朝著自己想要的解答前行。 
靈性成長的本身已經不再著重看得見的物理世界,而是超越物質的層面去
追探內心更多的可能性,它跨越物質性的工具去探索內心世界的內在資源,並
且在實作當中同步的調整自己的方法並探索更好的方向,因此靈性的學習某種
程度上,已經跨越了轉化學習三個歷程當中技術的旨趣與實踐的旨趣,直指解
放的旨趣。靈性的學習就好像我們自我發展的陰性一面,在我們的現實生活當
中總是強調著要能夠「看得見」的外部世界,可比擬為我們個體發展的陽性面;
而我們在這顆星球上如何對待我們自我發展的陰性面,可以觀看我們是否重視
人類自身的靈性學習。我們自身的靈性學習也好比女英雄的旅程一般,有著不
同於外部世界的陽性面發展要說,它是屬於陰性層面的柔性力量,其實正好是
拉住活在現實世界的我們,在理智而陽性層面的部分不要太快地往前衝,而可
以回頭檢視一下自己的真實需求,先傾聽自己內在的聲音。當我們在現實生活
當中無法獲得自己渴望的解答時,我們可以暫時先停下來,和我們自身的內在
對齊,再重新尋找方向與出發。不要一味的否定陰性面的內在力量,而是允許
我們擁抱自身的存在需要,倘若我們越接納陰性力量,我們能接納的自己越多,
我們因此便越是能向生命的更大可能性敞開,擁抱更加豐盈的自我,創造更加
豐盛的世界。 
貳、超越自我 
雖然在經歷了這些多年的內在心靈成長讓我突破了許多來自自我的限制,
然而我們還是得要回到現實生活當中來去觀看,這一位醫生帶給我的實質影響。
當一開始他說我可以參考「內在醫生」的意見,到後來他的態度卻變成「他是
我期望的投射」,這對我來說不啻是一層最後的打擊。我不斷去思索這中間故事
133 
的前因後果,最後我相信的是潛意識帶給我的訊息,也就是傾聽我內在真實的
聲音。在整件與靈魂溝通的事情當中,與對方或許夾雜著許多似是而非的連結,
而由於我深陷於其中而無法做出很好的界線識別,然而無論這些帶來給我的訊
息是來自於真實的醫生還是內在的醫生,它的存在都是來自於我的潛意識所要
我知曉的。它是來自於我這個主體的潛意識,因此它的某部分是屬於真實的故
事;而在屬於我的真實與對方的真實之中,存在著一個交會的地方,那一處光
明照耀的所在就是來自於內心對於這件事的洞見:我如何看待這一整件事情的
發生。當對方為了保護所謂的治療關係而必須做出相反的行為時,我的內心如
果能夠看見這一點對彼此的重要性,我也能夠發現這樣的(某種程度上是)謊
言的部分。醫生說的也是沒有錯的,有些部分是來自於我的期待,可是也有些
部分,是他所沒有承認的,否則他不會在一開始的時候,同意且允許我繼續那
樣與它對話下去;如果那是會誤導我的,他應該最開始就會阻止我這麼去做,
而不是放任它到最後一刻,才告訴我該「大夢初醒」! 
我必須承認這些對待於我自己的內心而言,是令人感到充滿憤怒的回應,
但回過頭來,我們要能夠看見這是一個人對自己的愛:這整個生命的追尋其實
它的出發點與看似的結束,都是來自於我的內在對於靈魂所渴望的擁有豐沛的
愛。當我們能夠好好的妥善處理這份來自看似命運的背叛,我們往往能向前更
進階的躍入新高度。允許生命的動能持續湧現與運轉它自己,我們的生命力有
時候往往來自於內在的不滿足,各種渴望的情緒來自於內在深處,想要獲得,
無論那是公平還是公義的對待。我所渴望的公平和正義,可能是來自於他的肯
認,但他此刻無法給予這樣的回答,他是做不到的。處理這種失落的感覺與其
再去追問對方索要一個什麼,不如接受我們的故事最多就只能到這一個結果。
接下來那些關於渴望愛的續篇,我們應該要回到現實生活當中去努力經營、繼
續尋找。在這段故事當中,我面臨到內在世界與外部世界的衝突與挑戰,為了
134 
要能夠勇敢的跨越出去,想獲得在彼此之中的真相有最大的交集,我所獲得答
案的方式就是透過塔羅或占卜牌卡向內探索,透過尋找別人協助自己釐清可能
的盲點,然後再和對方在現實當中所用以面對的態度互相參照,來獲得自我印
證。 
與其說我總是向外渴望確認些什麼,不如說這是真正的相信自己;「信任自
己」是一道光芒,也是我們身處在黑暗當中,唯一且最好的希望。透過勇敢地
與外部世界談論這件事情,並且與自己的真實對齊,我將那些內在的困惑及不
適之處與自我調適,將自己的視野躍升進入一個新的高地。在那個地方我可以
保有我自己原本的觀點,同時包含了對方對這件事情的理解,我們對這個事件
的觀點因此有了交集,而這樣的交會不是由於線性的領悟,而是多重面向的:
我可能由於我的內在限制而有的誤解、我可能由於看見真實而無法被對方接納
的地方、對方由於他的限制而有的觀點、他可能看見真實而無法被我接受的地
方。而由於有了較為全面的圖譜與看見,我們對整件事的接納態度於是可以找
到一處安放之地,可以讓這樣的故事在我們的生命當中找到他命運裡的位置,
讓它得以有一個歇腳的地方,讓我們不必一直要帶著它繼續往前,而可以卸下
這一路以來所肩負的重擔。當它完成了它自己的生命課題,這樣的命運週期也
得以結束。下一步,就是我們要繼續迎向的明天,也是我們在經歷了漫長的個
體化旅程之後,要帶著更新的自我去擁抱未來的人生。 
第二節 建議:科學與靈性 
在當今的社會探討二元性,科學與靈性的世界經常是互不相容的,他們對
於宇宙觀的基本認知常常相斥,甚至有一種互不相讓的意味。然而,這個世界
135 
真的就只能如此二元與分裂嗎?是否,二元與分裂也是宇宙合一與圓滿的一環
呢? 
壹、我的理科背景和人文素養 
我的求學生涯在大學畢業以前都是接受傳統分類的第三類組的科系所薰陶,
也因此有相當程度的生物科學的底蘊。在選擇科系的時候,曾經一度以為生物
科學就會是我的歸宿和嚮往的未來,最後我卻在自己的疾病當中發現,理性的
唯物主義並不能滿足我對生命世界當中的心理需求與靈性追尋的願望。我毅然
而然的捨下了自己曾經擁抱的大兵,開始投向心理學與社會學的領域學習。在
浸淫一段時間之後,受到相當的鼓舞,特別是對於權力關係的感觸,更讓我踏
上「自我力量」的議題和道路。就在我踏上自我追尋的理想當中,我受到了占
卜牌卡的啟發,當我剛開始接觸這方面的牌卡解讀時,我時常用文字記錄下自
己對牌面意義的所思及所感,這樣的日誌被某位治療師好友看到之後,她告訴
我做的很棒,而她也會開始這樣紀錄,我因而受到相當大的鼓舞,開始朝著牌
卡的世界漫遊。後來,更是由於「與神對話」的親身經驗,讓我從此典範移轉,
從一位不可知論者成為一個有神論的人。其實,現在回頭來看,信仰豈不是如
此嗎?科學家也是可以有自己的信仰的,他不一定只能信奉科學大旗,甚至可
以是一個虔誠的基督徒或修行的佛教徒,而科學,只是他們手中理解世界的一
項工具罷了。 
貳、我的心靈天賦 
在踏入身心靈的牌卡世界後,我先後修習了許多堂關於能量療癒的課程,
這些課程會讓我們對於心靈的狀態更加敏銳,無論是對自己還是對別人的。當
136 
然,我們首要面對的是自己的內在世界,當我們的內心能夠穩定的自我支持的
時候,我們對外的靈性療癒才會是基礎穩健的。在這些靈性的學習當中,也不
乏遇見許多想要反過來打壓科學研究的同修,他們對於當今的基礎醫學研究相
當排斥,而且認為許多介入是無效的治療,標榜所謂的自然療法的身心靈派別,
才能夠根治當前文明所誘發的疾病。然而,在我的內心深處,我實在明瞭,無
論是擁抱其中一方企圖壓抑另一方的發展,都是不利於我們的身心健全的。就
好像我們的物理世界需要科學的研究才能讓我們更加了解有形宇宙的運作邏輯,
也才會讓我們對於無形世界的玄妙與怪力亂神擁有基本的識別能力,使自己不
受無知與欺騙的擺佈而不自覺,更重要的是,這些基礎的邏輯能力才讓我們擁
有更多的理解力,來參透無形世界的運行法則。其實,如同「在上的,如同在
下」(As above, so below.),這句話也可以說是「在內的,如同在外」。雖然Jung
曾有句名言說到:「往外看的人作著夢,往內看的人醒著。」(Who looks outside, 
dreams; who looks inside, awakens.) 但當我們能夠內在如實的看著彼此、照見自
己如是,不論是往內看還是往外看,都如同往自己身上觀看一樣自然,我們所
遇見的外相,反映著的都是我們的內在成熟狀態。當我們能夠用這樣的態度來
面對自己的人生遭遇,我們的心靈就可以獲得更大的自由力量,去追尋我們在
人生當中真正渴望的自己。我們真正自我實現的,不見得會體現在外部世界,
更多的完滿與滿足,會活在我們內在的真實生命。 
參、科學與靈性的交會 
當我的生命帶給我的轉化讓我學習到「避免陷入二選一的困境」時,我的
內在世界同時被寬幅的打開,視野開始能夠同時涵融著兩種相異性的課題與事
物,例如:我同時是科學的也是靈性的、我同時是佛陀的也是基督的,我可以
站在自己的一個制高點上,坐看兩極相異的事物在我的世界上此起彼落的舞動。
137 
然而這並非高傲的想要從輕發落,相反地,我認為這是一種真正的謙卑,至少
是我的生命當中,尋找到的真實與謙虛。我並非站在這樣的高峰上想要睥睨世
界,而是發現唯有這樣的低谷才能夠海納百川。這樣的觀點與其說是新發現,
不如說是過去的我們不相信自己可以做到的「接納」。單純的只是去接納相異的
事物,我們就能夠找到屬於自己的「制低點」,只是人們往往渴望成為一個有威
望的山頭,而拒絕承擔屬於自己的命運課題。我們的命運其實有其共享的部分,
那就是每一個人都會需要經歷各種相反與相異的兩極地帶,往高處也往低處走
過之後,上了天堂與下了地獄之後,才會真正地發掘到自性所存在的地方。在
那個地方無異無我,生生滅滅又不生不滅,是屬於永恆的樂章。倘若我們有能
力將這樣的精神保存,允許這樣的內在理想化作為天際的一顆北極星,讓它在
十指不見的黑暗夜空當中照亮我們通往明日的隱而未顯之路,相信這樣的我們
將會擁有更大的能力去成為我們在生命中渴望成為的人,不論那是美好的、驚
喜的、開明的、寬容的,還是其他我們所想為這個世界所帶來的一切。允許我
們的黑暗發光,這樣我們就是走在整全自我的道路上,也就是走在兩極的相異
得以相遇的歸途,我們也就是走在科學與靈性得以交會的地方。 
138 
參考文獻 
王健文(2021)。回家的路:新舊世紀之交外省第二代作家的自我書寫。台灣
社會研究季刊,118,55-103。 
王嘉陵(2022)。大學靈性教育的學習內涵與實踐方向。台灣教育研究期刊,3
(2),129-145。 
田秀蘭(2005)。書寫與心靈.應用心理研究,26,11-13。 
吳明烈(2010)。UNESCO、OECD與歐盟終身學習關鍵能力之比較研究。教
育政策論壇,13(1),45-75。 
吳明烈(2015)。終身學習: 理念與實踐。五南。 
李安德、傅東華(1992)。超個人心理學:心理學的新典範。桂冠。 
李聿勛、唐耀棕(2013)。台灣的靈性學習課程-新時代入世精神的再現與教
育實踐。德明學報,37(2),63-76。 
李明芬、林金根(2010)。整全實踐社群的浮現-創生性社群的整全學習與整
全實踐。社區學習國際學術研討會會議論文集,43-69。國立政治大學。
1 月16-17日。 
李亮儒(2022)。成年初顯期靈性學習與靈性成長之經驗探究—以一貫道道親
為例(未發表的碩士論文)。國立臺中教育大學諮商與應用心理學系碩
士班。 
杜明勳(2003)。談靈性。護理雜誌,50(1),81-85。 
阮氏碧玄(2019)。經典學習與生命轉化關係之研究—以某讀書會的成員為例
(未發表的碩士論文)。國立臺北教育大學教育學系生命教育碩士班。 
林合鑫(2020)。一貫道信仰者生命意義觀點轉化學習之研究(未發表的碩士
論文)。國立暨南國際大學終身學習與人力資源發展碩士學位學程碩士
在職專班。 
林美和(2006)。成人發展,性別與學習。五南。 
林曉君(2017)。轉化學習理論發展、理論緊張與新方向之探討。成人及終身
教育學刊,29,1-42。https://doi.org/10.3966/181880012017120029001  
林耀堂(2014)。初探靈性教育理念融入天主教高等教育課程與教學的價值。
全人教育學報,12,1-32。https://doi.org/10.6272/jhe.2014.12.1  
洪曼茜、黃國彰、易欣榆(2021)從電影「82年生的金智英」探索女性個體化
歷程。輔導季刊,57(3),49-58。 
洪櫻純(2009)。老人靈性健康之開展與模式探詢(未發表的博士論文)。國
立臺灣師範大學社會教育學系。 
洪櫻純(2012)。老人靈性健康的阻力與助力分析:成功老化觀點。生命教育
研究,4(1),83-108。https://doi.org/10.6424/jle.201206.0083  
139 
秦秀蘭、李明芬、洪櫻純、蕭玉芬、林金根、陳志仁、張維純(2013)機構照
顧者實踐社群行動研究的引導與反思。從內變革:開創教與學的主體行
動國際研討會論文集。國立台灣師範大學。11月8-9日。 
夏允中、劉淑慧、王智弘、孫頌賢(2018)。靈性與安身立命:從四大存有問
題建構貫串靈性與生涯的本土化理論。中華輔導與諮商學報,51,1-26。
https://doi.org/10.3966/172851862018040051001  
孫效智(2009)。台灣生命教育的挑戰與願景。課程與教學,3(12),1-26。 
徐秀菊(2013)。全面多元的成人學習。T&D 飛訊,167,1-23。 
袁廣義(2021)。當代中國人的靈性追求與教會的應對方式。公教神學評論,1,
126-141。 
高淑清(2008)。質性研究的 18 堂課:首航初探之旅。麗文。 
教育部(2021)。學習社會白皮書。 
許慧聆(2020)。N 世代的靈性需求初探。台灣神學研究學院。 
陳向明(2007)。社會科學質的研究。五南圖書。  
陳秉華、黃奕暉、范嵐欣(2018)。融入宗教/靈性的心理諮商在臺灣:回顧
與展望。中華輔導與諮商學報,53,45-79。
https://doi.org/10.3966/172851862018100053003  
陳相如(2020)。占星學習者的自我轉化學習歷程(未發表的博士論文)。國
立臺灣師範大學社會教育學系。 
陳紅杏(2015)。中高齡者靈性學習歷程之研究:以參與正念減壓課程者為例。
國立中正大學成人及繼續教育系高齡者教育研究所。 
陳家倫(2005)。與諸神共舞:新時代運動的內涵與特徵。弘光人文社會學報,
3,426-470。https://doi.org/10.29933/shss.200508.0013  
楊惠(2021)。意識的起源。世界圖書。 
傅佩榮(2003)。走向世界的高峰:靈的世界。遠見天下。 
傅佩榮(2018)。哲學與人生。遠見天下。 
鈕文英(2021)。質性研究方法與論文寫作(三版)。雙葉書廊。 
黃秀娟(2017)。靈性、靈性學習與靈性照顧之探討。臺灣教育評論月刊,6
(2),35-38。 
黃素菲(2005)。我不看我時 “我” 在嗎?。應用心理研究,26,1-10。 
黃富順(2004)。成人學習。五南。 
黃靜怡(2022)。疾病後遇見自己-全身性紅斑性狼瘡患者生病經驗之自我敘
說探究(未發表的碩士論文)。國立臺中教育大學諮商與應用心理學系
碩士班。 
楊深耕(2003)。以馬濟洛 (Mezirow) 的轉化學習理論來看教師專業成長。教
育資料與研究,54,124-131。 
楊鳳麟(2022)。女性主義教育觀點的課程實踐分享。臺灣教育評論月刊,11
(6),118-121。 
140 
葉俊廷、李雅慧(2018)。從懷疑到信任:中高齡宗教信仰者的靈性學習歷程。
中華輔導與諮商學報,51,105-141。
https://doi.org/10.3966/172851862018040051004  
廖淑純(2011)。探究成人靈性轉化學習-以生涯轉換者為例(未發表的碩士
論文)。國立暨南國際大學終身學習與人力資源發展碩士學位學程碩士
在職專班。 
趙可式(1998)。精神衛生護理與靈性照護。護理雜誌,45(1),16-20。
https://doi.org/10.6224/jn.45.1.16  
趙燕、黃宗堅(2018)。從童話故事「森林中的老婦人」探索女性個體化的歷
程。輔導季刊,54(4),15-22。 
劉亞蘭(2021)。平等與差異:漫遊女性主義。三民書局股份有限公司。 
劉淑娟(1999)。老年人的靈性護理。護理雜誌,46(4),51-56。
https://doi.org/10.6224/jn.46.4.51  
蔣佩珍、甘桂安(2016)。助人工作的挑戰與轉化:從靈性角度出發。諮商與
輔導,365,23-26。 
鍾宜諠(2008)。參與靈性學習課程高齡者對自我概念與生命意義感之相關研
究(未出版之碩士論文)。國立中正大學高齡者教育所。 
Braden, G.(2022)。無量之網 2:正確祈禱,連結萬物為你效力!(達娃譯)。
橡實文化。(原著出版於2016年) 
Chinen, A. B.(2020)。喚醒世界:女性和英勇女性特質經典故事(丁舒偉譯)。 
天衛文化。(原著出版於1996年) 
Cooper, D(2010)。靈性法則之光(沈文玉譯)。生命潛能。(原著出版於
2004年) 
Cortright, B.(2005)。超個人心理治療:心理治療與靈性轉化的整合(易之新
譯)。心靈工坊。(原著出版於1997年) 
Estés, C. P.(2017)。與狼同奔的女人(吳菲菲譯)。心靈工坊。(原著出版於
1989 年) 
Hopcke, R. H.(2002)。導讀榮格(蔣韜譯)。立緒。(原著出版於1989年) 
Kim, J.-H.(2018)。理解敘說探究:以故事的雕琢與分析作為研究(張曉佩、
卓秀足譯)。心理。(原著出版於2015年) 
Kribbe, P(2010)。靈性煉金術: 激勵人心的約書亞靈訊(郭宇、林荊、李平
譯)。方智。(原著出版於2008年) 
Lefebvre, A.(1998)。超個人心理學:心理學的新典範(若水譯)。桂冠。
(原著出版於1992年) 
May, R.(2016)。哭喊神話:羅洛‧梅經典(朱侃如譯)。立緒。(原著出版於
1991 年) 
Merriam, S. B., & Caffarella, R. S.(2004)。終身學習全書:成人教育總論(楊
惠君譯)。商周出版。(原著出版於1999年) 
141 
Murdock, M.(2022)。女英雄的旅程(吳菲菲譯)。心靈工坊。(原著出版於
1990 年) 
Wilber, K.(2005)。萬法簡史(廖士德譯)。心靈工坊。(原著出版於1996
年) 
Wilcock, D.(2012)。源場:超自然關鍵報告(白樂、隋芃譯)。橡實文化。
(原著出版於2011年) 
