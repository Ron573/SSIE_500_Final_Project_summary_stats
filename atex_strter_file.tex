\documentclass[11pt]{article}

% ---------- 1.  PACKAGES ----------
\usepackage[margin=1in]{geometry}
\usepackage{booktabs}
\usepackage{siunitx}      % \num, \SI
\sisetup{round-mode=places,round-precision=3}
\usepackage{graphicx}
\usepackage{pgfplots}     % for Figure 1 (probability spectrum)
\usetikzlibrary{pgfplots.groupplots}
\pgfplotsset{compat=1.18}
\usepackage{hyperref}
\hypersetup{colorlinks=true,linkcolor=blue,citecolor=blue,urlcolor=blue}

% ---------- 2.  BIBLIOGRAPHY STUB ----------
\usepackage[backend=biber,style=authoryear]{biblatex}
\addbibresource{refs.bib}

% ---------- 3.  AUTO-IMPORT OF RESULTS ----------
% The Python script dumps a file results.tex that contains only
% \newcommand{\<macro>}{<number>} lines.  We input it here so all
% numbers are live.
\IfFileExists{results.tex}{\begin{table}[ht]
\centering
\caption{Aggregate statistics per file.\label{tab:agg}}
\begin{tabular}{l
                S[table-format=7.0]
                S[table-format=9.0]
                S[table-format=9.0]
                S[table-format=1.3]
                S[table-format=1.2e1]
                S[table-format=1.2e1]
                S[table-format=2.1]}
\toprule
{File} & {Chars} & {Bits$_\text{raw}$} & {Bits$_\text{Huff}$} &
{$\rho$} & {$E_\text{raw}$} & {$E_\text{Huff}$} & {$\Delta E$\,\%}\\
\midrule
\input{tables/file_rows.tex}
\midrule
\textbf{Mean} & & & &
\textbf{\MeanComprRatio} & & &
\textbf{\MeanEnergySaving}\\
\bottomrule
\end{tabular}
\end{table}

\begin{figure}[ht]
\centering
\input{figures/prob_spectrum.tex}
\caption{Complementary cumulative distribution of transition
probabilities for \emph{foo.txt}.  The heavy-tailed curve (blue) is
pre-compression; the near-Bernoulli curve (orange) is post-compression.}
\end{figure}

}{\message{results.tex not found}}

\title{High-Order Markov Analysis and Landauer-Limited Energy Savings\\
       under Huffman Compression}
\author{Your Name}
\date{\today}

\begin{document}
\maketitle
\begin{abstract}
We quantify the information-theoretic and thermodynamic implications of
Huffman compression on a GitHub corpus consisting of \textbf{\num{\Nfiles}}
plain-text files (total \textbf{\num{\NcharsTotal}} characters).
Third-order character-level Markov models are estimated \emph{pre} and
\emph{post} compression.  Compression achieves an average
\textbf{\num[round-precision=2]{\MeanComprRatio}} × reduction in bit-length,
lowering the Landauer erase-energy bound by
\textbf{\num{\MeanEnergySaving}\%}.  Despite this size reduction, the
post-compression chains retain significant third-order structure
(perplexity \textbf{\num{\MeanPostPPL}}).  We discuss the consequences for
cascade compression and green storage architectures.
\end{abstract}

\section{Introduction}
\input{sections/introduction.tex}

\section{Methodology}
\subsection{Corpus Acquisition}
All plaintext files in commit \texttt{\CommitHash} of
\href{https://github.com/\Repo}{\Repo} were analysed individually; no
cross-file concatenation was performed so that file-local statistics remain
interpretable by developers.

\subsection{Pre-processing Robustness Upgrades}
\begin{enumerate}
  \item \textbf{Encoding validation} — Each byte stream is decoded as
        UTF-8 with \texttt{strict} error policy.  Any failure aborts the
        pipeline and records the offending filename.
  \item \textbf{Normalisation} — Unicode canonical composition
        (NFC) is applied to avoid code-point duplication.
  \item \textbf{Whitespace policy} — All carriage returns are mapped to
        LF; trailing spaces are preserved because they influence entropy.
\end{enumerate}

\subsection{Third-Order Markov Modelling}
Counts $c(\mathbf{x}_{n-3}^{n-1}, x_n)$ are gathered in a
single pass; the alphabet is the set of observed Unicode code points
($k = \num{\AlphabetSize}$).  We apply Laplace $+1/k$ smoothing to avoid
singular probability rows, then convert to probabilities
$p = (c+1/k)/(\sum c + 1)$.

\subsection{Huffman Coding and Bit-Sequence Extraction}
For each file we derive a canonical Huffman code from the marginal
distribution $P(x)$ and emit the compressed bit-stream $B$.
The third-order analysis is then repeated on $B$ with
alphabet $\{0,1\}$.

\subsection{Landauer Bound Computation}
The minimum erasure energy is $E = |B|\;k_B T \ln 2$ with
$T=\SI{300}{\kelvin}$.  We report both absolute energies (joule) and the
percentage saving $\Delta E$ relative to the raw stream.

\subsection{Software and Reproducibility}
Python 3.11; deterministic seed $42$; script commit
\texttt{\ScriptHash}.  Running
\begin{verbatim}
$ python build_markov_and_landauer.py --batch repo/*.txt
\end{verbatim}
creates \texttt{markov\_pre.json}, \texttt{markov\_post.json} and a
\texttt{results.tex} file that this document auto-imports.



\section{Results}
\begin{table}[ht]
\centering
\caption{Aggregate statistics per file.\label{tab:agg}}
\begin{tabular}{l
                S[table-format=7.0]
                S[table-format=9.0]
                S[table-format=9.0]
                S[table-format=1.3]
                S[table-format=1.2e1]
                S[table-format=1.2e1]
                S[table-format=2.1]}
\toprule
{File} & {Chars} & {Bits$_\text{raw}$} & {Bits$_\text{Huff}$} &
{$\rho$} & {$E_\text{raw}$} & {$E_\text{Huff}$} & {$\Delta E$\,\%}\\
\midrule
\input{tables/file_rows.tex}
\midrule
\textbf{Mean} & & & &
\textbf{\MeanComprRatio} & & &
\textbf{\MeanEnergySaving}\\
\bottomrule
\end{tabular}
\end{table}

\begin{figure}[ht]
\centering
\input{figures/prob_spectrum.tex}
\caption{Complementary cumulative distribution of transition
probabilities for \emph{foo.txt}.  The heavy-tailed curve (blue) is
pre-compression; the near-Bernoulli curve (orange) is post-compression.}
\end{figure}



\section{Discussion}
\input{sections/discussion.tex}

\printbibliography
\end{document}

