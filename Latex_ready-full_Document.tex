% ====================================================================
%  High-Order Markov Analysis & Landauer-Limited Energy Savings
%  Manuscript – camera-ready template
% ====================================================================
\documentclass[11pt,a4paper]{article}

% -------------------------- packages --------------------------------
\usepackage[utf8]{inputenc}          % UTF-8 source
\usepackage[T1]{fontenc}
\usepackage{lmodern}
\usepackage{microtype}

\usepackage{geometry}
\geometry{margin=2.5cm}

\usepackage{booktabs}
\usepackage{siunitx}
\sisetup{detect-all,round-mode=places,round-precision=2}

\usepackage{graphicx}
\usepackage{subcaption}

\usepackage{pgfplots}
\pgfplotsset{compat=1.18}

\usepackage{hyperref}
\usepackage[capitalize,nameinlink]{cleveref}

\usepackage{csquotes}      % for biblatex
\usepackage[
    backend=biber,
    style=authoryear,
    maxcitenames=2,
    maxbibnames=99
]{biblatex}
\addbibresource{refs.bib}

% -------------------- auto-generated macros -------------------------
\begin{table}[ht]
\centering
\caption{Aggregate statistics per file.\label{tab:agg}}
\begin{tabular}{l
                S[table-format=7.0]
                S[table-format=9.0]
                S[table-format=9.0]
                S[table-format=1.3]
                S[table-format=1.2e1]
                S[table-format=1.2e1]
                S[table-format=2.1]}
\toprule
{File} & {Chars} & {Bits$_\text{raw}$} & {Bits$_\text{Huff}$} &
{$\rho$} & {$E_\text{raw}$} & {$E_\text{Huff}$} & {$\Delta E$\,\%}\\
\midrule
\input{tables/file_rows.tex}
\midrule
\textbf{Mean} & & & &
\textbf{\MeanComprRatio} & & &
\textbf{\MeanEnergySaving}\\
\bottomrule
\end{tabular}
\end{table}

\begin{figure}[ht]
\centering
\input{figures/prob_spectrum.tex}
\caption{Complementary cumulative distribution of transition
probabilities for \emph{foo.txt}.  The heavy-tailed curve (blue) is
pre-compression; the near-Bernoulli curve (orange) is post-compression.}
\end{figure}



% ------------------------- document ---------------------------------
\begin{document}

\title{High-Order Markov Analysis and\\
       Landauer-Limited Energy Savings under Huffman Compression}
\author{Your Name\thanks{Affiliation, address.  
        Email: \texttt{your.name@inst.edu}}}
\date{\today}
\maketitle

\begin{abstract}
We investigate the relationship between third-order character
dependencies in a text corpus and the thermodynamic minimum energy
required to erase the compressed representation.
Using canonical Huffman coding and Landauer's principle at
\SI{300}{\kelvin}, we achieve a mean compression ratio of
\MeanComprRatio\ (95 \% CI: \MeanComprRatioLo–\MeanComprRatioHi), which
translates to an average energy saving of
\SI{\MeanEnergySaving}{\percent} (95 \% CI:
\MeanEnergySavingLo–\MeanEnergySavingHi).
A χ² order-selection test (\(\chi^{2}_{\ChiTwoDF} =
\ChiTwo\), \(p = \ChiTwoP\)) confirms that third-order
dependencies explain significantly more variance than a unigram model.
\end{abstract}

\section{Introduction}
Modern storage and communication systems frequently employ reversible
or near-reversible encodings to minimise energy dissipation.  At the
thermodynamic limit, each bit erased costs at least \(k_{\mathrm{B}} T
\ln 2\) joules~\parencite{Landauer1961}.  This paper combines
information-theoretic analysis and physical considerations to quantify
the attainable savings on a real-world corpus.

\section{Related Work}
\label{sec:related}
Briefly review Markov entropy estimation, Huffman coding efficiency and
recent efforts to approach the Landauer limit in computing
hardware~\parencite{Bennett2003,Anders2019}.

\section{Methodology}
\label{sec:method}
\subsection{Corpus}
We analyse \Nfiles\ UTF-8 text files (\NcharsTotal\ characters) drawn
from \<describe source\>.

\subsection{Pipeline}
The entire workflow is encoded in
\texttt{build\_markov\_and\_landauer.py}.  Key steps:
\begin{enumerate}
  \item Unicode NFC normalisation and ingestion.
  \item Third-order (\(n = 3\)) character Markov counting; entropy
        \(H_{3}\) estimated per file.
  \item Canonical Huffman coding; compressed bit-stream emitted.
  \item Landauer energy computed as \(E = N_{\mathrm{bits}} k_{\mathrm{B}} T \ln 2\).
  \item \(2\,000\)-iteration bootstrap for corpus-level CIs
        (fixed seed = 42).
  \item χ² goodness-of-fit: \(n=3\) vs.\ unigram.
\end{enumerate}

\section{Results}
\label{sec:results}
\subsection{Per-file statistics}
\begin{table}[htbp]
  \centering
  \caption{Per-file compression and energy metrics.}
  \label{tab:perfile}
  \begin{tabular}{lrrrr}
    \toprule
    File & \#\,Chars & Raw bits & Huffman bits & Ratio \\ \midrule
    \input{results/tables/file_rows.tex}
    \bottomrule
  \end{tabular}
\end{table}

\subsection{Corpus aggregates}
The mean compression ratio
(\autoref{tab:perfile}) is \MeanComprRatio\ with a
\SI{\MeanEnergySaving}{\percent} average energy saving.

\subsection{Model order test}
The χ² statistic of \ChiTwo\ on \ChiTwoDF\ degrees of freedom
(\(p = \ChiTwoP\)) rejects the null that unigram frequencies alone
explain the observed third-order counts.

\begin{figure}[htbp]
  \centering
  \includegraphics[width=.75\linewidth]{figures/ratio_hist.pdf}
  \caption{Histogram of per-file compression ratios.}
  \label{fig:ratio-hist}
\end{figure}

\section{Discussion}
Interpret the practical implications for low-energy storage systems
and compare with prior art.

\section{Conclusion}
We provide a fully reproducible link between high-order statistical
structure and thermodynamic cost, achieving substantial energy savings
relative to raw UTF-8 storage.

\printbibliography

\end{document}

